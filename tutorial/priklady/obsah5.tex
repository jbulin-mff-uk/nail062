\section*{NAIL062 V\&P Logika: 5. sada příkladů -- Rezoluční metoda}
% po 5. přednášce


\subsection*{Cíle výuky:} Po absolvování cvičení student

    \begin{itemize}\setlength{\itemsep}{0pt}
        \item zná potřebné pojmy z rezoluční metody (rezoluční pravidlo, rezolventa, rezoluční důkaz/zamítnutí, rezoluční strom), umí je formálně definovat, uvést příklady
        \item umí pracovat s výroky v CNF a jejich modely v množinové reprezentaci
        \item umí sestrojit rezoluční zamítnutí dané (i nekonečné) CNF formule (existuje-li), a také nakreslit příslušný rezoluční strom
        \item zná pojem stromu dosazení, umí ho formálně definovat a pro konkrétní CNF formuli sestrojit
        \item umí aplikovat rezoluční metodu k řešení daného problému (slovní úlohy, aj.)
    \end{itemize}
    

\section*{Příklady na cvičení}


\begin{problem}

    Označme jako $\varphi$ výrok $\neg (p \vee q) \to (\neg p \wedge \neg q)$. Ukažte, že $\varphi$ je tautologie:
    \begin{enumerate}[(a)]
        \item Převeďte $\neg \varphi$ do CNF a zapište výsledný výrok jako formuli $S$ v množinové reprezentaci.
        \item Najděte rezoluční zamítnutí $S$.
    \end{enumerate}

    \begin{solution}

        \begin{enumerate}[(a)]
            \item Pomocí ekvivalentních úprav: $\neg\varphi=\neg(\neg (p \vee q) \to (\neg p \wedge \neg q))\sim\neg(\neg \neg (p \vee q) \lor (\neg p \wedge \neg q))\sim\neg(p \vee q \vee (\neg p \wedge \neg q))\sim \neg p \land \neg q \land  \neg (\neg p \wedge \neg q)\sim\neg p \land \neg q \land  (p \lor q)$
            $$
            S = \{\{\neg p\},\{\neg q\},\{p,q\}\}
            $$
            \item Rezoluční zamítnutí: $\{\neg p\},\{p,q\},\{q\},\{\neg q\},\square$ (nakreslete si rezoluční strom).
        \end{enumerate}
                    
    \end{solution}

\end{problem}


\begin{problem}

    Dokažte rezolucí, že v $T=\{\neg p \to \neg q,\neg q \to \neg r, (r\to p)\to s\}$ platí výrok $s$.

    \begin{solution}
        Teorii  $T\cup\{\neg s\}$ převedeme do CNF:, a zapíšeme v množinové reprezentaci. Máme $(r\to p)\to s\sim \neg(\neg r\lor p)\lor s\sim (r\land\neg p)\lor s\sim (r\lor s)\land (\neg p\lor s)$, ostatní axiomy se převedou snadno. Dostaneme: $S=\{\{p,\neg q\},\{q,\neg r\},\{r,s\},\{\neg p,s\},\{\neg s\}\}$. Rezoluční zamítnutí znázorníme rezolučním stromem:

        \begin{center}
            \begin{forest}
            for tree={grow=north}
            [$ \square $
                [$ \{\neg q\} $
                    [{$ \{p, \neg q\} $}]   
                    [{$ \{\neg p\} $}
                        [{$ \{\neg s\} $}]
                        [{$ \{\neg p, s\} $}]
                    ]
                ]
                [$ \{q\} $
                    [{$ \{q, \neg r\} $}]
                    [{$ \{r\} $}
                        [{$ \{\neg s\} $}]
                        [{$ \{r, s\} $}]
                    ]
                ]
            ]
            \end{forest}
        \end{center}
                    
    \end{solution}

\end{problem}


\begin{problem}
    
    Nechť prvovýroky $r$, $s$, $t$  reprezentují (po řadě), že \emph{``Radka / Sára / Tom je ve škole''} a označme $\mathbb{P}=\{r,s,t\}$. Víme, že:
    \begin{itemize}\it
        \item Není-li Tom ve škole, není tam ani Sára.
        \item Radka bez Sáry do školy nechodí.
        \item Není-li Radka ve škole, je tam Tom.
    \end{itemize}
    \begin{enumerate}[(a)]
        \item Formalizujte naše znalosti jako teorii $T$ v jazyce $\mathbb P$.
        \item Rezoluční metodou dokažte, že z $T$ vyplývá, že \emph{Tom je ve škole}: Napište formuli $S$ v množinové reprezentaci, která je nesplnitelná, právě když to platí, a najděte rezoluční zamítnutí $S$. Nakreslete rezoluční strom.
        \item Určete množinu modelů teorie $T$.
    \end{enumerate}

    \begin{solution}

        \begin{enumerate}[(a)]
            \item $T=\{\neg t\limplies\neg s, \neg(r\land\neg s), \neg r\limplies t\}$
            \item $S$ získáme z teorie $T\cup\{\neg t\}$ převodem do CNF: $S=\{\{t,\neg s\},\{\neg r,s\},\{r,t\},\{\neg t\}\}$
            \begin{center}
                \begin{forest}
                for tree={grow=north}
                [$ \square $
                    [{$ \{\neg s\} $}
                        [{$ \{\neg t\} $}]
                        [{$ \{t, \neg s\} $}]
                    ]
                    [$ \{s\} $
                        [{$ \{\neg r, s\} $}]
                        [{$ \{r\} $}
                            [{$ \{\neg t\} $}]
                            [{$ \{r, t\} $}]
                        ]
                    ]
                ]
                \end{forest}
            \end{center}
            \item Využijeme toho, že $T\models t$ (dokázali jsme v (b)). První a třetí axiom jsou díky tomu splněny, $T\sim\{t,\neg(r\land\neg s)\}$. Z toho snadno $\M(T)=\{(0,0,1),(0,1,1),(1,1,1)\}$.            
        \end{enumerate}
                    
    \end{solution}

\end{problem}


\begin{problem}
        
    Zkonstruujte \emph{strom dosazení} pro následující formuli. Na základě tohoto stromu sestrojte rezoluční zamítnutí, dle postupu z důkazu Věty o úplnosti rezoluce.
    $$
    S=\{\{p,r\},\{q,\neg r\},\{\neg q\},\{\neg p,t\},\{\neg s\},\{s,\neg t\}\}
    $$

    \begin{solution}
        Přednostně větvíme přes výrokové proměnné v jednotkových klauzulích. (Jakmile narazíme na prázdnou klauzuli, víme, že větev je sporná, zbytek formule nepotřebujeme, zde kvůli nedostatku místa nebudeme zapisovat.)
        \begin{center}
            \begin{forest}    
            [{$S$}
                [{$S^q=\{\{p,r\},\square,\dots\}$}, tikz={\node[fit to=tree,label=below:{$\otimes$}] {};}]
                [{$S^{\bar q}=\{\{p,r\},\{\neg r\},\{\neg p,t\},\{\neg s\},\{s,\neg t\}\}$}
                    [{$S^{\bar qr}=\{\square,\dots\}$}, tikz={\node[fit to=tree,label=below:{$\otimes$}] {};}]
                    [{$S^{\bar q\bar r}=\{\{p\},\{\neg p,t\},\{\neg s\},\{s,\neg t\}\}$}
                        [{$S^{\bar q \bar r p}=\{\{t\},\{\neg s\},\{s,\neg t\}\}$}
                            [{$S^{\bar q \bar r p s}=\{\{t\},\square\}$}, tikz={\node[fit to=tree,label=below:{$\otimes$}] {};}]
                            [{$S^{\bar q \bar r p \bar s}=\{\{t\},\{\neg t\}\}$}
                               [{$S^{\bar q \bar r p \bar s t}=\{\square\}$}, tikz={\node[fit to=tree,label=below:{$\otimes$}] {};}]
                               [{$S^{\bar q \bar r p \bar s \bar t}=\{\square\}$}, tikz={\node[fit to=tree,label=below:{$\otimes$}] {};}]                                
                            ]
                        ]
                        [{$S^{\bar q \bar r \bar p}=\{\square,\dots\}$}, tikz={\node[fit to=tree,label=below:{$\otimes$}] {};}]
                    ]
                ]
            ]
            \end{forest}
        \end{center}
        Strom dosazení dává ``návod'', jak sestrojit rezoluční zamítnutí (to je klíčem k důkazu věty o úplnosti rezoluce). Postupujeme od listů ke kořeni, neboli podle počtu proměnných ve formulích. Pro formule na listech stromu dosazení máme jednoprvková rezoluční zamítnutí $\square$.

        Formule $S^{\bar q \bar r p \bar s}=\{\{t\},\{\neg t\}\}$ má jednokrokové rezoluční zamítnutí:

        \begin{center}
            \begin{forest}    
            [{$\square$}
                [{$\{\neg t\}$}]
                [{$\{t\}$}]
            ]
            \end{forest}
        \end{center}

        Jak vzniklo? Z rezolučního zamítnutí $\square$ formule $S^{\bar q \bar r p \bar s t}$ vyrobíme rezoluční důkaz klauzule $\{\neg t\}$ z $S^{\bar q \bar r p \bar s}$, a to tak, že pro každý list, který vznikl odebráním literálu $\neg t$, vrátíme $\neg t$ do něj i do všech klauzulí nad ním. (Zde máme jen jeden list, což je zároveň kořen $\square$.)

        Analogicky vyrobíme rezoluční důkaz $\{t\}$ z $S^{\bar q \bar r p \bar s \bar t}$ (přidáváme do vrcholů literál $t$). A nakonec přidáme jeden rezoluční krok, který z $\{\neg t\}$ a $\{t\}$ odvodí $\square$. (Pokud by žádný list nevznikl odebráním literálu z klauzule z $S^{\bar q \bar r p \bar s}$, znamená to, že rezoluční zamítnutí, které máme, je už i rezolučním zamítnutím $S^{\bar q \bar r p \bar s}$.)

        Stejně postupujeme ve stromu výše, pro $S^{\bar q \bar r p}$, $S^{\bar q \bar r}$, $S^{\bar q}$, a nakonec pro $S$:

            \begin{forest}    
                [{$\square$}
                    [{$\{\neg s\}$}]
                    [{$\{s\}$}
                        [{$\{\neg t,s\}$}]
                        [{$\{t\}$}]
                    ]
                ]
            \end{forest}        
            \begin{forest}    
                [{$\square$}
                    [{$\{\neg p\}$}
                        [{$\{\neg s\}$}]
                        [{$\{s,\neg p\}$}
                            [{$\{\neg t,s\}$}]
                            [{$\{t,\neg p\}$}]
                        ]
                    ]
                    [{$\{p\}$}]
                ]
            \end{forest}
            \begin{forest}    
                [{$\square$}    
                    [{$\{\neg r\}$}]
                    [{$\{r\}$}
                        [{$\{\neg p\}$}
                            [{$\{\neg s\}$}]
                            [{$\{s,\neg p\}$}
                                [{$\{\neg t,s\}$}]
                                [{$\{t,\neg p\}$}]
                            ]
                        ]
                        [{$\{p,r\}$}]
                    ]
                ]
            \end{forest}
            \begin{forest}
                [{$\square$}  
                    [{$\{\neg q\}$}]
                    [{$\{q\}$}    
                        [{$\{\neg r,q\}$}]
                        [{$\{r\}$}
                            [{$\{\neg p\}$}
                                [{$\{\neg s\}$}]
                                [{$\{s,\neg p\}$}
                                    [{$\{\neg t,s\}$}]
                                    [{$\{t,\neg p\}$}]
                                ]
                            ]
                            [{$\{p,r\}$}]
                        ]
                    ]                    
                ]
            \end{forest}
        
        \medskip

        Ověřte, že výsledný strom opravdu reprezentuje rezoluční zamítnutí $S$. Všimněte si, jak jeho tvar kopíruje tvar stromu dosazení. (V našem případě je strom ``chlupatá cesta'', což obecně být nemusí, ale konstrukce funguje stejně.)
                    
    \end{solution}
    
\end{problem}
        
        
\section*{Další příklady k procvičení}
        

\begin{problem}
    
    Najděte rezoluční zamítnutí následujících výroků:
    \begin{enumerate}[(a)]
        \item $\neg(((p\to q)\to \neg q)\to \neg q)$
        \item $(p\leftrightarrow (q\to r))\wedge((p\leftrightarrow q)\wedge(p\leftrightarrow \neg r))$        
    \end{enumerate}
\end{problem}


\begin{problem}

    Tonia a Fabio nám popisují svůj nejnovější recept na nejlepší pizzu na světě.
    \begin{itemize}
        \item Tonia řekla: ``Do receptu patří ančovičky nebo bazalka nebo česnek.''
        \item Tonia také řekla: ``Jestli tam nepatří dušená šunka, nepatří tam ani bazalka.''
        \item Fabio řekl: ``Do receptu patří dušená šunka.'' 
        \item Fabio dále řekl: ``Nepatří tam ančovičky ani bazalka, ale patří tam česnek.''
    \end{itemize}
    Víme, že Tonia vždy mluví pravdu, zatímco Fabio vždy lže.

    \begin{enumerate}[(a)]
        \item Vyjádřete naše znalosti jako výrokovou teorii $T$ v jazyce $\mathbb P=\{a,b,c,d\}$, kde výrokové proměnné mají po řadě význam ``do receptu patří ančovičky/bazalka/česnek/dušená šunka''.
        \item Pomocí rezoluční metody dokažte, že z teorie $T$ vyplývá, že ``do receptu patří ančovičky''. Nakreslete rezoluční strom.
    \end{enumerate}

\end{problem}


\begin{problem}

    Celá čísla postihla záhadná nemoc šířící se (v diskrétních krocích) dle následujících pravidel (platících pro všechna čísla ve všech krocích).
    \begin{enumerate}[(i)]\it
        \item Zdravé číslo onemocní, právě když je právě jedno sousední číslo nemocné (v předchozím čase).
        \item Nemocné číslo se uzdraví, právě když je předchozí číslo nemocné (v předchozím čase).
        \item V čase $0$ bylo nemocné číslo $0$, ostatní čísla byla zdravá.
    \end{enumerate}
        %(Sousedními čísly čísla $i$ myslíme $i-1$ a $i+1$, předchozím číslem myslíme $i-1$.)
    \begin{enumerate}[(a)]
        \item Napište teorie $T_1, T_2, T_3$ vyjadřující (po řadě) tvrzení $(i), (ii), (iii)$ nad množinou prvovýroků $\mathbb{P}=\{p_i^t \mid i\in\mathbb{Z}, t\in\mathbb{N}_0\}$, kde prvovýrok $p_i^t$ vyjadřuje, že ``{\it číslo $i$ je v čase $t$ nemocné.}''
        \item Převeďte axiomy z $T_1, T_2, T_3$ do CNF a napište teorii $S$ v množinové reprezentaci, která je nesplnitelná, právě když $T_1 \cup T_2 \cup T_3 \models \neg p_1^2$, tj.: ``{\it Číslo $1$ je zdravé v čase $2$.}'' (Stačí převést jen konkrétní axiomy z $T_1,T_2,T_3$, ze kterých plyne $\neg p_1^2$, a do $S$ uvést jen příslušné klauzule.)
        \item Rezolucí dokažte, že $S$ je nesplnitelná. Zamítnutí znázorněte rezolučním stromem.
    \end{enumerate}

\end{problem}

        
\section*{K zamyšlení}


\begin{problem}
    Dokažte podrobně, že je-li $S=\{C_1,C_2\}$ splnitelná a $C$ je rezolventa $C_1$ a $C_2$, potom je i $C$ splnitelná.
\end{problem}
