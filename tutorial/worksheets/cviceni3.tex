\documentclass[a4paper,12pt]{article}

%% slide-specific

\usetheme[progressbar=frametitle]{metropolis}
%\usecolortheme{spruce}
%\metroset{block=fill}

% block indentation workaround
% map defaulf block to oldblock
\let\oldblock\block
\let\endoldblock\endblock
% change block by adding smallskip
\renewenvironment{block}[1]
    {\begin{oldblock}{#1}
        \smallskip
    }
    { 
    \end{oldblock}
    }

% define Metropolis colors    
\definecolor{mAlert}{HTML}{EB811B}
\definecolor{mExample}{HTML}{14B03D}
\definecolor{mBlock}{HTML}{23373b}

\usepackage[most]{tcolorbox}

%\newcommand{\myexample}[1]{\leavevmode\textcolor{mExample}{#1}}
\newcommand{\myalert}[1]{
\begin{tcolorbox}[colback=mAlert!10, enhanced, boxrule=0pt, boxsep=-1mm, frame hidden, left=2mm, right=2mm]
    {#1}  
\end{tcolorbox}
}
\newcommand{\myexample}[1]{
\begin{tcolorbox}[colback=mExample!10, enhanced, boxrule=0pt, boxsep=-1mm, frame hidden, left=2mm, right=2mm]
    {#1}  
\end{tcolorbox}
}
\newcommand{\myblock}[1]{
\begin{tcolorbox}[colback=mBlock!10, enhanced, boxrule=0pt, boxsep=-1mm, frame hidden, left=2mm, right=2mm]
    {#1}  
\end{tcolorbox}
}

\newcommand{\mystructure}[1]{\mathcal{#1}}



% \newcommand{\myexamplemath}[1]{
% \begin{tcolorbox}[colback=mExample!10, enhanced, boxrule=0pt, frame hidden]
%     \ensuremath{#1}  
% \end{tcolorbox}
% }


%% packages
\usepackage{amsmath,amssymb,amsthm}
\usepackage{booktabs}
\usepackage[czech]{babel}
\usepackage{enumerate}
\usepackage{forest}
\usepackage{multicol}
% \usepackage{tcolorbox}
\usepackage{tikz}
    \usetikzlibrary{arrows.meta}
%\usepackage[unicode]{hyperref}
\usepackage[utf8]{inputenc}
\usepackage{xfrac}

% %% theorems
% \theoremstyle{plain}
%     \newtheorem{theorem}{Věta}[section]
%     \newtheorem*{theorem-unnumbered}{Věta}
%     \newtheorem{proposition}[theorem]{Tvrzení}
%     \newtheorem{corollary}[theorem]{Důsledek}
%     \newtheorem{lemma}[theorem]{Lemma}
%     \newtheorem{observation}[theorem]{Pozorování}
% \theoremstyle{definition}
%     \newtheorem{definition}[theorem]{Definice}
%     \newtheorem*{algorithm}{Algoritmus}
% \theoremstyle{remark}
%     \newtheorem{remark}[theorem]{Poznámka}
%     \newtheorem{example}[theorem]{Příklad}
%     \newtheorem{exercise}{Cvičení}[chapter]
%     \newtheorem*{solution}{Řešení}

%% macros and definitions
\DeclareMathOperator{\Aut}{Aut}
\DeclareMathOperator{\Conseq}{Csq}
\DeclareMathOperator{\DeLO}{DeLO}
\DeclareMathOperator{\dom}{dom}
\DeclareMathOperator{\Fm}{Fm}
\DeclareMathOperator{\M}{M}
%\DeclareMathOperator{\Proof}{Proof}
\DeclareMathOperator{\rng}{rng}
\DeclareMathOperator{\Term}{Term}
\DeclareMathOperator{\Th}{Th}
\DeclareMathOperator{\Thm}{Thm}
\DeclareMathOperator{\Tree}{Tree}
\DeclareMathOperator{\Var}{Var}
\DeclareMathOperator{\VF}{VF}

\newcommand{\A}{\structure{A}}
\newcommand{\B}{\structure{B}}
\newcommand{\Con}{\mathit{Con}}
\newcommand{\disjointunion}{\mathbin{\dot{\sqcup}}}
\newcommand{\F}{\ensuremath{\mathrm{F}}}
\newcommand{\landsymb}{{\land}}
\newcommand{\lbin}{\mathbin{\square}}
\newcommand{\lbinsymb}{{\lbin}}
\newcommand{\liff}{\mathbin{\leftrightarrow}}
\newcommand{\liffsymb}{{\liff}}
\newcommand{\limplies}{\mathbin{\rightarrow}}
\newcommand{\limpliessymb}{{\limplies}}
\newcommand{\lorsymb}{{\lor}}
\newcommand{\Prf}{\mathit{Prf}}
\newcommand{\proves}{\vdash}
%\newcommand{\structure}[1]{\mathcal{#1}}
\newcommand{\todo}{[TODO]}
\newcommand{\T}{\ensuremath{\mathrm{T}}}
\newcommand{\union}{\mathbin{\cup}}



\begin{document}

\section*{NAIL062 V\&P Logika: 3. cvičení}
% po 2. přednášce

% podzim 2023: plán: 1, 2, 3ab, 4a, 5, 8, 9a, 10

\textbf{Témata:} 
Syntaxe a sémantika výrokové logiky. Převod do CNF a DNF. Univerzálnost logických spojek.


\medskip\begin{problem}
    Mějme teorii $T=\{\neg q \to (\neg p \vee q),\ \neg p \to q,\ r \to q\}$ v jazyce $\{p, q, r\}$.
    \begin{enumerate}
        \item Uveďte příklad následujícího: výrok pravdivý v $T$, lživý v $T$, nezávislý v $T$, splnitelný v $T$, a dvojice $T$-ekvivalentních výroků.
        \item Které z následujících výroků jsou pravdivé, lživé, nezávislé, splnitelné v $T$? $T$-ekvivalentní? 
        $$
        p, \ \neg q, \ \neg p\vee q, \ p\to r,\ \neg q\to r, \ p\vee q\vee r
        $$
    \end{enumerate}
\end{problem}


\medskip\begin{problem}
    Uvažme nekonečnou výrokovou teorii $T=\{p_i \to p_{i+1}\mid i\in \mathbb{N}\}$ nad $\mathrm{var}(T)$. 
    \begin{enumerate}
        \item Které výroky ve tvaru  $p_i \to p_j$ jsou důsledky $T$?
        \item Určete všechny modely $T$.
    \end{enumerate}
    \end{problem}
    
    
\medskip\begin{problem} Určete množinu modelů dané formule. Využijte toho, že je v DNF resp. v CNF.
\begin{enumerate}
    \item $(\neg p_1 \wedge \neg p_2)\vee( \neg p_1 \wedge p_2)\vee( p_1 \wedge \neg p_2)\vee( p_2 \wedge \neg p_3)$
    \item $(\neg p_1 \vee \neg p_2)\wedge( \neg p_1 \vee p_2)\wedge( p_1 \vee \neg p_2)\wedge( p_2 \vee \neg p_3)$
    \item $(p_1 \wedge  \neg p_2 \wedge  p_3 \wedge  \neg p_4 )\vee(p_2 \wedge  p_3 \wedge  \neg p_4 )\vee(\neg p_3)\vee(p_2 \wedge  p_4)\vee(p_1 \wedge  p_3 \wedge  p_5 )$
    \item $(p_1 \vee \neg p_2 \vee p_3 \vee \neg p_4 )\wedge(p_2 \vee p_3 \vee \neg p_4 )\wedge(\neg p_3)\wedge(p_2 \vee p_4)\wedge(p_1 \vee p_3 \vee p_5 )$
\end{enumerate}
\end{problem}
    
    
\medskip\begin{problem} Převeďte následující výroky do CNF a DNF 
\begin{enumerate}
    \item $(\neg p \vee q)\to (\neg q \wedge r)$,
    \item $(\neg p \to (\neg q \to r))\to p$,
\end{enumerate}
Proveďte to:
\begin{itemize}
    \item[(I)] sémanticky (pomocí pravdivostní tabulky),
    \item[(II)]  ekvivalentními úpravami.
\end{itemize}    

\end{problem}
     
    
\medskip\begin{problem} Najděte (co nejkratší) CNF a DNF reprezentace Booleovské funkce $\mathrm{maj}: {^3}2\to 2$, která vrací převládající hodnotu mezi 3 vstupy.
\end{problem}
    
    
\medskip\begin{problem} Najděte CNF a DNF reprezentaci $n$-ární parity, tj. Booleovské funkce $\mathrm{par}: {^n}2\to 2$,
která vrací XOR všech vstupních hodnot:
$$
\mathrm{par}(x_1,\dots,x_n)=(x_1+\dots+x_n)\bmod 2
$$
Zkuste to pro malé hodnoty $n$.
\end{problem}
    
    
\medskip\begin{problem} Buď $\mathbb P$ spočetně nekonečná množina prvovýroků. 
\begin{itemize}
    \item Ukažte, že již neplatí, že každou $K\subseteq \mathrm{M}_\mathbb P$ lze axiomatizovat výrokem v CNF i výrokem v DNF.
    \item  Uveďte příklad množiny modelů $K$, kterou nelze axiomatizovat ani výrokem v CNF, ani výrokem v DNF.
\end{itemize}
\end{problem}


\medskip\begin{problem}
    Ukažte, že $\wedge$ a $\vee$ nestačí k definování všech Booleovských operátorů, tj. že $\{\wedge,\vee\}$ není \emph{univerzální} množina logických spojek.
    \end{problem}
    
    \medskip\begin{problem} Jsou následující množiny logických spojek univerzální? Zdůvodněte.
    \begin{enumerate}
        \item $\{\downarrow\}$ kde $\downarrow$ je Peirce arrow (NOR),
        \item $\{\uparrow\}$ kde $\uparrow$ je Sheffer stroke (NAND),
        \item $\{\vee, \rightarrow, \leftrightarrow\}$,
        \item $\{\vee, \wedge, \rightarrow\}$.
    \end{enumerate}
    \end{problem}
    
    
    \medskip\begin{problem}
    Uvažte ternární Booleovský operátor $\mathrm{IFTE}(p, q, r)$ definovaný jako `if $p$ then $q$ else $r$'. 
    \begin{enumerate}
        \item Zkonstruujte pravdivostní tabulku.
        \item Ukažte, že všechny základní Booleovské operátory ($\neg, \to, \wedge,\vee,\dots$) lze vyjádřit pomocí IFTE a konstant TRUE a FALSE.
    \end{enumerate}  
    \end{problem}

   
\end{document}