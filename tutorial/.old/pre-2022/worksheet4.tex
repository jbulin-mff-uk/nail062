\documentclass{amsart}
\usepackage[utf8]{inputenc}
\usepackage{enumerate}
\usepackage{graphicx}

\usepackage{a4wide}

\theoremstyle{definition}
\newtheorem{problem}{Exercise}

\title{\sc Logic Tutorial: Worksheet 4}

\date{}


\begin{document}

\maketitle


\subsection*{Topics:} More on the Tableau method. Deduction Theorem. Compactness Theorem and its applications. Resolution method, set representation of CNF formulas, resolution closure, tree of reductions. Hilbert's calculus.

\medskip \hrule

\begin{problem} Propose suitable atomic tableaux for the Peirce arrow $\downarrow$ (NOR), the Sheffer stroke $\uparrow$ (NAND), and the ternary operator ``if p then q else r'' (IFTE).
\end{problem}\medskip

\begin{problem}
Prove directly (by tableau transformations) the deduction theorem, i.e. for every theory $T$ and propositions $\varphi$, $\psi$,
$$T \vdash \varphi\to \psi\text{\ \ if and only if\ \ }T,\varphi \vdash \psi.$$
\end{problem}\medskip

% \begin{problem}
% In the proof of the lemma on completeness, we verified that if an assignment $v$ agrees with every entry on a finished branch $B$ up to the depth $i$ of formation trees, then it also agrees with every entry of the form $T(\varphi \wedge \psi)$ or $F(\varphi \wedge \psi)$ on $B$ where $\varphi \wedge \psi$ has depth $i+1$. Verify the same fact for other logical connectives.
% \end{problem}\medskip

\hrule

\begin{problem}
Show that every countable planar graph is colourable by four colours.
\end{problem}\medskip

\begin{problem}
Show that every countable partial order can be extended to a total (linear) order.
\end{problem}\medskip

% \begin{problem}
% Let $\mathcal S$ be a countable nonempty family (a set) of nonempty finite sets. An \emph{injective selector} on $\mathcal S$ is an injective function $f\colon \mathcal{S} \to \bigcup \mathcal{S}$ such that $f(S)\in S$ for every $S\in \mathcal{S}$. Prove that $\mathcal{S}$ has an injective selector if and only if every nonempty finite part of $\mathcal{S}$ has an injective selector.
% \end{problem}\medskip

\hrule

\begin{problem}
Let $\varphi$ be the proposition $\neg (p \vee q) \to (\neg p \wedge \neg q)$.
\begin{enumerate}[(a)]
\item Transform $\neg \varphi$ into CNF and into set representation (clausal form).
\item Find a resolution refutation of $\neg \varphi$, i.e., a proof of $\varphi$.
\end{enumerate}
\end{problem}\medskip


\begin{problem}
Find resolution closures $\mathcal{R}(S)$ of the following formulas $S$.
\begin{enumerate}
\item $\{\{p,q\},\{\neg p, \neg q\},\{\neg p, q\}\}$
\item $\{\{p,q\},\{p,\neg q\},\{\neg p,\neg q\}\}$
\item $\{\{p,\neg q,r\},\{q,r\},\{\neg p, r\},\{q,\neg r\},\{\neg q\}\}$
\end{enumerate}
\end{problem}\medskip

\begin{problem}
Find resolution refutations of the following propositions.
\begin{enumerate}
\item $(p\leftrightarrow (q\to r))\wedge((p\leftrightarrow q)\wedge(p\leftrightarrow \neg r))$
\item $\neg(((p\to q)\to \neg q)\to \neg q)$
\end{enumerate}
\end{problem}\medskip


\begin{problem}
Prove by resolution that $s$ is valid in $T=\{\neg p \to \neg q,\neg q \to \neg r, (r\to p)\to s\}$.
\end{problem}\medskip


\hrule


\begin{problem}
Show that if $S=\{C_1,C_2\}$ is satisfiable and $C$ is a resolvent of $C_1$ and $C_2$, then $C$ is satisfiable as well.
\end{problem}\medskip

\begin{problem}
Find the \emph{tree of reductions} of the formula $S=\{\{p,r\},\{q,\neg r\},\{\neg q\},\{\neg p,t\},\{\neg s\},\{s,\neg t\}\}$.
\end{problem}\medskip


\begin{problem}Assume that the chemical substances MgO, H$_2$, O$_2$, and C are available to us and we can perform the following chemical reactions:
\begin{align*}&(1)\quad\text{MgO\ +\ H$_2$\ \ $\to$\ \ Mg\ +\ H$_2$O}\\
&(2)\quad\text{C\ +\ O$_2$\ \ $\to$\ \ CO$_2$}\\
&(3)\quad\text{CO$_2$\ +\ H$_2$O\ \ $\to$\ \ H$_2$CO$_3$}
\end{align*}
\begin{enumerate}
\item Represent the state of affairs as a proposition in a suitable language and transform it into a set representation.
\item Prove by (linear input) resolution that we can produce H$_2$CO$_3$.
\end{enumerate}
\end{problem}\medskip


\bigskip \hrule

\subsection*{Hilbert's calculus}
The \emph{Hilbert's propositional calculus} is a proof system for propositional logic where 
\begin{itemize}
    \item we only use the logical connectives $\neg,\to$
    \item we have the following (schemes of) \emph{logical axioms}:
    \begin{enumerate}[(i)]
        \item $\varphi \to (\psi \to \varphi)$
        \item $(\varphi\to (\psi \to \chi)) \to ((\varphi \to \psi)\to(\varphi \to \chi))$
        \item $(\neg \varphi \to \neg \psi)\to(\psi \to \varphi)$
    \end{enumerate}
    \item and the following \emph{rule of inference}:
    $$\frac{\varphi,\ \varphi \to \psi}{\psi}$$
    i.e. ``from $\varphi$ and $\varphi\to\psi$ infer $\psi$'' (called ``modus ponens'')
\end{itemize}
In Hilbert's calculus, a \emph{proof} of a proposition $\varphi$ from a theory $T$ is a finite sequence $\varphi_0,\dots,\varphi_n=\varphi$ of formulas such that for every $i\leq n$,
\begin{itemize}
\item $\varphi_i$ is a logical axiom, or 
\item $\varphi_i \in T$ (an axiom of the theory), or
\item $\varphi_i$ can be inferred from a pair of preceding propositions $\varphi_j$, $\varphi_k$ ($j<i,k<i$) by applying the rule of inference.
\end{itemize}
If such a proof exists, we write $T\ \vdash_H\ \varphi$.

\bigskip

\begin{problem}
Find proofs for the following:
\begin{enumerate}[(a)]
\item $\vdash_H\ p\to p$
\item $\{\neg p\}\ \vdash_H\ p\to q$
\item $\{\neg(\neg p)\}\ \vdash_H\ p$
\item $\{p\to q,q \to r\}\ \vdash_H\ p\to r$
\item $\{p, q \to (p\to r)\}\ \vdash_H\ q\to r$

\end{enumerate}
\end{problem}\medskip


\begin{problem}
Prove soundness of Hilbert's calculus:
\begin{itemize}
    \item Show that the logical axioms are tautologies.
    \item Show that modus ponens is sound, i.e., if $T\models\varphi$ and $T\models\varphi\to\psi$, then $T\models\psi$.
    \item Show that $T\ \vdash_H\ \varphi$ implies $T\models\varphi$.
\end{itemize}
\end{problem}\medskip

\begin{problem}
State and prove a deduction theorem for Hilbert's calculus.
\end{problem}\medskip

%\begin{problem}
%**How could we prove completeness of Hilbert's calculus?
%\end{problem}

\end{document}