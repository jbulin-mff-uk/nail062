\documentclass{beamer}

\input{slides-header.tex}

\title{Osmá přednáška}
\subtitle{NAIL062 Výroková a predikátová logika}
\author{Jakub Bulín (KTIML MFF UK)}
% \institute{KTIML MFF UK}
\date{Zimní semestr 2024}


\begin{document}


\maketitle


\begin{frame}{Osmá přednáška}

    \textbf{Program}
        \begin{itemize}
            \item korektnost a úplnost, kanonický model
            \item věta o kompaktnosti, Löwenheim-Skolemova věta
            \item hilbertovský kalkulus
        \end{itemize}

    \textbf{Materiály}

        \href{https://github.com/jbulin-mff-uk/nail062/raw/main/lecture/lecture-notes/lecture-notes.pdf}{\alert{\textbf{Zápisky z přednášky}}}, Sekce 7.4-7.6 z Kapitoly 7 (+ Sekce 4.8)

\end{frame}


\section{7.4 Korektnost a úplnost}


\begin{frame}{Korektnost a úplnost}
    
    \pause
    Stejně jako ve výrokové logice:
    
    \pause
    \begin{center}
        \alert{dokazatelnost} je totéž, co \alert{platnost}    
    \end{center}    

    \pause
    \begin{itemize}
        \item \alert{$T\proves\varphi\ \Rightarrow\T\models\varphi$} \hspace{0.5cm} (korektnost) \hfill {\it``co jsme dokázali, platí''}
        \item\alert{$T\models\varphi\ \Rightarrow\T\proves\varphi$}  \hspace{0.5cm} (úplnost) \hfill {\it ``co platí, lze dokázat''}
    \end{itemize} 

    \bigskip

    \pause
    (Důkazy mají stejnou strukturu, liší se jen v implementačních detailech pomocných lemmat.)
 
\end{frame}


\begin{frame}{Korektnost: pomocné lemma}

    \pause
    \vspace{-6pt}
    Model $\A$ se \alert{shoduje} \alert{s položkou $P$}, pokud
    $P=\mathrm{T}\varphi$ a $\A\models\varphi$, nebo $P=\mathrm{F}\varphi$ a $\A\not\models\varphi$, a \alert{s větví $V$}, shoduje-li s každou položkou na $V$.
    
    \smallskip

    \pause
    \myblock{
    \textbf{Lemma:}
        Shoduje-li se model $\A$ teorie $T$ (v jazyce $L$) s položkou v kořeni tabla z $T$, potom \myalertinline{lze $\A$ expandovat do jazyka $L_C$} (interpretovat symboly $c_i\in C$) tak, že se shoduje s některou větví v tablu.
    }
    
    \pause
    {\small NB: Stačí interpret. symboly $c_i$ vyskytující se na větvi, ostatní libovolně.}
   
    \pause
    \textbf{Důkaz:} \pause
    Indukcí podle konstrukce $\tau=\bigcup_{i\geq 0}\tau_i$ najdeme  posloupnost větví $V_0\subseteq V_1\subseteq\dots$ a expanzí $\A_i$ o konstanty na $V_i$ tak, že:\pause
     \begin{itemize}
        \item $V_i$ je větev v tablu $\tau_i$ shodující se s modelem $\A_i$\pause
        \item $V_{i+1}$ je prodloužením $V_i$ a $\A_{i+1}$ je expanzí $\A_i$\pause
     \end{itemize}
    Hledaná větev v $\tau$ je $V=\bigcup_{i\geq 0}V_i$, $L_C$-expanze $\A$ je `limita' $\A_i$: vyskytuje-li se $c\in C$ na $V_i$, interpretuj jako v $\A_i$, jinak libovolně.\pause

    \alert{Báze:} $\A_0=\A$ se shoduje s kořenem, tj. s (jednoprvkovou) $V_0$ v $\tau_0$.

\end{frame}


\begin{frame}{Pokračování důkazu pomocného lemmatu}

    \vspace{-6pt}\pause
    \alert{Indukční krok:} Pokud jsme neprodloužili $V_i$: $V_{i+1}=V_i$, $\A_{i+1}=\A_i$.

    \vspace{-3pt}

    \pause
    Pokud jsme připojili $\mathrm{T}\alpha$ (pro $\alpha\in T$) na konec $V_i$, definujeme $V_{i+1}$ jako tuto prodlouženou větev, $\A_{i+1}=\A_i$ (nepřidali jsme nový symbol). Protože $\A\models T$, máme i $\A_{i+1}\models\alpha$, tedy se shoduje.
  
    \pause
    Nechť $\tau_{i+1}$ vzniklo připojením atomického tabla pro $P$ na konec $V_i$.

    \vspace{-3pt}
    
    \pause
    \begin{itemize}
        \item \textbf{logická spojka:} $\A_{i+1}=\A_i$ se shoduje s kořenem atomického tabla, tedy i s některou větví, o tu prodloužíme $V_i$ na $V_{i+1}$\pause
        
        \medskip

        \item \textbf{typ ``}\textcolor{red}{svědek}\textbf{'':} SÚNO $P=\T(\exists x)\varphi(x)$: $\A_i\models(\exists x)\varphi(x)$, tedy \myalertinline{existuje $a\in A$, že $\A_i\models\varphi(x)[e(x/a)]$}. $V_{i+1}$ je prodloužení $V_i$ o nově přidanou $\T\varphi(x/c)$, $\A_{i+1}$ je expanze $\A_i$ o \myalertinline{
            $c^{\A_{i+1}}=a$
        }.\pause
        
        \medskip

        \item \textbf{typ ``}\textcolor{blue}{všichni}\textbf{'':} $V_{i+1}$ je prodloužení $V_i$ o atomické tablo. SÚNO nová položka $\T\varphi(x/t)$ pro nějaký $L_C$-term $t$. \pause 
        Model $\A_{i+1}$ je \myalertinline{libovolná expanze} $\A_i$ o nové symboly z $t$.    
        $\A_i\models(\forall x)\varphi(x)$ $\Rightarrow$ $\A_{i+1}\models(\forall x)\varphi(x)$ $\Rightarrow$ \myalertinline{
            $\A_{i+1}\models\varphi(x/t)$
        }, tedy se shoduje. \hfill\qedsymbol
    \end{itemize}

\end{frame}


\begin{frame}{Věta o korektnosti [tablo metody ve predikátové logice]}

    \pause
    \myblock{
    \textbf{Věta (O korektnosti):}
    Je-li sentence $\varphi$ tablo dokazatelná z teorie~$T$, potom je $\varphi$ pravdivá v~$T$, tj. \alert{$T\proves\varphi\ \Rightarrow\T\models\varphi$}.
    }

    \medskip

    \pause
    \textbf{Myšlenka důkazu:} Protipříklad by se [po vhodné interpretaci pomocných symbolů] shodoval s některou větví, ty jsou ale sporné.

    \medskip

    \pause
    \textbf{Důkaz:} \pause Sporem, nechť $T\not\models\varphi$, tj. existuje $\A\in\M(T)$, že $\A\not\models\varphi$.
    
    \pause
    Protože $T\proves\varphi$, existuje tablo důkaz $\varphi$ z $T$, což je sporné tablo z $T$ s položkou $\mathrm{F}\varphi$ v kořeni. 
        
    \pause
    Model $\A$ se shoduje s kořenem $\mathrm{F}\varphi$, tedy podle Lemmatu lze interpretovat symboly $c\in C$ tak, že se výsledná $L_C$-expanze $\A'$ shoduje s nějakou větví $V$. \pause Všechny větve jsou ale sporné, musela by se shodovat s $\T\psi$ a zároveň $\F\psi$ pro nějakou $L_C$-sentenci $\psi$.\hfill\qedsymbol

\end{frame}


\begin{frame}{Kanonický model: jazyk bez rovnosti}

    \pause
    opět z \alert{bezesporné dokončené} větve $V$ (tabla z $T$) vyrobíme model\\ \pause
    jeho doména? trik: \myalertinline{ze syntaktických objektů uděláme sémantické}

    \medskip

    \pause
    \myblock{
    Je-li $L=\langle\mathcal F,\mathcal R\rangle$ bez rovnosti, \alert{kanonický model} pro bezespornou dokončenou $V$ je $L_C$-struktura $\A=\langle A,\mathcal F^\mathcal A\cup C^\mathcal A,\mathcal R^\mathcal A\rangle$, kde:\pause
    \begin{itemize}
        \item doména $A$ je \myalertinline{množina všech konstantních $L_C$-termů}\pause
        \item pro $n$-ární relační symbol $R\in\mathcal R$ a ``$s_1$'', \dots, ``$s_n$'' z $A$:
        $$
        (\text{``$s_1$''},\dots,\text{``$s_n$''})\in R^\mathcal A\ \Leftrightarrow\ \text{na $V$ je položka $\T R(s_1,\dots,s_n)$}
        $$
        
        \pause
        \item pro $n$-ární funkční symbol $f\in\mathcal F$ a ``$s_1$'', \dots, ``$s_n$'' z $A$:        
        $$
        f^\mathcal A(\text{``$s_1$''},\dots,\text{``$s_n$''})=\text{``$f(s_1,\dots,s_n)$''}
        $$
        
        \pause
        \item speciálně, pro konstantní symbol $c$ máme $c^\mathcal A=\text{``$c$''}$
    \end{itemize}
    }

    %(``$t$'' píšeme pro zdůraznění, že term $t$ chápeme jako řetězec znaků)

    \pause
    (funkce $f^\mathcal A$  je ``vytvoření'' termu ze symbolu $f$ a vstupních termů)

\end{frame}


\begin{frame}{Příklad}

    \pause
    \myexampleinline{
        $T=\{(\forall x)R(f(x))\}$
    } v jazyce $L=\langle R,f,d \rangle$ bez rovnosti ($R$ unární relační, $f$ unární funkční, $d$ konstantní). Protipříklad: \myexampleinline{
        $T\not\models\neg R(d)$
    }

    \pause
    \begin{itemize}
        \item dokončené tablo z $T$ s položkou $\F\neg R(d)$ v kořeni má jedinou, bezespornou větev $V$\pause
        \item \alert{kanon. model:} $L_C$-struktura {$\A=\langle A,R^\A,f^\A,d^\A,c_0^\A,c_1^\A,\dots\rangle$}\pause
        \item doména je {$A=\{\text{``$d$''},\text{``$f(d)$''},\text{``$f(f(d))$''},\dots,\text{``$c_0$''},\text{``$f(c_0)$''},$ $\text{``$f(f(c_0))$''},\dots,\text{``$c_1$''},\text{``$f(c_1)$''},\text{``$f(f(c_1))$''},\dots\}$}\pause
        \item interpretace symbolů jsou:\pause
        \begin{itemize}
            \item $d^\A=\text{``$d$''}$\pause
            \item $c^\A_i=\text{``$c_i$''}$ pro všechna $i\in \mathbb N$\pause
            \item $f^\A(\text{``$d$''})=\text{``$f(d)$''}$, $f^\A(\text{``$f(d)$''})=\text{``$f(f(d))$''}$, \dots\pause
            \item $\alert{R^\A=A\setminus C}=\{\text{``$d$''},\text{``$f(d)$''},\text{``$f(f(d))$''},\dots,\text{``$f(c_0)$''},$ $\text{``$f(f(c_0))$''},\dots,\text{``$f(c_1)$''},\text{``$f(f(c_1))$''},\dots\}$.\pause
        \end{itemize}
        \item redukt na původní jazyk $L$: $\A'=\langle A, R^\A, f^\A, d^\A\rangle$
    \end{itemize}
        
\end{frame}


\begin{frame}{Kanonický model: jazyk s rovností}

    \pause
    \myblock{
    Je-li $L$ s rovností:\pause
    \begin{itemize}
        \item vezmeme kanonický model $\mathcal B$ pro $V$ jako by byl $L$ bez rovnosti\pause
        \item definujeme relaci $=^B$ stejně jako pro ostatní relační symboly:
        $$
        \text{``$s_1$''}=^B\text{``$s_2$''}\ \Leftrightarrow\ \text{na $V$ je položka $\T s_1=s_2$}
        $$

        \pause
        \item \alert{kanonický model} pro $V$ je faktorstruktura \alert{$\A=\B/_{=^B}$}
    \end{itemize} 
    }

    \medskip

    \pause
    \begin{itemize}
        \item tablo je nyní z teorie $T^*$ (rozšíření o axiomy rovnosti)\pause
        \item $=^B$ je opravdu kongruence struktury $\B$ a $=^\A$ je identita na $A$\pause
        \item \textbf{Pozorování:} pro lib. formuli $\varphi$ platí \alert{$\B\models\varphi$ právě když $\A\models\varphi$}\\
        (symbol $=$ interpretujeme jako $=^B$ v $\B$ a jako identitu v $\A$)\pause
    \end{itemize}

    \textbf{Všimněte si:}\pause
    \begin{itemize}
        \item v jazyce bez rovnosti je kanonický model spočetně nekonečný\pause
        \item v jazyce s rovností může být i konečný
    \end{itemize}

\end{frame}


\begin{frame}{Příklad}

    \pause
    \myexampleinline{
        $T=\{(\forall x)R(f(x)),(\forall x)(x=f(f(x)))\}$
    } $L=\langle R,f,d \rangle$ \myalertinline{s rovností}\\ opět chceme protipříklad ukazující, že \myexampleinline{
        $T\not\models\neg R(d)$
    }

    \pause
    \begin{itemize}
        \item dokončené tablo \alert{z $T^*$} pro $\F\neg R(d)$ má jedinou, bezespornou $V$\pause
        \item sestrojíme kanonický model jako by byl jazyk bez rovnosti:\pause
        $$
        \B=\langle B,R^\B,f^\B,d^\B,c_0^\B,c_1^\B,c_2^\B,\dots\rangle
        $$

        \pause
        \item \alert{`$=$' jako obyčejný symbol:} \pause $s_1=^B s_2$ $\Leftrightarrow$ $s_1=f(\cdots (f(s_2))\cdots)$ nebo \small$s_2=f(\cdots (f(s_1))\cdots)$ pro sudý počet~$f$ \pause  
        $$\hspace{-1cm}
        B/_{=^B} = \{[\text{``$d$''}]_{=^B},[\text{``$f(d)$''}]_{=^B},[\text{``$c_0$''}]_{=^B},[\text{``$f(c_0)$''}]_{=^B},[\text{``$c_1$''}]_{=^B},[\text{``$f(c_1)$''}]_{=^B},\dots\}
        $$    

        \pause
        \item \alert{kanonický model}: 
        $\A=\B/_{=^B}=\langle A,R^\A,f^\A,d^\A,c_0^\A,c_1^\A,c_2^\A,\dots\rangle$\pause
        \begin{itemize}
            \item $A=B/_{=^B}$, $d^\A=[\text{``$d$''}]_{=^B}$, $c^\A_i=[\text{``$c_i$''}]_{=^B}$ pro všechna $i\in \mathbb N$,\pause
            \item $f^\A([\text{``$d$''}]_{=^B})=[\text{``$f(d)$''}]_{=^B}$, $f^\A([\text{``$f(d)$''}]_{=^B})=[\text{``$f(f(d))$''}]_{=^B}=[\text{``$d$''}]_{=^B}$, \dots\pause
            \item $R^\A=A=B/_{=^B}$.\pause
        \end{itemize}        
        \item redukt na původní jazyk $L$: $\A'=\langle A, R^\A, f^\A, d^\A\rangle$
    \end{itemize}

\end{frame}


\begin{frame}{Úplnost: pomocné lemma}

    \pause
    \myblock{
    \textbf{Lemma:}
    Kanonický model pro (bezespornou, dokončenou) větev $V$ se shoduje s $V$.
    }

    \pause
    \textbf{Důkaz:} \pause
    \alert{Jazyk bez rovnosti:} indukcí podle struktury sentence v $P$\pause

    \begin{itemize}
        \item \textbf{atomická sentence:} stejně jako ve VL (\alert{báze indukce})\pause
        
        \medskip

        \item \textbf{logická spojka:} stejně jako ve VL\pause
        
        \medskip

        \item \textbf{typ ``}\textcolor{red}{svědek}\textbf{'':} \pause \alert{$P=\T(\exists x)\varphi(x)$}, potom je na $V$ i $T\varphi(x/c)$ pro nějaké $\text{``$c$''}\pause\in A$; z indukčního předpokladu $\A\models\varphi(x/c)$, tj. $\A\models\varphi(x)[e(x/\text{``$c$''})]$ tedy i $\A\models(\exists x)\varphi(x)$\pause

        \medskip

        \item \textbf{typ ``}\textcolor{blue}{všichni}\textbf{'':} \pause \alert{$P=\T(\forall x)\varphi(x)$}, na $V$ jsou i položky $T\varphi(x/t)$ pro každý konstantní $L_C$-term, tj. pro každý prvek $\text{``$t$''}\in A$; \pause z~ind. předpokladu je $\A\models\varphi(x/t)$, tj. $\A\models\varphi(x)[e(x/\text{``$t$''})]$ pro každé $\text{``$t$''}\in A$, tedy $\A\models(\forall x)\varphi(x)$
        
    \end{itemize}

    \pause
    \alert{Jazyk s rovností:} $\A=\B/_{=^B}$, pro $\B$ máme, zbytek z Pozorování \hfill\qedsymbol

\end{frame}


\begin{frame}{Věta o úplnosti}

    \pause
    \myblock{
    \textbf{Věta (O úplnosti):} Je-li sentence $\varphi$ pravdivá v teorii $T$, potom je tablo dokazatelná z $T$, tj. \alert{$T\models\varphi\ \Rightarrow\T\proves\varphi$}.
    }
    
    \smallskip

    \pause
    \textbf{Důkaz:} \pause
    Ukážeme, že libovolné dokončené (např. \alert{systematické}) tablo z $T$ s $\mathrm{F}\varphi$ v kořeni je nutně sporné, tedy je tablo důkazem.
    
    \pause
    Sporem: \alert{Není-li sporné}, má bezespornou (dokončenou) větev $V$, a dle Lemmatu se kanonický model $\A$ s větví $V$ shoduje. 
    
    \pause
    \myalert{
    Buď $\A'$ redukt $\A$ na jazyk teorie $T$ (zapomeň pomocné symboly).
    }

    \pause
    Protože je $V$ dokončená, obsahuje $\mathrm{T}\alpha$ pro všechny axiomy $T$. Model $\A$, \myalertinline{tedy i $\A'$}, splňuje všechny axiomy a máme \alert{$\A'\models T$}. 
    
    \pause
    Protože se ale $\A$, \myalertinline{tedy i $\A'$}, shoduje i s položkou $\mathrm{F}\varphi$ v kořeni, máme \alert{$\A'\not\models\varphi$}, což dává protipříklad, a máme \alert{$T\not\models\varphi$}, spor.\hfill\qedsymbol

\end{frame}


\section{7.5 Důsledky korektnosti a úplnosti}


\begin{frame}{$\proves\ =\ \models$}

    \pause
    Syntaktickou analogií \alert{důsledků} jsou \alert{teorémy}:
    $$
    \Thm_L(T)=\{\varphi\mid \varphi\text{ je $L$-sentence a } T\proves\varphi\}
    $$
    
    \pause
    Z korektnosti a úplnosti okamžitě dostáváme:
        \begin{itemize}
            \item $T\proves\varphi$ právě když $T\models\varphi$
            \item $\Thm_L(T)=\Conseq_L(T)$
        \end{itemize}
    
    \pause
    Všude můžeme nahradit `\alert{platnost}' pojmem `\alert{dokazatelnost}'.  Např:
    \begin{itemize}
        \item $T$ je \alert{sporná}, je-li v ní dokazatelný spor (tj. \alert{$T\proves\bot$})
        \item $T$ je \alert{kompletní}, je-li pro každou sentenci buď $T\proves\varphi$ nebo $T\proves\neg\varphi$, ale ne obojí (jinak by byla sporná)
    \end{itemize}

    \pause
    \myblock{
        \textbf{Věta (O dedukci):} $T,\varphi\proves\psi$ právě když  $T\proves\varphi\to\psi$.
    }

    \pause
    \textbf{Důkaz:} Stačí dokázat: $T,\varphi\models\psi\Leftrightarrow T\models\varphi\to\psi$. To je snadné.\hfill\qedsymbol

\end{frame}


\begin{frame}{Löwenheim-Skolemova věta \& Věta o kompaktnosti}
    
    \medskip
    
    \pause
    \myblock{
    \textbf{Věta (Löwenheim-Skolemova):}
    Je-li $L$ spočetný bez rovnosti, potom každá bezesporná $L$-teorie má spočetně nekonečný model.
    }

    \pause
    (Později ukážeme i verzi s rovností, kan. model může být konečný.)

    \pause
    \textbf{Důkaz:} V $T$ není dokazatelný spor. Dokončené tablo z $T$ s $\F\bot$ v kořeni tedy musí obsahovat bezespornou větev. Hledaný model je $L$-redukt kanonického modelu pro tuto větev.\hfill\qedsymbol

    \bigskip

    \pause
    Věta o kompaktnosti, vč. důkazu, je stejná jako ve výrokové logice:

    \pause
    \smallskip
    \myblock{
    \textbf{Věta (O kompaktnosti):}
    Teorie má model, právě když každá její konečná část má model.\vspace{2pt}  
    }  
    
    \pause
    \textbf{Důkaz:} Model teorie je modelem každé části. Naopak, pokud $T$ nemá model, je sporná, tedy $T\proves\bot$. Vezměme nějaký \alert{konečný} tablo důkaz $\bot$ z $T$. K jeho konstrukci stačí konečně mnoho axiomů $T$, ty tvoří konečnou podteorii $T'\subseteq T$, která nemá model.\hfill\qedsymbol

\end{frame}


\begin{frame}{Nestandardní model přirozených čísel}

    \pause
    \begin{itemize}
        \item $\underline{\mathbb N}=\langle\mathbb N,S,+,\cdot,0,\leq\rangle$ je \alert{standardní model} přirozených čísel\pause
        \item \alert{teorie struktury $\Th(\underline{\mathbb N})$:} všechny sentence \alert{pravdivé} v~$\underline{\mathbb N}$\pause
        \item \alert{$n$-tý numerál:} term $\underline n=S(S(\cdots (S(0)\cdots))$, kde $S$ je $n$-krát
    \end{itemize}
    
    \pause
    Přidáme nový konstantní symbol $c$ a vyjádříme, že je ostře větší než každý $n$-tý numerál:
    $$
    T=\Th(\underline{\mathbb N})\cup\{\underline n<c\mid n\in \mathbb N\}
    $$
        
    \pause
    \begin{itemize}
        \item každá konečná část $T$ má model\pause
        \item dle věty o kompaktnosti: i $T$ má model\pause
        \item říkáme mu \alert{nestandardní model} (označme $\A$)\pause
        \item platí v něm tytéž sentence, které platí ve standardním modelu\pause
        \item ale zároveň obsahuje prvek $c^\A$, který je větší než každé $n\in \mathbb N$ (tzn. větší než hodnota termu $\underline n$ v nestandardním modelu $\A$)
    \end{itemize}    

\end{frame}


% \section{7.6 Hilbertovský kalkulus v predikátové logice}

% \begin{frame}{Hilbertovský kalkulus v predikátové logice}

%     \vspace{-6pt}
%     \begin{itemize}
%         \item používá jen $\neg$ a $\limplies$, dokazuje lib. formule (nejen sentence)
%         \item \alert{schémata log. axiomů} ($\varphi,\psi,\chi$ formule, $t$ term, $x$ proměnná)
%         \begin{enumerate}[(i)]
%             \item $\varphi \limplies (\psi \limplies \varphi)$
%             \item $(\varphi\limplies (\psi \limplies \chi))\limplies ((\varphi \limplies \psi)\limplies(\varphi \limplies \chi))$
%             \item $(\neg \varphi \limplies \neg \psi)\limplies(\psi \limplies \varphi)$            
%         \end{enumerate}

%         \medskip

%         \myalert{
%         \begin{enumerate}[(iv)]
%             \item $(\forall x)\varphi \limplies \varphi(x/t)$ \hfill je-li $t$ substituovatelný za $x$ do $\varphi$
%             \item $(\forall x)(\varphi \to \psi) \limplies (\varphi \limplies (\forall x)\psi)$ \hfill není-li $x$ volná ve $\varphi$
%         \end{enumerate}
        
%         a navíc \textbf{axiomy rovnosti}, je-li jazyk s rovností
%         }

%         \medskip

%         \item \alert{odvozovací pravidla}:\vspace{-6pt}
%         \begin{center}
%             \begin{minipage}{0.45\textwidth}
%                 $$
%                 \frac{\varphi, \varphi \limplies \psi}{\psi}\ \text{(modus ponens)}
%                 $$
%             \end{minipage}          
%             \begin{minipage}{0.45\textwidth}
%                 \myalertmath{
%                 $$
%                 \frac{\varphi}{(\forall x)\varphi} \ \text{(generalizace)} 
%                 $$
%                 }
%             \end{minipage}            
%         \end{center}        
        
%         \item \alert{hilbertovský důkaz} formule $\varphi$ z $T$ je \alert{konečná} posloupnost $\varphi_0, \dots, \varphi_n=\varphi$, kde $\varphi_i$ je \alert{logický axiom} (vč. axiomů rovnosti), \alert{axiom teorie}, nebo lze odvodit z předchozích pomocí pravidel
%         \item existuje-li, píšeme \alert{$T\proves_H\varphi$}
%     \end{itemize}

% \end{frame}


% \begin{frame}{Korektnost a úplnost}

%     \myblock{
%     \textbf{Věta (o korektnosti hilbertovského kalkulu):}
%     $T\proves_H\varphi \Rightarrow T\models\varphi$
%     }

%     \medskip

%     \textbf{Důkaz:} Indukcí dle délky důkazu: každá $\varphi_i$ (vč. $\varphi_n=\varphi$) platí v $T$
%     \begin{itemize}
%         \item logické axiomy (vč. axiomů rovnosti) jsou tautologie, platí v $T$
%         \item axiomy z $T$ jistě v $T$ také platí
%          \item modus ponens i generalizace jsou \alert{korektní} inferenční pravidla:
%         \begin{itemize}
%             \item je-li $T\models\varphi$ a $T\models\varphi\to\psi$, potom $T\models\psi$
%             \item je-li $T\models\varphi$, potom $T\models(\forall x)\varphi$
%             \hfill\qedsymbol
%         \end{itemize}
%     \end{itemize}

%     \bigskip
    
%     \myblock{
%     \textbf{Věta (o úplnosti hilbertovského kalkulu):}
%     $T\models\varphi\ \Rightarrow\ T\proves_H\varphi$
%     }

%     Důkaz vynecháme.
    
% \end{frame}


\section{Hilbertovský kalkulus}


\begin{frame}{Hilbertovský deduktivní systém}

    \pause
    \begin{itemize}
        \item jiný, původní dokazovací systém \pause
        \item používá jen logické spojky $\neg$, $\limplies$ \pause
        \item \alert{schémata logických axiomů} ($\varphi,\psi,\chi$ jsou lib. výroky/formule) \pause
        \begin{enumerate}[(i)]
            \item $\varphi \limplies (\psi \limplies \varphi)$\pause
            \item $(\varphi\limplies (\psi \limplies \chi))\limplies ((\varphi \limplies \psi)\limplies(\varphi \limplies \chi))$
            \item $(\neg \varphi \limplies \neg \psi)\limplies(\psi \limplies \varphi)$\\ \pause
            v predikátové logice navíc: \pause
            \item $(\forall x)\varphi \limplies \varphi(x/t)$ \hfill je-li $t$ substituovatelný za $x$ do $\varphi$ \pause
            \item $(\forall x)(\varphi \to \psi) \limplies (\varphi \limplies (\forall x)\psi)$ \hfill není-li $x$ volná ve $\varphi$ \pause
            \item \textbf{axiomy rovnosti}, je-li jazyk s rovností \pause
        \end{enumerate}

        \item \alert{odvozovací pravidla}:\hfill v predikátové logice navíc:
        \vspace{-6pt}
        \begin{center}
            \begin{minipage}{0.45\textwidth}
                $$
                \frac{\varphi, \varphi \limplies \psi}{\psi}\ \text{(modus ponens)}
                $$
            \end{minipage}          
            \begin{minipage}{0.45\textwidth}
                \myalertmath{
                $$
                \frac{\varphi}{(\forall x)\varphi} \ \text{(generalizace)} 
                $$
                }
            \end{minipage}            
        \end{center}      
        
    \end{itemize}

\end{frame}


\begin{frame}{Hilbertovský důkaz}

    \begin{itemize}[<+->]
        \item \alert{hilbertovský důkaz} výroku $\varphi$ z teorie $T$ je \alert{konečná} posloupnost $\varphi_0, \dots, \varphi_n=\varphi$, ve které pro každé $i\le n$:
        \begin{itemize}
        \item $\varphi_i$ je \alert{logický axiom}, nebo
        \item $\varphi_i$ je \alert{axiom teorie} ($\varphi_i \in T$), nebo
        \item $\varphi_i$ lze odvodit z předchozích pomocí \alert{odvozovacího pravidla}
        \end{itemize}
        \item existuje-li hilbertovský důkaz, píšeme: \alert{$T\proves_H\varphi$}
    \end{itemize}

\end{frame}


\begin{frame}{Příklad (jen ve výrokové logice)}

    \pause
    Ukažme, že pro teorii $T=\{\neg\varphi\}$ a pro libovolný výrok $\psi$ platí:  
    \myexamplemath{  
    $$
    T\proves_H\varphi\limplies\psi
    $$
    }

    \pause
    Hilbertovský důkaz:
    
    \begin{enumerate}\it
        \item $\neg\varphi$\hfill axiom teorie
        \item $\neg \varphi \limplies (\neg \psi \limplies \neg \varphi)$\hfill logický axiom (i)
        \item $\neg \psi \limplies \neg \varphi$\hfill modus ponens na 1. a 2.
        \item $(\neg \psi \limplies \neg \varphi)\limplies(\varphi \limplies \psi)$ \hfill logický axiom (iii)
        \item $\varphi \limplies \psi$ \hfill modus ponens na 3. a 4.
    \end{enumerate}    

\end{frame}


\begin{frame}{Korektnost a úplnost}

    \pause
    \myblock{
    \textbf{Věta (o korektnosti hilbertovského kalkulu):}
    $T\proves_H\varphi \Rightarrow T\models\varphi$
    }

    \medskip

    \pause
    \textbf{Důkaz:} Indukcí dle délky důkazu: každá $\varphi_i$ (vč. $\varphi_n=\varphi$) platí v $T$\pause
    \begin{itemize}
        \item logické axiomy (vč. axiomů rovnosti) jsou tautologie, platí v $T$\pause
        \item axiomy z $T$ jistě v $T$ také platí\pause
         \item modus ponens i generalizace jsou \alert{korektní} inferenční pravidla:\pause
        \begin{itemize}
            \item je-li $T\models\varphi$ a $T\models\varphi\to\psi$, potom $T\models\psi$\pause
            \item je-li $T\models\varphi$, potom $T\models(\forall x)\varphi$
            \hfill\qedsymbol
        \end{itemize}
    \end{itemize}

    \bigskip
    
    \pause
    \myblock{
    \textbf{Věta (o úplnosti hilbertovského kalkulu):}
    $T\models\varphi\ \Rightarrow\ T\proves_H\varphi$
    }

    Důkaz vynecháme.
    
\end{frame}


\end{document}
