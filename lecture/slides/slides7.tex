\documentclass{beamer}

\input{slides-header.tex}

\title{Sedmá přednáška}
\subtitle{NAIL062 Výroková a predikátová logika}
\author{Jakub Bulín (KTIML MFF UK)}
% \institute{KTIML MFF UK}
\date{Zimní semestr 2024}


\begin{document}


\maketitle


\begin{frame}{Sedmá přednáška}

    \textbf{Program}
        \begin{itemize}
            \item extenze teorií, extenze o definice
            \item definovatelnost a databázové dotazy
            \item vztah výrokové a predikátové logiky
            \item tablo metoda v predikátové logice, jazyky s rovností
        \end{itemize}

    \textbf{Materiály}

        \href{https://github.com/jbulin-mff-uk/nail062/raw/main/lecture/lecture-notes/lecture-notes.pdf}{\alert{\textbf{Zápisky z přednášky}}}, Sekce 6.7-6.9 z Kapitoly 6, Sekce 7.1-7.3 z Kapitoly 7

\end{frame}


\section{6.7 Extenze teorií}


\begin{frame}{Extenze teorie}

    Stejně jako ve výrokové logice, je-li $T$ teorie v jazyce $L$:\pause 

    \myblock{
        \begin{itemize}
            \item \alert{extenze:} $T'$ v jazyce $L'\supseteq L$ splňující $\Conseq_L(T)\subseteq\Conseq_{L'}(T')$\pause 
            \item \alert{jednoduchá:} $L'=L$\pause 
            \item \alert{konzervativní:} $\Conseq_L(T)=\Conseq_L(T')=\Conseq_{L'}(T')\cap \Fm_L$\pause 
            \item \alert{ekvivalentní:} $T'$ extenzí $T$ a $T$ extenzí $T'$ (obě v témž jazyce)\pause 
        \end{itemize}
    }
    
    Jsou-li $T,T'$ ve stejném jazyce $L$:\pause 
    \begin{itemize}
        \item $T'$ je extenze $T$, právě když $\M_L(T')\subseteq\M_L(T)$\pause 
        \item $T'$ je ekvivalentní s $T$, právě když $\M_L(T')=\M_L(T)$\pause 
    \end{itemize}
       
    Zvětšíme-li jazyk:\pause 
    \begin{itemize}
        \item \alert{ve výrokové logice:} přidáváme/zapomínáme hodnoty pro nové prvovýroky\pause 
        \item \alert{v predikátové logice:} expandujeme/redukujeme modely (přidáváme/zapomínáme nové relace, funkce, konstanty)
    \end{itemize}

\end{frame}


\begin{frame}{Extenze teorie: sémantický popis}

    \myblock{
        Mějme jazyky $L\subseteq L'$, $L$-teorii $T$ a $L'$-teorii $T'$:\pause 
        \begin{enumerate}[(i)]
            \item $T'$ je \alert{extenzí} $T$ $\Leftrightarrow$ $L$-redukt každého modelu $T'$ je model $T$\pause 
            \item $T'$ je \alert{konzervativní extenzí} $T$ $\Leftrightarrow$ $T'$ je extenzí $T$, a každý model $T$ lze expandovat do $L'$ na nějaký model $T'$\pause 
        \end{enumerate}        
    }
    
    {\small \textbf{Poznámka:} Důkaz \alert{(ii) \Large $\Rightarrow$} vynecháme (technický problém: model, který

    \vspace{-11pt} nelze expandovat $\rightsquigarrow$ $L$-sentence platná v $T$ ale ne v $T'$)}\pause 

    \vspace{-3pt}
    \textbf{Důkaz:}\pause 
    \alert{(i) \Large $\Rightarrow$} Buď $\A'$ model $T'$, $\A$ jeho $L$-redukt. \pause Protože $T'$ je extenzí, platí v ní, tedy i v $\A'$, každý axiom $\varphi\in T$. Ten ale obsahuje jen symboly z $L$, tedy platí i v $\A$.
    
    \pause 
    \vspace{-3pt}
    \alert{(i) \Large $\Leftarrow$} \textbf{Mějme:} $L$-sentenci $\varphi$,  $T\models\varphi$. \textbf{Chceme:} $T'\models\varphi$. \pause Pro lib. model $\A'\in\M_{L'}(T')$ víme, že jeho $L$-redukt $\A$ je modelem $T$, tedy $\A\models\varphi$. Z toho plyne i $\A'\models\varphi$ (opět $\varphi$ je v $L$).
    
    \pause 
    \vspace{-3pt}
    \alert{(ii) \Large $\Leftarrow$} \textbf{Mějme:} $L$-sentenci $\varphi$,  $T'\models\varphi$. \textbf{Chceme:} $T\models\varphi$. \pause Každý $\A\in\M_L(T)$ lze expandovat na nějaký $\A'\in\M_{L'}(T')$. Víme, že $\A'\models\varphi$, takže i $\A\models\varphi$. Tím jsme dokázali $T\models\varphi$.    
    \hfill\qedsymbol
    
\end{frame}


\begin{frame}{Extenze o definice (neformálně)}

    \pause 
    \begin{itemize}
        %\item speciální případ extenze: \alert{extenze o definici}
        \item přidáme nový symbol, jehož význam je jednoznačně daný \alert{definující formulí} (jako procedura/funkce v programování)\pause 
        \item pro relační symboly jednoduché, pro funkční symboly musíme navíc zaručit \alert{existenci} a \alert{jednoznačnost} funkční hodnoty\pause 
    \end{itemize}

    Ukážeme:\pause 
    \begin{itemize}
        \item je to konzervativní extenze, dokonce každý model původní teorie lze \alert{jednoznačně} expandovat na model nové teorie\pause 
        \item každou formuli používající nové symboly lze přepsat na  formuli v původním jazyce (tak, že jsou v extenzi ekvivalentní)
    \end{itemize}

\end{frame}


\begin{frame}{Definice relačního symbolu}
    
    \pause 
    nový $n$-ární relační symbol $R$ lze definovat lib. formulí $\psi(x_1,\dots,x_n)$

    \pause 
    \begin{itemize}
        \item teorii v jazyce s rovností lze \myexampleinline{rozšířit o symbol $\neq$} \alert{definovaný} formulí \alert{$\neg x_1=x_2$}; tj. požadujeme, aby:\myalertinline{
            $x_1\neq x_2\liff\neg x_1=x_2$
            }\pause 
        \item teorii uspořádání lze \myexampleinline{rozšířit o $<$} definovaný formulí \alert{$x_1\leq x_2\land \neg x_1=x_2$}; tj. platí:\myalertinline{
            $x_1<x_2 \liff x_1\leq x_2\land \neg x_1=x_2$
        }\pause 
        \item v aritmetice lze \myexampleinline{zavést $\leq$} takto:\myalertinline{
            $x_1\leq x_2\liff(\exists y)(x_1+y=x_2)$
        }\pause 
        \item v uspořádaném stromu lze zavést unární predikát \myexampleinline{
            $\mathrm{Leaf}(x)$
        }:
        \myalertinline{
            $\mathrm{Leaf}(x)\liff\neg(\exists y)(x<_Ty)$
        }\pause 
    \end{itemize}

    \medskip

    \myblock{
    Mějme teorii $T$ a formuli $\psi(x_1,\dots,x_n)$ v jazyce $L$. Označme jako $L'$ rozšíření jazyka $L$ o nový $n$-ární relační symbol $R$. \pause \alert{Extenze teorie $T$ o definici $R$ formulí $\psi$} je $L'$-teorie:\pause 
    $$
    T'=T\cup\{R(x_1,\dots,x_n)\ \liff\ \psi(x_1,\dots,x_n)\}
    $$
    }

\end{frame}


\begin{frame}{Definice relačního symbolu: vlastnosti}
    
    \pause 
    \myblock{
        \textbf{Tvrzení:}
        \begin{enumerate}[(i)]
            \item $T'$ je konzervativní extenze $T$.\pause 
            \item Pro každou $L'$-formuli $\varphi'$ existuje $L$-formule $\varphi$ taková, že $T'\models\varphi'\liff\varphi$.
        \end{enumerate}
        
    }

    \pause 
    \textbf{Důkaz:} \pause \alert{(i)} ihned ze sémantického popisu extenzí, neboť zřejmě každý model $T$ lze \alert{jednoznačně} expandovat na model $T'$

    \pause 
    \alert{(ii)} atomickou podformuli s novým symbolem $R$, tj. tvaru \alert{$R(t_1,\dots,t_n)$}, nahradíme formulí
    $$
    \psi'(x_1/t_1,\dots,x_n/t_n)
    $$\pause 
    kde $\psi'$ je \alert{varianta $\psi$ zaručující substituovatelnost} všech termů (např. přejmenujeme všechny vázané proměnné $\psi$ na zcela nové)\hfill\qedsymbol

\end{frame}


\begin{frame}{Definice funkčního symbolu: příklady}
    
    \pause 
    \myblock{
    vztah $f(x_1,\dots,x_n)=y$ definujeme formulí $\psi(x_1,\dots,x_n,y)$; pro každý vstup $(x_1,\dots,x_n)$ musí \alert{existovat jednoznačný} výstup $y$
    }

    \bigskip

    \pause 
    1. \myexampleinline{Teorie grup:} binární funkční symbol \alert{$-_b$} pomocí $+$ a unárního $-$
        
        \myalertmath{
        $$
        x_1 -_b x_2 = y\ \liff\ x_1 + (-x_2) = y
        $$
        }

    \pause 
    \begin{itemize}
        \item zřejmě pro každá $x,y$ \alert{existuje} \alert{jednoznačné} $z$ splňující definici  
    \end{itemize}
        
    \bigskip
    
    \pause 
    2. \myexampleinline{Teorie \alert{lineárních uspořádání}:} binární funkční symbol \alert{$\min$}
        
        \myalertmath{
        \vspace{-12pt}
        $$
        \min(x_1,x_2)=y\ \liff\ y\leq x_1\land y\leq x_2\land (\forall z)(z\leq x_1\land z\leq x_2\limplies z\leq y)
        $$
        }

    \pause 
    \begin{itemize}
        \item existence a jednoznačnost platí díky linearitě ($x\leq y\lor y\leq x$)\pause 
        \item pouze v teorii uspořádání by nešlo o dobrou definici: $\min^\A(a_1,a_2)$ nemusí existovat
    \end{itemize}        
    
\end{frame}


\begin{frame}{Definice funkčního symbolu: definice}

    \pause 
    \myblock{
    Mějme teorii $T$ a formuli $\psi(x_1,\dots,x_n,y)$ v jazyce $L$. Označme  $L'$ rozšíření $L$ o nový $n$-ární funkční symbol $f$. Nechť platí:\pause 
    \begin{itemize}
        \item $T\models(\exists y)\psi(x_1,\dots,x_n,y)$ \hfill \alert{\footnotesize (existence)}\pause 
        \item $T\models\psi(x_1,\dots,x_n,y)\land \psi(x_1,\dots,x_n,z)\limplies y=z$ \hfill \alert{\footnotesize (jednoznačnost)}\pause 
    \end{itemize}
    Potom \alert{extenze teorie $T$ o definici $f$ formulí $\psi$} je $L'$-teorie:

    \vspace{-12pt}
    $$
    T'=T\cup\{f(x_1,\dots,x_n)=y\ \liff\ \psi(x_1,\dots,x_n,y)\}
    $$
    }

    \pause 
    \begin{itemize}
        \item $\psi$ definuje v modelu $(n+1)$-ární relaci, ta \alert{musí být funkcí} \pause 
        \item je-li $\psi$ tvaru $t(x_1,\dots,x_n)=y$ pro term $t$, vždy to platí
    \end{itemize}

    \pause 
    \myblock{
        \textbf{Tvrzení:}
        \begin{enumerate}[(i)]
            \item $T'$ je konzervativní extenze $T$.\pause 
            \item Pro každou $L'$-formuli $\varphi'$ existuje $L$-formule $\varphi$ taková, že $T'\models\varphi'\liff\varphi$.
        \end{enumerate}        
    }

    \pause 
    \textbf{Důkaz:} \alert{(i)} modely $T$ lze \alert{jednoznačně} expandovat na modely $T'$

\end{frame}


\begin{frame}{Pokračování důkazu}

    \pause 
    \alert{(ii)} stačí pro jediný výskyt symbolu $f$, jinak induktivně (je-li více vnořených výskytů $f(\dots f(\dots)\dots)$, potom od vnitřních k vnějším)

    \pause 
    \begin{enumerate}
    \item  nahradíme term $f(t_1,\dots,t_n)$ \alert{novou} proměnnou $z$: \alert{výsledek $\varphi^*$} \pause 
    \item $\varphi$ zkonstruujeme takto: \myalertinline{
     $(\exists z)(\varphi^*\land \psi'(x_1/t_1,\dots,x_n/t_n,y/z))$
     }
     (kde $\psi'$ je varianta $\psi$ zaručující substituovatelnost)
    \end{enumerate}\pause 

    Ukážeme, že pro libovolný model $\A\models T'$ a ohodnocení $e$ platí:
    $$
    \A\models\varphi'[e]\ \text{ právě když }\ \A\models\varphi[e]
    $$\pause 
      
    Označme \alert{$a=(f(t_1,\dots,t_n))^\A[e]$}. Díky existenci a jednoznačnosti:
    $$
    \A\models\psi'(x_1/t_1,\dots,x_n/t_n,y/z)[e]\ \text{ právě když }\ e(z)=a 
    $$\pause 
    Máme tedy: $\A\models\varphi'[e]$ $\Leftrightarrow$ $\A\models\varphi^*[e(z/a)]$ $\Leftrightarrow$ $\A\models\varphi[e]$ \hfill\qedsymbol    

\end{frame}


\begin{frame}{Definice konstantního symbolu}
    
    \begin{itemize}
        \item \alert{speciální případ:} funkční symbol arity $0$\pause 
        \item extenze o definici konstantního symbolu $c$ formulí $\psi(y)$:\pause
        $$
        \alert{T'=T\cup\{c=y\liff \psi(y)\}}
        $$
        
        \pause
        \item musí platit \myalertinline{
            $T\models (\exists y)\psi(y)$
        } a \myalertinline{
            $T\models\psi(y)\land\psi(z)\limplies y=z$
        }\pause 
        \item platí stejná tvrzení
    \end{itemize}

    \pause 
    \myexampleinline{1. teorie v jazyce aritmetiky}, rozšíříme o definici symbolu $1$ formulí $\psi(y)$ tvaru \alert{$y=S(0)$}, přidáme tedy axiom \myalertinline{
            $1=y\ \liff\ y=S(0)$
            }

    \pause 
    \myexampleinline{2. teorie těles}, nový symbol $\frac{1}{2}$, definice formulí \alert{$y\cdot (1+1)=1$}, tj. přidáním 
        \myalertinline{
            $\frac{1}{2}=y\ \liff\ y\cdot (1+1)=1$
        }?\pause 
        \begin{itemize}
            \item \myalertinline{není extenze o definici!} neplatí existence: v tělese \alert{charakteristiky 2}, např. $\mathbb Z_2$, nemá rovnice $y\cdot (1+1)=1$ řešení\pause 
            \item ale v teorii těles \alert{charakteristiky různé od 2}, tj. přidáme-li axiom $\neg (1+1=0)$, už ano; např. v $\mathbb Z_3$ máme $\frac{1}{2}^{\mathbb Z_3}=2$
        \end{itemize}        

\end{frame}


\begin{frame}{Extenze o definice}

    \pause 
    \myblock{
        $L'$-teorie $T'$ je \alert{extenzí}  $L$-teorie $T$ \alert{o definice}, pokud vznikla postupnou extenzí o definice relačních a funkčních (vč. konstantních) symbolů.
    } 

    \medskip
    
    \pause 
    \myblock{
    \textbf{Tvrzení:} (snadno indukcí)\pause 
        \begin{itemize}
            \item Každý model $T$ lze jednoznačně expandovat na model $T'$.\pause 
            \item $T'$ je konzervativní extenze $T$.\pause 
            \item Pro $L'$-formuli $\varphi'$ existuje $L$-formule $\varphi$, že $T'\models\varphi'\liff\varphi$.
        \end{itemize}
    }

    \bigskip

    \pause 
    \myexampleinline{
        Příklad: $T=\{(\exists y)(x+y=0),(x+y=0)\land(x+z=0)\limplies y=z\}$
    }\pause  
    
    $L=\langle +,0,\leq\rangle$ s rovností, zavedeme \alert{\large $<$} a unární \alert{\large $-$} přidáním axiomů:\pause 
    \begin{align*}
        T'= T\cup\{ -x=y\ &\liff\ x+y=0,\\
        x<y\ &\liff\ x\leq y\land\neg(x=y)\}
    \end{align*}\pause 
    Formule \alert{$-x<y$} v jazyce $L'=\langle +,-,0,\leq,<\rangle$ s rovností je v $T'$ ekvivalentní formuli:\pause 
    \myalertinline{
        $(\exists z)((z\leq y\land\neg(z=y))\land x+z=0)$
    }

\end{frame}


\section{6.8 Definovatelnost ve struktuře}


\begin{frame}{Definovatelné množiny}

    \pause
    \begin{itemize}
        \item formule $\varphi$ s jednou volnou proměnnou $x$ \dots ``vlastnost'' prvků \pause
        \item ve struktuře \alert{definuje} množinu prvků, které vlastnost splňují (tj. prvků $a$ takových, že $\varphi$ platí při ohodnocení kde $e(x)=a$)\pause
        \item $\varphi(x,y)$ definuje binární relaci, atp.\pause
    \end{itemize}
    
    \myblock{
    Množina \alert{definovaná} $\varphi(x_1,\dots,x_n)$ \alert{ve struktuře} $\A$ (v témž jazyce):
    \vspace{-6pt}
    $$
    \alert{\varphi^\A(x_1,\dots,x_n)}=\{(a_1,\dots,a_n)\in A^n\mid\A\models\varphi[e(x_1/a_1,\dots,x_n/a_n)]\}
    $$
    \vspace{-16pt}\pause
    }

    Zkráceně píšeme: \myalertinline{
        $\varphi^\A(\bar x)=\{\bar a\in A^n\mid\A\models\varphi[e(\bar x/\bar a)]\}$
    }\pause

    \begin{itemize}
        \item formule \myexampleinline{
            $\neg(\exists y)E(x,y)$
         } definuje \myexampleinline{ v daném grafu } množinu všech \alert{izolovaných} vrcholů\pause
        \item \myexampleinline{
            $(\exists y)(y\cdot y=x)\land\neg (x=0)$
        } definuje \myexampleinline{
            v tělese $\underline{\mathbb R}$
        } množinu všech kladných reálných čísel\pause
        \item \myexampleinline{
            $x\leq y\land \neg (x=y)$
         } definuje v \myexampleinline{
            uspořádané množině $\langle S,\leq^S\rangle$
          } relaci \alert{ostrého uspořádání} $<^S$
    \end{itemize}

\end{frame}


\begin{frame}{Definovatelnost s parametry}

    \pause
    \begin{itemize}
        \item vlastnosti prvků relativně k jiným prvkům? nelze čistě syntakticky, ale můžeme dosadit prvky jako \alert{parametry}\pause
        \item zápis \alert{$\varphi(\bar x,\bar y)$}: volné proměnné $x_1,\dots,x_n,y_1,\dots,y_k$\pause
    \end{itemize}

	\myblock{
    Mějme $\varphi(\bar x,\bar y)$ (kde $|\bar x|=n$, $|\bar y|=k$), 
    strukturu $\A$ (v témž jazyce), $\bar b\in A^k$. Množina \alert{definovaná} $\varphi(\bar x,\bar y)$ \alert{s parametry} $\bar b$ \alert{ve struktuře} $\A$:
    $$
    \varphi^{\A,\bar b}(\bar x,\bar y)=\{\bar a\in A^n\mid\A\models\varphi[e(\bar x/\bar a,\bar y/\bar b)]\}
    $$
    
    \pause
    Pro $B\subseteq A$ označíme $\mathrm{Df}^n(\A,B)$ množinu všech množin definovatelných v $\A$ s parametry pocházejícími z $B$.
    }

    \pause
    \textbf{Pozorování:} $\mathrm{Df}^n(\A,B)$ je uzavřená na doplněk, průnik, sjednocení, a obsahuje $\emptyset$ a $A^n$: je to \alert{podalgebra potenční algebry} $\mathcal P(A^n)$.

    \pause
    \medskip
    Např. pro \myexampleinline{
        $\varphi(x,y)=E(x,y)$
     } a \myexampleinline{
        vrchol $v\in V(\mathcal G)$
      } je $\varphi^{\mathcal G,v}(x,y)$ množina všech sousedů vrcholu $v$.
    
\end{frame}


\begin{frame}{Aplikace: databázové dotazy}
    
    \pause
    \begin{itemize}
        \item \alert{relační databáze}: jedna nebo více \alert{tabulek}, také \alert{relace}\pause
        \item řádky tabulky jsou \alert{záznamy (records)}, také \alert{tice (tuples)}\pause
        \item struktura v čistě relačním jazyce\pause
    \end{itemize}

    \medskip

    {\ttfamily\scriptsize

        \hspace{2cm}{\bf\small Movies}
        
        \hspace{2cm}\begin{tabular}{lll}
            title          & director    & actor \\ \hline
            Forrest Gump   & R. Zemeckis & T. Hanks      \\
            Philadelphia   & J. Demme    & T. Hanks      \\
            Batman Returns & T. Burton   & M. Keaton     \\
            \vdots         & \vdots      & \vdots
        \end{tabular} \pause
 
        \hspace{2cm}{\bf\small Program}

        \hspace{2cm}\begin{tabular}{lll}
            cinema         & title          & time   \\ \hline
            Atlas          & Forrest Gump   & 20:00  \\
            Lucerna        & Forrest Gump   & 21:00  \\
            Lucerna        & Philadelphia   & 18:30  \\
            \vdots         & \vdots         & \vdots
        \end{tabular}

    }

\end{frame}


\begin{frame}{Příklad SQL dotazu}

    \pause
    \begin{itemize}
        \item  SQL dotaz v nejjednodušší formě je formule (pomineme např. \alert{agregační funkce})\pause
        \item výsledek je množina definovaná touto formulí (s parametry) \pause
       \end{itemize}
    
    \begin{center}
        \myexampleinline{``Kdy a kde můžeme vidět film s Tomem Hanksem?''}\pause
    \end{center}

    \vspace{-6pt}
        
    {\tt\footnotesize
    \alert{select} Program.cinema, Program.time \alert{from} Program, Movies \alert{where} Program.title = Movies.title  \alert{and} Movies.actor = `T. Hanks' 
    }  \pause
    
    \begin{itemize}
        \item výsledek je množina \myalertinline{
            $\varphi^{\text{Database},\text{`T. Hanks'}} (x_\mathrm{cinema},x_\mathrm{time},y_\mathrm{actor})$
        }\pause
        \item definovaná ve struktuře \alert{$\text{Database}=\langle D, \mathrm{Program}, \mathrm{Movies}\rangle$}\pause
        \item jejíž doména je {\small $D=\{\text{\tt`Atlas'},\text{\tt`Lucerna'},\dots,\text{\tt`M. Keaton'}\}$}\pause
        \item s parametrem \alert{\small\tt`T. Hanks'}, \pause
        \item definující formule \alert{\small $\varphi(x_\mathrm{cinema},x_\mathrm{time},y_\mathrm{actor})$}:\pause
        
        \medskip
        \myexampleamsmath{
        \begin{align*}
            (\exists z_\mathrm{title})(\exists z_\mathrm{director})(&\mathrm{Program}(x_\mathrm{cinema},z_\mathrm{title},x_\mathrm{time}) \land \\ &\mathrm{Movies}(z_\mathrm{title},z_\mathrm{director},y_\mathrm{actor}))          
        \end{align*}        
        }  
    \end{itemize}

\end{frame}


\section{6.9 Vztah výrokové a predikátové logiky}


\begin{frame}{Teorie Booleových algeber \hfill $L=\langle -,\landsymb,\lorsymb,\bot,\top\rangle$ s rovností}

    \myblock{\vspace{-9pt}
    \begin{columns}
        
        \footnotesize

        \column{0.5\textwidth}
        \begin{itemize}
            \item \alert{asociativita} $\land$ a $\lor$:
            \vspace{-9pt}\begin{align*}
                x\land(y\land z) &=(x\land y)\land z\\
                x\lor(y\lor z) &=(x\lor y)\lor z
            \end{align*}\vspace{-24pt}
            \item \alert{komutativita} $\land$ a $\lor$:
            \vspace{-9pt}\begin{align*}
                x\land y &= y\land x\\
                x\lor y &= y\lor x
            \end{align*}\vspace{-24pt}
            \item \alert{distributivita} $\land$ vůči $\lor$, $\lor$ vůči $\land$:
            \vspace{-9pt}\begin{align*}
                x\land(y\lor z) &= (x\land y)\lor (x\land z)\\
                x\lor(y\land z) &= (x\lor y)\land (x\lor z)
            \end{align*}\vspace{-24pt}
        \end{itemize}

        \column{0.5\textwidth}
        \begin{itemize}
            \item \alert{absorpce}:
            \vspace{-9pt}\begin{align*}
                x\land(x\lor y) &= x\\
                x\lor(x\land y) &= x
            \end{align*}\vspace{-24pt}
            \item \alert{komplementace}:
            \vspace{-9pt}\begin{align*}
                x\land(-x) &= \bot \\
                x\lor(-x) &= \top
            \end{align*}\vspace{-24pt}
            \item \alert{netrivialita}:
            \vspace{-9pt}\begin{align*}
                -(\bot &= \top)
            \end{align*}
        \end{itemize}
                

    \end{columns}
    \vspace{9pt}
    }

    \pause
    \begin{itemize}
        \item dualita: záměnou $\land$ s $\lor$ a $\bot$ s $\top$ získáme tytéž axiomy \pause   
        \item nejmenší model: \alert{2-prvková B. algebra} $\langle \{0,1\},f_\neg,f_\land,f_\lor,0,1\rangle$\pause
        \item konečné modely, až na \alert{izomorfismus} ($f^n$ je $f$ po složkách):
        \myalertmath{
        $$
        \langle \{0,1\}^n,f_\neg^n,f_\land^n,f_\lor^n,(0,\dots,0),(1,\dots,1)\rangle
        $$
        }
        
        \pause
        \item jsou izomorfní \alert{potenčním algebrám} $\mathcal P(\{1,\dots,n\})$ pomocí bijekce mezi podmnožinami a  charakteristickými vektory
    \end{itemize}
    
\end{frame}


\begin{frame}{Vztah výrokové a predikátové logiky}
   
    \pause
    \begin{itemize}
        \item výrokovou logiku lze `simulovat' v predikátové logice v teorii Booleových algeber\pause
        \item výroky jsou \alert{Booleovské termy}, konstanty $\bot,\top$ představují pravdu a lež\pause
        \item pravdivostní hodnota výroku (při daném pravdivostním ohodnocení) je hodnota termu v 2-prvkové Booleově algebře\pause
        \item kromě toho, \alert{algebra výroků} daného výrokového jazyka nebo teorie je Booleovou algebrou (i pro nekonečné jazyky)        
    \end{itemize}
    
\end{frame}


\begin{frame}{Na druhou stranu\dots}

    \pause
    \begin{itemize}
        \item máme-li \alert{otevřenou} formuli $\varphi$ (bez rovnosti), můžeme reprezentovat atomické formule pomocí prvovýroků, a získat tak výrok, který platí, právě když platí $\varphi$\pause
        \item viz Kapitola 8: Rezoluce v predikátové logice, kde se nejprve zbavíme kvantifikátorů pomocí tzv. \alert{Skolemizace}\pause
        \item výrokovou logiku lze také zavést jako fragment logiky predikátové, pokud povolíme \alert{nulární relace}\pause
        \item $A^0=\{\emptyset\}$, tedy na libovolné množině jsou právě dvě nulární relace $R^A\subseteq A^0$: $R^A=\emptyset=0$ a $R^A=\{\emptyset\}=\{0\}=1$
    \end{itemize}
        
\end{frame}


\section{\sc Kapitola 7: Tablo metoda v predikátové logice}


\section{7.1 Neformální úvod}

\begin{frame}{Úvodní příklady: dva tablo důkazy}
    
    \pause
    \begin{minipage}{.49\textwidth}
        \centering
        \scalebox{0.87}{
        \begin{forest}
            for tree={math content}
            [\F(\exists x)\neg P(x)\limplies\neg(\forall x)P(x)
                [\textcolor{red}{\T(\exists x)\neg P(x)}
                    [\F\neg(\forall x)P(x)
                        [\textcolor{blue}{\T(\forall x)P(x)}
                            [\T\neg P(c_0)
                                [\F P(c_0)
                                    [\textcolor{blue}{\T(\forall x)P(x)}
                                        [\T P(c_0), tikz={\node[fit to=tree,label=below:$\otimes$] {};}]
                                    ]
                                ]
                            ]                
                        ]
                    ]
                ]
            ]
        \end{forest}
        }
    \end{minipage}\pause
    \begin{minipage}{.49\textwidth}
        \centering
        \scalebox{0.87}{
        \begin{forest}
            for tree={math content}
            [\F\neg(\forall x)P(x)\limplies(\exists x)\neg P(x)
                [\T\neg(\forall x) P(x)
                    [\textcolor{blue}{\F(\exists x)\neg P(x)}
                        [\textcolor{red}{\F(\forall x)P(x)}
                            [\F P(c_0)
                                [\textcolor{blue}{\F (\exists x)\neg P(x)}
                                    [\F\neg P(c_0)
                                        [\T P(c_0), tikz={\node[fit to=tree,label=below:$\otimes$] {};}]
                                    ]
                                ]
                            ]                
                        ]
                    ]
                ]
            ]
        \end{forest}
        }
    \end{minipage}

    % \begin{itemize}\footnotesize
    %     \item $c_0$ je \alert{pomocný konstantní symbol} (přidáme do jazyka)
    %     \item kvantifikátory: položky typu ``\textcolor{red}{svědek}'' vs. typu ``\textcolor{blue}{všichni}''
    % \end{itemize}

\end{frame}


\begin{frame}{Tablo metoda v predikátové logice}

    \pause
    \begin{itemize}
        \item opět vždy předpokládáme, že jazyk $L$ je spočetný
        (nejprve bez rovnosti, později metodu rozšíříme pro rovnost)\pause
        \item v položkách musí být \alert{sentence}: pravdivostní hodnota nesmí záviset na ohodnocení (ale můžeme vzít \alert{generální uzávěry})\pause
        \item \alert{redukce položek}: stejná atomická tabla pro logické spojky (kde $\varphi,\psi$ jsou sentence), ale čtyři nové případy \alert{pro kvantifikátory}:\pause
        \begin{itemize}
            \item typ ``\textcolor{red}{svědek}'': položky tvaru \textcolor{red}{$\mathrm{T}(\exists x)\varphi(x)$} a \textcolor{red}{$\mathrm{F}(\forall x)\varphi(x)$}\pause
            \item typ ``\textcolor{blue}{všichni}'': položky tvaru \textcolor{blue}{$\mathrm{T}(\forall x)\varphi(x)$} a \textcolor{blue}{$\mathrm{F}(\exists x)\varphi(x)$}\pause
        \end{itemize}
        \item kvantifikátor nelze odstranit, $\varphi(x)$ by typicky nebyla sentence\pause
        \item místo toho za $x$ \alert{substituujeme} \alert{konstantní term} $t$: \myalertinline{
            $\varphi(x/t)$
        }\pause
        \item jaký? podle typu položky (``\textcolor{red}{svědek}'' vs. ``\textcolor{blue}{všichni}'')
        % \begin{itemize}
        %     \item jazyk rozšíříme o pomocné konstantní symboly
        %     \item typ ``\textcolor{red}{svědek}'': nový pomocný konstantní symbol, reprezentuje `svědka'
        %     \item typ ``\textcolor{blue}{všichni}'': jakýkoliv konstantní term (na bezesporné dokončené větvi musíme substituovat všechny -- )
        % \end{itemize}
       
    \end{itemize}        

\end{frame}


\begin{frame}{Redukce položek s kvantifikátorem}

    \pause
    \begin{itemize}
        \item jazyk $L$ rozšíříme o spočetně mnoho \alert{nových (pomocných) konstantních symbolů} $C=\{c_0,c_1,c_2,\dots\}$, 
        %(ale píšeme i $c,d,\dots$)
        označíme \alert{$L_C$}\pause
        \item vždy máme k dispozici \alert{nový}, dosud \alert{nepoužitý} symbol $c\in C$\pause
        
        \medskip

        \item \textbf{typ} ``\textcolor{red}{svědek}''\textbf{:} dosadíme \alert{nový} $c\in C$ (dosud na větvi není)\pause
        \begin{itemize}
            \item pro $\T(\exists x)\varphi(x)$ tedy máme $\T\varphi(x/c)$\pause
            \item $c$ hraje roli prvku, který položku `splňuje'\pause
        \end{itemize}

        \medskip

        \item \textbf{typ} ``\textcolor{blue}{všichni}''\textbf{:} substituujeme \alert{libovolný} konstantní $L_C$-term\pause
        \begin{itemize}
            \item pro $\T(\forall x)\varphi(x)$ tedy máme $\T\varphi(x/t)$\pause
            \item bezesporná větev je \alert{dokončená} jen pokud \alert{dosadíme všechny $t$} (`použijeme vše, co víme')\pause
        \end{itemize}

        \medskip

        \item \textbf{konvence:} kořeny atomických tabel nekreslíme \alert{kromě položek typu} ``\textcolor{blue}{všichni}'' (po jednom dosazení ještě nejsme hotovi!)\pause
        
        \medskip
        
        \item \textbf{typický postup:} nejprve zredukujeme položky typu ``\textcolor{red}{svědek}'', poté zjistíme, co `\alert{o svědcích říkají}' položky typu ``\textcolor{blue}{všichni}'' 
        
    \end{itemize}

\end{frame}


\section{7.2 Formální definice}


\begin{frame}{Jazyk, položky, atomická tabla}

    \begin{itemize}[<+->]
        \item buď $L$ \alert{spočetný} jazyk \alert{bez rovnosti}.
        \item označme $L_C$ rozšíření $L$ o spočetně mnoho nových \alert{pomocných} konstantních symbolů $C=\{c_i\mid i\in \mathbb N\}$
        \item zvolme očíslování konstantních $L_C$-termů: $\{t_i\mid i\in\mathbb N\}$
        \item mějme nějakou $L$-teorii $T$ a $L$-sentenci $\varphi$
        \item \alert{položka} je nápis $\T\varphi$ nebo $\F\varphi$, kde $\varphi$ je $L_C$-sentence
        \item položky tvaru $\T(\exists x)\varphi(x)$ a $\F(\forall x)\varphi(x)$ jsou \alert{typu} ``\textcolor{red}{svědek}''
        \item položky tvaru $\T(\forall x)\varphi(x)$ a $\F(\exists x)\varphi(x)$ jsou \alert{typu} ``\textcolor{blue}{všichni}'' 
        \item \alert{atomická tabla} jsou násl. položkami označkované stromy:
    \end{itemize}

\end{frame}


\begin{frame}{Atomická tabla pro kvantifikátory}

    $\varphi$ je libovolná $L_C$-sentence, $x$ proměnná, $t_i$ konstantní $L_C$-term,
    $c_i\in C$ je nový pomocný konstantní symbol (při konstrukci tabla nesměl dosud být na dané větvi)
    
    \begin{center}
        \begin{tabular}{@{}c||c|c@{}}
            & $\forall$ & $\exists$ \\ \midrule \midrule
            True
            &  
            \textcolor{blue}{
            \begin{forest}
                [$\T(\forall x)\varphi(x)$ [$\T\varphi(x/t_i)$]]
            \end{forest}
            }
            &
            \textcolor{red}{  
            \begin{forest}
                [$\T(\exists x)\varphi(x)$ [$\T\varphi(x/c_i)$]]
            \end{forest}
            }
            \\ \midrule
            False 
            &  
            \textcolor{red}{
            \begin{forest}
                [$\F(\forall x)\varphi(x)$ [$\F\varphi(x/c_i)$]]
            \end{forest}
            }
            &  
            \textcolor{blue}{
            \begin{forest}
                [$\F(\exists x)\varphi(x)$ [$\F\varphi(x/t_i)$]]
            \end{forest} 
            }
        \end{tabular}
    \end{center}    

\end{frame}


\begin{frame}{Atomická tabla pro logické spojky}

    $\varphi$ a $\psi$ jsou libovolné $L_C$-sentence
    
    \begin{center}
        \scalebox{0.9}{
        \begin{tabular}{@{}c||c|c|c|c|c@{}}
            & $\neg$ & $\land$ & $\lor$ & $\limplies$ & $\liff$  \\ \midrule \midrule
            True
            &  
            \begin{forest}
            [$\mathrm{T}\neg\varphi$ [$\mathrm{F}\varphi$]]
            \end{forest}
            &  
            \begin{forest}
            [$\mathrm{T}\varphi\land\psi$ [$\mathrm{T}\varphi$ [$\mathrm{T}\psi$]]]
            \end{forest}
            & 
            \begin{forest}
            [$\mathrm{T}\varphi\lor\psi$ [$\mathrm{T}\varphi$] [$\mathrm{T}\psi$]]
            \end{forest}
            &
            \begin{forest}
            [$\mathrm{T}\varphi\limplies\psi$ [$\mathrm{F}\varphi$] [$\mathrm{T}\psi$]]
            \end{forest}
            &  
            \begin{forest}
            [$\mathrm{T}\varphi\liff\psi$ [$\mathrm{T}\varphi$ [$\mathrm{T}\psi$]] [$\mathrm{F}\varphi$ [$\mathrm{F}\psi$]]]
            \end{forest}
            \\ \midrule
            False 
            & 
            \begin{forest}
            [$\mathrm{F}\neg\varphi$ [$\mathrm{T}\varphi$]]
            \end{forest}
            &
            \begin{forest}
            [$\mathrm{F}\varphi\land\psi$ [$\mathrm{F}\varphi$] [$\mathrm{F}\psi$]]
            \end{forest}
            &
            \begin{forest}
            [$\mathrm{F}\varphi\lor\psi$ [$\mathrm{F}\varphi$ [$\mathrm{F}\psi$]]]
            \end{forest}
            &
            \begin{forest}
            [$\mathrm{F}\varphi\limplies\psi$ [$\mathrm{T}\varphi$ [$\mathrm{F}\psi$]]]
            \end{forest}
            &
            \begin{forest}
            [$\mathrm{F}\varphi\liff\psi$ [$\mathrm{T}\varphi$ [$\mathrm{F}\psi$]] [$\mathrm{F}\varphi$ [$\mathrm{T}\psi$]]]
            \end{forest}
        \end{tabular}
        }
    \end{center}
    
\end{frame}


\begin{frame}{Formální definice tabla}
    
    \pause
    \begin{itemize}
        \item \alert{konečné tablo z teorie $T$} je uspoř., položkami označ. strom zkonstruovaný aplikací konečně mnoha následujících pravidel:\pause
        \begin{itemize}
            \item jednoprvkový strom s libovolnou položkou je tablo z teorie $T$\pause
            \item pro libovolnou položku $P$ na libovolné větvi $V$ můžeme na konec větve $V$ připojit atomické tablo pro položku $P$\pause
            
            \medskip
            
            \myalert{je-li $P$ typu ``\textcolor{red}{svědek}'', můžeme použít jen $c_i\in C$, který dosud na $V$ není (pro typ ``\textcolor{blue}{všichni}'' lze použít lib. konst. $L_C$-term $t_i$)\pause
            }

            \medskip
            
            \item na konec libovolné větve můžeme připojit položku $\mathrm{T}\alpha$ pro libovolný axiom $\alpha\in T$\pause
        \end{itemize}
        \item \alert{tablo z teorie $T$} je buď konečné, nebo i nekonečné: v tom případě je spočetné a definujeme ho jako $\tau=\bigcup_{i\geq 0}\tau_i$, kde:\pause
        \begin{itemize}
            \item $\tau_i$ jsou konečná tabla z $T$\pause
            \item $\tau_0$ je jednoprvkové tablo\pause
            \item $\tau_{i+1}$ vzniklo z $\tau_i$ v jednom kroku\pause
        \end{itemize}
        \item \alert{tablo pro položku $P$} je tablo, které má položku $P$ v kořeni\pause
    \end{itemize}
   
    \textbf{konvence:} kořen atom. tabla nezapisujeme \myalertinline{není-li $P$ typu ``\textcolor{blue}{všichni}''} 

\end{frame}


\begin{frame}{Dokončené a sporné tablo}

    \begin{itemize}[<+->]
        \item Tablo je \alert{sporné}, pokud je každá jeho větev sporná.
        \item Větev je \alert{sporná}, pokud obsahuje položky $\mathrm{T}\psi$ a $\mathrm{F}\psi$ pro nějakou \myalertinline{sentenci} $\psi$, jinak je \alert{bezesporná}.
        \item Tablo je \alert{dokončené}, pokud je každá jeho větev dokončená.
        \item Větev je \alert{dokončená}, pokud je sporná, nebo
        \begin{itemize}[<+->]
            \item každá její položka je na této větvi \alert{redukovaná},
            \item a zároveň obsahuje položku $\mathrm{T}\alpha$ pro každý axiom $\alpha\in T$.
        \end{itemize}
         
        \item Položka $P$ je \alert{redukovaná} na větvi $V$ procházející $P$, pokud 
        \begin{itemize}[<+->]
            \item je tvaru $\mathrm{T}\psi$ resp. $\mathrm{F}\psi$ pro \myalertinline{atomickou sentenci}, nebo
            \item \myalertinline{není typu ``\textcolor{blue}{všichni}'' a} vyskytuje se na $V$ jako kořen atomického tabla (tj., typicky, již došlo k jejímu rozvoji na $V$)\myalertinline{, nebo}
        \end{itemize}
        \myalert{
        \begin{itemize}
            \item je typu ``\textcolor{blue}{všichni}'' a všechny její \alert{výskyty} na větvi $V$ jsou na $V$ \alert{redukované}.
        \end{itemize}
        }
    \end{itemize}

\end{frame}


\begin{frame}{Kdy je výskyt položky typu ``všichni'' redukovaný?}
    
    \pause
    Výskyt položky $P$ typu ``\textcolor{blue}{všichni}'' na $V$ je \alert{$i$-tý}, má-li právě $i-1$ předků označených $P$, a \alert{$i$-tý výskyt je redukovaný} na $V$, pokud\pause
    \begin{itemize}
        \item $P$ má $(i+1)$-ní výskyt na $V$, a zároveň\pause
        \item na $V$ je položka \alert{$\T\varphi(x/t_i)$} (je-li $P=\T(\forall x)\varphi(x)$) resp. \alert{$\F\varphi(x/t_i)$} (je-li $P=\F(\exists x)\varphi(x)$), kde $t_i$ je $i$-tý konstantní $L_C$-term (tj., typicky, už jsme za $x$ substituovali $t_i$)\pause
    \end{itemize} 


    \textbf{NB:} je-li položka typu ``\textcolor{blue}{všichni}'' na $V$ redukovaná, má na $V$ nekonečně výskytů, a dosadili jsme všechny konstantní $L_C$-termy

\end{frame}


\begin{frame}{Tablo důkaz a tablo zamítnutí}

    \begin{itemize}[<+->]
        \item \alert{tablo důkaz} \myalertinline{sentence} $\varphi$ z teorie $T$ je \alert{sporné} tablo z teorie $T$ s položkou $\mathrm{F}\varphi$ v kořeni
        \item pokud existuje, je $\varphi$ \alert{(tablo) dokazatelný} z $T$, píšeme \alert{$T\proves\varphi$}
        \item podobně, \alert{tablo zamítnutí} je sporné tablo s $\mathrm{T}\varphi$ v kořeni
        \item existuje-li, je $\varphi$ \alert{(tablo) zamítnutelný} z $T$, tj. platí \alert{$T\proves\neg\varphi$}
    \end{itemize}

\end{frame}


\begin{frame}{Příklad: tablo důkaz (v logice)}

    \centering
    \scalebox{0.64}{
        \begin{forest}
            for tree={math content}
            [\F(\forall x)(P(x) \limplies Q(x)) \limplies ((\forall x)P(x) \limplies (\forall x)Q(x))
                [\textcolor{blue}{\T(\forall x)(P(x) \limplies Q(x))}
                    [\F(\forall x)P(x) \limplies (\forall x)Q(x)
                        [\textcolor{blue}{\T(\forall x)P(x)}
                            [\textcolor{red}{\F(\forall x)Q(x)}
                                [\F Q(c_0)
                                    [\textcolor{blue}{\T(\forall x)P(x)}
                                        [\T P(c_0)
                                            [\textcolor{blue}{\T(\forall x)(P(x) \limplies Q(x))}
                                                [\T P(c_0)\limplies Q(c_0)
                                                    [\F P(c_0), tikz={\node[fit to=tree,label=below:$\otimes$] {};}]
                                                    [\T Q(c_0), tikz={\node[fit to=tree,label=below:$\otimes$] {};}]            
                                                ]
                                            ]
                                        ]
                                    ]
                                ]
                            ]                
                        ]
                    ]
                ]
            ]
        \end{forest}
    }

\end{frame}


\begin{frame}{Ještě příklad ($\varphi,\psi$ jsou formule s jedinou volnou proměnnou $x$)}

    \centering
    \scalebox{0.73}{
        \centering
        \begin{forest}
        for tree={math content}
        [\F(\forall x)(\varphi(x) \land \psi(x)) \liff((\forall x)\varphi (x) \land (\forall x)\psi(x))
            [\textcolor{blue}{\T(\forall x)(\varphi(x) \land \psi(x))}
                [\F(\forall x)\varphi (x) \land (\forall x)\psi(x)
                    [\textcolor{red}{\F(\forall x)\varphi (x)}
                        [\F\varphi(c_0)
                            [\textcolor{blue}{\T(\forall x)(\varphi(x) \land \psi(x))}
                                [\T\varphi(c_0) \land \psi(c_0)
                                    [\T\varphi(c_0)
                                        [\T\psi(c_0), tikz={\node[fit to=tree,label=below:$\otimes$] {};}]
                                    ]
                                ]
                            ]
                        ]
                    ]
                    [\textcolor{red}{\F(\forall x)\psi(x)}
                        [\F\psi(c_0)
                            [\textcolor{blue}{\T(\forall x)(\varphi(x) \land \psi(x))}
                                [\T\varphi(c_0) \land \psi(c_0)
                                    [\T\varphi(c_0)
                                        [\T\psi(c_0), tikz={\node[fit to=tree,label=below:$\otimes$] {};}]
                                    ]
                                ]
                            ]
                        ]
                    ]
                ]
            ]
            [\textcolor{red}{\F(\forall x)(\varphi(x) \land \psi(x))}
                [\T(\forall x)\varphi (x) \land (\forall x)\psi(x)
                    [\textcolor{blue}{\T(\forall x)\varphi (x)}
                        [\textcolor{blue}{\T(\forall x)\psi(x)}
                            [\F(\varphi(c_0) \land \psi(c_0))
                                [\F\varphi(c_0)
                                    [\textcolor{blue}{\T(\forall x)\varphi (x)}
                                        [\T\varphi(c_0), tikz={\node[fit to=tree,label=below:$\otimes$] {};}]
                                    ]
                                ]
                                [\F\psi(c_0)
                                    [\textcolor{blue}{\T(\forall x)\psi (x)}
                                        [\T\psi(c_0), tikz={\node[fit to=tree,label=below:$\otimes$] {};}]
                                    ]
                                ]
                            ]                
                        ]
                    ]
                ]
            ]
        ]
        \end{forest}
    }

    \vspace{-4pt}
    \footnotesize
    ($c_0$ lze použít jako \alert{nový} ve všech případech: \alert{na dané větvi} se dosud nevyskytuje)

\end{frame}


\begin{frame}{Systematické tablo}

    \pause
    \vspace{-6pt}
    musí někdy zredukovat každou položku, použít každý axiom, a nově ve všech položkách typu ``\textcolor{blue}{všichni}'' \alert{dosadit každý $L_C$ term $t_i$ }

    \pause
    \myblock{
    \alert{Systematické tablo} z $T=\{\alpha_0,\alpha_1,\alpha_2,\dots\}$ pro položku $R $ je $\tau=\bigcup_{i\geq 0}\tau_i$, kde $\tau_0$ je jednoprvkové s položkou $R$, a pro $i\geq 0$:

    \pause
    \begin{itemize}
        \item buď $P$ nejlevější položka v co nejmenší úrovni, která není redukovaná na nějaké bezesporné větvi procházející $P$ \myalertinline{(resp. je-li typu ``\textcolor{blue}{všichni}'', její \alert{výskyt} není redukovaný)}\pause
        \item nejprve definujeme $\tau_i'$ vzniklé z $\tau_i$ připojením atomického tabla pro $P$ na každou bezespornou větev procházející~$P$, kde\pause
        
        \smallskip
        
        \myalert{
        je-li $P$ typu ``\textcolor{blue}{všichni}'' a má-li ve vrcholu $k$-tý výskyt, dosadíme $k$-tý $L_C$-term $t_k$,
        je-li typu ``\textcolor{red}{svědek}'', substituujeme $c_i\in C$ s nejmenším $i$, které na větvi zatím není
        }

        \pause
        \item pokud taková položka $P$ neexistuje, potom $\tau_i'=\tau_i$\pause
        \item $\tau_{i+1}$ vznikne z $\tau_i'$ připojením $\mathrm{T}\alpha_{i+1}$ na vš. bezesporné větve (pokud už jsme použili všechny axiomy, definujeme $\tau_{i+1}=\tau_i'$)
    \end{itemize} 
    }

\end{frame}


\begin{frame}{Konečnost a systematičnost důkazů}

    \pause
    \myblock{
        \textbf{Lemma:} Systematické tablo je dokončené.
    }

    \pause
    \textbf{Důkaz:} $k$-tý výskyt položky typu ``\textcolor{blue}{všichni}'' redukujeme když na něj narazíme: připojíme $(k+1)$-ní výskyt a dosadíme $k$-tý $L_C$-term $t_k$. Zbytek důkazu jako ve výrokové logice.\hfill\qedsymbol

    \bigskip

    \pause
    Neprodlužujeme-li sporné větve (což nemusíme), je sporné tablo vždy konečné. Důkaz stejný jako ve výrokové logice:

    \pause
    \myblock{
        \textbf{Důsledek (Konečnost důkazů):}
    Pokud $T\proves\varphi$, potom existuje i konečný tablo důkaz $\varphi$ z $T$.
    }

    \bigskip

    \pause
    Stejně jako ve výrokové logice z důkazu plyne:

    \pause
    \myblock{
        \textbf{Důsledek (Systematičnost důkazů):}
        Pokud $T\proves\varphi$, potom systematické tablo je (konečným) tablo důkazem $\varphi$ z $T$.
    }
    
\end{frame}


\section{7.3 Jazyky s rovností}


\begin{frame}{Rovnost}

    \pause
    $1+0=0+1$? identita celých čísel, výrazů, množin, unifikovatelnost termů (v Prologu), \dots

    \pause
    Tablo je čistě \alert{syntaktický} objekt, ale \alert{$=^\A$} má být \alert{identita} na $A$. Jak toho docílit?

    \pause
    Mějme dokončenou bezespornou větev tabla s položkou \alert{$\T c_1=c_2$}. V \alert{kanonickém modelu} musí platit nejen \alert{$(c_1^\A,c_2^\A)\in {=^\A}$}, ale také:

    \pause
    \begin{itemize}
        \item $c_2^\A =^\A c_1^\A$\pause
        \item $f^\A(c_1^\A) =^\A f^\A(c_2^\A)$\pause
        \item $c_1^\A\in P^\A$ právě když $c_2^\A\in P^\A$
    \end{itemize}

    \pause
    To vynutíme přidáním \alert{axiomů rovnosti}, $=^\A$ bude \alert{kongruence}~$\A$ (ekvivalence, která se chová dobře k funkcím a relacím). 
    
    \pause
    Poté vezmeme \alert{faktorstrukturu} $\B=\A/_{=^\A}$, v ní už je $=^\B$ \alert{identita}.

\end{frame}


\begin{frame}{Kongruence a faktorstruktura}
    
    \pause    
    Buď $\sim$ ekvivalence na $A$, $f\colon A^n\to A$, $R\subseteq A^n$. Říkáme, že $\sim$ je:\pause
    \begin{itemize}
        \item \alert{kongruence pro $f$}, pokud pro všechna $a_i,b_i\in A$ taková, že \myalertinline{
            $a_i\sim b_i$ ($1\leq i\leq n$)
         }, platí \myalertinline{
            $f(a_1,\dots,a_n)\sim f(b_1,\dots,b_n)$
        }\pause
        \item \alert{kongruence pro $R$}, pokud pro všechna $a_i,b_i\in A$ taková, že \myalertinline{
            $a_i\sim b_i$ ($1\leq i\leq n$)
         }, platí \myalertinline{
            $R(a_1,\dots,a_n)$ $\Leftrightarrow$ $R(b_1,\dots,b_n)$
        }\pause
    \end{itemize}    
    \alert{Kongruence} struktury $\A$ je ekvivalence na $A$, která je kongruencí pro všechny funkce a relace $\A$. \pause

    \alert{Faktorstruktura (podílová struktura)} $\A$ podle $\sim$ je struktura \alert{$\A/_\sim$ } v témž jazyce, doména $A/_\sim$ je množina všech rozkladových tříd $A$ podle $\sim$, funkce a relace definujeme \alert{pomocí reprezentantů}:\pause
    
    \myalert{
    \begin{itemize}
        \item $f^{\A/_\sim}([a_1]_\sim,\dots,[a_n]_\sim)=[f^\A(a_1,\dots,a_n)]_\sim$\pause
        \item $R^{\A/_\sim}([a_1]_\sim,\dots,[a_n]_\sim)$ $\Leftrightarrow$ $R^\A(a_1,\dots,a_n)$
    \end{itemize}
    }
        
\end{frame}


\begin{frame}{Axiomy rovnosti}

    \myblock{
    \alert{Axiomy rovnosti} pro jazyk $L$ s rovností:\pause
    \begin{enumerate}[(i)]
        \item $x=x$\pause
        \item pro každý $n$-ární funkční symbol $f$ jazyka $L$:
        $$
        x_1=y_1\land\cdots\land x_n=y_n\limplies f(x_1,\dots,x_n)=f(y_1,\dots,y_n)
        $$

        \pause
        \item pro každý $n$-ární relační symbol $R$ jazyka $L$ \alert{včetně rovnosti}:
        $$
        x_1=y_1\land\cdots\land x_n=y_n\limplies (R(x_1,\dots,x_n)\limplies R(y_1,\dots,y_n))
        $$ 
    \end{enumerate}
    }

    \pause
    \begin{itemize}
        \item symetrie a tranzitivita plynou z (iii) pro $=$ (dokažte si)\pause
        \item z axiomů $(i)$ a $(iii)$ tedy plyne, že relace $=^\A$ je ekvivalence\pause
        \item axiomy $(ii)$ a $(iii)$ vyjadřují, že $=^\A$ je kongruence\pause
    \end{itemize}

    V tablo metodě pro jazyk s rovností implicitně přidáme axiomy rovnosti (přesněji jejich generální uzávěry, potřebujeme sentence).

\end{frame}


\begin{frame}{Tablo důkaz s rovností}

    \pause
    Je-li $T$ teorie v jazyce $L$ s rovností, označme jako $T^*$ rozšíření $T$ o generální uzávěry axiomů rovnosti pro $L$. 

    \pause
    \begin{itemize}
        \item \alert{tablo důkaz} z teorie $T$ je \alert{tablo důkaz} z $T^*$\pause
        \item podobně pro tablo zamítnutí, a obecně jakékoliv tablo z $T$
    \end{itemize}
 
    \pause
    \textbf{Pozorování:}\pause
    \begin{itemize}
        \item Je-li $\A\models T^*$, potom i $\A/_{=^\A}\models T^*$, a ve struktuře $\A/_{=^\A}$ je symbol rovnosti interpretován jako identita.\pause
        \item Na druhou stranu, v každém modelu, ve kterém je symbol rovnosti interpretován jako identita, platí axiomy rovnosti.
    \end{itemize}
     
    \pause
    (Použijeme při konstrukci \alert{kanonického modelu} v důkazu úplnosti.)

\end{frame}


\end{document}
