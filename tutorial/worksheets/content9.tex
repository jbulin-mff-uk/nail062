\section*{NAIL062 V\&P Logika: 9. sada příkladů -- Příprava na rezoluci v PL}
% po 5. přednášce


\subsection*{Cíle výuky:} Po absolvování cvičení student

    \begin{itemize}\setlength{\itemsep}{0pt}
        \item umí převádět formule do prenexní normální formy (PNF)
        \item rozumí pojmu Skolemova varianta, umí skolemizovat danou teorii
        \item umí převést danou otevřenou teorii do CNF, zapsat v množinové reprezentaci
        \item zná Herbrandovu větu, umí ji demonstrovat na příkladě, popsat Herbrandův model
    \end{itemize}
    

\section*{Příklady na cvičení}


\begin{problem} 
    
    Převeďte následující formule do PNF. Poté najděte jejich Skolemovy varianty.
    \begin{enumerate}[(a)]
        \item $(\forall y)((\exists x)P(x,y)\to Q(y,z))\wedge (\exists y)((\forall x)R(x,y)\vee Q(x,y))$
        \item $(\exists x)R(x,y)\leftrightarrow (\forall y)P(x,y)$
        \item $\neg((\forall x)(\exists y)P(x,y)\to (\exists x)(\exists y)R(x,y))\wedge(\forall x)\neg(\exists y)Q(x,y)$
    \end{enumerate}

    \begin{solution}

        \begin{enumerate}[(a)]
            \item Volné proměnné jsou $x,z$. (Můžete si nakreslit strom formule.) Nejprve nahradíme formuli variantou, kde přejmenujeme vázané proměnné aby byly různé, a různé od volných. Dostáváme ekvivalentní formuli:
            $$
            (\forall y_1)((\exists x_1)P(x_1,y_1)\to Q(y_1,z))\wedge (\exists y_2)((\forall x_2)R(x_2,y_2)\vee Q(x,y_2))
            $$
            Nyní postupujeme aplikací pravidel pro převod do PNF, podle stromu formule (posouváme kvantifikátory nahoru). Pořadí vytýkání volíme tak, že nejprve vytkneme ty kvanfitikátory, které nakonec (až budou v kvantifikátorovém prefixu, tj. nahoře u kořene stromu) budou existenční, a teprve potom ty, které budou univerzální (abychom zbytečně nevytvořili `závislost' ve Skolemově variantě). Dostáváme:
            \begin{align*}
                &(\forall y_1)((\exists x_1)P(x_1,y_1)\to Q(y_1,z))\wedge (\exists y_2)((\forall x_2)R(x_2,y_2)\vee Q(x,y_2))\\
                \sim\ & (\forall y_1)(\forall x_1)(P(x_1,y_1)\to Q(y_1,z))\wedge (\exists y_2)(\forall x_2)(R(x_2,y_2)\vee Q(x,y_2))\\
                \sim\ & (\exists y_2)((\forall y_1)(\forall x_1)(P(x_1,y_1)\to Q(y_1,z))\wedge (\forall x_2)(R(x_2,y_2)\vee Q(x,y_2)))\\
                \sim\ & (\exists y_2)(\forall y_1)(\forall x_1)(\forall x_2)((P(x_1,y_1)\to Q(y_1,z))\wedge (R(x_2,y_2)\vee Q(x,y_2)))
            \end{align*}
            (Pozor na pravidla pro implikaci: při vytknutí z antecedentu se mění kvantifikátor, alternativně lze nejprve přepsat implikaci $\varphi\limplies\psi$ jako $\neg\varphi\lor\psi$. Buďte také opatrní, aby proměnná ve vytknutém kvantifikátoru nebyla volná v druhé části formule, tomu jsme ale zajistili přejmenováním.)

            Nezapomeňte, že pro skolemizaci potřebujeme sentenci, tj. generální uzávěr formule:
            $$
            (\forall x)(\forall z)(\exists y_2)(\forall y_1)(\forall x_1)(\forall x_2)((P(x_1,y_1)\to Q(y_1,z))\wedge (R(x_2,y_2)\vee Q(x,y_2)))
            $$
            Skolemova varianta je potom:
            $$
            (\forall x)(\forall z)(\forall y_1)(\forall x_1)(\forall x_2)((P(x_1,y_1)\to Q(y_1,z))\wedge (R(x_2,f(x,z))\vee Q(x,f(x,z))))
            $$
            Zde $f$ je nový, binární funkční symbol. (Pozor, skolemizujeme-li teorii, všechny funkční symboly použité při skolemizaci všech axiomů musí být nové, navzájem různé.)

            Skolemova varianta je z definice sentence, ale i její otevřené jádro $(P(x_1,y_1)\to Q(y_1,z))\wedge (R(x_2,f(x,z))\vee Q(x,f(x,z)))$ je ekvisplnitelné (ale typicky ne ekvivalentní!) s původní formuli.
            
            \item Postupujeme stejně jako v (a), ale ekvivalenci nejprve přepíšeme na dvojici implikací:
            \begin{align*}
                &(\exists x)R(x,y)\leftrightarrow (\forall y)P(x,y)\\
                \sim\ &((\exists x)R(x,y)\limplies (\forall y)P(x,y))\land((\forall y)P(x,y)\limplies (\exists x)R(x,y))\\
                \sim\ &((\exists x_1)R(x_1,y)\limplies (\forall y_1)P(x,y_1))\land((\forall y_2)P(x,y_2)\limplies (\exists x_2)R(x_2,y))\\
                \sim\ &(\exists x_2)(\exists y_2)(\forall x_1)(\forall y_1)((R(x_1,y)\limplies P(x,y_1))\land(P(x,y_2)\limplies R(x_2,y)))\\
                \sim\ &(\forall x)(\forall y)(\exists x_2)(\exists y_2)(\forall x_1)(\forall y_1)((R(x_1,y)\limplies P(x,y_1))\land(P(x,y_2)\limplies R(x_2,y)))
            \end{align*}
            Skolemova varianta ($f,g$ jsou nové, binární funkční symboly):
            $$
            (\forall x)(\forall y)(\forall x_1)(\forall y_1)((R(x_1,y)\limplies P(x,y_1))\land(P(x,g(x,y))\limplies R(f(x,y),y)))
            $$

            \item Všimněte si, že formule je sentence. Obdobným postupem:
            \begin{align*}
            &\neg((\forall x)(\exists y)P(x,y)\to (\exists x)(\exists y)R(x,y))\wedge(\forall x)\neg(\exists y)Q(x,y)\\
            \sim\ &\neg((\forall x)(\exists y)P(x,y)\to (\exists x')(\exists y')R(x',y'))\wedge(\forall x'')\neg(\exists y'')Q(x'',y'')\\
            \sim\ & (\forall x)(\exists y)(\forall x')(\forall y')(\forall x'')(\forall y'')
            (    
                \neg(P(x,y)\to R(x',y'))\wedge\neg Q(x'',y'') 
            )
            \end{align*}
            
        \end{enumerate}
                    
    \end{solution}

\end{problem}


\begin{problem} 
    
    Převeďte na ekvisplnitelnou CNF formuli, zapište v množinové reprezentaci.

    \begin{enumerate}[(a)]
        \item $(\forall y)(\exists x)P(x,y)$
        \item $\neg (\forall y)(\exists x)P(x,y)$
        \item $\neg (\exists x)((P(x)\to P(c))\wedge (P(x)\to P(d)))$
        \item $(\exists x)(\forall y)(\exists z)(P(x,z)\wedge P(z,y) \to R(x,y))$
    \end{enumerate}

    \begin{solution}
        Nejprve vytvoříme Skolemovu variantu (viz předchozí příklad), poté vezmeme její otevřené jádro, a převedeme do CNF pomocí ekvivalentních úprav (stejně jako ve výrokové logice):
        \begin{enumerate}[(a)]
            \item $(\forall y)(\exists x)P(x,y)$ už je sentence v PNF, Skolemova varianta: $(\forall y)(P(f(y),y))$ ($f$ nový, unární funkční symbol). Otevřené jádro: $P(f(y),y)$, CNF v množinové reprezentaci: 
            $$
            S=\{\{P(f(y),y)\}\}
            $$
            \item $\neg (\forall y)(\exists x)P(x,y)\sim(\exists y)(\forall x)\neg P(x,y)$, Skolemova varianta $(\forall x)\neg P(x,c)$ ($c$ nový konstantní symbol), CNF: $S=\{\{\neg P(x,c)\}\}$
            \item Symboly $c,d$ chápejme jako konstantní symboly, ne proměnné (dle konvence), jde tedy už o sentenci. Převod do PNF je snadný:
            $$
            \neg (\exists x)((P(x)\to P(c))\wedge (P(x)\to P(d)))\sim(\forall x)\neg((P(x)\to P(c))\wedge (P(x)\to P(d)))
            $$
            To už je univerzální sentence, skolemizace není potřeba (je sama svojí skolemovou variantou). Odstraníme $(\forall x)$ a převedeme do CNF: $(P(x)\lor P(x))\land (P(x)\lor \neg P(d))\land (\neg P(c)\lor P(x))\land (\neg P(c)\lor \neg P(d))$, po zjednodušení $P(x)\land (\neg P(c)\lor \neg P(d))$, množinová reprezentace: $S=\{\{P(x)\},\{\neg P(c),\neg P(d)\}\}$
            \item Skolemova varianta: $(\forall y)(P(c,f(y))\wedge P(f(y),y) \to R(c,y))$, CNF v množinové reprezentaci: $S=\{\{\neg P(c,f(y)),\neg P(f(y),y), R(c,y)\}\}$.
        \end{enumerate}
    \end{solution}

\end{problem}

      
\begin{problem}
    
    Nechť $T=\{(\exists x)R(x), (\exists y)\neg P(x,y), (\exists y)(\forall z)(\neg R(x)\vee P(y,z))\}$ je teorie jazyka $L=\langle P,R\rangle$ bez rovnosti. Najděte otevřenou teorii $T'$ ekvisplnitelnou s $T$. Převeďte $T'$ do CNF a výslednou formuli $S$ zapište v množinové reprezentaci.

    \begin{solution}
        Axiomy už jsou v PNF, ale pro skolemizaci potřebujeme sentence (generální uzávěry): $T\sim\{(\exists x)R(x), (\forall x)(\exists y)\neg P(x,y), (\forall x)(\exists y)(\forall z)(\neg R(x)\vee P(y,z))\}$. Skolemizujeme (všechny symboly musí být nové):
        $\{R(c), (\forall x)\neg P(x,f(x)), (\forall x)(\forall z)(\neg R(x)\vee P(g(y),z))\}$. Odstraníme univerzální kvanfitikátory:
        $$T'=\{R(c), \neg P(x,f(x)), \neg R(x)\vee P(g(y),z)\}$$
        Ta už je v CNF, množinový zápis: $S=\{\{R(c)\},\{\neg P(x,f(x))\},\{\neg R(x),P(g(y),z)\}\}$
        
        Všimněte si, že $S$ je nesplnitelná: to vidíme `na úrovni výrokové logiky', substitujeme-li do druhé klauzule $\{x/g(c)\}$ a do třetí klauzule $\{x/c,y/c,z/f(g(c))\}$, dostáváme základní instance:
        $$
        S'=\{\{R(c)\},\{\neg P(g(c),f(g(c)))\},\{\neg R(c),P(g(c),f(g(c)))\}\}
        $$
        Díky ekvisplnitelnosti s $S$ je tedy nesplinitelná i původní teorie $T$.        
    \end{solution}

\end{problem}


\begin{problem}

    Nechť $T=\{\varphi_1,\varphi_2\}$ je teorie v jazyce $L=\langle R\rangle$ s~rovností, kde:
    \begin{align*}
    \varphi_1=&\quad (\exists y)R(y,x)\\
    \varphi_2=&\quad (\exists z)(R(z,x)\wedge R(z,y)\wedge (\forall w)(R(w,x) \wedge R(w,y)\to R(w,z)))
    \end{align*}
    \begin{enumerate}[(a)]
        \item Pomocí skolemizace sestrojte otevřeně axiomatizovanou teorii $T'$ (případně v širším jazyce $L'$) ekvisplnitelnou s $T$.
        \item Buď $\mathcal{A}=\langle\mathbb{N},R^A\rangle$, kde $(n,m)\in R^A$ právě když $n$ dělí $m$.  Nalezněte expanzi $\mathcal{A}'$ $L$-struktury $\mathcal{A}$ do jazyka $L'$ takovou, že $\mathcal{A}'\models T'$. (Množina všech přirozených čísel $\mathbb N$ obsahuje nulu, viz ISO 80000-2:2019.)
    \end{enumerate}

    \begin{solution}
        \begin{enumerate}[(a)]
            \item Nejprve skolemizujeme:
            \begin{itemize}
                \item $\varphi_1\sim (\forall x)(\exists y)R(y,x)$, Skolemova varianta: 
                $(\forall x)R(f(x),x)$
                \item $\varphi_2\sim (\forall x)(\forall y)(\exists z)(\forall w)(R(z,x)\wedge R(z,y)\wedge (R(w,x) \wedge R(w,y)\to R(w,z)))$, Skolemova varianta: $(\forall x)(\forall y)(\forall w)(R(g(x,y),x)\wedge R(g(x,y),y)\wedge (R(w,x) \wedge R(w,y)\to R(w,g(x,y))))$
            \end{itemize}
            Nakonec odstraníme kvantifikátorové prefixy, a třetí axiom ve formě konjunkce pro přehlednost rozdělíme:
            $$
            T'=\{R(f(x),x), R(g(x,y),x), R(g(x,y),y), R(w,x) \wedge R(w,y)\to R(w,g(x,y))\}
            $$
            \item Uvědomíme si význam axiomů: první říká, že každé číslo má nějakého dělitele, a druhý vyjadřuje existenci největšího společného dělitele. Potřebujeme odpovídajícím způsobem interpretovat funkční symboly, např.:
            \begin{itemize}
                \item $f^{\mathcal A'}(n)=n$ (pro všechna $n\in\mathbb N$)
                \item $g^{\mathcal A'}(n,m)=\gcd(n,m)$ (pro všechna $n,m\in\mathbb N$),
            \end{itemize}
            Dostáváme tedy strukturu $\mathcal A'=\langle\mathbb N,R^\mathcal A,f^{\mathcal A'},g^{\mathcal A'}\rangle$ v jazyce $L'=\langle R,f,g\rangle$ s rovností, kde $R^\mathcal A$ je relace dělitelnosti jako výše.

            Všimněte si, že tato $\mathcal A'$ není jediná možná. Mohli bychom také zvolit $f^{\mathcal A'}(n)=1$. To je důvod, proč skolemizací dostaneme ekvisplnitelnou, ale typicky ne ekvivalentní teorii.
        \end{enumerate}
    \end{solution}

\end{problem}


\begin{problem} 
    
    Sestrojte Herbrandův model dané teorie, nebo najděte nesplnitelnou konjunkci základních instancí jejích axiomů ($c,d$ jsou konstantní symboly v daném jazyce).    
    \begin{enumerate}[(a)]
        \item $T=\{\neg P(x)\vee Q(f(x),y), \neg Q(x,d), P(c)\}$
        \item $T=\{\neg P(x)\vee Q(f(x),y), Q(x,d), P(c)\}$
        \item $T=\{P(x,f(x)),\neg P(x,g(x))\}$
        \item $T=\{P(x,f(x)),\neg P(x,g(x)), P(g(x),f(y)) \to P(x,y)\}$
    \end{enumerate}

    \begin{solution}
        \begin{enumerate}[(a)]
            \item $T$ je sporná. Nesplnitelná je následující konjunkce základních instancí axiomů:
            $$
            (\neg P(c)\lor Q(f(c),d))\land \neg Q(f(c),d)\land P(c)
            $$
            To lze snadno ověřit rezolucí `na úrovni výrokové logiky', tj. každou atomickou formuli chápeme jako prvovýrok: označíme-li $P(c)$ jako $p$ a $Q(f(c),d)$ jako $q$, máme $(\neg p\lor q)\land \neg q\land p$.

            Jak můžeme tyto základní instance axiomů najít? Herbrandova věta dává návod: sestrojíme (konečný) tablo důkaz sporu z teorie $T_\mathrm{ground}$ sestávající ze všech základních instancí axiomů $T$. Alternativně, použijeme stejný postup jako při hledání modelů: sestrojíme \emph{tablo zamítnutí} některého z axiomů $\varphi_\mathrm{ground}\in T_\mathrm{ground}$, tj. sporné tablo z teorie $T_\mathrm{ground}$ s položkou $\mathrm{T}\varphi_\mathrm{ground}$ v kořeni. Při vhodné volbě axiomů dojdeme ke spornému tablu snadno:
            \begin{center}
                \scalebox{1}{
                \begin{forest}
                    for tree={math content}
                    [\textcolor{blue}{\T P(c)}
                        [\textcolor{blue}{\T \neg Q(f(c),d)}
                            [{\F Q(f(c),d)}
                                [\textcolor{blue}{\T\neg P(c)\lor Q(f(c),d)}
                                    [\T\neg P(c)
                                        [\F P(c), tikz={\node[fit to=tree,label=below:$\otimes$] {};}]
                                    ]
                                    [{\T Q(f(c),d)}, tikz={\node[fit to=tree,label=below:$\otimes$] {};}]
                                ]
                            ]
                        ]
                    ]
                \end{forest}
                }
            \end{center}  
            Axiomy $T_\mathrm{ground}$ (tj. základní instance axiomů $T$), teré jsme v tablo zamítnutí použili (modře), patří do nesplnitelné konjunkce výše.

            V příští sadě příkladů si ale ukážeme lepší postup: nesplnitelnou konjunkci základních instancí axiomů můžeme také získat z \emph{rezolučního zamítnutí} $S$ (vzniklého převodem $T$ do CNF).

            \item Tato teorie není sporná, má tedy Herbrandův model (model, jehož univerzum tvoří všechny konstantní termy jazyka, a kde interpretace funkčních symbolů odpovídají ``vytvoření termu'' aplikací daného symbolu). 
            
            V takto jednoduchém případě můžeme Herbrandův model zkonstruovat sami: 
            $$
            \mathcal H=\langle H;P^\mathcal H,Q^\mathcal H,f^\mathcal H,c^\mathcal H,d^\mathcal H\rangle
            $$
            \begin{itemize}
                \item $\mathcal H=\{\text{``c''},\text{``d''},\text{``f(c)''},\text{``f(d)''},\text{``f(f(c))''},\text{``f(f(d))''},\dots\}$
                \item $c^\mathcal H=\text{``c''}$, $d^\mathcal H=\text{``d''}$
                \item $f^\mathcal H(\text{``t''})=\text{``f(t)''}$
                \item $P^\mathcal H=H$
                \item $Q^\mathcal H=H\times H$
            \end{itemize}
            Relace $P^\mathcal H=H$ a $Q^\mathcal H=H\times H$ jsme zvolili jako úplnou unární resp. binární relaci, aby bylo snadno vidět, že všechny axiomy jsou splněny. Zbytek je dán z definice Herbrandova modelu.

            Vysvětlíme si také postup daný důkazem Herbrandovy věty. Podobně jako výše bychom zkonstruovali dokončené tablo z $T_\mathrm{ground}$ pro položku $\T P(c)$ (například), ale tentokrát tablo nebude sporné (teorie $T_\mathrm{ground}$ není sporná). Herbrandův model získáme z libovolné (dokončené) bezesporné větve, a to stejným postupem jako kanonický model, s tím rozdílem, že do jazyka nepřidáváme pomocné konstantní symboly $c_0,c_1,\dots$ jako jsme to dělali v tablo metodě. Všimněte si, že nejsou potřeba: protože je teorie $T$ \emph{otevřená}, nejsou v ní ani v $T_\mathrm{ground}$ žádné kvantifikátory, tedy nemusíme redukovat žádné položky typu ``svědek''. (Můžete si zkusit kousek tabla zkonstruovat, ale vyjde poměrně složité.)
            
            \item Není sporná, v Herbrandově modelu lze zvolit např. $P^\mathcal H=H\times\{\text{``f(t)''}\mid t\in H\}$, tj. v relaci jsou takové dvojice termů, kde druhý začíná funkčním symbolem $f$.
            
            \item Teorie je sporná. V axiomech nevidíme žádný konstantní symbol, proto si musíme jeden do jazyka přidat, buďme tedy v jazyce $L=\langle P,f,g,c\rangle$ (bez rovnosti). Sporná, nesplnitelná konjunkce základních instancí axiomů je např.:
            $$
            P(g(c),f(g(c)))\land \neg P(c,g(c))\land (P(g(c),f(g(c))) \to P(c,g(c)))
            $$
        
        \end{enumerate}

    \end{solution}

\end{problem}

        
\section*{Další příklady k procvičení}


\begin{problem}

    Teorie těles $T$ jazyka $L=\langle +,-,\cdot,0,1\rangle$ obsahuje jeden axiom $\varphi$, který není otevřený: $x\neq 0\ \to\ (\exists y)(x\cdot y=1)$. Víme, že $T\models 0\cdot y=0$ a $T\models\ (x\ne 0\ \wedge\ x\cdot y=1\ \wedge\ x\cdot z=1)\ \to\ y=z$.
    \begin{enumerate}[(a)]
        \item Najděte Skolemovu variantu $\varphi_S$ formule $\varphi$ s novým funkčním symbolem $f$.
        \item Uvažme teorii $T'$ vzniklou z $T$ nahrazením $\varphi$ za $\varphi_S$. Platí $\varphi$ v $T'$?
        \item Lze každý model $T$ \emph{jednoznačně} rozšířit na model $T'$?
    \end{enumerate}
    Nyní uvažme formuli $\psi=x\cdot y=1\vee  (x=0 \wedge y=0)$.
    \begin{enumerate}[(a)]
        \setcounter{enumi}{3}
        \item Platí v $T$ axiomy existence a jednoznačnosti pro $\psi(x,y)$ a proměnnou $y$?
        \item Sestrojte extenzi $T''$ teorie $T$ o definici symbolu $f$ formulí $\psi$.
        \item Je $T''$ ekvivalentní teorii $T'$?
        \item Najděte $L$-formuli, která je v $T''$-ekvivalentní s formulí:
        $f(x\cdot y)=f(x)\cdot f(y)$
    \end{enumerate}

\end{problem}


\medskip\begin{problem} Víme, že platí následující:
    \begin{itemize}\it
        \item Je-li cihla na (jiné) cihle, potom není na zemi.
        \item Každá cihla je na (jiné) cihle nebo na zemi.
        \item Žádná cihla není na cihle, která by byla na (jiné) cihle.
    \end{itemize}
    Chceme dokázat rezolucí následující tvrzení: {\it ``Je-li cihla na (jiné) cihle, spodní cihla je na zemi.''}. Sestrojte příslušnou CNF formuli $S$, a pokuste se najít i její rezoluční zamítnutí.
\end{problem}

        
\section*{K zamyšlení}

    
\begin{problem}
    
    Skolemova varianta nemusí být ekvivalentní původní formuli, ověřte, že platí:
    \begin{enumerate}[(a)]
        \item $\models (\forall x)P(x,f(x)) \to (\forall x)(\exists y)P(x,y)$
        \item $\not\models (\forall x)(\exists y)P(x,y)\to (\forall x)P(x,f(x))$
    \end{enumerate}

\end{problem}















 


