\documentclass[a4paper]{article}
\usepackage{a4wide}
\usepackage{amssymb}
\usepackage[utf8]{inputenc}
\usepackage[english]{babel}

\begin{document}

\thispagestyle{empty}


\begin{center}
    \large{Sample tutorial test: Propositional logic}    
\end{center}

Time limit: 45 minutes. Total points: 100.

\bigskip

\begin{enumerate}
\item Consider three brothers, each of whom either always tells the truth or always lies.
\begin{enumerate}
\item[$(i)$] The eldest says: \emph{``Both my brothers are liars.''}
\item[$(ii)$] The middle one says: \emph{``The youngest is a liar.''}
\item[$(iii)$] The youngest says: \emph{``The eldest is a liar.''}
\end{enumerate}
Let the propositional variables $p_1$, $p_2$, $p_3$ represent (in order) that \emph{``the eldest / middle / youngest brother is truthful''} and denote $\mathbb{P}=\{p_1,p_2,p_3\}$.
\begin{enumerate}
\item Write formulas (in the form of equivalences) $\varphi_1$, $\varphi_2$, $\varphi_3$ over $\mathbb{P}$ representing the knowledge derived (in order) from $(i)$, $(ii)$, $(iii)$. {\it (15 points)}
\item Write a theory $S$ in set notation obtained by converting $\varphi_1$, $\varphi_2$, $\varphi_3$, $p_3$ or their negations into CNF, which is unsatisfiable if and only if it follows from statements $(i)$, $(ii)$, $(iii)$ that \emph{``the youngest is truthful''}. {\it (15 points)}
\item Prove by resolution that $S$ is unsatisfiable. Represent the resolution refutation with a resolution tree. {\it (20 points)}
\end{enumerate}

\item Let $T=\{(\neg p \wedge q) \to r,\ (q \to r) \leftrightarrow p\}$ be a theory over $\mathbb{P}=\{p,q,r\}$.
\begin{enumerate}
    \item Using the tableaux method, determine all models of theory $T$. {\it (20 points)}
    \item Is $T$ an extension of the theory $S=\{q \to p\}$ over $\{p,q\}$? Is $T$ a conservative extension of $S$? Justify. {\it (15 points)}
    \item Determine how many pairwise inequivalent propositions over $\mathbb{P}$ are there that are independent in both $S$ and $T$. Justify. {\it (15 points)}
\end{enumerate}

\end{enumerate}

\end{document}
