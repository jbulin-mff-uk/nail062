\documentclass{beamer}

%% slide-specific

\usetheme[progressbar=frametitle]{metropolis}
%\usecolortheme{spruce}
%\metroset{block=fill}

% define Metropolis colors    
\definecolor{mAlert}{HTML}{EB811B}
\definecolor{mExample}{HTML}{14B03D}
\definecolor{mBlock}{HTML}{23373b}

% my blocks
\setlength\fboxsep{0pt}%

\newcommand{\myblock}[1]{\colorbox{mBlock!12}{\begin{minipage}{\linewidth}#1\end{minipage}}}
\newcommand{\myblockmath}[1]{\colorbox{mBlock!12}{\begin{minipage}{\linewidth}\vspace{-6pt}#1\end{minipage}}}
\newcommand{\myblockamsmath}[1]{\colorbox{mBlock!12}{\begin{minipage}{\linewidth}\vspace{-6pt}#1\end{minipage}}}
\newcommand{\myblockinline}[1]{\colorbox{mBlock!12}{#1}}
\newcommand{\myexample}[1]{\colorbox{mExample!12}{\begin{minipage}{\linewidth}#1\end{minipage}}}
\newcommand{\myexamplemath}[1]{\colorbox{mExample!12}{\begin{minipage}{\linewidth}\vspace{-6pt}#1\end{minipage}}}
\newcommand{\myexampleamsmath}[1]{\colorbox{mExample!12}{\begin{minipage}{\linewidth}\vspace{-18pt}#1\end{minipage}}}
\newcommand{\myexampleinline}[1]{\colorbox{mExample!12}{#1}}
\newcommand{\myalert}[1]{\colorbox{mAlert!12}{\begin{minipage}{\linewidth}#1\end{minipage}}}
\newcommand{\myalertmath}[1]{\colorbox{mAlert!12}{\begin{minipage}{\linewidth}\vspace{-6pt}#1\end{minipage}}}
\newcommand{\myalertamsmath}[1]{\colorbox{mAlert!12}{\begin{minipage}{\linewidth}\vspace{-18pt}#1\end{minipage}}}
\newcommand{\myalertinline}[1]{\colorbox{mAlert!12}{#1}}

%% other
\newcommand{\mystructure}[1]{\mathcal{#1}}


%% packages
\usepackage{amsmath,amssymb,amsthm}
\usepackage{bookmark}
\usepackage{booktabs}
\usepackage{cancel}
\usepackage[czech]{babel}
\usepackage{enumerate}
\usepackage[T1]{fontenc}
\usepackage{forest}
\usepackage{lmodern}
\usepackage{multicol}
% \usepackage{tcolorbox}
\usepackage{tikz}
    \usetikzlibrary{arrows.meta}
%\usepackage[unicode]{hyperref}
\usepackage[utf8x]{inputenc}
\usepackage{xfrac}


% %% theorems
% \theoremstyle{plain}
%     \newtheorem{theorem}{Věta}[section]
%     \newtheorem*{theorem-unnumbered}{Věta}
%     \newtheorem{proposition}[theorem]{Tvrzení}
%     \newtheorem{corollary}[theorem]{Důsledek}
%     \newtheorem{lemma}[theorem]{Lemma}
%     \newtheorem{observation}[theorem]{Pozorování}
% \theoremstyle{definition}
%     \newtheorem{definition}[theorem]{Definice}
%     \newtheorem*{algorithm}{Algoritmus}
% \theoremstyle{remark}
%     \newtheorem{remark}[theorem]{Poznámka}
%     \newtheorem{example}[theorem]{Příklad}
%     \newtheorem{exercise}{Cvičení}[chapter]
%     \newtheorem*{solution}{Řešení}

%% macros and definitions
\DeclareMathOperator{\Aut}{Aut}
\DeclareMathOperator{\Conseq}{Csq}
\DeclareMathOperator{\DeLO}{DeLO}
\DeclareMathOperator{\dom}{dom}
\DeclareMathOperator{\Fm}{Fm}
\DeclareMathOperator{\M}{M}
%\DeclareMathOperator{\Proof}{Proof}
\DeclareMathOperator{\rng}{rng}
\DeclareMathOperator{\Term}{Term}
\DeclareMathOperator{\Th}{Th}
\DeclareMathOperator{\Thm}{Thm}
\DeclareMathOperator{\Tree}{Tree}
\DeclareMathOperator{\Var}{Var}
\DeclareMathOperator{\VF}{VF}

\newcommand{\A}{\mystructure{A}}
\newcommand{\B}{\mystructure{B}}
\newcommand{\Con}{\mathit{Con}}
\newcommand{\disjointunion}{\mathbin{\dot{\sqcup}}}
\newcommand{\F}{\ensuremath{\mathrm{F}}}
\newcommand{\landsymb}{{\land}}
\newcommand{\lbin}{\mathbin{\square}}
\newcommand{\lbinsymb}{{\lbin}}
\newcommand{\liff}{\mathbin{\leftrightarrow}}
\newcommand{\liffsymb}{{\liff}}
\newcommand{\limplies}{\mathbin{\rightarrow}}
\newcommand{\limpliessymb}{{\limplies}}
\newcommand{\lorsymb}{{\lor}}
\newcommand{\Prf}{\mathit{Prf}}
%\newcommand{\structure}[1]{\mathcal{#1}}
\newcommand{\todo}{[TODO]}
\newcommand{\T}{\ensuremath{\mathrm{T}}}
\newcommand{\union}{\mathbin{\cup}}

\DeclareRobustCommand\proves{\mathrel{|}\joinrel\mkern-.5mu\mathrel{-}}

\title{Pátá přednáška}
\subtitle{NAIL062 Výroková a predikátová logika}
\author{Jakub Bulín (KTIML MFF UK)}
% \institute{KTIML MFF UK}
\date{Zimní semestr 2023}


\begin{document}


\frame{\titlepage}


\begin{frame}{Pátá přednáška}

    \textbf{Program}
        \begin{itemize}
            \item věta o kompaktnosti
            \item hilbertovský kalkulus
            \item rezoluční metoda
            \item korektnost a úplnost rezoluce
            \item LI-rezoluce a Horn-SAT
        \end{itemize}

    \textbf{Materiály}

        \href{https://github.com/jbulin-mff-uk/nail062/raw/main/lecture/lecture-notes/lecture-notes.pdf}{\alert{\textbf{Zápisky z přednášky}}}, Sekce 4.7-4.8 z Kapitoly 4, Kapitola 5

\end{frame}


\section{4.7 Věta o kompaktnosti}


\begin{frame}{Věta o kompaktnosti}

    % \begin{itemize}
    %     \item `kompaktnost' v kalkulu: každá posloupnost má konvergentní podposloupnost 
    %     \item představte si `konvergenci' zvětšujících se konečných částí k nekonečnému celku
    % \end{itemize}

    %Tvrzení o nekonečných objektech/procesech $\leftrightsquigarrow$ tvrzení o (všech) konečných částech. 

    \myblock{
        \textbf{Věta (O kompaktnosti):} Teorie má model, právě když každá její konečná část má model.
    }
    \smallskip

    \textbf{Důkaz:} \alert{$\Rightarrow$ Snadné:} Model $T$ je zjevně modelem každé její části.
    
    \alert{$\Leftarrow$ Nepřímo:} buď $T$ sporná, najdeme spornou konečnou $T'\subseteq T$.

    Z \alert{úplnosti} víme, že $T\proves\bot$, tedy existuje i \alert{konečný} tablo důkaz $\tau$ výroku $\bot$ z $T$. Konstrukce $\tau$ má konečně mnoho kroků, použili jsme tedy jen konečně 
    mnoho axiomů z $T$. Definujme:
    
    \myalertmath{
    $$
    T'=\{\alpha\in T\mid \mathrm{T}\alpha\text{ je položka v tablu $\tau$}\}
    $$
    }   

    Tedy $\tau$ je tablo jen z teorie $T'$, máme tablo důkaz $T'\proves\bot$, dle \alert{korektnosti} je $T'$ sporná.\hfill\qedsymbol

\end{frame}


\begin{frame}{Aplikace kompaktnosti}

    \myblock{
        \textbf{Důsledek:} Spočetně nekonečný graf je bipartitní, právě když je každý jeho konečný podgraf bipartitní.
    }

%     \textbf{Důkaz:} Každý podgraf bipartitního grafu je zjevně také bipartitní. Ukažme opačnou implikaci. Graf je bipartitní, právě když je obarvitelný 2 barvami. Označme barvy $0,1$.

%     Sestrojíme výrokovou teorii $T$ v jazyce $\mathbb P=\{p_v\mid v\in V(G)\}$, kde hodnota výrokové proměnné $p_v$ reprezentuje barvu vrcholu $v$.
%     $$  
%         T=\{p_u\limplies\neg p_v\mid \{u,v\}\in E(G)\}
%     $$
%     Zřejmě platí, že $G$ je bipartitní, právě když $T$ má model. Podle Věty o kompaktnosti stačí ukázat, že každá konečná část $T$ má model. Vezměme tedy konečnou $T'\subseteq T$. Buď $G'$ podgraf $G$ indukovaný na množině vrcholů, o kterých se zmiňuje teorie $T'$, tj. $V(G')=\{v\in V(G)\mid p_v\in\Var(T')\}$. Protože je $T'$ konečná, je $G'$ také konečný, a podle předpokladu je 2-obarvitelný. Libovolné 2-obarvení $V(G')$ ale určuje model teorie $T'$.\hfill\qedsymbol

%     Tvrzení o nekonečných objektech/procesech $\leftrightsquigarrow$ tvrzení o (všech) konečných částech. Například:

    
% Následující jednoduchou aplikaci Věty o kompaktnosti můžete chápat jako šablonu, kterou následuje i mnoho dalších, složitějších aplikací této věty. 

% Základem této techniky je popis požadované vlastnosti nekonečného objektu pomocí (nekonečné) výrokové teorie. Dále si všimněte, jak z konečné části teorie sestrojíme konečný podobjekt mající danou vlastnost (v našem případě konečný podgraf, který je bipartitní).
    

\end{frame}


\end{document}


