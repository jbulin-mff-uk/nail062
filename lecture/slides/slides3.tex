\documentclass{beamer}

%% slide-specific

\usetheme[progressbar=frametitle]{metropolis}
%\usecolortheme{spruce}
%\metroset{block=fill}

% define Metropolis colors    
\definecolor{mAlert}{HTML}{EB811B}
\definecolor{mExample}{HTML}{14B03D}
\definecolor{mBlock}{HTML}{23373b}

% my blocks
\setlength\fboxsep{0pt}%

\newcommand{\myblock}[1]{\colorbox{mBlock!12}{\begin{minipage}{\linewidth}#1\end{minipage}}}
\newcommand{\myblockmath}[1]{\colorbox{mBlock!12}{\begin{minipage}{\linewidth}\vspace{-6pt}#1\end{minipage}}}
\newcommand{\myblockamsmath}[1]{\colorbox{mBlock!12}{\begin{minipage}{\linewidth}\vspace{-6pt}#1\end{minipage}}}
\newcommand{\myblockinline}[1]{\colorbox{mBlock!12}{#1}}
\newcommand{\myexample}[1]{\colorbox{mExample!12}{\begin{minipage}{\linewidth}#1\end{minipage}}}
\newcommand{\myexamplemath}[1]{\colorbox{mExample!12}{\begin{minipage}{\linewidth}\vspace{-6pt}#1\end{minipage}}}
\newcommand{\myexampleamsmath}[1]{\colorbox{mExample!12}{\begin{minipage}{\linewidth}\vspace{-18pt}#1\end{minipage}}}
\newcommand{\myexampleinline}[1]{\colorbox{mExample!12}{#1}}
\newcommand{\myalert}[1]{\colorbox{mAlert!12}{\begin{minipage}{\linewidth}#1\end{minipage}}}
\newcommand{\myalertmath}[1]{\colorbox{mAlert!12}{\begin{minipage}{\linewidth}\vspace{-6pt}#1\end{minipage}}}
\newcommand{\myalertamsmath}[1]{\colorbox{mAlert!12}{\begin{minipage}{\linewidth}\vspace{-18pt}#1\end{minipage}}}
\newcommand{\myalertinline}[1]{\colorbox{mAlert!12}{#1}}

%% other
\newcommand{\mystructure}[1]{\mathcal{#1}}


%% packages
\usepackage{amsmath,amssymb,amsthm}
\usepackage{bookmark}
\usepackage{booktabs}
\usepackage{cancel}
\usepackage[czech]{babel}
\usepackage{enumerate}
\usepackage[T1]{fontenc}
\usepackage{forest}
\usepackage{lmodern}
\usepackage{multicol}
% \usepackage{tcolorbox}
\usepackage{tikz}
    \usetikzlibrary{arrows.meta}
%\usepackage[unicode]{hyperref}
\usepackage[utf8x]{inputenc}
\usepackage{xfrac}


% %% theorems
% \theoremstyle{plain}
%     \newtheorem{theorem}{Věta}[section]
%     \newtheorem*{theorem-unnumbered}{Věta}
%     \newtheorem{proposition}[theorem]{Tvrzení}
%     \newtheorem{corollary}[theorem]{Důsledek}
%     \newtheorem{lemma}[theorem]{Lemma}
%     \newtheorem{observation}[theorem]{Pozorování}
% \theoremstyle{definition}
%     \newtheorem{definition}[theorem]{Definice}
%     \newtheorem*{algorithm}{Algoritmus}
% \theoremstyle{remark}
%     \newtheorem{remark}[theorem]{Poznámka}
%     \newtheorem{example}[theorem]{Příklad}
%     \newtheorem{exercise}{Cvičení}[chapter]
%     \newtheorem*{solution}{Řešení}

%% macros and definitions
\DeclareMathOperator{\Aut}{Aut}
\DeclareMathOperator{\Conseq}{Csq}
\DeclareMathOperator{\DeLO}{DeLO}
\DeclareMathOperator{\dom}{dom}
\DeclareMathOperator{\Fm}{Fm}
\DeclareMathOperator{\M}{M}
%\DeclareMathOperator{\Proof}{Proof}
\DeclareMathOperator{\rng}{rng}
\DeclareMathOperator{\Term}{Term}
\DeclareMathOperator{\Th}{Th}
\DeclareMathOperator{\Thm}{Thm}
\DeclareMathOperator{\Tree}{Tree}
\DeclareMathOperator{\Var}{Var}
\DeclareMathOperator{\VF}{VF}

\newcommand{\A}{\mystructure{A}}
\newcommand{\B}{\mystructure{B}}
\newcommand{\Con}{\mathit{Con}}
\newcommand{\disjointunion}{\mathbin{\dot{\sqcup}}}
\newcommand{\F}{\ensuremath{\mathrm{F}}}
\newcommand{\landsymb}{{\land}}
\newcommand{\lbin}{\mathbin{\square}}
\newcommand{\lbinsymb}{{\lbin}}
\newcommand{\liff}{\mathbin{\leftrightarrow}}
\newcommand{\liffsymb}{{\liff}}
\newcommand{\limplies}{\mathbin{\rightarrow}}
\newcommand{\limpliessymb}{{\limplies}}
\newcommand{\lorsymb}{{\lor}}
\newcommand{\Prf}{\mathit{Prf}}
%\newcommand{\structure}[1]{\mathcal{#1}}
\newcommand{\todo}{[TODO]}
\newcommand{\T}{\ensuremath{\mathrm{T}}}
\newcommand{\union}{\mathbin{\cup}}

\DeclareRobustCommand\proves{\mathrel{|}\joinrel\mkern-.5mu\mathrel{-}}

\title{Třetí přednáška}
\subtitle{NAIL062 Výroková a predikátová logika}
\author{Jakub Bulín (KTIML MFF UK)}
% \institute{KTIML MFF UK}
\date{Zimní semestr 2024}


\begin{document}


\maketitle


\begin{frame}{Třetí přednáška}

    \textbf{Program}
        \begin{itemize}
            \item problém splnitelnosti, SAT solvery
            \item 2-SAT a implikační graf
            \item Horn-SAT a jednotková propagace
            \item algoritmus DPLL
            \item úvod do tablo metody
        \end{itemize}

    \textbf{Materiály}

        \href{https://github.com/jbulin-mff-uk/nail062/raw/main/lecture/lecture-notes/lecture-notes.pdf}{\alert{\textbf{Zápisky z přednášky}}}, Kapitola 3, Sekce 4.1-4.2 z Kapitoly 4

\end{frame}


\section{\sc Kapitola 3: Problém splnitelnosti}


\begin{frame}{Problém splnitelnosti Booleovských formulí}
   
    \textbf{Problém SAT:}
    \begin{itemize}
        \item vstup: výrok $\varphi$ v CNF
        \item otázka: je $\varphi$ splnitelný? 
    \end{itemize}

    \alert{univerzální problém}: každou teorii nad konečným jazykem lze převést do CNF

    \alert{Cook-Levinova věta}: SAT je NP-úplný (důkaz: formalizuj výpočet nedeterministického Turingova stroje ve výrokové logice)

    ale některé \emph{fragmenty} jsou v P, efektivně řešitelné, např. 2-SAT a Horn-SAT (viz Sekce 3.2 a 3.3)

    \alert{praktický problém}: moderní \emph{SAT solvery} (viz Sekce 3.1) se používají v řadě odvětví aplikované informatiky, poradí si s obrovskými instancemi
    
\end{frame}


\section{3.1 SAT solvery}


\begin{frame}{SAT solvery}

    \begin{itemize}
        \item existují od 60. let 20. století, v 21. století dramatický rozvoj dnes až $~10^8$ proměnných, viz \href{http://www.satcompetition.org}{\alert{www.satcompetition.org}}.
        \item nejčastěji založeny na jednoduchém \alert{algoritmu DPLL} (viz Sekce 3.4), umí i najít řešení (model)
        \item různá rozšíření, např. \href{https://en.wikipedia.org/wiki/Conflict-driven_clause_learning}{\alert{Conflict-driven clause learning (CDCL)}}
        \item řada technologií pro efektivnější řešení instancí pocházejících z různých aplikačních domén, heuristiky pro řízení prohledávání (za použití ML, NN) --- desítky tisíc řádků kódu
    \end{itemize}
    
    \textbf{Praktická ukázka: boardomino}

    \smallskip

    \myexample{
    Lze pokrýt šachovnici s chybějícími dvěma protilehlými rohy perfektně pokrýt kostkami domina?
    }

    \smallskip

    těžká instance SATu (proč?), jak zakódovat? 
    
    řešič \href{http://www.labri.fr/perso/lsimon/glucose/}{\alert{Glucose}}, formát vstupu: \href{http://people.sc.fsu.edu/~jburkardt/data/cnf/cnf.html}{\alert{DIMACS CNF}}
    
\end{frame}


\section{3.2 2-SAT a implikační graf}


\begin{frame}{2-SAT vs. 3-SAT}

    \begin{itemize}
        \item \alert{$k$-CNF}: CNF a každá klauzule nejvýše $k$ literálů
        \item \alert{$k$-SAT}: je daný $k$-CNF výrok splnitelný?
        \item $k$-SAT je NP-úplný pro $k\geq 3$ (ke každému výroku lze sestrojit \alert{ekvisplnitelný} 3-CNF výrok)
        \item ale \myalertinline{2-SAT je v P, dokonce řešitelný v lineárním čase}
        \item algoritmus využívá tzv. \alert{implikační graf}:         
        \begin{itemize}
            \item 2-klauzule $p\lor q$ je ekvivalentní $\neg p\limplies q$ a také $\neg q\limplies p$
            \item $p\sim p\lor p$ je ekvivalentní $\neg p\limplies p$
            \item vrcholy jsou literály
            \item hrany dané implikacemi
            \item \textbf{myšlenka}: ohodnotíme-li vrchol 1, všude kam se dostaneme po hranách (\alert{komponenta} silné souvislosti) musí být také 1
        \end{itemize}        
    \end{itemize}    
    
\end{frame}


\begin{frame}{Implikační graf}

    \myalertamsmath{
        \begin{align*}
            V(\mathcal G_\varphi) =& \{p, \neg p \mid p\in\Var(\varphi)\},\\
            E(\mathcal G_\varphi) =& \{(\overline{\ell_1},\ell_2), (\overline{\ell_2},\ell_1)\mid \ell_1\lor \ell_2\text{ je klauzule }\varphi\}\union{}\\ 
            &\{(\overline{\ell},\ell)\mid \ell \text{ je jednotková klauzule }\varphi\}
        \end{align*}
    }    

    \medskip
    \myexamplemath{\vspace{-9pt}\small
    $$
        (\neg p_1 \lor p_2) \land (\neg p_2 \lor \neg p_3) 
        \land  (p_1\lor p_3) \land (p_3\lor \neg p_4)\land  (\neg p_1\lor p_5)\land (p_2\lor p_5)\land p_1\land \neg p_4    
    $$    
    }
    \vspace{-18pt}
    \begin{columns}
        \begin{column}{0.45\textwidth}
        
        \begin{itemize}
            \item najdeme komponenty silné souvislosti
            \item literály v komponentě musí být ohodnoceny stejně (jinak ``$1\limplies 0$'')
            \item pokud má nějaká komponenta opačné literály, je $\varphi$ nesplnitelný
            \item jinak sestrojíme model
        \end{itemize}
        
        \end{column}
        \begin{column}{0.55\textwidth}
            \scalebox{0.85}{
            \begin{tikzpicture}[every node/.style={circle,fill=gray!5,draw,minimum width=1cm,node distance=2cm}
            ]
            \node[fill=red!10] (-5) {$\neg p_5$};
            \node[fill=blue!10,right of=-5] (-1) {$\neg p_1$};
            \node[fill=blue!10,above of=-1] (-2) {$\neg p_2$};
            \node[fill=blue!10,below of=-1] (3) {$p_3$};    
            \node[fill=green!10,left of=3] (4) {$p_4$};
            \node[fill=blue!70,right of=-1] (1) {$p_1$}; 
            \node[fill=blue!70,above of=1] (2) {$p_2$};
            \node[fill=blue!70,below of=1] (-3) {$\neg p_3$};       
            \node[fill=red!70,right of=1] (5) {$p_5$};            
            \node[fill=green!70,right of=-3] (-4) {$\neg p_4$};
            \draw[-{Latex[length=2.5mm]}] (-5) -- (-1);
            \draw[-{Latex[length=2.5mm]}] (-1) -- (3);
            \draw[-{Latex[length=2.5mm]}] (3) edge[bend right] (-2);
            \draw[-{Latex[length=2.5mm]}] (-2) -- (-1);
            \draw[-{Latex[length=2.5mm]}] (4) -- (3);
            \draw[-{Latex[length=2.5mm]}] (-1) -- (1);
            \draw[-{Latex[length=2.5mm]}] (-3) -- (1);
            \draw[-{Latex[length=2.5mm]}] (2) edge[bend left] (-3);
            \draw[-{Latex[length=2.5mm]}] (1) -- (2);
            \draw[-{Latex[length=2.5mm]}] (-5) edge[bend left=70] (2);
            \draw[-{Latex[length=2.5mm]}] (4) edge[bend right] (-4);
            \draw[-{Latex[length=2.5mm]}] (1) -- (5);
            \draw[-{Latex[length=2.5mm]}] (-2) edge[bend left=70] (5);
            \draw[-{Latex[length=2.5mm]}] (-3) -- (-4);
            \end{tikzpicture}
            }
        \end{column}    
    \end{columns}

\end{frame}


\begin{frame}{Konstrukce modelu}

    \myalert{\textbf{Všimněte si:} stačí, aby z žádné komponenty ohodnocené 1 nevedla hrana do komponenty ohodnocené 0}


    provedeme \alert{kontrakci komponent}, výsledný graf $\mathcal G_\varphi^\ast$ je \alert{acyklický}

    \vspace{-9pt}
    \begin{center}
        \scalebox{0.7}{
            \begin{tikzpicture}[every node/.style={circle,fill=gray!5,draw,minimum width=1cm,node distance=2cm}
            ]
            \node[fill=red!10] (C1) {$C_1$};
            \node[fill=blue!10,below right of=C1] (C3) {$C_3$};
            \node[fill=green!10,below left of=C3] (C2) {$C_2$};
            \node[fill=blue!70,right of=C3] (barC3) {$\overline{C_3}$}; 
            \node[fill=red!70,above right of=barC3] (barC1) {$\overline{C_1}$};                    
            \node[fill=green!70,below right of=barC3] (barC2) {$\overline{C_2}$};
            \draw[-{Latex[length=2.5mm]}] (C1) -- (C3);
            \draw[-{Latex[length=2.5mm]}] (C2) -- (C3);
            \draw[-{Latex[length=2.5mm]}] (C3) -- (barC3);
            \draw[-{Latex[length=2.5mm]}] (barC3) -- (barC1);
            \draw[-{Latex[length=2.5mm]}] (barC3) -- (barC2);
            \draw[-{Latex[length=2.5mm]}] (C1) edge[bend left] (barC3);
            \draw[-{Latex[length=2.5mm]}] (C3) edge[bend left] (barC1);
            \draw[-{Latex[length=2.5mm]}] (C2) edge[bend right] (barC2);
            \end{tikzpicture}
        }
    \end{center}
    \vspace{-9pt}

    najdeme nějaké \alert{topologické uspořádání}; v něm najdeme nejlevější dosud neohodnocenou komponentu, ohodnotíme ji 0, opačnou komponentu ohodnotíme 1, a opakujeme

    \vspace{-9pt}
    \begin{center}
    \scalebox{0.7}{
        \begin{tikzpicture}[every node/.style={circle,fill=gray!5,draw,minimum width=1cm,node distance=2cm}
            ]
            \node[fill=red!10,label={below:0}] (C1) {$C_1$};
            \node[fill=green!10,right of=C1,label={below:0}] (C2) {$C_2$};
            \node[fill=blue!10,right of=C2,label={below:0}] (C3) {$C_3$};
            \node[fill=blue!70,right of=C3,label={below:1}] (barC3) {$\overline{C_3}$}; 
            \node[fill=green!70,right of=barC3,label={below:1}] (barC2) {$\overline{C_2}$};
            \node[fill=red!70,right of=barC2,label={below:1}] (barC1) {$\overline{C_1}$};                    
            
            \draw[-{Latex[length=2.5mm]}] (C1) edge[bend left] (C3);
            \draw[-{Latex[length=2.5mm]}] (C2) -- (C3);
            \draw[-{Latex[length=2.5mm]}] (C3) -- (barC3);
            \draw[-{Latex[length=2.5mm]}] (barC3) edge[bend left] (barC1);
            \draw[-{Latex[length=2.5mm]}] (barC3) -- (barC2);
            \draw[-{Latex[length=2.5mm]}] (C1) edge[bend left] (barC3);
            \draw[-{Latex[length=2.5mm]}] (C3) edge[bend left] (barC1);
            \draw[-{Latex[length=2.5mm]}] (C2) edge[bend left] (barC2);
            \end{tikzpicture}
        }
    \end{center}

\end{frame}


\begin{frame}{Shrnutí}
        
    \myblock{\textbf{Tvrzení:} $\varphi$ je splnitelný, právě když žádná silně souvislá komponenta v $\mathcal G_\varphi$ neobsahuje dvojici opačných literálů.}
    
    \textbf{Důkaz:} $\Rightarrow$ literály v komponentě musí být ohodnoceny stejně

    $\Leftarrow$ ohodnocení zkonstruované výše je model $\varphi$:
    \begin{itemize}
        \item \alert{jednotková} klauzule $\ell$ platí \myalertinline{%
            kvůli hraně ${\ell}\limplies\ell$}, komponenta s $\overline{\ell}$ byla ohodnocena dříve, a to 0, takže $v(\ell)=1$
        \item podobně pro \alert{2-klauzuli} $\ell_1\lor\ell_2$, \myalertinline{%
            máme hrany $\overline{\ell_1}\limplies\ell_2$, $\overline{\ell_2}\limplies\ell_1$}   
        
        pokud jsme $\ell_1$ ohodnotili dříve než $\ell_2$, museli jsme jako první narazit na komponentu s $\overline{\ell_1}$ a ohodnotit ji 0, tedy $\ell_1$ platí; v~opačném případě symetricky platí $\ell_2$ \hfill\qedsymbol
    \end{itemize}


    \myblock{\textbf{Důsledek:} 2-SAT je řešitelný v lineárním čase, včetně konstrukce modelu (pokud existuje).}

    \textbf{Důkaz:}
    Komponenty silné souvislosti i topologické uspořádání najdeme v čase $\mathcal O(|V|+|E|)$, stačí je projít jednou \hfill\qedsymbol

\end{frame}


\section{3.3 Horn-SAT a jednotková propagace}


\begin{frame}{Horn-SAT}

    \begin{itemize}
        \item \alert{hornovská klauzule}: \emph{nejvýše jeden *pozitivní* literál}
        $$
        \alert{\neg p_1\lor \neg p_2\lor \dots \lor \neg p_n\lor q}\ \sim\ (p_1\land p_2\land\dots \land p_n)\limplies q
        $$
        základ logického programování (Prolog \myblockinline{\texttt{q:-p1,p2,...,pn.}})
        \item \myalertinline{Horn-SAT}, tj. splnitelnost \alert{hornovského} výroku (konjunkce hornovských klauzulí) \myalertinline{je opět v P, v lineárním čase}
        \item algoritmus využívá tzv. \alert{jednotkovou propagaci}:  
        \begin{itemize}
            \item jednotková klauzule vynucuje hodnotu výrokové proměnné
            \item tím můžeme výrok zjednodušit, např. pro $\neg p$ ($p=0$):\\
            odstraníme klauzule s literálem $\neg p$, už jsou splněné\\
            odstraníme literál $p$ (nemůže být splněný)
            \item žádná jednotková klauzule $\Rightarrow$ každá klauzule má \alert{aspoň jeden negativní literál} $\Rightarrow$ vše nastavíme na 0
        \end{itemize}    
    \end{itemize}    
    
\end{frame}


\begin{frame}{Jednotková propagace}
    
    \myexamplemath{\vspace{-12pt}%
        $$\varphi=(\neg p_1\lor p_2)\land(\neg p_1\lor\neg p_2\lor p_3)\land(\neg p_2\lor\neg p_3)\land(\neg p_5\lor \neg p_4)\land \alert{p_4}$$}

    \begin{itemize}
        \item nastav $v(p_4)=1$, odstraň klauzule obsahující literál $p_4$, z ostatních klauzulí odstraň $\neg p_4$
        $$
        \varphi^{p_4}=(\neg p_1\lor p_2)\land(\neg p_1\lor\neg p_2\lor p_3)\land (\neg p_2\lor\neg p_3)\land\alert{\neg p_5}
        $$
        \item nastav $v(p_5)=0$, proveď jednotkovou propagaci $\neg p_5$
        $$
        (\varphi^{p_4})^{\neg p_5}=(\neg p_1\lor p_2)\land(\neg p_1\lor\neg p_2\lor p_3)\land(\neg p_2\lor\neg p_3)
        $$
        \item už žádná jednotková klauzule, v každé klauzuli alespoň dva literály ale \alert{nejvýše jeden pozitivní, tj. alespoň jeden negativní}: $v(p_1)=v(p_2)=v(p_3)=0$, model $v=(0,0,0,1,0)$
    \end{itemize}

    \myalertmath{
        $$
        \varphi^\ell=\{C\setminus\{\overline{\ell}\}\mid C\in\varphi,\ell\notin C\}\qquad\text{(množinový zápis)}
        $$
    }

    \myblock{
        \textbf{Pozorování:} $\varphi^\ell$ neobsahuje $\ell$ ani $\overline{\ell}$, modely = modely $\varphi$ splňující $\ell$
    }

    \myexampleinline{%
        $\psi=p\land (\neg p\lor q)\land (\neg q\lor r)\land\neg r$} je nesplnitelný, co se stane?

\end{frame}


\begin{frame}{Algoritmus pro Horn-SAT}

    \textbf{vstup}: výrok $\varphi$ v Hornově tvaru,\\ \textbf{výstup}: model $\varphi$ nebo informace, že $\varphi$ není splnitelný
    \begin{enumerate}
        \item Pokud $\varphi$ obsahuje dvojici opačných jednotkových klauzulí $\ell,\overline{\ell}$, není splnitelný.
        \item Pokud $\varphi$ neobsahuje žádnou jednotkovou klauzuli, je splnitelný, ohodnoť všechny zbývající proměnné 0.
        \item Pokud $\varphi$ obsahuje jednotkovou klauzuli $\ell$, ohodnoť literál $\ell$ hodnotou 1, proveď jednotkovou propagaci, nahraď $\varphi$ výrokem~$\varphi^\ell$, a vrať se na začátek.
    \end{enumerate}

    \myblock{
    \textbf{Tvrzení:} Algoritmus je korektní.\\
    \textbf{Důsledek:} Horn-SAT lze řešit v lineárním čase.\vspace{3pt}
    }

    \textbf{Důkaz:} {\small Korektnost plyne z pozorování a z diskuze. V každém kroku stačí projít, výrok zkrátíme (kvadratický horní odhad, ale při vhodné implementaci lineární)}\hfill\qedsymbol

\end{frame}


\section{3.4 DPLL algoritmus pro řešení problému SAT}


\begin{frame}{Algoritmus DPLL (Davis-Putnam-Logemann-Loveland, 1961)}

    \textbf{myšlenka:} \alert{čistý výskyt} $p$  buď jen v pozitivních nebo jen v negativních literálech $\Rightarrow$ lze mu nastavit příslušnou hodnotu!

    \myalert{
    DPLL = jednotková propagace + čistý výskyt + větvení (rekurze)
    }
    \textbf{vstup}: výrok $\varphi$ v CNF,\\ 
    \textbf{výstup}: model $\varphi$ nebo informace, že  $\varphi$ není splnitelný
    \begin{enumerate}                
        \item Dokud $\varphi$ obsahuje jednotkovou klauzuli $\ell$, ohodnoť literál $\ell$ hodnotou 1, proveď \alert{jednotkovou propagaci}, nahraď $\varphi$ výrokem~$\varphi^\ell$.
         \item Dokud existuje literál $\ell$, který má ve $\varphi$ \alert{čistý výskyt}, ohodnoť $\ell$ hodnotou 1, a odstraň klauzule obsahující $\ell$.
        \item Pokud $\varphi$ neobsahuje žádnou klauzuli, je splnitelný.
        \item Pokud $\varphi$ obsahuje prázdnou klauzuli, není splnitelný.
        \item Jinak zvol dosud neohodnocenou výrokovou proměnnou $p$, a \alert{zavolej algoritmus rekurzivně} na $\varphi\land p$ a na $\varphi\land \neg p$.
    \end{enumerate}

\end{frame}


\begin{frame}{Ukázkový běh}

    \myexampleamsmath{\vspace{3pt}\small
        \begin{align*}
            &(\neg p\lor q\lor \neg r)\land(\neg p\lor\neg q\lor\neg s)\land(p\lor \neg r\lor \neg s)\land \\ &(q\lor \neg r\lor s)\land (p\lor s)\land(p\lor\neg s)\land(q\lor s)
        \end{align*}
    }
    
    žádná jednotková klauzule, $\neg r$ má \alert{čistý výskyt}: nastav  $v(r)=0$ a odstraň klauzule obsahující $\neg r$:
    $$
    (\neg p\lor\neg q\lor\neg s)\land (p\lor s)\land(p\lor\neg s)\land(q\lor s)
    $$
    už žádný čistý výskyt, rekurzivně zavolej na:
    \begin{enumerate}
        \item $(\neg p\lor\neg q\lor\neg s)\land (p\lor s)\land(p\lor\neg s)\land(q\lor s)\alert{\land p}$
        \item $(\neg p\lor\neg q\lor\neg s)\land (p\lor s)\land(p\lor\neg s)\land(q\lor s)\alert{\land \neg p}$
    \end{enumerate}
    a pokračuj dále v obou větvích výpočtu
    \begin{gather*}
        \vdots \\
        \M_\varphi=\{(1,a,0,b,c)\mid a,b,c\in\{0,1\}\}
    \end{gather*}
    
\end{frame}


\section{\sc Kapitola 4: Metoda analytického tabla}


\section{4.1 Formální dokazovací systémy}


\begin{frame}{Formální dokazovací systém}

    chceme zjistit, zda výrok platí [\alert{$T\models\varphi$}], a to čistě syntakticky, aniž bychom se zabývali sémantikou: najít \alert{(formální) důkaz} [\alert{$T\proves\varphi$}]

    \alert{důkaz} je konečný syntaktický objekt vycházející z $\varphi$ a axiomů $T$

    dokazování lze dělat \alert{algoritmicky} (pokud máme algoritmický přístup k axiomům $T$, která může být nekonečná), a lze rychle algoritmicky \alert{ověřit}, zda je daný objekt opravdu korektní důkaz

    \begin{itemize}
        \item \alert{korektnost}: ``co dokážu, platí'' \hfill \myalertinline{%
            $T\proves\varphi\ \Rightarrow\ T\models\varphi$}
        \item \alert{úplnost}: ``dokážu vše, co platí'' \hfill \myalertinline{%
            $T\models\varphi\ \Rightarrow\ T\proves\varphi$}
    \end{itemize}
    (korektnost je nutná, úplnost ne: rychlý dokazovací systém může být praktický i když není úplný)

    ukážeme si: \emph{tablo metodu}, \emph{hilbertovský kalkulus}, \emph{rezoluční metodu}
    
    \myblock{nutný předpoklad: \alert{jazyk musí být spočetný} (potom i $T$ je spočetná)}

\end{frame}


\section{4.2 Úvod do tablo metody}


\begin{frame}{Tablo metoda neformálně}

    nejprve případ $T=\emptyset$, tedy dokazujeme, že $\varphi$ platí \emph{v logice}

    \alert{tablo} je strom představující \alert{hledání protipříkladu} (modelu $v\not\models\varphi$), když všechny větve \alert{selžou}, máme důkaz (sporem)

    labely: \alert{položky} $\mathrm{T}\psi,\mathrm{F}\psi$ (určují, zda na dané větvi platí výrok $\psi$)

    kořen \alert{$\mathrm{F}\varphi$}, dále rozvíjíme \alert{redukcí} položek (podle struktury výroků v nich), aby platil \alert{invariant}:

    \myalert{
        Každý model, který se \emph{shoduje} s položkou v kořeni (tj. ve kterém neplatí $\varphi$), se musí \emph{shodovat} i s některou větví tabla (tj. splňovat všechny požadavky vyjádřené položkami na této větvi).
    }

    je-li na větvi \alert{$\mathrm{T}\psi$} a zároveň \alert{$\mathrm{F}\psi$}, potom \alert{selhala} (je \alert{sporná}), pokud všechny větve selhaly, je tablo \alert{sporné}, je to \alert{důkaz} $T\proves\varphi$

    pokud nějaká větev neselhala a je \alert{dokončená} (vše na ní zredukované), lze z ní zkonstruovat model, ve kterém $\varphi$ neplatí

\end{frame}


\begin{frame}{Příklad: tablo důkaz $((p\limplies q)\limplies p)\limplies p$}

    \begin{columns}
    
        \column{0.32\textwidth}

        \centering

        \begin{forest}
        [$\mathrm{F}((p\limplies q)\limplies p)\limplies p$
            [$\mathrm{T}(p\limplies q)\limplies p$
                [$\mathrm{F}p$
                    [$\mathrm{F}p\limplies q$
                        [$\mathrm{T}p$ 
                            [$\mathrm{F}q$, tikz={\node[fit to=tree,label=below:$\otimes$] {};}]
                        ]                
                    ]
                    [$\mathrm{T}p$, tikz={\node[fit to=tree,label=below:$\otimes$] {};}
                    ]
                ]
            ]
        ]
        \end{forest}

        \column{0.68\textwidth}

        \begin{itemize}
            \item \alert{důkaz sporem}: v kořeni příznak F
            \item redukujeme položku tvaru \alert{$\mathrm{F}\varphi_1\limplies\varphi_2$}:
            \item pokud $v\not\models\varphi_1\limplies\varphi_2$, nutně $v\models\varphi_1$ a zároveň $v\not\models\varphi_2$
            \item proto na větev připojíme položky \alert{$\mathrm{T}(p\limplies q)\limplies p$} a \alert{$\mathrm{F}p$}, invariant platí
            \item redukce položky \alert{$\mathrm{T}(p\limplies q)\limplies p$}: model se shoduje s \alert{$\mathrm{F}(p\limplies q)$} nebo s \alert{$\mathrm{T}p$}, \alert{rozvětvi!}
            \item redukce \alert{$\mathrm{F}(p\limplies q)$}: připoj  \alert{$\mathrm{T}p$} a \alert{$\mathrm{F}q$}
            \item všechny větve sporné, protipříklad neexistuje, tedy máme tablo důkaz, píšeme: \alert{$\proves ((p\limplies q)\limplies p)\limplies p$}
        \end{itemize}

    \end{columns}

\end{frame}


\begin{frame}{Příklad: tablo pro $\mathrm{F}(\neg q\lor p)\limplies p$}

    \begin{columns}
    
        \column{0.32\textwidth}

        \centering

        \begin{forest}
            [$\mathrm{F}(\neg q\lor p)\limplies p$
                [$\mathrm{T}\neg q\lor p$
                    [$\mathrm{F}p$
                        [$\mathrm{T}\neg q$
                            [$\mathrm{F}q$, tikz={\node[fit to=tree,label=below:$\checkmark$] {};}]
                        ]
                        [$\mathrm{T}p$, tikz={\node[fit to=tree,label=below:$\otimes$] {};}
                        ]
                    ]
                ]
            ]
        \end{forest}

        \column{0.68\textwidth}

        \begin{itemize}
            \item tablo je dokončené, ale není sporné
            \item tedy nejde o důkaz
            \item levá větev dává protipříklad: model \alert{$v=(0,0)$} ve kterém výrok neplatí
            \item invariant říká, že existuje-li protipříklad, shoduje se s některou větví
            \item tato větev nemůže být sporná
            \item tak se dokáže \alert{korektnost} tablo metody
        \end{itemize}

    \end{columns}

\end{frame}


\begin{frame}{Poznámky}

    \begin{itemize}
        \item Jak redukujeme položky? \begin{itemize}
            \item Připojíme příslušné \alert{atomické tablo} (viz následující slide) na konec všech bezesporných větví procházejících vrcholem.
        \end{itemize}
        \item Co když dokazujeme v nějaké teorii $T$? \begin{itemize}
            \item Připojíme položky $\mathrm{T}\alpha$ pro (všechny) axiomy $\alpha\in T$.
        \end{itemize}
        \item Co když je $T$ nekonečná? \begin{itemize}
            \item Tablo může být nekonečné. 
            \item Ale vyjde-li sporné, lze sestrojit jiné, které je konečné a také sporné. (``Existuje-li důkaz, existuje konečný důkaz.'')
        \end{itemize}
    \end{itemize}

\end{frame}


\begin{frame}{Atomická tabla}

        \begin{tabular}{@{}c||c|c|c|c|c@{}}
         & $\neg$ & $\land$ & $\lor$ & $\limplies$ & $\liff$  \\ \midrule \midrule
        True
        &  
        \begin{forest}
        [$\mathrm{T}\neg\varphi$ [$\mathrm{F}\varphi$]]
        \end{forest}
        &  
        \begin{forest}
        [$\mathrm{T}\varphi\land\psi$ [$\mathrm{T}\varphi$ [$\mathrm{T}\psi$]]]
        \end{forest}
        & 
        \begin{forest}
        [$\mathrm{T}\varphi\lor\psi$ [$\mathrm{T}\varphi$] [$\mathrm{T}\psi$]]
        \end{forest}
        &
        \begin{forest}
        [$\mathrm{T}\varphi\limplies\psi$ [$\mathrm{F}\varphi$] [$\mathrm{T}\psi$]]
        \end{forest}
        &  
        \begin{forest}
        [$\mathrm{T}\varphi\liff\psi$ [$\mathrm{T}\varphi$ [$\mathrm{T}\psi$]] [$\mathrm{F}\varphi$ [$\mathrm{F}\psi$]]]
        \end{forest}
        \\ \midrule
        False 
        & 
        \begin{forest}
        [$\mathrm{F}\neg\varphi$ [$\mathrm{T}\varphi$]]
        \end{forest}
        &
        \begin{forest}
        [$\mathrm{F}\varphi\land\psi$ [$\mathrm{F}\varphi$] [$\mathrm{F}\psi$]]
        \end{forest}
        &
        \begin{forest}
        [$\mathrm{F}\varphi\lor\psi$ [$\mathrm{F}\varphi$ [$\mathrm{F}\psi$]]]
        \end{forest}
        &
        \begin{forest}
        [$\mathrm{F}\varphi\limplies\psi$ [$\mathrm{T}\varphi$ [$\mathrm{F}\psi$]]]
        \end{forest}
        &
        \begin{forest}
        [$\mathrm{F}\varphi\liff\psi$ [$\mathrm{T}\varphi$ [$\mathrm{F}\psi$]] [$\mathrm{F}\varphi$ [$\mathrm{T}\psi$]]]
        \end{forest}
        \end{tabular}
    
\end{frame}


\begin{frame}{Konstrukce tabel z příkladů}
    
    \begin{center}
    \scalebox{0.7}{

    \begin{columns}
        
        \column{0.5\textwidth}

        \centering
            \begin{forest}
            [$\mathrm{F}((p\limplies q)\limplies p)\limplies p$
                [\textcolor{blue}{$\mathrm{F}((p\limplies q)\limplies p)\limplies p$}, tikz={\node[fit=()(!1)(!ll),rectangle,draw=blue!20,minimum size=12pt] {};}
                    [$\mathrm{T}(p\limplies q)\limplies p$
                        [$\mathrm{F}p$
                            [\textcolor{blue}{$\mathrm{T}(p\limplies q)\limplies p$}, tikz={\node[fit=()(!1)(!l),rectangle,draw=blue!20,minimum size=12pt] {};}
                                [$\mathrm{F}p\limplies q$
                                    [\textcolor{blue}{$\mathrm{F}p\limplies q$}, tikz={\node[fit=()(!1)(!ll),rectangle,draw=blue!20,minimum size=12pt] {};}
                                        [$\mathrm{T}p$ 
                                            [$\mathrm{F}q$, tikz={\node[fit to=tree,label=below:$\otimes$] {};}]
                                        ]
                                    ]
                                ]
                                [$\mathrm{T}p$, tikz={\node[fit to=tree,label=below:$\otimes$] {};}
                                ]
                            ]
                        ]
                    ]
                ]
            ]
            \end{forest}

        \column{0.5\textwidth}

        \centering

            \begin{forest}
            [$\mathrm{F}(\neg q\lor p)\limplies p$
                [\textcolor{blue}{$\mathrm{F}(\neg q\lor p)\limplies p$}, tikz={\node[fit=()(!1)(!ll),rectangle,draw=blue!20,minimum size=12pt] {};}
                    [$\mathrm{T}\neg q\lor p$
                        [$\mathrm{F}p$
                            [\textcolor{blue}{$\mathrm{T}(\neg q\lor p)$}, tikz={\node[fit=()(!1)(!l),rectangle,draw=blue!20,minimum size=12pt] {};}
                                [$\mathrm{T}\neg q$
                                    [\textcolor{blue}{$\mathrm{T}\neg q$}, tikz={\node[fit=()(!1)(!l),rectangle,draw=blue!20,minimum size=12pt] {};}
                                        [$\mathrm{F}q$, tikz={\node[fit to=tree,label=below:$\checkmark$] {};}]
                                    ]
                                ]
                                [$\mathrm{T}p$, tikz={\node[fit to=tree,label=below:$\otimes$] {};}
                                ]
                            ]
                        ]
                    ]
                ]
            ]
            \end{forest}

    \end{columns}

    }


    \alert{konvence:} kořeny atomických tabel (\textcolor{blue}{modře}) nezakreslujeme   

    \end{center}    

\end{frame}


\begin{frame}{O stromech}
    
    \begin{itemize}
        \item \alert{strom} je $T\neq\emptyset$ s částečným uspořádáním $<_T$, které má nejmenší prvek (\alert{kořen}) a množina předků libovolného vrcholu je \alert{dobře uspořádaná} (každá její neprázdná podmnožina má nejmenší prvek, to zakáže nekonečné klesající řetězce předků)
        \item \alert{větev} je maximální lineárně uspořádaná podmnožina $T$.
        \item \alert{uspořádaný strom} má navíc lineární uspořádání $<_L$ množiny synů každého vrcholu (říkáme mu \alert{pravolevé}, $<_T$ je \alert{stromové})
        \item \alert{označkovaný strom} má navíc funkci $\mathrm{label}\colon T\to\mathrm{Labels}$        
    \end{itemize}

    \myblock{\textbf{Königovo lemma:} Nekonečný, konečně větvící strom má nekonečnou větev.}

\end{frame}


\end{document}

