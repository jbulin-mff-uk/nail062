\documentclass[a4paper]{article}

\usepackage{a4wide}
\usepackage{amsmath}
\usepackage{amssymb}
\usepackage{amsthm}
\usepackage{enumerate}
\usepackage{tikz}
\usepackage[utf8]{inputenc}


\theoremstyle{definition}
\newtheorem{problem}{Příklad}
\newtheorem*{ukol}{Domácí úkol}


\begin{document}

\section*{NAIL062 V\&P Logika: 11. cvičení}


\textbf{Témata:}
Unifikace. Rezoluce v predikátové logice.


\medskip\begin{problem}
 
\end{problem}


\medskip\begin{problem} Víme, že platí následující:
    \begin{itemize}
        \item Je-li cihla na (jiné) cihle, potom není na zemi.
        \item Každá cihla je na (jiné) cihle nebo na zemi.
        \item Žádná cihla není na cihle, která by byla na (jiné) cihle.
    \end{itemize}
    Vyjádřete tato fakta ve vhodném jazyce logiky prvního řádu a dokažte rezolucí následující tvrzení: ``Je-li cihla na (jiné) cihle, spodní cihla je na zemi.''
\end{problem}
        
    
\medskip\begin{problem} Víme, že platí následující:
    \begin{enumerate}[(a)]
        \item Každý holič holí všechny, kdo neholí sami sebe
        \item Žádný holič neholí nikoho, kdo holí sám sebe.
    \end{enumerate}
    Vyjádřete tato fakta ve vhodném jazyce logiky prvního řádu a dokažte rezolucí, že neexistují žádní holiči.
\end{problem}


\medskip\begin{problem}
Jsou dána následující tvrzení o~proběhlém genetickém experimentu:
\begin{enumerate}[(i)]
    \item Každá ovce byla buď porozena jinou ovcí, nebo byla naklonována (avšak nikoli oboje zároveň).
    \item Žádná naklonovaná ovce neporodila.
\end{enumerate}
Chceme ukázat rezolucí, že pak:
\begin{enumerate}[(iii)]
    \item Pokud ovce porodila, byla sama porozena.
\end{enumerate}
Konkrétně:
\begin{enumerate}[(a)]
    \item Uvedená tvrzení vyjádřete \underline{sentencemi} $\varphi_1$, $\varphi_2$, $\varphi_3$ v jazyce $L=\langle P,K\rangle$ bez rovnosti, kde $P$ je binární relační symbol, $K$ je unární relační symbol a $P(x,y)$, $K(x)$ značí, že \emph{``ovce $x$ porodila ovci $y$''} a \emph{``ovce $x$ byla naklonována''}. {\it (15b)}    
    \item S využitím Skolemizace těchto formulí nebo jejich negací sestrojte množinu klauzulí $S$ (může být ve větším jazyce), která je nesplnitelná, právě když  $\{\varphi_1, \varphi_2\} \models \varphi_3$. Zapište ji v množinové reprezentaci. {\it (15b)}
    \item Najděte rezoluční zamítnutí $S$, znázorněte je rezolučním stromem. U každého kroku uveďte použitou unifikaci. {\it (30b)}
\end{enumerate}
\end{problem}
        
    
\medskip\begin{problem}
    Ukažte, že daná množina klauzulí je zamítnutelná (rezolucí). Popište zamítnutí pomocí rezolučního stromu. V každém kroku rezoluce napište použitou unifikaci a podtrhněte rezolvované literály.
    \begin{align*}
        S=\{
            &\{P(a,x,f(y)),P(a,z,f(h(b))),\neg Q(y,z)\},\\
            &\{\neg Q(h(b),w),H(w,a)\},\\
            &\{\neg P(a,w,f(h(b))),H(x,a)\},\\
            &\{P(a,u,f(h(u))),H(u,a),Q(h(b),b)\},\\
            &\{\neg H(v,a)\}
        \}
    \end{align*}
\end{problem}


\medskip\begin{ukol}[3 body]


Kromě tohoto úkolu se připravte na zápočtový test. Vyřešte vzorový test (na webu).
\end{ukol}

\end{document}