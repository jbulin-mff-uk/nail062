\section*{NAIL062 V\&P Logika: 6. sada příkladů -- Základy predikátové logiky}
% po 5. přednášce


\subsection*{Výukové cíle:} Po absolvování cvičení student

\begin{itemize}\setlength{\itemsep}{0pt}
    \item rozumí pojmu struktura, signatura, umí je formálně definovat a uvést příklady
    \item rozumí pojmům syntaxe predikátové logiky (jazyk, term, atomická formule, formule, teorie, volná proměnná, otevřená formule, sentence, instance, varianta) umí je formálně definovat a uvést příklady
    \item rozumí pojmům sémantiky predikátové logiky (hodnota termu, pravdivostní hodnota, platnost [při ohodnocení], model, pravdivost/lživost v modelu/v teorii, nezávislost [v teorii], důsledek teorie) umí je formálně definovat a uvést příklady
    \item rozumí pojmu kompletní teorie a jeho souvislosti s elementární ekvivalencí struktur, umí obojí definovat, aplikovat na příkladě
    \item zná základní příklady teorií (teorie grafů, uspořádání, algebraické teorie)
    \item umí popsat modely dané teorie    
\end{itemize}
    

\section*{Příklady na cvičení}


\begin{problem}
    
    Jsou následující formule variantami formule $(\forall x)(x<y \vee (\exists z)(z=y \wedge z\ne x))$?
    \begin{enumerate}[(a)]
        \item $(\forall z)(z<y \vee (\exists z)(z=y \wedge z\ne z))$
        \item $(\forall y)(y<y \vee (\exists z)(z=y \wedge z\ne y))$
        \item $(\forall u)(u<y \vee (\exists z)(z=y \wedge z\ne u))$
    \end{enumerate}

    \begin{solution}
        
    \end{solution}

\end{problem}


\begin{problem}

    Mějme strukturu $\mathcal{A}=(\{a,b,c,d\},\vartriangleright^{A})$ v jazyce s jediným binárním relačním symbolem $\vartriangleright$, kde $\vartriangleright^{A}=\{(a,c), (b,c), (c,c), (c,d)\}$. 
    \begin{enumerate}[I.]
        \item Které z následujících formulí jsou pravdivé v $\mathcal A$? 
        \item Pro každou z nich najděte strukturu $\mathcal{B}$ (existuje-li) takovou, že $\mathcal{B}\models \varphi$ právě když $\mathcal{A}\not\models \varphi$.
    \end{enumerate}    
    \begin{enumerate}[(a)]
       \item $x \vartriangleright y$
       \item $(\exists x)(\forall y)(y \vartriangleright x)$
       \item $(\exists x)(\forall y)((y \vartriangleright x) \to (x \vartriangleright x))$
       \item $(\forall x)(\forall y)(\exists z)((x \vartriangleright z)\wedge(z \vartriangleright y))$
       \item $(\forall x)(\exists y)((x \vartriangleright z)\vee(z \vartriangleright y))$
    \end{enumerate}

    \begin{solution}
        
    \end{solution}

\end{problem}


\begin{problem}

    Dokažte (sémanticky) nebo najděte protipříklad: Pro každou strukturu $\mathcal{A}$, formuli $\varphi$, a sentenci $\psi$,
    \begin{enumerate}[(a)]
    \item $\mathcal{A}\models (\psi \to (\exists x)\varphi) \Leftrightarrow \mathcal{A}\models (\exists x)(\psi \to \varphi)$
    \item $\mathcal{A}\models (\psi \to (\forall x)\varphi) \Leftrightarrow \mathcal{A}\models (\forall x)(\psi \to \varphi)$
    \item $\mathcal{A}\models ((\exists x)\varphi \to \psi) \Leftrightarrow \mathcal{A}\models (\forall x)(\varphi \to \psi)$
    \item $\mathcal{A}\models ((\forall x)\varphi \to \psi ) \Leftrightarrow \mathcal{A}\models (\exists x)(\varphi \to \psi)$
    \end{enumerate}
    Platí to i pro každou formuli $\psi$ s volnou proměnnou $x$? A pro každou formuli $\psi$ ve které $x$ není volná?

    \begin{solution}
                    
    \end{solution}

\end{problem}


\begin{problem}

    Rozhodněte, zda je $T$ (v jazyce $L = \langle U, f \rangle$ s rovností) kompletní. Existují-li, napište dva elementárně neekvivalentní modely, a dvě neekviv. kompletní jednoduché extenze:
    \begin{enumerate}[(a)]
        \item  $T = \{U(f(x)),\ \neg x=y,\ x =y\vee y=z\}$
        \item $T = \{U(f(x)),\ \neg (\forall x)(\forall y)x=y,\ x =y\vee y=z\}$
        \item  $T = \{U(f(x)),\ \neg x=f(x),\ \neg (\forall x)(\forall y)x=y,\ x =y\vee y=z\}$
        \item  $T = \{U(f(x)),\ \neg (\forall x) x=f(x),\ \neg (\forall x)(\forall y)x=y,\ x =y\vee y=z\}$
    \end{enumerate}


    \begin{solution}
                    
    \end{solution}

\end{problem}
        
        
\section*{Další příklady k procvičení}


\begin{problem}

    Určete volné a vázané výskyty proměnných v následujících formulích. Poté je převeďte na varianty, ve kterých nebudou proměnné s volným i vázaným výskytem zároveň.
    \begin{enumerate}[(a)]
        \item $(\exists x)(\forall y)P(y,z) \vee (y=0)$
        \item $(\exists x)(P(x) \wedge (\forall x)Q(x)) \vee (x=0)$
        \item $(\exists x)(x>y) \wedge (\exists y)(y>x)$
    \end{enumerate}

\end{problem}


\begin{problem}
    
    Označme $\varphi$ formuli $(\forall x)((x=z) \vee (\exists y)(f(x)=y) \vee (\forall z)(y=f(z)))$. Které z následujících termů jsou substituovatelné do $\varphi$?
    \begin{enumerate}[(a)]
        \item term $z$ za proměnnou $x$, term $y$ za proměnnou $x$,
        \item term $z$ za proměnnou $y$, term $g(f(y),w)$ za proměnnou $y$,
        \item term $x$ za proměnnou $z$, term $y$ za proměnnou $z$,
    \end{enumerate}

\end{problem}


\begin{problem}

    Jsou následující sentence pravdivé / lživé / nezávislé (v logice)?

    \begin{enumerate}[(a)]
        \item $(\exists x)(\forall y)(P(x) \vee \neg P(y))$
        \item $(\forall x)(P(x)\to Q(f(x))) \wedge (\forall x)P(x) \wedge (\exists x)\neg Q(x)$
        \item $(\forall x)(P(x) \vee Q(x)) \to ((\forall x)P(x) \vee (\forall x)Q(x))$
        \item $(\forall x)(P(x)\to Q(x)) \to ((\exists x)P(x)\to(\exists x)Q(x))$
        \item $(\exists x)(\forall y)P(x,y) \to (\forall y)(\exists x)P(x,y)$
    \end{enumerate}

\end{problem}


\begin{problem}
    
    Rozhodněte, zda následující platí pro každou formuli $\varphi$. Dokažte (sémanticky, z definic) nebo najděte protipříklad.
    \begin{enumerate}[(a)]
       \item $\varphi \models (\forall x)\varphi$
       \item $\models \varphi \to (\forall x)\varphi$
       \item $\varphi \models (\exists x)\varphi$
       \item $\models \varphi \to (\exists x)\varphi$
    \end{enumerate}

    \begin{solution}
                    
    \end{solution}

\end{problem}


\section*{K zamyšlení}


\begin{problem}

    Buď $L=\langle +, -, 0\rangle$ jazyk teorie grup (s rovností). Teorie grup $T$ sestává z těchto axiomů:
    \begin{align*}
    x+(y+z)&=(x+y)+z\\
    0+x&=x=x+0\\
    x+(-x)&=0=(-x)+x
    \end{align*}
    Rozhodněte, zda jsou následující formule pravdivé / lživé / nezávislé v $T$. Zdůvodněte.
    \begin{enumerate}[(a)]
        \item $x+y=y+x$
        \item $x+y=x\ \rightarrow\ y=0$
        \item $x+y=0\ \rightarrow\ y=-x$
        \item $-(x+y)=(-y)+(-x)$
    \end{enumerate}

\end{problem}



