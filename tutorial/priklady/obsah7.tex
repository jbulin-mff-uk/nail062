\section*{NAIL062 V\&P Logika: 7. sada příkladů -- Vlastnosti struktur a teorií}
% po 5. přednášce


\subsection*{Cíle výuky:} Po absolvování cvičení student

    \begin{itemize}\setlength{\itemsep}{0pt}
        \item rozumí pojmu podstruktura, generovaná podstruktura, expanze, redukt umí je najít
        \item rozumí pojmu expanze a redukt struktury, umí je formálně definovat, uvést příklady
        \item rozumí pojmům [jednoduchá, konzervativní] extenze, umí zformulovat definice, i příslušné sémantické kritérium (jak pro expanze, tak i pro redukty), aplikovat na příkladě
        \item rozumí pojmu extenze o definice, umí ho formálně definovat, uvést příklady
        \item umí rozhodnout, zda je daná teorie extenzí o definice, sestrojit extenzi o danou definici
        \item rozumí pojmu definovatelnosti ve struktuře, umí najít definovatelné podmnožiny/relace
    \end{itemize}
    

\section*{Příklady na cvičení}


\begin{problem}

    Uvažme $\underline{\mathbb Z_4}=\langle\{0,1,2,3\};+,-,0 \rangle$ kde $+$ je binární sčítání modulo $4$ a $-$ je unární funkce, která vrací \emph{inverzní} prvek $+$ vzhledem k \emph{neutrálnímu} prvku $0$.
    \begin{enumerate}[(a)]      
        \item Je $\underline{\mathbb Z_4}$ model teorie grup (tj. je to \emph{grupa})?
        \item Určete všechny podstruktury $\underline{\mathbb Z_4}\langle a\rangle$ generované nějakým $a\in \mathbb Z_4$.
        \item Obsahuje $\underline{\mathbb Z_4}$ ještě nějaké další podstruktury?
        \item Je každá podstruktura $\underline{\mathbb Z_4}$ modelem teorie grup?
        \item Je každá podstruktura $\underline{\mathbb Z_4}$ elementárně ekvivalentní $\underline{\mathbb Z_4}$?
        %\item Je každá podstruktura \emph{komutativní} grupy (tj. grupy, která splňuje $x+y=y+x$) také komutativní grupa?
    \end{enumerate}

    \begin{solution}

        \begin{enumerate}[(a)]
            \item Ano, lze ověřit, že $\underline{\mathbb Z_4}$ splňuje všechny axiomy teorie grup ($+$ je asociativní, $0$ je neutrální vůči $+$, $-x$ je inverzní prvek k $x$ vůči $+$ a $0$).
            \item $\underline{\mathbb Z_4}\langle 0\rangle=\underline{\mathbb Z_4}\restriction\{0\}$ (\emph{triviální grupa}), $\underline{\mathbb Z_4}\langle 1\rangle=\underline{\mathbb Z_4}\langle 3\rangle=\underline{\mathbb Z_4}$, $\underline{\mathbb Z_4}\langle 2\rangle=\underline{\mathbb Z_4}\restriction\{0,2\}$ (dvouprvková grupa \emph{izomorfní} grupě $\underline{\mathbb Z_4}$).
            \item Ne, jakmile máme prvek $1$ nebo $3$, generovaná podstruktura už je celá $\underline{\mathbb Z_4}$.
            \item Ano, teorie grup je \emph{otevřená}, proto podstruktury modelů (grup) jsou také modely (\emph{podgrupy}).
            \item Ne, jazyk teorie grup je s rovností, konečná velikost modelu lze popsat sentencí, tedy různě velké konečné modely nemohou být elementárně ekvivalentní. Velikost ale nepotřebujeme, stačí nám ``grupové vlasnosti'', např. sentence $(\forall x)x=0$ odliší triviální grupu $\underline{\mathbb Z_4}\restriction\{0\}$ od dvouprvkové grupy $\underline{\mathbb Z_4}\restriction\{0,2\}$ i od $\underline{\mathbb Z_4}$, a např. $(\forall x)x+x=0$ platí v $\underline{\mathbb Z_4}\restriction\{0,2\}$ ale ne v $\underline{\mathbb Z_4}$.
        \end{enumerate}
                    
    \end{solution}

\end{problem}


\begin{problem}

    Buď $\underline{\mathbb{Q}}=\langle\mathbb{Q};+,-,\cdot,0,1 \rangle$ těleso racionálních čísel se standardními operacemi.
    \begin{enumerate}[(a)]                
        \item Existuje redukt $\underline{\mathbb{Q}}$, který je modelem teorie grup?
        \item Lze redukt $\langle\mathbb{Q},\cdot,1\rangle$ rozšířit na model teorie grup?
        \item Obsahuje $\underline{\mathbb{Q}}$ podstrukturu, která není elementárně ekvivalentní $\underline{\mathbb{Q}}$?
        \item Označme $\mathrm{Th}(\underline{\mathbb{Q}})$ množinu všech sentencí pravdivých v $\underline{\mathbb{Q}}$. Je $\mathrm{Th}(\underline{\mathbb{Q}})$ kompletní teorie?
    \end{enumerate}

    \begin{solution}

        \begin{enumerate}[(a)]
            \item Ano, $\underline{\mathbb{Q}}=\langle\mathbb{Q};+,-,0\rangle$.
            \item Ne, prvek $1$ (interpretace symbolu $0$ z jazyka teorie grup) není neutrální prvek vzhledem k operaci $\cdot$ (interpretaci symbolu $+$), protože $1\cdot 0=0\neq 1$.
            \item Ano, např. $\underline{\mathbb{Q}}\restriction\mathbb Z =\langle\mathbb{Z};+,-,\cdot,0,1 \rangle$ (\emph{okruh celých čísel}), neplatí v něm existence inverzních prvků vůči násobení pro všechny nenulové prvky, tj. sentence $(\forall x)(\neg x=0\to (\exists y)x\cdot y=1)$ (např. číslo 2 nemá v celých číslech inverz, ale v racionálních ano, $\frac{1}{2}$). (Z toho plyne, že teorie těles nemůže být otevřeně axiomatizova[tel]ná, jinak by podstruktura tělesa musela být také tělesem.)
            \item Ano, tzv. \emph{teorie struktury} je vždy kompletní: Pro každou sentenci $\psi$ platí, že $\mathrm{Th}(\underline{\mathbb{Q}})\models\psi\ \Leftrightarrow\ \underline{\mathbb{Q}}\models\psi$, pokud to neplatí, máme $\underline{\mathbb{Q}}\models\neg\psi$ (je to \emph{sentence}) tedy $\mathrm{Th}(\underline{\mathbb{Q}})\models\neg\psi$.
        \end{enumerate}
                    
    \end{solution}

\end{problem}



\begin{problem}

    Mějme teorii $T=\{x=c_1 \vee x=c_2 \vee x=c_3\}$ v jazyce $L=\langle c_1,c_2,c_3\rangle$ s rovností.
    \begin{enumerate}[(a)]     
        \item Je $T$ kompletní?
        \item Kolik má teorie $T$ jednoduchých extenzí, až na ekvivalenci? Kolik je kompletních? Napište všechny kompletní a alespoň tři nekompletní.
        \item Je teorie $T'=T\cup\{x=c_1 \vee x=c_4\}$ v jazyce $L'=\langle c_1,c_2,c_3,c_4\rangle$ extenzí $T$? Je $T'$ jednoduchá extenze $T$? Je $T'$ konzervativní extenze $T$?
    \end{enumerate}

    \begin{solution}
        
        Teorie říká, že každý prvek je jednou ze tří konstant. Ty ale nemusí být různé. Nejprve najděme všechny modely až na izomorfismus, je jich pět (nakreslete si je):
        \begin{itemize}
            \item $\mathcal A_1=\langle\{0\};0,0,0\rangle$ (jednoprvkový model, $c_1^{\mathcal A_1}=c_2^{\mathcal A_1}=c_3^{\mathcal A_1}=0$)
            \item $\mathcal A_2=\langle\{0,1\};0,0,1\rangle$ (dvouprvkový model, $c_1^{\mathcal A_2}=c_2^{\mathcal A_2}\neq c_3^{\mathcal A_2}$)
            \item $\mathcal A_3=\langle\{0,1\};0,1,1\rangle$ (dvouprvkový model, $c_1^{\mathcal A_3}\neq c_2^{\mathcal A_3}=c_3^{\mathcal A_3}$) 
            \item $\mathcal A_4=\langle\{0,1\};0,1,0\rangle$ (dvouprvkový model, $c_1^{\mathcal A_4}=c_3^{\mathcal A_4}\neq c_2^{\mathcal A_4}$)
            \item $\mathcal A_5=\langle\{0,1,2\};0,1,2\rangle$ (trojprvkový model, konstanty jsou různé)
        \end{itemize}
        \begin{enumerate}[(a)]
            \item Není kompletní, např. sentence $c_1=c_2$ je v $T$ nezávislá: platí v $\mathcal A_1$,  neplatí v $\mathcal A_3$. (Neboli, dle sémantického kritéria, modely $\mathcal A_1$ a $\mathcal A_3$ nejsou elementárně ekvivalentní).
            \item Jednoduché extenze odpovídají podmnožinám $\{\mathcal A_1,\mathcal A_2,\mathcal A_3,\mathcal A_4,\mathcal A_5\}$, je jich 32, kompletní odpovídají jednoprvkovým podmnožinám, je jich 5.
            
            Jednoduché extenze, které nejsou kompletní:
            \begin{itemize}
                \item $T$ \hfill modely $\mathcal A_1,\mathcal A_2,\mathcal A_3,\mathcal A_4,\mathcal A_5$
                \item $T\cup\{x=y\lor x=z\}$ \hfill modely $\mathcal A_1,\mathcal A_2,\mathcal A_3,\mathcal A_4\phantom{,\mathcal A_5}$
                \item $T\cup\{(\exists x)(\exists y)\neg x=y\}$ \hfill modely $\phantom{\mathcal A_1,}\mathcal A_2,\mathcal A_3,\mathcal A_4,\mathcal A_5$\\
                (Pozor: $(\exists x)(\exists y)\neg x=y\sim\neg(\forall x)(\forall y)x=y\not\sim\neg x=y\sim(\forall x)(\forall y)\neg x=y$.)
                
                \item[\vdots]
                
                \item $\{x=x\land\neg x=x\}$ \hfill sporná teorie, nemá model
            \end{itemize}

            Jednoduché kompletní extenze:
            \begin{itemize}
                \item $\mathrm{Th}(\mathcal A_1)\sim\{x=y\}$
                \item $\mathrm{Th}(\mathcal A_2)\sim\{(\exists x)(\exists y)\neg x=y,x=y\lor x=z,c_1=c_2,\neg c_2=c_3\}$
                \item $\mathrm{Th}(\mathcal A_3)\sim\{(\exists x)(\exists y)\neg x=y,x=y\lor x=z,\neg c_1=c_2,c_2=c_3\}$
                \item $\mathrm{Th}(\mathcal A_4)\sim\{(\exists x)(\exists y)\neg x=y,x=y\lor x=z,c_1=c_3,\neg c_1=c_2\}$
                \item $\mathrm{Th}(\mathcal A_5)\sim\{x=c_1 \vee x=c_2 \vee x=c_3,\neg (c_1=c_2\lor c_1=c_3\lor c_2=c_3)\}$
            \end{itemize}    
            
            \item Teorie navíc říká, že každý prvek je buď interpretací symbolu $c_1$ nebo $c_4$. Modely tedy mají nejvýše dva prvky, až na izomorfismus jsou to:
            \begin{itemize}
                \item $\mathcal A_1'=\langle\{0\};0,0,0,0\rangle$
                \item $\mathcal A_2'=\langle\{0,1\};0,0,1,1\rangle$
                \item $\mathcal A_3'=\langle\{0,1\};0,1,1,1\rangle$
                \item $\mathcal A_4'=\langle\{0,1\};0,1,0,1\rangle$                
            \end{itemize}
            Teorie $T'$ je extenzí $T$, platí v ní všechny důsledky teorie $T$, sémanticky: restrikce modelů $T'$ na původní jazyk $L$ jsou modely $T$ (např. restrikcí modelu $\mathcal A_1'$ na $L$ je $\mathcal A_1$). Není to jednoduchá extenze, zvětšili jsme jazyk. 
            
            Není to ani konzervativní extenze, např. sentence $(\forall x)(\forall y)(\forall z)(x=y\lor x=z)$ je sentence původního jazyka $L$, platí v $T'$ ale neplatila v $T$. Sémanticky: (tříprvkový) model $\mathcal A_5$ teorie $T$ nelze expandovat do jazyka $L'$ na model teorie $T'$, neboli restrikcí modelů $T'$ na původní jazyk $L$ nedostaneme všechny modely $T$.

        \end{enumerate}
                    
    \end{solution}

\end{problem}


\begin{problem}

    Buď $T'$ extenze teorie $T=\{(\exists y)(x+y=0),(x+y=0)\wedge (x+z=0)\rightarrow y=z\}$ v jazyce $L=\langle +,0,\le\rangle$ s rovností o definice $<$ a unárního $-$ s axiomy
    \begin{align*}
        -x=y\ \ &\leftrightarrow\ \ x+y=0\\
        x<y\ \ &\leftrightarrow\ \ x\le y\ \wedge\ \neg(x=y)
    \end{align*}
    Najděte formule v jazyce $L$, které jsou ekvivalentní v $T'$ s následujícími formulemi.
        
    (a) $(-x)+x=0$ \hfill (b) $x+(-y)<x$ \hfill (c) $-(x+y)<-x$\hfill{}

    \begin{solution}

        Všimněte si, že axiomy teorie vyjadřují existenci a jednoznačnost pro definici funkčního symbolu $-$, jde tedy o korektní extenzi o definice. Postupujeme dle (důkazu) tvrzení z přednášky:
        \begin{enumerate}[(a)]
            \item $(\exists z)(x+z=0\land z+x=0)$ (Podformule $x+z=0$ říká, že `$z$ je $-x$' a druhá, že `$(-x)+x=0$'.)
            \item Nejprve nahradíme definicí term $-y$: 
            $$
            (\exists z)(y+z=0\land x+z<x)
            $$
            Nyní relační symbol $<$:
            $$
            (\exists z)(y+z=0\land x+z\leq z\land\neg(x+z=z))
            $$
            \item $(\exists u)(\exists v)((x+y)+u=0\land x+v=0\land u\leq v\land \neg u=v)$ (Kde `$u$ je $-(x+y)$' a `$v$ je $-x$'.)
        \end{enumerate}
                    
    \end{solution}
    
\end{problem}


\begin{problem}

    Mějme jazyk $L=\langle F \rangle$ s rovností, kde $F$ je binární funkční symbol. Najděte formule definující následující množiny (bez parametrů):
    \begin{enumerate}[(a)]
        \item interval $(0,\infty)$ v $\mathcal A=\langle\mathbb R, \cdot\rangle$ kde $\cdot$ je násobení reálných čísel
        \item množina $\{(x, 1/x)\mid x\neq 0\}$ ve stejné struktuře $\mathcal A$
        \item množina všech nejvýše jednoprvkových podmnožin $\mathbb N$ v $\mathcal B=\langle\mathcal P(\mathbb N),\cup\rangle$
        \item množina všech prvočísel v $\mathcal C=\langle \mathbb N\cup\{0\}, \cdot\rangle$
    \end{enumerate}

    \begin{solution}

        \begin{enumerate}[(a)]
            \item $(\exists y)F(y,y)=x\land \neg (\forall y)F(x,y)=x$ (Číslo $x$ je čtverec, a není to nula.)
            \item $(\exists z)(F(x,y)=z\land(\forall u)F(z,u)=u)$ (Součin je roven jedné.)
            \item $(\forall y)(\forall z)(F(y,z)=x\to y=x\lor z=x)\land\neg(\forall y)F(x,y)=y$ (Kdykoliv je množina sjednocením dvou množin, je rovna jedné z nich. A není prázdná.)
            \item $(\forall y)(\forall z)(F(y,z)=x\to y=x\lor z=x)\land\neg(\forall y)F(x,y)=x$ (Kdykoliv je součin dvou čísel roven prvočíslu, je jedno z nich rovno prvočíslu, a prvočíslo není nula.)
        \end{enumerate}
                    
    \end{solution}

\end{problem}
        
        
\section*{Další příklady k procvičení}


\begin{problem}

    Buď $T=\{\neg E(x,x), E(x,y)\to E(y,x), (\exists x)(\exists y)(\exists z)(E(x,y)\wedge E(y,z)\wedge E(x,z)\wedge \neg(x=y\vee y=z\vee x=z)),\varphi\}$ teorie v jazyce $L=\langle E\rangle$ s rovností, kde $E$ je binární relační symbol a $\varphi$ vyjadřuje, že ``existují právě čtyři prvky''.
    \begin{enumerate}[(a)]
        \item Uvažme rozšíření $L'=\langle E,c\rangle$ jazyka o nový konstantní symbol $c$. Určete počet (až na ekvivalenci) teorií $T'$ v jazyce $L'$, které jsou extenzemi teorie $T$. 
        \item Má $T$ nějakou \emph{konzervativní} extenzi v jazyce $L'$? Zdůvodněte.
    \end{enumerate}

\end{problem}


\begin{problem}

    Nechť $T=\{x=f(f(x)),\varphi, \neg c_1 = c_2\}$ je teorie jazyka $L=\langle f,c_1,c_2\rangle$ s rovností, kde $f$ je unární funkční, $c_1,c_2$ jsou konstantní symboly a axiom $\varphi$ vyjadřuje, že ``existují právě $3$ prvky''.
    \begin{enumerate}[(a)]    
        \item Určete, kolik má teorie $T$ navzájem neekvivalentních jednoduchých kompletních extenzí. Napište dvě z nich. {\it (3b)}
        \item Nechť $T'=\{x=f(f(x)),\varphi,\neg f(c_1)=f(c_2)\}$ je teorie stejného jazyka, axiom $\varphi$ je stejný jako výše. Je $T'$ extenze $T$? Je $T$ extenze $T'$? Pokud ano, jde o konzervativní extenzi? Uveďte zdůvodnění. {\it (2b)}
    \end{enumerate}
    
\end{problem}


\begin{problem}

    Mějme jazyk $L=\langle P,R,f,c,d\rangle$ s rovností a následující dvě formule:
    \begin{align*}
        \varphi:\quad P(x,y) &\leftrightarrow R(x,y) \wedge \neg x=y\\
        \psi:\quad P(x,y) &\to P(x,f(x,y)) \wedge P(f(x,y),y)
    \end{align*}
    Uvažme následující $L$-teorii:
    \begin{align*}
        T=\{&\varphi,\ \psi,\ \neg c=d,\\
        &R(x,x),\\ 
        &R(x,y) \wedge R(y,x) \to x=y,\\
        &R(x,y) \wedge R(y,z) \to R(x,z),\\
        &R(x,y) \vee R(y,x)\}    
    \end{align*}
    

    \begin{enumerate}[(a)]
        \item Nalezněte expanzi struktury $\langle \mathbb{Q},\le \rangle$ do jazyka $L$ na model teorie $T$.
        \item Je sentence $(\forall x)R(c,x)$ pravdivá/lživá/nezávislá v $T$? Zdůvodněte všechny tři odpovědi.
        \item Nalezněte dvě neekvivalentní kompletní jednoduché extenze $T$ nebo zdůvodněte, proč neexistují.
        \item Nechť $T'=T\setminus\{\varphi,\psi\}$ je teorie v jazyce $L'=\langle R,f,c,d\rangle$. Je teorie $T$ konzervativní extenzí teorie $T'$? Uveďte zdůvodnění.
    \end{enumerate}

\end{problem}

        
\section*{K zamyšlení}


\begin{problem}

    Nechť $T_n = \{\neg c_i = c_j | 1 \leq i < j \leq n\}$ označuje teorii jazyka $L_n = \langle c_1, \dots, c_n \rangle$ s rovností, kde $c_1, \dots, c_n$ jsou konstantní symboly.
    \begin{enumerate}[(a)]   
        \item Pro dané konečné $k \geq 1$ určete počet $k$-prvkových modelů teorie $T_n$ až na izomorfismus. 
        \item Určete počet spočetných modelů teorie $T_n$ až na izomorfismus. 
        \item Pro jaké dvojice hodnot $n$ a $m$ je $T_n$ extenzí $T_m$? Pro jaké je konzervativní extenzí? Zdůvodněte.
    \end{enumerate}

\end{problem}



