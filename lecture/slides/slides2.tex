\documentclass{beamer}


%% slide-specific

\usetheme[progressbar=frametitle]{metropolis}
%\usecolortheme{spruce}
%\metroset{block=fill}

% define Metropolis colors    
\definecolor{mAlert}{HTML}{EB811B}
\definecolor{mExample}{HTML}{14B03D}
\definecolor{mBlock}{HTML}{23373b}

% my blocks
\setlength\fboxsep{0pt}%

\newcommand{\myblock}[1]{\colorbox{mBlock!12}{\begin{minipage}{\linewidth}#1\end{minipage}}}
\newcommand{\myblockmath}[1]{\colorbox{mBlock!12}{\begin{minipage}{\linewidth}\vspace{-6pt}#1\end{minipage}}}
\newcommand{\myblockamsmath}[1]{\colorbox{mBlock!12}{\begin{minipage}{\linewidth}\vspace{-6pt}#1\end{minipage}}}
\newcommand{\myblockinline}[1]{\colorbox{mBlock!12}{#1}}
\newcommand{\myexample}[1]{\colorbox{mExample!12}{\begin{minipage}{\linewidth}#1\end{minipage}}}
\newcommand{\myexamplemath}[1]{\colorbox{mExample!12}{\begin{minipage}{\linewidth}\vspace{-6pt}#1\end{minipage}}}
\newcommand{\myexampleamsmath}[1]{\colorbox{mExample!12}{\begin{minipage}{\linewidth}\vspace{-18pt}#1\end{minipage}}}
\newcommand{\myexampleinline}[1]{\colorbox{mExample!12}{#1}}
\newcommand{\myalert}[1]{\colorbox{mAlert!12}{\begin{minipage}{\linewidth}#1\end{minipage}}}
\newcommand{\myalertmath}[1]{\colorbox{mAlert!12}{\begin{minipage}{\linewidth}\vspace{-6pt}#1\end{minipage}}}
\newcommand{\myalertamsmath}[1]{\colorbox{mAlert!12}{\begin{minipage}{\linewidth}\vspace{-18pt}#1\end{minipage}}}
\newcommand{\myalertinline}[1]{\colorbox{mAlert!12}{#1}}

%% other
\newcommand{\mystructure}[1]{\mathcal{#1}}


%% packages
\usepackage{amsmath,amssymb,amsthm}
\usepackage{bookmark}
\usepackage{booktabs}
\usepackage{cancel}
\usepackage[czech]{babel}
\usepackage{enumerate}
\usepackage[T1]{fontenc}
\usepackage{forest}
\usepackage{lmodern}
\usepackage{multicol}
% \usepackage{tcolorbox}
\usepackage{tikz}
    \usetikzlibrary{arrows.meta}
%\usepackage[unicode]{hyperref}
\usepackage[utf8x]{inputenc}
\usepackage{xfrac}


% %% theorems
% \theoremstyle{plain}
%     \newtheorem{theorem}{Věta}[section]
%     \newtheorem*{theorem-unnumbered}{Věta}
%     \newtheorem{proposition}[theorem]{Tvrzení}
%     \newtheorem{corollary}[theorem]{Důsledek}
%     \newtheorem{lemma}[theorem]{Lemma}
%     \newtheorem{observation}[theorem]{Pozorování}
% \theoremstyle{definition}
%     \newtheorem{definition}[theorem]{Definice}
%     \newtheorem*{algorithm}{Algoritmus}
% \theoremstyle{remark}
%     \newtheorem{remark}[theorem]{Poznámka}
%     \newtheorem{example}[theorem]{Příklad}
%     \newtheorem{exercise}{Cvičení}[chapter]
%     \newtheorem*{solution}{Řešení}

%% macros and definitions
\DeclareMathOperator{\Aut}{Aut}
\DeclareMathOperator{\Conseq}{Csq}
\DeclareMathOperator{\DeLO}{DeLO}
\DeclareMathOperator{\dom}{dom}
\DeclareMathOperator{\Fm}{Fm}
\DeclareMathOperator{\M}{M}
%\DeclareMathOperator{\Proof}{Proof}
\DeclareMathOperator{\rng}{rng}
\DeclareMathOperator{\Term}{Term}
\DeclareMathOperator{\Th}{Th}
\DeclareMathOperator{\Thm}{Thm}
\DeclareMathOperator{\Tree}{Tree}
\DeclareMathOperator{\Var}{Var}
\DeclareMathOperator{\VF}{VF}

\newcommand{\A}{\mystructure{A}}
\newcommand{\B}{\mystructure{B}}
\newcommand{\Con}{\mathit{Con}}
\newcommand{\disjointunion}{\mathbin{\dot{\sqcup}}}
\newcommand{\F}{\ensuremath{\mathrm{F}}}
\newcommand{\landsymb}{{\land}}
\newcommand{\lbin}{\mathbin{\square}}
\newcommand{\lbinsymb}{{\lbin}}
\newcommand{\liff}{\mathbin{\leftrightarrow}}
\newcommand{\liffsymb}{{\liff}}
\newcommand{\limplies}{\mathbin{\rightarrow}}
\newcommand{\limpliessymb}{{\limplies}}
\newcommand{\lorsymb}{{\lor}}
\newcommand{\Prf}{\mathit{Prf}}
%\newcommand{\structure}[1]{\mathcal{#1}}
\newcommand{\todo}{[TODO]}
\newcommand{\T}{\ensuremath{\mathrm{T}}}
\newcommand{\union}{\mathbin{\cup}}

\DeclareRobustCommand\proves{\mathrel{|}\joinrel\mkern-.5mu\mathrel{-}}

\title{Druhá přednáška}
\subtitle{NAIL062 Výroková a predikátová logika}
\author{Jakub Bulín (KTIML MFF UK)}
% \institute{KTIML MFF UK}
\date{Zimní semestr 2023}


\begin{document}


\maketitle


\begin{frame}{Druhá přednáška}

    \textbf{Program}
        \begin{itemize}
            \item sémantika výrokové logiky (pokračování)
            \item normální formy
            \item vlastnosti a důsledky teorií
            \item extenze teorií
        \end{itemize}   

    \textbf{Materiály}

        \href{https://github.com/jbulin-mff-uk/nail062/raw/main/lecture/lecture-notes/lecture-notes.pdf}{\alert{\textbf{Zápisky z přednášky}}}, Sekce 2.2.5- z Kapitoly 2

\end{frame}


\begin{frame}{Další sémantické pojmy}

    \begin{itemize}
        \item výrok $\varphi$ (nad $\mathbb P$) je \alert{pravdivý}, \alert{tautologie}, \alert{platí (v~logice)}, \alert{$\models \varphi$}, pokud platí v každém modelu, $\M_\mathbb P(\varphi)=\M_\mathbb P$
        \item \alert{lživý}, \alert{sporný}, pokud nemá žádný model, $\M_\mathbb P(\varphi)=\emptyset$
        
        \myalertinline{\it (Být \alert{lživý} není totéž, co nebýt \alert{pravdivý}!)}

        \item \alert{nezávislý}, pokud platí v nějakém modelu a neplatí v nějakém jiném modelu, tj.\ není pravdivý ani lživý, $\emptyset\subsetneq\M_\mathbb P(\varphi)\subsetneq\M_\mathbb P$
        \item \alert{splnitelný}, pokud má nějaký model, tj.\ není lživý, $\M_\mathbb P(\varphi)\neq\emptyset$
    \end{itemize}
    výroky $\varphi,\psi$ (ve stejném jazyce) jsou \alert{(logicky) ekvivalentní}, \alert{$\varphi\sim\psi$}, pokud mají stejné modely, tj.\
    \myalertinline{
    $
    \varphi\sim\psi\ \Leftrightarrow\ \M_\mathbb P(\varphi)=\M_\mathbb P(\psi)
    $
    }

    \myexample{
    \begin{itemize}
        \item pravdivé jsou např.: $\top$, $p\lor q\liff q\lor p$
        \item lživé: $\bot$, $(p\lor q)\land (p\lor \neg q)\land \neg p$
        \item nezávislé a také splnitelné: $p$, $p\land q$
        \item ekvivalentní: $p\sim p\lor p$, $p\limplies q\sim \neg p\lor q$, $\neg p \limplies (p\limplies q) \sim \top $
    \end{itemize}
    }
    
\end{frame}


\begin{frame}{Sémantické pojmy vzhledem k teorii}

    \textbf{relativně k dané teorii $T$} (omezíme se na její modely):
    \begin{itemize}
        \item \alert{pravdivý/platí v $T$}, \alert{důsledek $T$}, \alert{$T \models \varphi$} je-li \myalertinline{$\M_\mathbb P(T)\subseteq \M_\mathbb P(\varphi)$}
        \item \alert{lživý/sporný v $T$} pokud $\M_\mathbb P(\varphi)\cap\M_\mathbb P(T)=\M_\mathbb P(T,\varphi)=\emptyset$.
        \item \alert{nezávislý v $T$} pokud $\emptyset\subsetneq\M_\mathbb P(T,\varphi)\subsetneq\M_\mathbb P(T)$,
        \item \alert{splnitelný v $T$}, \alert{konzistentní s $T$} pokud $\M_\mathbb P(T,\varphi)\neq\emptyset$ (platí v alespoň jednom modelu $T$)
        \item $\varphi$ a $\psi$ jsou \alert{ekvivalentní v $T$}, \alert{$T$-ekvivalentní}, \alert{$\varphi\sim_T\psi$} platí-li v~týchž modelech $T$, tj. \myalertinline{
    $
    \varphi\sim_T\psi\ \Leftrightarrow\ \M_\mathbb P(T,\varphi)=\M_\mathbb P(T,\psi)
    $}
    \end{itemize}

    \myexample{např. pro $T=\{p\lor q,\neg r\}$:
        \begin{itemize}
            \item výroky $q\lor p$, $\neg p\lor\neg q\lor \neg r$ jsou pravdivé v $T$
            \item výrok $(\neg p\land\neg q)\lor r$ je v $T$ lživý
            \item výroky $p\liff q, p\land q$ jsou v $T$ nezávislé, a také splnitelné
            \item platí $p\sim_T p\lor r$ (ale $p\not\sim p\lor r$)
        \end{itemize}      
    }

\end{frame}


\begin{frame}{Univerzálnost logických spojek}

    množina logických spojek je \alert{univerzální}, pokud:
    \begin{itemize}        
        \item každá booleovská funkce je pravdivostní funkcí nějakého výroku vybudovaného z těchto spojek
        \item ekvivalentně: každá množina modelů nad konečným jazykem je množinou modelů nějakého výroku
    \end{itemize}

    \myblock{\textbf{Tvrzení:} $\{\neg, \landsymb,\lorsymb\}$ a $\{\neg, \limpliessymb\}$ jsou univerzální.}
    
    {\footnotesize [Důkaz na příštím slidu.]}


    Další zajímavé logické spojky:
    \begin{itemize}
        \item \alert{Shefferova spojka} (NAND, $\uparrow$)\hfill $p\uparrow q \sim \neg (p\land q)$,
        \item \alert{Pierceova spojka} (NOR, $\downarrow$)\hfill $p\downarrow q \sim \neg (p\lor q)$,
        \item \alert{Exclusive-OR} (XOR, $\oplus$)\hfill $p\oplus q \sim (p\lor q)\land\neg(p\land q)$
    \end{itemize}
    \myexampleinline{
        např. $\{\uparrow\}$ je univerzální, $\{\land,\lor\}$ není
    }

\end{frame}


\begin{frame}{Důkaz, že $\{\neg, \landsymb,\lorsymb\}$ a $\{\neg, \limpliessymb\}$ jsou univerzální}  

    Mějme $f\colon \{0,1\}^n\to \{0,1\}$, resp. $M=f^{-1}[1]\subseteq \{0,1\}^n$
 
    \textbf{Pro jediný model:} \myalertinline{
        $\varphi_v=\text{`musím být model $v$'}$
    }

    \begin{itemize}
        \item příklad: \myexampleinline{$v=(1,0,1,0)\ \rightsquigarrow\ \varphi_v=p_1\land \neg p_2 \land p_3\land \neg p_4$}
        \item obecně: $v=(v_1,\dots,v_n)$, použijeme značení $p^1=p$, $p^0=\neg p$
        $$
        \varphi_v = p_1^{v_1}\land p_2^{v_2}\land \dots\land p_n^{v_n}=\bigwedge_{i=1}^n p_i^{v(p_i)}=\bigwedge_{p\in\mathbb P}p^{v(p)}
        $$    
    \end{itemize}
    
    \textbf{Pro více modelů:} \myalertinline{`musím být alespoň jeden z modelů z $M$'}
    $$
    \varphi_M = \bigvee_{v\in M}\varphi_v=\bigvee_{v\in M}\bigwedge_{p\in\mathbb P}p^{v(p)}
    $$

    Zřejmě $\M(\varphi_M)=M$ neboli $f_{\varphi_M,\mathbb P}=f$, a $\varphi_M$ používá jen $\{\neg, \landsymb,\lorsymb\}$. Protože $p\land q\sim \neg (p\limplies \neg q)$ a $p\lor q\sim \neg p\limplies q$, mohli bychom $\varphi_M$ ekvivalentně vyjádřit i pomocí  $\{\neg, \limpliessymb\}$. \hfill\qedsymbol   

\end{frame}


\section{2.3 Normální formy}


\begin{frame}{CNF a DNF}

    \begin{itemize}
        \item \alert{literál} je prvovýrok nebo jeho negace, 
        $\bar \ell$ je \alert{opačný literál} k $\ell$ (pro \emph{pozitivní} $\ell=p$ je $\bar \ell=\neg p$, pro \emph{negativní}  $\ell=\neg p$  je $\bar \ell=p$)
        \item \alert{klauzule} je disjunkce literálů $C=\ell_1\lor\ell_2\lor\dots\lor\ell_n$ 
        (\emph{jednotková klauzule} je samotný literál, \emph{prázdná klauzule} je $\bot$)
        \item výrok je v \alert{konjunktivní normální formě} (\alert{CNF}) je-li konjunkcí klauzulí (prázdný CNF výrok je $\top$)
        \item \alert{elementární konjunkce} je konjunkce literálů $E=\ell_1\land\dots\land\ell_n$ (\emph{jednotková} el. konjunkce je samotný literál,  \emph{prázdná} je $\top$)
        \item výrok je v \alert{disjunktivní normální formě} (\alert{DNF}) je-li disjunkcí elementárních konjunkcí (prázdný DNF výrok je $\bot$)
    \end{itemize}

    například:
    \myexample{
        \begin{itemize}
            \item $(p\lor q)\land (p\lor \neg q)\land \neg p$ je v CNF
            \item $\neg p\lor (p\land q)$ je v DNF
            \item $\varphi_v$ je v CNF i DNF, $\varphi_M$ je v DNF
        \end{itemize}

    }

\end{frame}


\begin{frame}{O dualitě}

    \myblock{zaměníme-li $0\leftrightsquigarrow 1$, negace zůstává stejná, z $\landsymb$ se stává $\lorsymb$ a naopak}
    \vspace{-12pt}

    \begin{itemize}
        \item $\varphi$ nad $\{\neg,\landsymb,\lorsymb\}$, zaměníme-li $\landsymb,\lorsymb$ a znegujeme-li prvovýroky: \alert{duální} $\psi\sim\neg\varphi$, modely $\varphi$ jsou nemodely $\psi$, $f_\psi(\neg x)=\neg f_\varphi(x)$
        \item CNF a DNF jsou duální pojmy
        \item \emph{pravdivost} je duální k \emph{nesplnitelnosti} 
    \end{itemize}

    \textbf{Pozorování:} Výrok v CNF je \alert{pravdivý}, právě když každá klauzule má dvojici opačných literálů.

    \textbf{Duálně:} Výrok v DNF je \alert{nesplnitelný}, pokud každá elementární konjunkce má dvojici opačných literálů.
       
\end{frame}


\begin{frame}{Převod do normální formy: sémanticky (příklad)}

    mějme výrok \myexampleinline{$\varphi=p\liff (q\lor \neg r)$}

    jeho modely jsou \myexampleinline{
        $M=\{(0,0,1),(1,0,0),(1,1,0),(1,1,1)\}$
    }

    najdeme DNF a CNF výroky se stejnými modely, tj. ekvivalentní~$\varphi$

    DNF sestrojíme tak, že pro každý model přidáme elementární konjunkci vynucující právě tento model:    
    \myexamplemath{\vspace{-12pt}
    $$
    \varphi_{\mathrm{DNF}}=(\neg p\land\neg q\land r)\lor (p\land\neg q\land\neg r) \lor (p\land q\land\neg r)\lor (p\land q\land r)
    $$
    }

    při konstrukci CNF potřebujeme \alert{nemodely} $\varphi$: 
    \myexamplemath{$$\overline{M}=\{(0,0,0),(0,1,0),(0,1,1),(1,0,1)\}$$
    }

    každá klauzule zakáže jeden nemodel: 
    \myexamplemath{\vspace{-12pt}
    $$
    \varphi_{\mathrm{CNF}}=(p\lor q\lor r)\land (p\lor\neg q\lor r) \land (p\lor \neg q\lor\neg r)\land (\neg p\lor q\lor\neg r)
    $$ 
    }  

\end{frame}


\begin{frame}{Převod do normální formy: sémanticky}
    
    \myblock{
        \textbf{Tvrzení:} Buď $\mathbb P$ konečný, $M\subseteq\M_\mathbb P$ libovolná. Potom existují DNF a CNF výroky $\varphi_{\mathrm{DNF}},\varphi_{\mathrm{CNF}}$, že 
    \alert{$M=\M_\mathbb P(\varphi_{\mathrm{DNF}})=\M_\mathbb P(\varphi_{\mathrm{CNF}})$}.

    \vspace{-18pt}
    \begin{align*}
        \varphi_{\mathrm{\mathrm{DNF}}} &= \bigvee_{v\in M}\bigwedge_{p\in\mathbb P}p^{v(p)}\\
        \varphi_{\mathrm{CNF}} &= \bigwedge_{v\in \overline{M}}\bigvee_{p\in\mathbb P}\overline{p^{v(p)}}=\bigwedge_{v\notin M}\bigvee_{p\in\mathbb P}p^{1-v(p)}
    \end{align*}
    }

    \textbf{Důkaz:} $\varphi_{\mathrm{DNF}}=\varphi_M$ říká  \alert{`jsem jeden z modelů z $M$'}

    $\varphi_{\mathrm{CNF}}$ říká \alert{`nejsem žádný z nemodelů z $M$'}, je duální k $\varphi'_{\mathrm{DNF}}=\varphi_{\overline{M}}$ pro doplněk $M$, nebo přímo: modely klauzule $C_v=\bigvee_{p\in\mathbb P}p^{1-v(p)}$ jsou $\M_C=\M_\mathbb P\setminus\{v\}$, tedy každá klauzule zakáže jeden nemodel\hfill\qedsymbol

    \medskip
    \myblock{
    \textbf{Důsledek:} Každý výrok (v libovolném, i nekonečném jazyce $\mathbb P$) je ekvivalentní nějakému výroku v CNF a nějakému výroku v DNF.
    }

    \textbf{Důkaz:} použijeme konečný jazyk $\mathbb P'=\Var(\varphi)$, $M=\M_{\mathbb P'}(\varphi)$\hfill\qedsymbol

\end{frame}


\begin{frame}{Převod do normální formy: syntakticky}

    Hledat všechny modely je neefektivní, lze i syntakticky pomocí \alert{ekvivalentních úprav}.

    \textbf{Pozorování:} Nahradíme-li podvýrok $\psi$ výroku $\varphi$ ekvivalentním $\psi'$, výsledný výrok $\varphi'$ je také ekvivalentní $\varphi$.

    \textbf{Postup úprav:}
    \begin{enumerate}
        \item přepiš ekvivalenci a implikaci pomocí $\neg,\landsymb,\lorsymb$
        \item přesuň negace dolů (k listům) ve stromu výroku pomocí de Morganových pravidel, odstraň dvojité negace
        \item přesuň dolů disjunkce (pro CNF) resp. konjunkce (pro DNF) pomocí distributivity $\landsymb$ a $\lorsymb$
        \item případně zjednoduš (odstranění duplicit, tautologií apod.)
    \end{enumerate}

    Důkaz, že funguje: indukcí dle struktury výroku

\end{frame}


\begin{frame}{Převod do normální formy: syntakticky (příklad)}

    mějme opět výrok \myexampleinline{$\varphi=p\liff (q\lor \neg r)$}

    \begin{itemize}
        \item přepsat ekvivalence a implikace
        \begin{align*}
            p\liff (q\lor \neg r) &\sim (p\limplies (q\lor \neg r)) \land ((q\lor \neg r) \limplies p)\\
            &\sim (\neg p\lor q\lor \neg r) \land (\neg (q\lor \neg r) \lor p)
        \end{align*}
        \item negace dolů
        $$
        (\neg p\lor q\lor \neg r) \land ( (\neg q\land r) \lor p)
        $$
        \item do CNF (+ seřadíme prvovýroky v klauzulích)
        \myexamplemath{
        $$
        (\neg p\lor q\lor \neg r) \land (p\lor \neg q) \land (p \lor r) 
        $$
        }

        \item do DNF (+ zjednodušení)
        \myexamplemath{
        $$
        (\neg p \land \neg q\land r) \lor (p\land q \land r) \lor (p\land \neg r)
        $$
        }
    \end{itemize}

\end{frame}


\begin{frame}{Ekvivalentní úpravy}

    \begin{itemize}
        \item Implikace a ekvivalence:
        \begin{itemize}
            \item[] $\varphi\limplies\psi\sim\neg\varphi\lor\psi$
            \item[] $\varphi\liff\psi\sim(\neg\varphi\lor\psi)\land(\neg\psi\lor\varphi)$
        \end{itemize}
        \item Negace:
        \begin{itemize}
            \item[] $\neg(\varphi\land\psi)\sim\neg\varphi\lor\neg\psi$
            \item[] $\neg(\varphi\lor\psi)\sim\neg\varphi\land\neg\psi$
            \item[] $\neg\neg\varphi\sim\varphi$
        \end{itemize}
        \item Konjunkce (převod do DNF):
        \begin{itemize}
            \item[] $\varphi \land (\psi\lor\chi) \sim (\varphi\land\psi)\lor (\varphi\land\chi)$
            \item[] $(\varphi \lor \psi)\land\chi \sim (\varphi\land\chi)\lor (\psi\land\chi)$
        \end{itemize}
        \item Disjunkce (převod do CNF):
        \begin{itemize}
            \item[] $\varphi \lor (\psi\land\chi) \sim (\varphi\lor\psi)\land (\varphi\lor\chi)$
            \item[] $(\varphi \land \psi)\lor\chi \sim (\varphi\lor\chi)\land (\psi\lor\chi)$
        \end{itemize}
    \end{itemize}

\end{frame}


\section{2.4 Vlastnosti a důsledky teorií}


\begin{frame}{Vlastnosti teorií}
 
    \begin{itemize}
        \item \alert{sporná}: $T\models\bot$, ekvivalentně: nemá model, platí v ní vše
        \item \alert{bezesporná} (\alert{splnitelná}): není sporná, tj. má model
        \item \alert{kompletní}: bezesporná + každý výrok je v ní pravdivý nebo lživý (nemá nezávislé výroky), ekvivalentně: \myalertinline{právě jeden model}
        \item \alert{ekvivalence teorií}: $\alert{T\sim T'}$ právě když $\M_\mathbb P(T)=\M_\mathbb P(T')$ (různé \emph{axiomatizace} týchž vlastností)
    \end{itemize}   

    \myexample{
        \begin{itemize}
            \item $T_1=\{p,p\limplies q,\neg q\}$ je sporná
            \item $T_2=\{p\lor q,r\}$ je bezesporná, ale není kompletní, např. $p\land q$ je v ní nezávislý: platí v modelu $(1,1,1)$, neplatí v $(1,0,1)$ 
            \item $T_2\cup\{\neg p\}$ je kompletní, jediným modelem je $(0,1,1)$.
            \item ekvivalentní teorie: $\{p\limplies q,r\}\sim\{(\neg p\lor q)\land r\}$
        \end{itemize}
    }

\end{frame}


\begin{frame}{Důsledky teorií}

    \myblock{
    Buď $T$ teorie v jazyce $\mathbb P$. \alert{Množina všech důsledků} $T$ \alert{v jazyce $\mathbb P'$}:\vspace{-6pt}
    $$
    \alert{\Conseq_{\mathbb P'}(T)}=\{\varphi\in\VF_{\mathbb P'}\mid T\models \varphi\}
    $$
    }

    pokud $\mathbb P'=\mathbb P$:
    \myalertinline{%
        $\Conseq_\mathbb P(T)=\{\varphi\in\VF_\mathbb P\mid \M_\mathbb P(T)\subseteq \M_\mathbb P(\varphi)\}$}

    \bigskip

    \myblock{
    \textbf{Tvrzení:} Jsou-li $T,T'$ teorie a $\varphi,\varphi_1,\dots,\varphi_n$ výroky v jazyce $\mathbb P$:
        \begin{enumerate}[(i)]       
            \item $T\subseteq\Conseq_\mathbb P(T)$
            \item $\Conseq_\mathbb P(T)=\Conseq_\mathbb P(\Conseq_\mathbb P(T))$
            \item pokud $T\subseteq T'$, potom $\Conseq_\mathbb P(T)\subseteq\Conseq_\mathbb P(T')$
            \item $\varphi\in\Conseq_\mathbb P(\{\varphi_1,\dots,\varphi_n\})$ právě když $\models(\varphi_1\land \dots\land\varphi_n)\to\varphi$
        \end{enumerate}
    }

    \textbf{Důkaz:} snadný, použijte následující:
    \begin{itemize}
    \item $\M(\Conseq(T))=\M(T)$
    \item je-li $T\subseteq T'$ potom $\M(T)\supseteq\M(T')$
    \item $\models\psi\limplies\varphi$ právě když $\M(\psi)\subseteq\M(\varphi)$\hfill \qedsymbol
    \end{itemize}

\end{frame}


\begin{frame}{Extenze teorií: neformálně}

    \alert{Extenze} teorie $T$ je jakákoliv teorie, která splňuje vše, co platí v $T$
    \begin{itemize}
        \item dodatečné požadavky o systému: \alert{jednoduchá extenze}
        \item přidání nových částí k systému (a v původním platí totéž, co předtím): \alert{konzervativní extenze}
    \end{itemize}

    \myexample{
    Úvodní příklad o barvení grafů: 
    \begin{itemize}
        \item $T_3$ (úplná obarvení s hranovou podmínkou) je jednoduchou extenzí teorie $T_1$ (částečná obarvení bez ohledu na hrany)
        \item $T_3'$ (přidání nového vrcholu) je konzervativní, ale ne jednoduchou extenzí $T_3$
        \item $T_3'$ je extenze $T_1$, která není ani jednoduchá, ani konzervativní
    \end{itemize}
    }

\end{frame}


\begin{frame}{Extenze teorií: formálně}

    \myblock{Buď $T$ v jazyce $\mathbb P$. \alert{Extenze} teorie $T$ je libovolná teorie $T'$ v jazyce $\mathbb P'\supseteq\mathbb P$ splňující $\Conseq_\mathbb P(T)\subseteq\Conseq_{\mathbb P'}(T')$,
    
    \begin{itemize}
        \item \alert{jednoduchá}: $\mathbb P'=\mathbb P$
        \item \alert{konzervativní}: $\Conseq_\mathbb P(T)=\Conseq_\mathbb P(T')=\Conseq_{\mathbb P'}(T')\cap \VF_\mathbb P$\\
        \begin{center}
            \it ``nové důsledky musí obsahovat nové prvovýroky''
        \end{center}
            
    \end{itemize}
    }

    {\small
    \textbf{Pozorování:}\\
    1. $T'$ je jednoduchá extenze $T$, právě když $\mathbb P'=\mathbb P$ a $\M_\mathbb P(T')\subseteq\M_\mathbb P(T)$

    2. $T'$ je extenze $T$, právě když $\M_{\mathbb P'}(T')\subseteq\M_{\alert{\mathbb P'}}(T)$. Tj. \alert{restrikce}
    modelů $T'$ na $\mathbb P$ musí být modely $T$:
    \alert{$
    \{v{\restriction_\mathbb P}\mid v\in\M_{\mathbb P'}(T')\}\subseteq\M_\mathbb P(T)
    $}

    3. $T'$ je konzervativní extenze $T$, je-li to extenze a navíc každý model $T$ lze \alert{expandovat} na model $T'$ (tj. \emph{každý} model $T$ získáme restrikcí \emph{nějakého} modelu $T'$ na jazyk $\mathbb P$):
    \alert{$
    \{v{\restriction_\mathbb P}\mid v\in\M_      {\mathbb P'}(T')\}=\M_\mathbb P(T)
    $}

    4. $T'$ je extenze $T$ a zároveň $T$ je extenze $T'$, právě když $T\sim T'$     

    5. \alert{Kompletní jednoduché extenze} $T$ \alert{odpovídají modelům} $T$ (až na $\sim$)
    }

\end{frame}


\begin{frame}{Extenze teorií: příklad}

    \small
    \begin{itemize}
        \item mějme \myexampleinline{%
            $T=\{p\limplies q\}$ v jazyce $\mathbb P=\{p,q\}$}, teorie $T_1=\{p\land q\}$ v jazyce $\mathbb P$ \alert{je jednoduchá} extenze $T$: $\M_\mathbb P(T_1)\subseteq\M_\mathbb P(T)$
        %$\M_\mathbb P(T_1)=\{(1,1)\}\subseteq\{(0,0),(0,1),(1,1)\}=\M_\mathbb P(T)$
        \item $T_1$ je kompletní, až na ekvivalenci všechny jednoduché kompletní extenze $T$ jsou: $T_1$, $T_2=\{\neg p,q\}$, a $T_3=\{\neg p,\neg q\}$
        \item teorie \myexampleinline{%
            $T'=\{p\liff (q\land r)\}$ v $\mathbb P'=\{p,q,r\}$} je extenzí teorie $T$: $\mathbb P\subseteq\mathbb P'$ a $\M_{\mathbb P'}(T')\subseteq\M_{\mathbb P'}(T)$, restrikce modelů $T'$ na $\mathbb P$ jsou $\{(0,0),(0,1),(1,1)\}\subseteq\M_\mathbb P(T)$
        % {\footnotesize
        % \begin{align*}
        %     \M_{\mathbb P'}(T')&=\{(0,0,0),(0,0,1),(0,1,0),(1,1,1)\}\\ 
        %     &\subseteq\{(0,0,0),(0,0,1),(0,1,0),(0,1,1),(1,1,0),(1,1,1)\}=\M_{\mathbb P'}(T)     
        % \end{align*}
        %}
        % restrikce modelů $T'$ na $\mathbb P$ jsou {\small $\{(0,0),(0,1),(1,1)\}\subseteq\M_\mathbb P(T)$}
        
        \item protože dokonce $\{(0,0),(0,1),(1,1)\}=\M_\mathbb P(T)$, každý model $T$ lze rozšířit na model $T'$, $T'$ \alert{je konzervativní} extenze $T$
        \item každý výrok v jazyce $\mathbb P$ platí v $T$, právě když platí v $T'$, ale výrok $p\limplies r$ je novým důsledkem: platí v $T'$ ale ne v $T$
        \item teorie \myexampleinline{%
            $T''=\{\neg p\lor q,\neg q\lor r,\neg r\lor p\}$ v jazyce $\mathbb P'$} je extenze $T$, ale \alert{není konzervativní}, neboť v ní platí $p\liff q$, což neplatí v $T$ (nebo proto, že model $(0, 1)$ teorie $T$ nelze rozšířit na model teorie $T''$)
    \end{itemize}
 
\end{frame}


\end{document}


