\section*{NAIL062 P\&P Logic: Worksheet 6 -- Basics of predicate logic}
% after Lecture 6


\subsection*{Teaching goals:} After completing, the student

\begin{itemize}\setlength{\itemsep}{0pt}
    \item understands the notion of a structure and a signature, can formally define them and provide examples
    \item understands the notions of the syntax of predicate logic (language, term, atomic formula, formula, theory, free variable, open formula, sentence, instance, variant), can formally define them and provide examples
    \item understands the notions of the semantics of predicate logic (value of a term, truth value, validity [under an assignment], model, truth/falsity in a model/in a theory, independence [in a theory], consequence of a theory), can formally define them and provide examples
    \item understands the notion of a complete theory and its relation to elementary equivalence of structures, can define both and apply them to examples
    \item knows basic examples of theories (graph theories, orders, algebraic theories)
    \item can describe models of a given theory    
\end{itemize}
    

\section*{In-class problems}


\begin{problem}
    
    Are the following formulas variants of the formula $(\forall x)(x<y \vee (\exists z)(z=y \wedge z\ne x))$?
    \begin{enumerate}[(a)]
        \item $(\forall z)(z<y \vee (\exists z)(z=y \wedge z\ne z))$
        \item $(\forall y)(y<y \vee (\exists z)(z=y \wedge z\ne y))$
        \item $(\forall u)(u<y \vee (\exists z)(z=y \wedge z\ne u))$
    \end{enumerate}

    \begin{solution}
        Let $\psi=(x<y \vee (\exists z)(z=y \wedge z\ne x))$, so the formula is $(\forall x)\psi$.

        \begin{enumerate}[(a)]
            \item No, $z$ is not substitutable for $x$ into $\psi$: a new bound occurrence would be created.
            \item No, $y$ has a free occurrence in $\psi$.
            \item Yes, $u$ is a fresh variable: in that case one can always form a variant.
        \end{enumerate}        
        
    \end{solution}

\end{problem}


\begin{problem}

    Let $\mathcal{A}=(\{a,b,c,d\};\vartriangleright^{A})$ be a structure in the language with a single binary relation symbol $\vartriangleright$, where $\vartriangleright^{A}=\{(a,c), (b,c), (c,c), (c,d)\}$. 
    \begin{enumerate}[I.]
        \item Which of the following formulas are true in $\mathcal A$? 
        \item For each of them find a structure $\mathcal{B}$ (if one exists) such that $\mathcal{B}\models \varphi$ iff $\mathcal{A}\not\models \varphi$.
    \end{enumerate}    
    \begin{enumerate}[(a)]
       \item $x \vartriangleright y$
       \item $(\exists x)(\forall y)(y \vartriangleright x)$
       \item $(\exists x)(\forall y)((y \vartriangleright x) \to (x \vartriangleright x))$
       \item $(\forall x)(\forall y)(\exists z)((x \vartriangleright z)\wedge(z \vartriangleright y))$
       \item $(\forall x)(\exists y)((x \vartriangleright z)\vee(z \vartriangleright y))$
    \end{enumerate}

    \begin{solution}

        We can visualize the structure as oriented edges.
        \begin{enumerate}[(a)]
            \item I. No — intuitively the formula would state that the relation $\vartriangleright^\mathcal A$ contains all pairs (edges); from the definition $\mathrm{PH}^\mathcal A(x \vartriangleright y)[e]=0$ for example for $e(x)=a$, $e(y)=a$.\\            
            II. For example $\mathcal B_0=(\{0\};\vartriangleright^{\mathcal B_0})$ with $\vartriangleright^{\mathcal B_0}=\{(0,0)\}$.
            \item I. No — intuitively the graph has no sink; from the definition: $\mathrm{PH}^\mathcal A(\varphi)=\max_{u\in A}\mathrm{PH}^\mathcal A((\forall y)(y \vartriangleright x))[e(x/u)]=\max_{u\in A}\min_{v\in A}\mathrm{PH}^\mathcal A(y \vartriangleright x)[e(x/u,y/v)]=0$, e.g. for $u=a$ we may take $v=a$.\\
            II. For example $\mathcal B_0$ as above.
            \item I. Yes (evaluate $x$ for instance by the element $a$), the antecedent is not satisfied for any valuation of $y$, hence the implication is always true.\\
            II. For example $\mathcal B_1=(\{0,1\};\vartriangleright^{\mathcal B_1})$ with $\vartriangleright^{\mathcal B_1}=\{(0,1)\}$.
            \item I. No. II: For example $\mathcal B_0$.
            \item I. No. II: For example $\mathcal B_0$.  
        \end{enumerate}
        
    \end{solution}

\end{problem}


\begin{problem}

    Prove (semantically) or find a counterexample: For every structure $\mathcal{A}$, formula $\varphi$, and sentence $\psi$,
    \begin{enumerate}[(a)]
    \item $\mathcal{A}\models (\psi \to (\exists x)\varphi) \Leftrightarrow \mathcal{A}\models (\exists x)(\psi \to \varphi)$
    \item $\mathcal{A}\models (\psi \to (\forall x)\varphi) \Leftrightarrow \mathcal{A}\models (\forall x)(\psi \to \varphi)$
    \item $\mathcal{A}\models ((\exists x)\varphi \to \psi) \Leftrightarrow \mathcal{A}\models (\forall x)(\varphi \to \psi)$
    \item $\mathcal{A}\models ((\forall x)\varphi \to \psi ) \Leftrightarrow \mathcal{A}\models (\exists x)(\varphi \to \psi)$
    \end{enumerate}
    Does it hold for every formula $\psi$ with a free variable $x$? And for every formula $\psi$ in which $x$ is not free?

    \begin{solution}

        (a) It would be simpler to use the tableau method, but we want to practice a semantic proof. Intuitively, since $\psi$ is a sentence, the valuation of $x$ does not play a role in computing the truth value of $\psi$, so the equivalence holds. Compute from the definitions: $\mathcal{A}\models (\psi \to (\exists x)\varphi)$ holds iff it holds under every valuation $e:\mathrm{Var}\to\mathcal A$. Compute the truth value. Use the fact that $f_\to(a,b)=\max(1-a,b)$:
            \begin{align*}
                &\mathrm{PH}^\mathcal A(\psi \to (\exists x)\varphi)[e]\\
                =&f_\to(\mathrm{PH}^\mathcal A(\psi)[e], \mathrm{PH}^\mathcal A((\exists x)\varphi)[e])\\
                =&\max(1-\mathrm{PH}^\mathcal A(\psi)[e], \mathrm{PH}^\mathcal A((\exists x)\varphi)[e])\\
                =&\max(1-\mathrm{PH}^\mathcal A(\psi)[e], \max_{a\in A}\mathrm{PH}^\mathcal A(\varphi)[e(x/a)])\\                
            \end{align*}
            Similarly for the formula on the right:
            \begin{align*}
                &\mathrm{PH}^\mathcal A((\exists x)(\psi \to \varphi))[e]\\
                =&\max_{a\in A}\mathrm{PH}^\mathcal A(\psi \to \varphi)[e(x/a)]\\
                =&\max_{a\in A}(\max(1-\mathrm{PH}^\mathcal A(\psi)[e(x/a)], \mathrm{PH}^\mathcal A(\varphi)[e(x/a)]))
            \end{align*}
            Because $\psi$ is a sentence, it does not contain a free occurrence of the variable $x$, hence $\mathrm{PH}^\mathcal A(\psi)[e(x/a)]=\mathrm{PH}^\mathcal A(\psi)[e]$. From this we see that:
            \begin{align*}
                =&\max_{a\in A}(\max(1-\mathrm{PH}^\mathcal A(\psi)[e], \mathrm{PH}^\mathcal A(\varphi)[e(x/a)]))\\
                =&\max(1-\mathrm{PH}^\mathcal A(\psi)[e], \max_{a\in A}(\mathrm{PH}^\mathcal A(\varphi)[e(x/a)]))
            \end{align*}
            Both truth values are the same, so the equivalence holds. For this argument it is sufficient that $x$ is not free in $\psi$. 
            
            If $x$ is free in $\psi$, the equivalence does not hold. For example in language $L=\langle c\rangle$ with equality, where $c$ is a constant symbol:
            \begin{itemize}
                \item $\varphi$ is $\neg x=x$,
                \item $\psi$ is $x=c$,
                \item $\mathcal A=(\{0,1\};0)$ (i.e. $c^\mathcal A=0$).
            \end{itemize}
            We have $\mathcal A\not\models (x=c\to (\exists x) \neg x=x)$, because it does not hold under the valuation $e(x)=0$. But $\mathcal A\models(\exists x)(x=c\to \neg x=x)$, because $x$ can be assigned the element $1$, and then the antecedent is not satisfied.

        (b), (c), (d) are solved similarly. 
                    
    \end{solution}

\end{problem}


\begin{problem}

    Decide whether $T$ (in the language $L = \langle U, f \rangle$ with equality) is complete. If they exist, give two elementarily non-equivalent models, and two non-equivalent complete simple extensions:
    \begin{enumerate}[(a)]
        \item  $T = \{U(f(x)),\ \neg x=y,\ x=y\vee x=z\vee y=z\}$
        \item $T = \{U(f(x)),\ \neg (\forall x)(\forall y)x=y,\ x=y\vee x=z\vee y=z\}$
        \item  $T = \{U(f(x)),\ \neg x=f(x),\ \neg (\forall x)(\forall y)x=y,\ x=y\vee x=z\vee y=z\}$
        \item  $T = \{U(f(x)),\ \neg (\forall x) x=f(x),\ \neg (\forall x)(\forall y)x=y,\ x=y\vee x=z\vee y=z\}$
    \end{enumerate}

    \begin{solution}
        \begin{enumerate}[(a)]
            \item Beware, this theory is inconsistent. Note that $\neg x=y$ is inconsistent: it is not true in any model, because it fails under the valuation $e(x)=a,e(y)=a$ for any element $a\in A$. (It is equivalent to its universal closure $(\forall x)(\forall y)\neg x=y$.) An inconsistent theory is not complete by definition, and all its extensions are also inconsistent, so it has no complete simple extension.
            \item Not complete. Informally, $T$ says that the model has exactly two elements, and the outputs of $f^\mathcal A$ must lie inside $U^\mathcal A$. From this we know $U^\mathcal A\neq\emptyset$. If it is a one-element set, we have a single model (up to isomorphism); if it is two-element, we have in total three pairwise non-isomorphic (and thus elementarily non-equivalent) models (where $f^\mathcal A$ has no fixed point, has one fixed point, or has two fixed points, i.e. is the identity):
            \begin{itemize}
                \item $\mathcal A_1=(\{0,1\};U^\mathcal A_1,f^\mathcal A_1)$ where $U^\mathcal A_1=\{0\}$ and $f^\mathcal A_1=\{(0,0),(1,0)\}$, i.e. $f^\mathcal A_1(0)=0$, $f^\mathcal A_1(1)=0$
                \item $\mathcal A_2=(\{0,1\};\{0,1\},\{(0,1),(1,0)\})$,
                \item $\mathcal A_3=(\{0,1\};\{0,1\},\{(0,0),(1,0)\})$,
                \item $\mathcal A_4=(\{0,1\};\{0,1\},\{(0,0),(1,1)\})$.
            \end{itemize} 
            (Draw the pictures!) The corresponding complete simple extensions can be written as $\mathrm{Th}(\mathcal A_i)$ for $i=1,2,3,4$. Or:
            \begin{itemize}
                \item $T_1=T\cup\{\neg (\forall x) U(x)\}$,
                \item $T_2=T\cup\{U(x),\neg f(x)=x\}$,
                \item $T_3=T\cup\{U(x),(\exists x)f(x)=x,(\exists x)\neg f(x)=x\}$,
                \item $T_4=T\cup\{U(x),f(x)=x\}$.
            \end{itemize}
            \item Similarly, it expresses that the model has exactly two elements and $f$ has no fixed point. It is complete: there is a single model up to isomorphism, namely $\mathcal A_2$.
            \item The model has exactly two elements and $f$ has at least one fixed point. It is not complete; its models up to isomorphism are $\mathcal A_3$ and $\mathcal A_4$.
            

        \end{enumerate}
                    
    \end{solution}

\end{problem}

        
        
\section*{Extra practice}


\begin{problem}

    Determine the free and bound occurrences of variables in the following formulas. Then convert them to variants in which no variable occurs both free and bound at the same time.
    \begin{enumerate}[(a)]
        \item $(\exists x)(\forall y)P(y,z) \vee (y=0)$
        \item $(\exists x)(P(x) \wedge (\forall x)Q(x)) \vee (x=0)$
        \item $(\exists x)(x>y) \wedge (\exists y)(y>x)$
    \end{enumerate}

\end{problem}


\begin{problem}
    
    Let $\varphi$ denote the formula $(\forall x)((x=z) \vee (\exists y)(f(x)=y) \vee (\forall z)(y=f(z)))$. Which of the following terms are substitutable into $\varphi$?
    \begin{enumerate}[(a)]
        \item the term $z$ for the variable $x$, the term $y$ for the variable $x$,
        \item the term $z$ for the variable $y$, the term $g(f(y),w)$ for the variable $y$,
        \item the term $x$ for the variable $z$, the term $y$ for the variable $z$,
    \end{enumerate}

\end{problem}


\begin{problem}

    Are the following sentences true / false / independent (in logic)?

    \begin{enumerate}[(a)]
        \item $(\exists x)(\forall y)(P(x) \vee \neg P(y))$
        \item $(\forall x)(P(x)\to Q(f(x))) \wedge (\forall x)P(x) \wedge (\exists x)\neg Q(x)$
        \item $(\forall x)(P(x) \vee Q(x)) \to ((\forall x)P(x) \vee (\forall x)Q(x))$
        \item $(\forall x)(P(x)\to Q(x)) \to ((\exists x)P(x)\to(\exists x)Q(x))$
        \item $(\exists x)(\forall y)P(x,y) \to (\forall y)(\exists x)P(x,y)$
    \end{enumerate}

\end{problem}


\begin{problem}
    
    Decide whether the following hold for every formula $\varphi$. Prove (semantically, from the definitions) or provide a counterexample.
    \begin{enumerate}[(a)]
       \item $\varphi \models (\forall x)\varphi$
       \item $\models \varphi \to (\forall x)\varphi$
       \item $\varphi \models (\exists x)\varphi$
       \item $\models \varphi \to (\exists x)\varphi$
    \end{enumerate}

\end{problem}


\section*{For further thought}


\begin{problem}

    Let $L=\langle +, -, 0\rangle$ be the language of group theory (with equality). The theory of groups $T$ consists of the following axioms:
    \begin{align*}
    x+(y+z)&=(x+y)+z\\
    0+x&=x=x+0\\
    x+(-x)&=0=(-x)+x
    \end{align*}
    Decide whether the following formulas are true / false / independent in $T$. Provide justification.
    \begin{enumerate}[(a)]
        \item $x+y=y+x$
        \item $x+y=x\ \rightarrow\ y=0$
        \item $x+y=0\ \rightarrow\ y=-x$
        \item $-(x+y)=(-y)+(-x)$
    \end{enumerate}

\end{problem}




