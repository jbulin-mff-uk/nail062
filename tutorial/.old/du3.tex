\documentclass{amsart}
\usepackage[utf8]{inputenc}
\usepackage{enumerate}
\usepackage{url}
\usepackage{hyperref}

\usepackage{a4wide}
\theoremstyle{definition}
\newtheorem{task}{Úkol}


\title{\sc Cvičení z logiky: Domácí úkol č. 3}
%\author{Termín odevzdání: úterý 22. prosince v 10:40}


\begin{document}

\maketitle

\thispagestyle{empty}

Úkol odevzdávejte v Moodle. Ponechte si dostatečný čas pro odevzdání, tak aby vám krátkodobé technické potíže s Moodle nezabránily úkol odevzdat. Pozdě odevzdané úkoly nebudou hodnoceny, kromě případů hodných zvláštního zřetele. Odevzdané řešení musí být vaše vlastní, není dovoleno hledat nápovědy ani řešení konzultovat s kýmkoliv kromě mne. Své odpovědi dostatečně podrobně zdůvodněte, uveďte všechny pomocné výpočty apod.



\medskip
\begin{task}
Mějme teorii $T=\{A(x)\leftrightarrow \neg B(x),A(x)\wedge A(y)\to x=y,B(x)\wedge B(y)\to x=y\}$ v jazyce $L=\langle A,B\rangle$ s rovností, kde  $A$ a $B$ jsou unární relační symboly. Označme $T'$ univerzální uzávěr teorie $T$. Nechť $\varphi$ je následující sentence:
$$
(\forall x)(\forall y)(\forall z)(x=y\vee y=z\vee x=z)
$$

Najděte tablo důkaz sentence $\varphi$ z teorie $T'$.
\end{task}


\medskip
\begin{task}
Nechť $T$ označuje teorii těles jazyka $L=\langle +,-,\cdot,0,1 \rangle$ s rovností a $\mathcal{A}=\langle\mathbb{R},+,-,\cdot,0,1 \rangle$ je (standardní) těleso reálných čísel.
\begin{enumerate}[(a)]
\item Napište formuli jazyka $L$, která v struktuře $\mathcal{A}$ definuje (bez parametrů) množinu $\{\sqrt{2}\}$.
\item Je teorie $T'=T \cup \{f(x)=y \leftrightarrow y\cdot y=x\}$ (korektní) extenzí teorie $T$ o definici funkčního symbolu $f$? Uveďte zdůvodnění.
\item Je teorie $T'$ konzervativní extenzí teorie $T$?
\end{enumerate}
\end{task}

\medskip
\begin{task}
Uvažte následující tvrzení o králících:
\begin{enumerate}[(i)]
    \item Bob a Bobek jsou králíci.
    \item Králíci spí nebo jedí mrkev.
    \item Je-li králík hladový, nemůže spát.
    \item Hladový králík jí mrkev.
\end{enumerate}

\begin{enumerate}[(a)]
\item Formalizujte tvrzení (i)--(iv) po řadě jako \underline{sentence} $\varphi_1,\varphi_2,\varphi_3,\varphi_4$ v predikátové logice v jazyce $L=\langle H, J, K, S, b_1, b_1 \rangle$ bez rovnosti, kde $H,J,K,S$ jsou unární relační symboly (označující po řadě ``být hladový'', ``jíst mrkev'', ``být králík'', ``spát'') a $b_1,b_2$ jsou konstantní symboly označující Boba a Bobka.
\item Sestrojte dokončené tablo z teorie $T=\{\varphi_1,\varphi_2,\varphi_3\}$ s položkou $F\varphi_4$ v kořeni. 
\item Je sentence $\varphi_4$ pravdivá/lživá/nezávislá v teorii $T$? Zdůvodněte. 
\item Má teorie $T$ úplnou konzervativní extenzi? Zdůvodněte. 
\item Uvažme teorii $T'=T\cup \{(\forall x)K(x),(\forall x)S(x)\}$. Kolik má teorie $T'$ dvouprvkových modelů (až na elementární ekvivalenci)? Zdůvodněte. 
\end{enumerate}
\end{task}


\medskip
\begin{task}
Nechť $T=\{(\exists x)P(x,x), (\forall x)(\exists y)R(x,y),  (\forall u)(\forall v)((\forall x)(\exists y)R(x,y) \to \neg P(u,v))\}$ je teorie jazyka $L=\langle P,R\rangle$ bez rovnosti, kde $P,R$ jsou binární relační symboly.
\begin{enumerate}[(a)]
\item Skolemizací nalezněte k $T$ otevřenou ekvisplnitelnou teorii $T'$ (nad vhodně rozšířeným jazykem). 
\item Převeďte $T'$ na ekvivalentní teorii $S$ v CNF. Zapište $S$ v množinové reprezentaci.
\item Nalezněte rezoluční zamítnutí teorie $S$. Zakreslete ho ve formě rezolučního stromu. U každého kroku uveďte použitou unifikaci.
\item Nalezněte konjunkci základních instancí axiomů $S$, která je nesplnitelná. {\it Nápověda: využijte unifikace z (c).} 
\item Je sentence $(\forall x)P(x,x)$ pravdivá / lživá / nezávislá v $T$? Uveďte zdůvodnění. 
\end{enumerate}
\end{task}



\end{document}
