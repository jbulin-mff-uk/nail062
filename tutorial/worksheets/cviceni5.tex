\documentclass[a4paper]{article}

\usepackage{a4wide}
\usepackage{amsmath}
\usepackage{amssymb}
\usepackage{amsthm}
\usepackage{enumitem}
    \setlist[enumerate]{label=(\alph*),itemsep=3pt,topsep=6pt}
    \setlist[itemize]{itemsep=3pt,topsep=6pt}
\usepackage{multicol}
\usepackage{tikz}
\usepackage[utf8]{inputenc}


\theoremstyle{definition}
\newtheorem{problem}{Příklad}
\newtheorem*{ukol}{Domácí úkol}


\begin{document}

\section*{NAIL062 V\&P Logika: 5. cvičení}


\textbf{Témata:} 
Tablo metoda ve výrokové logice. 


\medskip\begin{problem}
    Pomocí tablo metody dokažte následující výroky:
    \begin{enumerate}
    \item $(p\to (q \to q))$
    \item $p \leftrightarrow \neg \neg  p$
    \item $\neg (p \vee q) \leftrightarrow (\neg p \wedge \neg q)$
    \item $(p \to q) \leftrightarrow (\neg q \to \neg p)$    
    \end{enumerate}
\end{problem} 
   

\medskip\begin{problem}
    Pomocí tablo metody dokažte nebo najděte protipříklad ve formě \emph{kanonického} modelu pro bezespornou větev.
    \begin{enumerate}
    \item $\{ \neg q,\ p \vee q\} \models p$
    \item $\{ q \to p,\ r \to q,\ (r \to p) \to s\} \models s$
    \item $\{ p \to r,\ p \vee q,\ \neg s \to \neg q\} \models r \to s$
    \end{enumerate}
\end{problem}
  

\medskip\begin{problem}
    Pomocí tablo metody určete všechny modely následujících teorií:
    \begin{enumerate}
    \item $\{(\neg p \vee q) \to (\neg q \wedge r)\}$
    \item $\{\neg q \to (\neg p \vee q),\ \neg p \to q,\ r \to q\}$
    \item $\{ q \to p,\ r \to q,\ (r \to p) \to s\}$
    \end{enumerate}
\end{problem}

\medskip\begin{problem}
Aladin našel v jeskyni dvě truhly, A a B. Ví, že každá truhla obsahuje buď poklad, nebo smrtonosnou past.
\begin{itemize}
\item Na truhle A je nápis: {\it ``Alespoň jedna z těchto dvou truhel obsahuje poklad.''}
\item Na truhle B je nápis: {\it ``V truhle A je smrtonosná past.''}
\end{itemize}
Aladin ví, že buď jsou oba nápisy pravdivé, nebo jsou oba lživé.
\begin{enumerate}
    \item Vyjádřete Aladinovy informace jako teorii $T$ nad vhodně zvolenou množinou výrokových proměnných $\mathbb P$. (Vysvětlete význam jednotlivých výrokových proměnných v $\mathbb P$.)
    \item Pomocí tablo metody najděte všechny modely teorie $T$.
    \item Může Aladin zvolit truhlu tak, aby si byl jistý, že bude obsahovat poklad? Pokud ano, kterou?
\end{enumerate}
\end{problem}


\medskip\begin{problem} % a bit longer - don't do in class
V prezidentských volbách kandidují pan A a pan B.
\begin{itemize}
\item Pan A říká: {\it ``Budu zvolen nebo pan B lže.''}
\item {Pan B říká: {\it ``Pan A nebude zvolen nebo lžu.''}
\item Bude zvolen právě jeden z nich.
\end{itemize}
\begin{enumerate}
\item Formalizujte naše znalosti jako teorii $T$ v jazyce $\mathbb P=\{z_a,z_b,p_a,p_b\}$, kde $z_a$ resp. $z_b$ znamená, že zvolen bude pan A resp. pan B, a $p_a$ resp. $p_b$ znamená, že A resp. B mluví pravdu.
\item Sestrojte dokončená tabla z teorie $T$ s položkami $\mathrm{F}z_a$ resp. $\mathrm{F}z_b$ v kořeni. Jaký z těchto tabel můžeme učinit závěr?
\item Uveďte příklad výroku nad $\mathbb{P}$, který je v teorii $T$ nezávislý, anebo zdůvodněte, proč takový výrok neexistuje.
\item Existuje teorie $S$ nad $\{z_a,z_b\}$ taková, že $T$ je konzervativní extenzí $S$? Uveďte příklad, nebo zdůvodněte, proč ne.
\end{enumerate}
\end{problem}


\medskip\begin{problem}
    Uvažme nekonečnou výrokovou teorii (a) $T=\{p_{i+1} \to p_i\mid i\in \mathbb{N}\}$ (b) $T=\{p_i \to p_{i+1}\mid i\in \mathbb{N}\}$. Pomocí tablo metody najděte všechny modely $T$, a to tak, že sestrojíte tablo z $T$ s položkou $\mathrm{T}p_0\to p_1$ v kořeni. Je každý model $T$ kanonickým modelem pro některou z větví tohoto tabla? Pokuste se sestrojit také \emph{systematické} tablo.
\end{problem}




\medskip\begin{problem} 
    Navrhněte vhodná atomická tabla pro Peirceovu spojku $\downarrow$ (NOR), pro Shefferovu spojku $\uparrow$ (NAND), a pro $\oplus$ (XOR).
\end{problem}


\medskip\begin{problem}
Dokažte přímo (transformací tabel) větu o dedukci, tj. že pro každou teorii $T$ a výroky $\varphi$, $\psi$ platí
$$T \vdash \varphi\to \psi\text{\ \ právě když\ \ }T,\varphi \vdash \psi.$$
\end{problem}\medskip


\medskip\begin{ukol}[3 body]{\,}

\begin{enumerate}[label=\arabic*.]
\item Pomocí tablo metody:
\begin{enumerate}
    \item dokažte, že následující výrok je tautologie: 
    $$(p \to (q \to r)) \to ((p\to q)\to (p \to r))$$
    \item dokažte nebo najděte protipříklad ve formě \emph{kanonického} modelu pro bezespornou větev:
    $$\{ p \to r,\ p \vee q,\ \neg s \to \neg q\} \models r \to s$$
    \item určete všechny modely:
    $$\{ q \to p,\ r \to q,\ (r \to p) \to s\}$$
\end{enumerate}
\item Navrhněte vhodná atomická tabla pro ternární operátor ``if p then q else r'' (IFTE). Ukažte, že souhlasí-li model s kořenem vašich atomických tabel, souhlasí i s některou větví.
\end{enumerate} 
\end{ukol}

\end{document}