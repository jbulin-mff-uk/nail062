\documentclass[a4paper,12pt]{article}

%% slide-specific

\usetheme[progressbar=frametitle]{metropolis}
%\usecolortheme{spruce}
%\metroset{block=fill}

% block indentation workaround
% map defaulf block to oldblock
\let\oldblock\block
\let\endoldblock\endblock
% change block by adding smallskip
\renewenvironment{block}[1]
    {\begin{oldblock}{#1}
        \smallskip
    }
    { 
    \end{oldblock}
    }

% define Metropolis colors    
\definecolor{mAlert}{HTML}{EB811B}
\definecolor{mExample}{HTML}{14B03D}
\definecolor{mBlock}{HTML}{23373b}

\usepackage[most]{tcolorbox}

%\newcommand{\myexample}[1]{\leavevmode\textcolor{mExample}{#1}}
\newcommand{\myalert}[1]{
\begin{tcolorbox}[colback=mAlert!10, enhanced, boxrule=0pt, boxsep=-1mm, frame hidden, left=2mm, right=2mm]
    {#1}  
\end{tcolorbox}
}
\newcommand{\myexample}[1]{
\begin{tcolorbox}[colback=mExample!10, enhanced, boxrule=0pt, boxsep=-1mm, frame hidden, left=2mm, right=2mm]
    {#1}  
\end{tcolorbox}
}
\newcommand{\myblock}[1]{
\begin{tcolorbox}[colback=mBlock!10, enhanced, boxrule=0pt, boxsep=-1mm, frame hidden, left=2mm, right=2mm]
    {#1}  
\end{tcolorbox}
}

\newcommand{\mystructure}[1]{\mathcal{#1}}



% \newcommand{\myexamplemath}[1]{
% \begin{tcolorbox}[colback=mExample!10, enhanced, boxrule=0pt, frame hidden]
%     \ensuremath{#1}  
% \end{tcolorbox}
% }


%% packages
\usepackage{amsmath,amssymb,amsthm}
\usepackage{booktabs}
\usepackage[czech]{babel}
\usepackage{enumerate}
\usepackage{forest}
\usepackage{multicol}
% \usepackage{tcolorbox}
\usepackage{tikz}
    \usetikzlibrary{arrows.meta}
%\usepackage[unicode]{hyperref}
\usepackage[utf8]{inputenc}
\usepackage{xfrac}

% %% theorems
% \theoremstyle{plain}
%     \newtheorem{theorem}{Věta}[section]
%     \newtheorem*{theorem-unnumbered}{Věta}
%     \newtheorem{proposition}[theorem]{Tvrzení}
%     \newtheorem{corollary}[theorem]{Důsledek}
%     \newtheorem{lemma}[theorem]{Lemma}
%     \newtheorem{observation}[theorem]{Pozorování}
% \theoremstyle{definition}
%     \newtheorem{definition}[theorem]{Definice}
%     \newtheorem*{algorithm}{Algoritmus}
% \theoremstyle{remark}
%     \newtheorem{remark}[theorem]{Poznámka}
%     \newtheorem{example}[theorem]{Příklad}
%     \newtheorem{exercise}{Cvičení}[chapter]
%     \newtheorem*{solution}{Řešení}

%% macros and definitions
\DeclareMathOperator{\Aut}{Aut}
\DeclareMathOperator{\Conseq}{Csq}
\DeclareMathOperator{\DeLO}{DeLO}
\DeclareMathOperator{\dom}{dom}
\DeclareMathOperator{\Fm}{Fm}
\DeclareMathOperator{\M}{M}
%\DeclareMathOperator{\Proof}{Proof}
\DeclareMathOperator{\rng}{rng}
\DeclareMathOperator{\Term}{Term}
\DeclareMathOperator{\Th}{Th}
\DeclareMathOperator{\Thm}{Thm}
\DeclareMathOperator{\Tree}{Tree}
\DeclareMathOperator{\Var}{Var}
\DeclareMathOperator{\VF}{VF}

\newcommand{\A}{\structure{A}}
\newcommand{\B}{\structure{B}}
\newcommand{\Con}{\mathit{Con}}
\newcommand{\disjointunion}{\mathbin{\dot{\sqcup}}}
\newcommand{\F}{\ensuremath{\mathrm{F}}}
\newcommand{\landsymb}{{\land}}
\newcommand{\lbin}{\mathbin{\square}}
\newcommand{\lbinsymb}{{\lbin}}
\newcommand{\liff}{\mathbin{\leftrightarrow}}
\newcommand{\liffsymb}{{\liff}}
\newcommand{\limplies}{\mathbin{\rightarrow}}
\newcommand{\limpliessymb}{{\limplies}}
\newcommand{\lorsymb}{{\lor}}
\newcommand{\Prf}{\mathit{Prf}}
\newcommand{\proves}{\vdash}
%\newcommand{\structure}[1]{\mathcal{#1}}
\newcommand{\todo}{[TODO]}
\newcommand{\T}{\ensuremath{\mathrm{T}}}
\newcommand{\union}{\mathbin{\cup}}



\begin{document}

\section*{NAIL062 V\&P Logika: 13. cvičení}
% po 11. a 12. přednášce

\textbf{Témata:}
(Zápočtový test z predikátové logiky.) Vybraná témata z teorie modelů.



\medskip\begin{problem}
    Buď $T=\{(\forall x)(\exists y) S(y)=x,\ S(x)=S(y)\to x=y\}$ teorie v~jazyce $L=\langle S\rangle$ s~rovností, kde $S$ je unární funkční symbol.
    \begin{enumerate}
    \item Buď $\mathcal{R}=\langle\mathbb{R},S\rangle$, kde $S(r)=r+1$ pro $r\in\mathbb{R}$. Právě pro která $r\in\mathbb{R}$ je množina $\{r\}$ definovatelná v~$\mathcal{R}$ z~parametru $0$?
    \item Je teorie $T$ otevřeně axiomatizovatelná? Uveďte zdůvodnění.
    \item Je extenze $T'$ teorie $T$ o~axiom $S(x)=x$ $\omega$-kategorická teorie? Je $T'$ kompletní?
    \item Pro která $0<n\in\mathbb{N}$ existuje $L$-struktura $\mathcal{B}$ velikosti $n$ elementárně ekvivalentní s~$\mathcal{R}$? Existuje spočetná struktura $\mathcal{B}$ elementárně ekvivalentní s~$\mathcal{R}$?
    \end{enumerate}
\end{problem}


\medskip\begin{problem}
Uvažme následující graf:

\begin{center}
    \begin{tikzpicture}[every node/.style={circle,fill=blue!10,draw,minimum size=0.5cm,node distance=1.5cm}]
        \node (1) {$1$};
        \node[right of=1] (2) {$2$};
        \node[below of=2] (3) {$3$};
        \node[left of=3] (4) {$4$};
        \path[draw] (1) -- (2) -- (3) -- (4) -- (1) -- (3);
        \node[right of=3] (5) {$5$};
        \path[draw] (2) -- (5) -- (3);
    \end{tikzpicture}
\end{center}

    \begin{enumerate}
        \item Najděte všechny automorfismy.
        \item Které podmnožiny množiny vrcholů $V$ jsou definovatelné? Uveďte definující formule. {\it (Nápověda: Využijte (a).)}
        \item Které binární relace na $V$ jsou definovatelné?
    \end{enumerate} 





\end{problem}



\medskip\begin{problem}
    Nechť $T = \{U(x) \to U(f(x)), (\exists x)U(x), \neg (f(x) = x), \varphi\}$ je teorie v jazyce $L = \langle U, f \rangle$ s rovností, kde $U$ je unární relační symbol, $f$ je unární funkční symbol a $\varphi$ vyjadřuje, že ``existují maximálně 4 prvky''.
    \begin{enumerate}
    \item Je teorie $T$ extenzí teorie $S = \{ (\exists x)(\exists y)(\neg x = y \land U(x) \land U(y)), \varphi \}$ v jazyce $L' = \langle U \rangle$? Je konzervativní extenzí? Zdůvodněte.
    \item Je teorie $T$ otevřeně axiomatizovatelná? Zdůvodněte.
    \end{enumerate} 
\end{problem}

\medskip\begin{problem}
Nechť $T=\{\varphi\}$ je teorie jazyka $L=\langle U, c \rangle$ s rovností, kde $U$ je unární relační symbol, $c$ je konstantní symbol a axiom $\varphi$ vyjadřuje \emph{``Existuje alespoň $5$ prvků, pro které platí U(x).''}
\begin{enumerate}
\item Nalezněte dvě neekvivalentní jednoduché kompletní extenze teorie $T$ nebo zdůvodněte, proč neexistují.
\item Je teorie $T$ otevřeně axiomatizovatelná? Uveďte zdůvodnění.
\end{enumerate}
\end{problem}


\medskip\begin{problem} %move to tutorial 8?
    Buď $T=\{(\forall x)(\exists y) S(y)=x,\ S(x)=S(y)\to x=y\}$ teorie v~jazyce $L=\langle S\rangle$ s~rovností, kde $S$ je unární funkční symbol.
    \begin{enumerate}
    \item Nalezněte extenzi $T'$ teorie $T$ o definici nového unárního funkčního symbolu $P$ takovou, že $T' \models S(S(x))=y \leftrightarrow P(P(y))=x$. {\it (2b)}
    \item Je teorie $T'$ otevřeně axiomatizovatelná? Uveďte zdůvodnění. {\it (2b)}
    \end{enumerate} 
\end{problem} 


\medskip\begin{problem}
    Nechť $T$ je extenze teorie $DeLO^-$ (tj. hustých lineárních uspořádání s minimálním prvkem a bez maximálního prvku) o nový axiom $c \le d$ v jazyce $L=\langle \le,c,d\rangle$ s rovností, kde $c$, $d$ jsou nové konstantní symboly.
    \begin{enumerate}
    \item Jsou sentence $(\exists x)(x\le d \wedge x \ne d)$ a $(\forall x)(x \le d)$ pravdivé / lživé / nezávislé v $T$? Uveďte zdůvodnění.
    \item Napište dvě neekvivalentní jednoduché kompletní extenze teorie $T$.
    \end{enumerate} 
\end{problem}


\end{document}