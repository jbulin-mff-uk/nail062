\documentclass{beamer}

\input{slides-header.tex}

\title{Desátá přednáška}
\subtitle{NAIL062 Výroková a predikátová logika}
\author{Jakub Bulín (KTIML MFF UK)}
% \institute{KTIML MFF UK}
\date{Zimní semestr 2024}


\begin{document}


\maketitle


\begin{frame}{Desátá přednáška}

    \textbf{Program}
        \begin{itemize}            
            \item unifikace, unifikační algoritmus
            \item rezoluční pravidlo, rezoluční důkaz
            \item korektnost rezoluce
            \item lifting lemma a úplnost rezoluce
        \end{itemize}    

    \textbf{Materiály}

        \href{https://github.com/jbulin-mff-uk/nail062/raw/main/lecture/lecture-notes/lecture-notes.pdf}{\alert{\textbf{Zápisky z přednášky}}}, Sekce 8.4-8.6 z Kapitoly 8

\end{frame}


\section{8.4 Unifikace}


\begin{frame}{Příklady substitucí}

    \pause
    Místo \alert{všech základních} použijeme \alert{`vhodné'} substituce (\alert{unifikace}):

    \pause
    \myexampleinline{
        1. $\{P(x),Q(x,a)\}$ a $\{\neg P(y),\neg Q(b,y)\}$
        } \vspace{-6pt}
        \pause
        \begin{itemize}
            \item substitucí \myalertinline{
            $\{x/b,y/a\}$
            } získáme $\{P(b),Q(b,a)\}$ a $\{\neg P(a),\neg Q(b,a)\}$, z nich rezolucí $\{P(b),\neg P(a)\}$\pause 
            \item nebo \myalertinline{
            $\{x/y\}$
            } a rezolucí přes $P(y)$ máme $\{Q(y,a),\neg Q(b,y)\}$\pause 
            \item šlo by např. \myalertinline{
                $\{x/a\}$
            }, získat $\{Q(a,a),\neg Q(b,a)\}$, ale to je \alert{horší}   
        \end{itemize}         
      
    \pause
    \myexampleinline{
        2. $\{P(x),Q(x,z)\}$ a $\{\neg P(y),\neg Q(f(y),y)\}$
    } \vspace{-6pt}
        \pause 
        \begin{itemize}
            \item lze použít \myalertinline{
                $\{x/f(a),y/a,z/a\}$
                }, získat $\{P(f(a)),Q(f(a),a)\}$ a $\{\neg P(a),\neg Q(f(a),a)\}$, rezolucí \alert{$\{P(f(a)),\neg P(a)\}$}\pause 
            \item \alert{lepší} je \myalertinline{
                $\{x/f(z),y/z\}$
            }, dává $\{P(f(z)),Q(f(z),z)\}$ a $\{\neg P(z),\neg Q(f(z),z)\}$, rezolventu \alert{$\{P(f(z)),\neg P(z)\}$}\pause 
            \item proč lepší? \alert{obecnější}, rezolventa `říká více': {\small$\{P(f(a)),\neg P(a)\}$} je důsledkem {\small$\{P(f(z)),\neg P(z)\}$}, ale nejsou ekvivalentní\pause 
            \item $\{x/f(a),y/a,z/a\}$ získáme \alert{složením} $\{x/f(z),y/z\}$ a $\{z/a\}$
        \end{itemize}            

\end{frame}


\begin{frame}{Substituce formálně}

    \pause 
    \myblock{
    \begin{itemize}
        \item \alert{substituce} je konečná množina \alert{$\sigma=\{x_1/t_1,\dots,x_n/t_n\}$}, kde $x_i$ jsou navzájem různé proměnné, $t_i$ jsou termy, $t_i$ není $x_i$\pause 
        \begin{itemize}
            \item \alert{základní}: všechny termy $t_i$ jsou konstantní\pause 
            \item \alert{přejmenování proměnných}: vš. $t_i$ navzájem různé proměnné\pause 
        \end{itemize}
        \item \alert{výraz} je term nebo literál (atomická formule nebo její negace)\pause 
        \item \alert{instance} výrazu $E$ \alert{při substituci} $\sigma=\{x_1/t_1,\dots,x_n/t_n\}$, \alert{$E\sigma$}: simultánně nahradíme  všechny výskyty $x_i$ za termy $t_i$\pause 
        \item pro množinu výrazů $S$ je \alert{$S\sigma=\{E\sigma\mid E\in S\}$}\pause 
    \end{itemize}
    }

    \bigskip
    
    \begin{itemize}
        \item simultánně proto, aby výskyt $x_i$ v termu $t_j$ nevedl ke zřetězení\pause 
        \item např. \myexampleinline{
            $S=\{P(x),R(y,z)\}$, $\sigma=\{x/f(y,z),y/x,z/c\}$
        }\pause
        $$
        S\sigma=\{P(f(y,z)),R(x,c)\}
        $$
    \end{itemize}

\end{frame}


\begin{frame}{Skládání substitucí}
    
    \pause 
    \begin{itemize}
        \item substituce lze \alert{skládat}, \alert{$\sigma\tau$} znamená nejprve $\sigma$ a potom~$\tau$\pause 
        \item chceme, aby platilo \alert{$E(\sigma\tau)=(E\sigma)\tau$}, pro libovolný výraz $E$\pause 
        \item např. pro výraz \myexampleinline{
            $E=P(x,w,u)$
        } a substituce
        $$
        \sigma=\{x/f(y),w/v\}\qquad\tau=\{x/a,y/g(x),v/w,u/c\}
        $$
        máme $E\sigma=P(f(y),v,u)$ a $(E\sigma)\tau=P(f(g(x)),w,c)$, takže:
        $$
        \sigma\tau=\{x/f(g(x)),y/g(x),v/w,u/c\}
        $$
        \item \pause skládání není komutativní, $\sigma\tau$ je (typicky) jiná než $\tau\sigma$, zde
        $$
        \tau\sigma=\{x/a,y/g(f(y)),u/c,w/v\}
        $$ 
        \item \pause ale je asociativní (takže nemusíme psát závorky v $\sigma_1\sigma_2\cdots\sigma_n$)\pause 
    \end{itemize}

    \myblock{
        Buď $\sigma=\{x_1/t_1,\dots,x_n/t_n\}$ a $\tau=\{y_1/s_1,\dots,y_m/s_m\}$, označme $X=\{x_1,\dots,x_n\}$ a $Y=\{y_1,\dots,y_m\}$. \alert{Složení $\sigma$ a $\tau$} je substituce
        \vspace{-6pt}
        $$
        \sigma\tau=\{x_i/t_i\tau\mid x_i\in X,x_i\neq t_i\tau\}\cup\{y_j/s_j\mid y_j\in Y\setminus X\}
        $$        
    }

\end{frame}


\begin{frame}{Vlastnosti skládání}

    \myblock{
        \textbf{Tvrzení:} Pro libovolné substituce $\sigma$, $\tau$, $\varrho$ a výraz $E$ platí:\pause 
        \vspace{-6pt}
        \begin{center}
            (i)  $(E\sigma)\tau=E(\sigma\tau)$ \qquad\pause
            (ii) $(\sigma\tau)\varrho=\sigma(\tau\varrho)$\pause
        \end{center}
        
    }

    \medskip

    \textbf{Důkaz:} \alert{(i)} Buď $\sigma=\{x_1/t_1,\dots,x_n/t_n\}$ a $\tau=\{y_1/s_1,\dots,y_m/s_m\}$. Stačí pro $E$ proměnnou (substituce nemění ostatní symboly):\pause
    \begin{itemize}
        \item pro $E=x_i$ je $E\sigma=t_i$ a $(E\sigma)\tau=t_i\tau=E(\sigma\tau)$\pause
        \item pro $E=y_j\notin X$ je $E\sigma=E$ a $(E\sigma)\tau=E\tau=s_j=E(\sigma\tau)$\pause
        \item je-li $E$ jiná proměnná, potom $(E\sigma)\tau=E=E(\sigma\tau)$.\pause
    \end{itemize}
    \alert{(i)} opakovaným užitím (i) máme pro lib. výraz, tedy i proměnnou:\pause
    $$
    E((\sigma\tau)\varrho)=(E(\sigma\tau))\varrho=((E\sigma)\tau)\varrho=(E\sigma)(\tau\varrho)=E(\sigma(\tau\varrho))
    $$\pause
    Z toho plyne, že $(\sigma\tau)\varrho$ a $\sigma(\tau\varrho)$ jsou touž substitucí. \pause
    
    (Podrobněji, zřejmě platí: $\pi=\{z_1/v_1,\dots,z_k/v_k\}$ právě když $z_i\pi=v_i$ a $E\pi=E$ je-li $E$ proměnná různá od všech $z_i$.) \hfill\qedsymbol

\end{frame}


\begin{frame}{Unifikace}
    
    \pause
    \myblock{
    \begin{itemize}
        \item \alert{unifikace} pro $S=\{E_1,\dots,E_n\}$ je substituce $\sigma$ taková, že $E_1\sigma=E_2\sigma=\cdots =E_n\sigma$, tj.  $S\sigma$ obsahuje jediný výraz\pause
        \item pokud má $S$ unifikaci, je \alert{unifikovatelná}\pause
        \item unifikace pro $S$ je \alert{nejobecnější}, pokud pro každou unifikaci $\tau$ pro $S$ existuje substituce $\lambda$ taková, že $\tau=\sigma\lambda$\pause        
    \end{itemize}
    }
    
    NB: různé nejobecnějších unifikace pro $S$ se liší jen přejmenováním proměnných\pause
    
    Např. pro \myexampleinline{
        $S=\{P(f(x),y),P(f(a),w)\}$
    }\pause

    \begin{itemize}
        \item $\sigma=\{x/a,y/w\}$ je nejobecnější unifikace\pause
        \item $\tau=\{x/a,y/b,w/b\}$ je unifikace, ale není nejobecnější, nelze z ní získat např. unifikaci $\varrho=\{x/a, y/c, w/c\}$\pause
        \item z nejobecnější unifikace $\sigma$ získáme $\tau=\sigma\lambda$ pro $\lambda=\{w/b\}$
    \end{itemize}

\end{frame}


\begin{frame}{Unifikační algoritmus}
   
    \pause
    \vspace{-6pt}
    \begin{itemize}
        \item postupně od začátku výrazů aplikuje substituce %tak, aby se výrazy stávaly více podobnými
        \pause
        \item buď $p$ nejlevější pozice, na které se nějaké dva výrazy z $S$ liší\pause
        \item \alert{$D(S)$} je množina všech podvýrazů začínajících na pozici $p$\pause
        \item \myexampleinline{\small
            $S=\{P(x,y),P(f(x),z),P(z,f(x))\}, p=3, D(S)=\{x,f(x),z\}$
        }  
    \end{itemize}
    
    \medskip

    \pause
    \myblock{
    \textbf{vstup}: konečná množina výrazů $S\neq\emptyset$\\ \pause
    \textbf{výstup:} nejobecnější unifikace $\sigma$ nebo info, že není unifikovatelná
    \pause

    \begin{enumerate}[(1)]\setcounter{enumi}{-1}
        \item nastav $S_0:=S$, $\sigma_0:=\emptyset$, $k:=0$\pause
        \item pokud $|S_k|=1$, vrať $\sigma=\sigma_0\sigma_1\cdots \sigma_k$\pause
        \item zjisti, zda je v $D(S_k)$ proměnná $x$ a term $t$ \alert{neobsahující} $x$\pause
        \item pokud ano, nastav $\sigma_{k+1}:=\{x/t\}$, $S_{k+1}:=S_k\sigma_{k+1}$, $k:=k+1$, a jdi na (1)\pause
        \item pokud ne, odpověz, že $S$ není unifikovatelná\pause
    \end{enumerate}
    }

    NB: hledání $x$ a $t$ v kroku (2) je relativně výpočetně náročné

\end{frame}


\begin{frame}{Ukázkový běh}

    \pause
    \vspace{-6pt}
    \myexampleinline{\small
    $S=S_0=\{P(f(y,g(z)),h(b)),\ P(f(h(w),g(a)),t),\ P(f(h(b),g(z)),y)\}$
    }

    \vspace{-1pt}
    
    \pause
    \alert{($k=0$)} $|S_0|>1$, $D(S_0)=\{y,h(w),h(b)\}$, proměnná $y$ není v $h(w)$, nastavíme $\sigma_1:=\{y/h(w)\}$ a $S_1=S_0\sigma_1$
    
    \pause
    \myexampleinline{\small
    $S_1=\{P(f(h(w),g(z)),h(b)),\ P(f(h(w),g(a)),t),\ P(f(h(b),g(z)),h(w))\}$
    }

    \vspace{-1pt}

    \pause
    \alert{($k=1$)} $D(S_1)=\{w,b\}$, $\sigma_2=\{w/b\}$, $S_2=S_1\sigma_2$    
    
    \pause
    \myexampleinline{\small
    $S_2=\{P(f(h(b),g(z)),h(b)),\ P(f(h(b),g(a)),t)\}$
    }

    \vspace{-1pt}
    \pause
    \alert{($k=2$)} $D(S_2)=\{z,a\}$, $\sigma_3=\{z/a\}$, $S_3=S_2\sigma_3$    
    
    \pause
    \myexampleinline{\small
    $S_3=\{P(f(h(b),g(a)),h(b)),\ P(f(h(b),g(a)),t)\}$
    }

    \vspace{-1pt}

    \pause
    \alert{($k=3$)} $D(S_3)=\{h(b),t\}$, $\sigma_4=\{t/h(b)\}$, $S_4=S_3\sigma_4$    
    
    \pause
    \myexampleinline{\small
    $S_4=\{P(f(h(b),g(a)),h(b))\}$
    }

    \vspace{-1pt}

    \pause
    \alert{($k=4$)} $|S_4|=1$, nejobecnější unifikace pro $S$ je
    $\sigma=\sigma_1\sigma_2\sigma_3\sigma_4=$\\    
    $\{y/h(w)\}\{w/b\}\{z/a\}\{t/h(b)\}=$\myalertinline{
        $\{y/h(b),w/b,z/a,t/h(b)\}$
    }    

\end{frame}


\begin{frame}{Důkaz korektnosti}
    
    \pause
    \myblock{
    \textbf{Tvrzení:}
    Unifikační algoritmus je korektní. Pro sestrojenou $\sigma$ navíc platí, že je-li $\tau$ libovolná unifikace, potom $\tau=\sigma\tau$.
    }
    
    \pause
    \textbf{Důkaz:} Algoritmus vždy skončí, neboť v každém kroku eliminuje proměnnou. Skončí-li neúspěchem, nelze unifikovat $S_k$, tedy ani $S$.
    
    \pause
    Odpoví-li $\sigma=\sigma_0\sigma_1\cdots\sigma_k$, zjevně jde o unifikaci. Zbývá dokázat, že je nejobecnější, k tomu stačí dokázat vlastnost `navíc': \pause Buď $\tau$ lib. unifikace pro $S$. Indukcí pro $0\leq i\leq k$ ukážeme \alert{$\tau=\sigma_0\sigma_1\cdots\sigma_i\tau$}\pause
    
    \alert{(báze indukce)} Pro $i=0$ je $\sigma_0=\emptyset$, $\tau=\sigma_0\tau$ tedy platí triviálně.\pause

    \alert{(indukční krok)} Buď $\sigma_{i+1}=\{x/t\}$. Ukažme, že pro lib. proměnnou platí: \alert{$u\sigma_{i+1}\tau=u\tau$} \pause Z toho okamžitě plyne i $\tau=\sigma_0\sigma_1\cdots\sigma_i\sigma_{i+1}\tau$.\pause
    
    Pro \alert{$u\neq x$} je $u\sigma_{i+1}=u$, tedy i $u\sigma_{i+1}\tau=u\tau$. Je-li \alert{$u=x$}, máme $u\sigma_{i+1}=x\sigma_{i+1}=t$. Protože $\tau$ unifikuje $S_i=S\sigma_0\sigma_1\cdots\sigma_i$ a $x,t\in D(S_i)$, $\tau$ unifikuje i $x$ a $t$, tzn. $t\tau=x\tau$, tj. $u\sigma_{i+1}\tau=u\tau$.\hfill\qedsymbol

\end{frame}


\section{8.5 Rezoluční metoda}


\begin{frame}{Příklad rezolučního kroku}

    \pause
    Chceme-li ukázat $T\models\varphi$, skolemizací najdeme CNF formuli $S$ ekvisplnitelnou s $T\cup\{\neg\varphi\}$. Stačí najít rezoluční zamítnutí $S$.

    \pause
    Jediným podstatným rozdílem bude \alert{rezoluční pravidlo}. 

    \pause
    Rezolventou dvojice klauzulí bude klauzule, kterou lze odvodit aplikací \alert{(nejobecnější) unifikace}. Nejprve příklad:


    \pause
    \myexampleinline{
        $C_1=\{P(x),Q(x,y),Q(x,f(z))\}, C_2=\{\neg P(u),\neg Q(f(u),u)\}$
    } 

    \pause
    Vyberme z $C_1$ \alert{oba} pozitivní literály začínající $Q$, z $C_2$ negativní. 

    \pause
    $S=\{Q(x,y),Q(x,f(z)),Q(f(u),u)\}$ lze unifikovat pomocí nejobecnější unifikace \myalertinline{
        $\sigma=\{x/f(f(z)),y/f(z),u/f(z)\}$
    }

    \pause
    \begin{itemize}
        \item $C_1\sigma=\{P(f(f(z))),Q(f(f(z)),f(z))\}$\pause
        \item $C_2\sigma=\{\neg P(f(z)),\neg Q(f(f(z)),f(z))\}$
    \end{itemize}

    \pause
    z nich odvodíme rezolventu $C=\{P(f(f(z))),\neg P(f(z))\}$
    
\end{frame}


\begin{frame}{Rezoluční pravidlo}

    \pause
    \myblock{
    Mějme klauzule $C_1$ a $C_2$ s disjunktními množinami proměnných tvaru
    $$
    C_1=C_1'\sqcup \{A_1,\dots,A_n\},\quad C_2=C_2'\sqcup \{\neg B_1,\dots,\neg B_m\}
    $$
    \pause
    kde $n,m\ge 1$ a $S=\{A_1,\dots,A_n,B_1,\dots,B_m\}$ lze unifikovat. \pause Buď $\sigma$ nejobecnější unifikace $S$. \pause \alert{Rezolventa} $C_1$ a $C_2$ je potom klauzule
    $$
    C=C_1'\sigma \cup C_2'\sigma
    $$
    }

    \medskip

    \pause
    \begin{itemize}
        \item Disjunktní množ. proměnných získáme přejmenováním. Proč? Z~\myexampleinline{
            $\{\{P(x)\},\{\neg P(f(x))\}\}$
        } odvodíme $\square$, nahradíme-li $\{P(x)\}$ klauzulí $\{P(y)\}$. Ale $S=\{P(x),P(f(x))\}$ není unifikovatelná.\pause
    
        \item Proč potřebujeme z klauzule odstranit více literálů najednou? \myexampleinline{
            $S=\{\{P(x),P(y)\},\{\neg P(x),\neg P(y)\}\}$
        } je zamítnutelná, ale nemá zamítnutí, které by v každém kroku odstranilo jen jeden.  
    \end{itemize}
        
\end{frame}


\begin{frame}{Rezoluční důkaz}

    \pause
    \myblock{
    \alert{Rezoluční důkaz (odvození)} klauzule $C$ z formule $S$ je konečná posloupnost klauzulí $C_0,C_1,\dots,C_n=C$ taková, že pro každé $i$ je buď
    \vspace{-12pt}\pause
    \begin{itemize}
        \item $C_i=C_i'\sigma$ pro nějakou 
            \myalertinline{
                $C'_i\in S$ a přejmenování proměnných $\sigma$
            }\pause 
        \item nebo $C_i$ je rezolventou nějakých $C_j,C_k$ kde $j<i$ a $k<i$.\pause
    \end{itemize}
    Existuje-li, je $C$ \alert{rezolucí dokazatelná} z $S$, \alert{$S\proves_R C$}. \alert{(Rezoluční) zamítnutí}  $S$ je rez. důkaz $\square$ z $S$, potom je $S$ \alert{(rezolucí) zamítnutelná}.
    }

\end{frame}


\begin{frame}{Příklad}

    \pause
    \myexampleamsmath{\small
    \smallskip
    \begin{align*}
        S=\{&\{\neg P(x,y),\neg P(y,z), P(x,z)\},\{\neg P(x,x)\},\\
        & \{\neg P(x,y),P(y,x)\},\{P(x,f(x))\}\}
    \end{align*} 
    }

    \pause
    rezoluční zamítnutí:
    
    \pause
    \myalertamsmath{\small
    \smallskip
    \begin{align*}
        &\{\neg P(x,y),\neg P(y,z), P(x,z)\},\ 
        \{P(x',f(x'))\},\ 
        \textcolor{blue}{\{\neg P(f(x),z),P(x,z)\}},\\
        &\{\neg P(x,y),P(y,x)\},\ 
        \textcolor{blue}{\{P(f(x'),x')\}},\ 
        \textcolor{blue}{\{P(x,x)\}},\ 
        \{P(x',x')\},\ 
        \textcolor{blue}{\square}  
    \end{align*}
    }

    \medskip

    \pause
    rezoluční strom:

    \pause
    \medskip

    \scalebox{0.85}{
    \hspace{-1cm}\begin{forest}
        for tree={l=1.5cm, grow=north}
        [{$ \square $}, label=left:{\footnotesize\textcolor{blue}{$x'/x$}}
            [{$ \{\neg P(x',x')\} $}]
            [{$ \{P(x,x)\} $}, label=left:{\footnotesize\textcolor{blue}{$z/x,x'/x$}}
                [{$ \{P(f(x'),x')\} $}, label=right:{\footnotesize\textcolor{blue}{$x/x',y/f(x')$}}
                    [{$ \{P(x',f(x'))\} $}]
                    [{$ \{\neg P(x,y),P(y,x)\} $}]            
                ]
                [{$ \{\neg P(f(x),z),P(x,z)\} $}, label=left:{\footnotesize\textcolor{blue}{$y/f(x'),x'/x$}}
                    [{$ \{P(x',f(x'))\} $}]
                    [{$ \{\neg P(x,y),\neg P(y,z), P(x,z) \} $}]                
                ]
            ]
        ]
    \end{forest}
    }    

\end{frame}


\section{8.6 Korektnost a úplnost}


\begin{frame}{Korektnost rezolučního kroku}

    \pause
    \myblock{
    \textbf{Tvrzení:}
    Mějme klauzule $C_1$, $C_2$ a jejich rezolventu $C$. Platí-li v~nějaké struktuře $\A$ klauzule $C_1$ a $C_2$, potom v ní platí i $C$.
    }

    \pause
    \textbf{Důkaz:} Buď $C_1=C_1'\sqcup \{A_1,\dots,A_n\}$, $C_2=C_2'\sqcup \{\neg B_1,\dots,\neg B_m\}$, a $C=C_1'\sigma \cup C_2'\sigma$, kde $S\sigma=\{A_1\sigma\}$ (a $\sigma$ je
    nejobecnější). \pause Klauzule jsou otevřené formule, proto platí i jejich instance:     
    $$
    \A\models C_1\sigma\ \text{ a }\ \A\models C_2\sigma
    $$     
    \pause Po aplikaci unifikace máme: \pause
    \begin{align*}        
        C_1\sigma&=C_1'\sigma \cup \{A_1\sigma\} \\ 
        C_2\sigma&=C_2'\sigma \cup \{\neg A_1\sigma\}      
    \end{align*}
    \pause Chceme ukázat, že $\A\models C[e]$ pro lib. ohodnocení $e$.
    \begin{itemize}
        \item \pause Je-li \alert{$\A\models A_1\sigma[e]$}, potom $\A\not\models\neg A_1\sigma[e]$ a musí \alert{$\A\models C_2'\sigma[e]$}. Tedy i $\A\models C[e]$. \pause
        \item Je-li \alert{$\A\not\models A_1\sigma[e]$}, musí být \alert{$\A\models C_1'\sigma[e]$} a opět $\A\models C[e]$. \hfill\qedsymbol
    \end{itemize}    

\end{frame}


\begin{frame}{Korektnost rezoluce}

    \pause
    \myblock{
    \textbf{Věta (O korektnosti rezoluce):}
    Pokud je CNF formule $S$ rezolucí zamítnutelná, potom je nesplnitelná.
    }

    \medskip

    \pause
    \textbf{Důkaz:}
        Víme, že $S\proves_R\square$, vezměme tedy nějaký rezoluční důkaz $\square$ z $S$. \pause Kdyby existoval model $\A\models S$, díky korektnosti rezolučního pravidla bychom dokázali (indukcí podle délky důkazu) i $\A\models\square$, \pause což ale není možné. \hfill\qedsymbol

\end{frame}


\begin{frame}{Lifting lemma}

    \pause
    úplnost rezoluce dokážeme převedením na případ výrokové logiky: rezoluční důkaz `na úrovni VL' je možné `zvednout' na úroveň PL

    \medskip

    \pause
    \myblock{
    \textbf{Lifting lemma:}
        Buďte $C_1$ a $C_2$ klauzule s disj. množ. proměnných, $C^*_1$ a $C^*_2$ jejich základní instance, $C^*$ rezolventa $C^*_1$ a $C^*_2$. Potom $C_1$ a $C_2$ mají rezolventu $C$ takovou, že $C^*$ je základní instance $C$.
    }

    \pause
    (důkaz na příštím slidu)

    \bigskip

    \pause
    \myblock{
    \textbf{Důsledek:}
        Buď $S$ CNF formule a označme $S^*$ množinu všech jejích základních instancí. Pokud $S^*\proves_R C^*$ pro nějakou základní klauzuli $C^*$ (`na úrovni VL'), potom existuje klauzule $C$ a základní substituce $\sigma$ taková, že $C^*=C\sigma$ a $S\proves_R C$ (`na úrovni PL').
    }
    
    \pause
    \textbf{Důkaz:} Snadno z Lifting lemmatu indukcí dle délky důkazu.\hfill\qedsymbol

\end{frame}


\begin{frame}{Důkaz Lifting lemmatu}

    \pause
    Nechť \alert{$C^*_1=C_1\tau_1$} a \alert{$C^*_2=C_2\tau_2$}, $\tau_1$ a $\tau_2$ zákl. substituce nesdílející žádnou proměnnou. Najdeme rezolventu $C$, že \alert{$C^*=C\tau_1\tau_2$}.

    \pause
    Buď $C^*$ rezolventa $C_1^*$ a $C_2^*$ přes literál $P(t_1,\dots,t_k)$. Víme, že:\pause
    \begin{align*}
        C_1&=C_1' \sqcup \{A_1,\dots,A_n\},\text{ kde }\{A_1,\dots,A_n\}\tau_1=\{P(t_1,\dots,t_k)\}\\
        C_2&=C_2' \sqcup \{\neg B_1,\dots,\neg B_m\},\{\neg B_1,\dots,\neg B_m\}\tau_2=\{\neg P(t_1,\dots,t_k)\}
    \end{align*}
    \pause Tedy $(\tau_1\tau_2)$ unifikuje $S=\{A_1,\dots,A_n,B_1,\dots,B_m\}$. Buď $\sigma$ nejob. unifikace pro $S$ z Unifikačního algoritmu. Zvolme \alert{$C=C_1'\sigma \cup C_2'\sigma$}.

    \vspace{-24pt}
    
    \pause
    \begin{align*}
        &\alert{C\tau_1\tau_2
        =} (C_1'\sigma \cup C_2'\sigma)\tau_1\tau_2
        =C_1'\sigma\tau_1\tau_2 \cup C_2'\sigma\tau_1\tau_2
        \textcolor{red}{=}C_1'\tau_1\tau_2 \cup C_2'\tau_1\tau_2\\ &
        \textcolor{blue}{=}C_1'\tau_1 \cup C_2'\tau_2
        =(C_1\setminus\{A_1,\dots,A_n\})\tau_1\cup (C_2\setminus\{\neg B_1,\dots,\neg B_m\})\tau_2\\
        &=(C_1^*\setminus\{P(t_1,\dots,t_k)\})\cup(C_2^*\setminus \{\neg P(t_1,\dots,t_k)\})\alert{=C^*}
    \end{align*}
    
    \pause
    Zde \textcolor{red}{=} plyne z vlastnosti `navíc' Unif. algoritmu $(\tau_1\tau_2)=\sigma(\tau_1\tau_2)$, \\a \textcolor{blue}{=} z 
    toho, že jde o základní substituce nesdílející proměnnou.\hfill\qedsymbol    

\end{frame}


\begin{frame}{Úplnost rezoluce}

    \pause
    \myblock{
    \textbf{Věta (O úplnosti rezoluce):}
        Je-li CNF formule $S$ nesplnitelná, potom je zamítnutelná rezolucí.
    }

    \medskip

    \pause
    \textbf{Důkaz:} \pause
    Množina $S^*$ všech základních instancí klauzulí z $S$ je také nesplnitelná (důsledek Herbrandovy věty). \pause Úplnost \alert{výrokové} rezoluce dává $S^*\proves_R\square$ (`na úrovni VL').\pause 
    
    Z důsledku Lifting lemmatu dostáváme klauzuli $C$ a základní substituci $\sigma$ takové, že $C\sigma=\square$ a $S\proves_R C$ (`na úrovni PL'). \pause
    
    Ale protože prázdná klauzule $\square$ je instancí $C$, musí být $C=\square$. Tím jsme našli rezoluční zamítnutí $S\proves_R \square$.        
    \hfill\qedsymbol

\end{frame}


\end{document}
