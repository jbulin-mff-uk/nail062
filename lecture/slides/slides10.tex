\documentclass{beamer}

%% slide-specific

\usetheme[progressbar=frametitle]{metropolis}
%\usecolortheme{spruce}
%\metroset{block=fill}

% define Metropolis colors    
\definecolor{mAlert}{HTML}{EB811B}
\definecolor{mExample}{HTML}{14B03D}
\definecolor{mBlock}{HTML}{23373b}

% my blocks
\setlength\fboxsep{0pt}%

\newcommand{\myblock}[1]{\colorbox{mBlock!12}{\begin{minipage}{\linewidth}#1\end{minipage}}}
\newcommand{\myblockmath}[1]{\colorbox{mBlock!12}{\begin{minipage}{\linewidth}\vspace{-6pt}#1\end{minipage}}}
\newcommand{\myblockamsmath}[1]{\colorbox{mBlock!12}{\begin{minipage}{\linewidth}\vspace{-6pt}#1\end{minipage}}}
\newcommand{\myblockinline}[1]{\colorbox{mBlock!12}{#1}}
\newcommand{\myexample}[1]{\colorbox{mExample!12}{\begin{minipage}{\linewidth}#1\end{minipage}}}
\newcommand{\myexamplemath}[1]{\colorbox{mExample!12}{\begin{minipage}{\linewidth}\vspace{-6pt}#1\end{minipage}}}
\newcommand{\myexampleamsmath}[1]{\colorbox{mExample!12}{\begin{minipage}{\linewidth}\vspace{-18pt}#1\end{minipage}}}
\newcommand{\myexampleinline}[1]{\colorbox{mExample!12}{#1}}
\newcommand{\myalert}[1]{\colorbox{mAlert!12}{\begin{minipage}{\linewidth}#1\end{minipage}}}
\newcommand{\myalertmath}[1]{\colorbox{mAlert!12}{\begin{minipage}{\linewidth}\vspace{-6pt}#1\end{minipage}}}
\newcommand{\myalertamsmath}[1]{\colorbox{mAlert!12}{\begin{minipage}{\linewidth}\vspace{-18pt}#1\end{minipage}}}
\newcommand{\myalertinline}[1]{\colorbox{mAlert!12}{#1}}

%% other
\newcommand{\mystructure}[1]{\mathcal{#1}}


%% packages
\usepackage{amsmath,amssymb,amsthm}
\usepackage{bookmark}
\usepackage{booktabs}
\usepackage{cancel}
\usepackage[czech]{babel}
\usepackage{enumerate}
\usepackage[T1]{fontenc}
\usepackage{forest}
\usepackage{lmodern}
\usepackage{multicol}
% \usepackage{tcolorbox}
\usepackage{tikz}
    \usetikzlibrary{arrows.meta}
%\usepackage[unicode]{hyperref}
\usepackage[utf8x]{inputenc}
\usepackage{xfrac}


% %% theorems
% \theoremstyle{plain}
%     \newtheorem{theorem}{Věta}[section]
%     \newtheorem*{theorem-unnumbered}{Věta}
%     \newtheorem{proposition}[theorem]{Tvrzení}
%     \newtheorem{corollary}[theorem]{Důsledek}
%     \newtheorem{lemma}[theorem]{Lemma}
%     \newtheorem{observation}[theorem]{Pozorování}
% \theoremstyle{definition}
%     \newtheorem{definition}[theorem]{Definice}
%     \newtheorem*{algorithm}{Algoritmus}
% \theoremstyle{remark}
%     \newtheorem{remark}[theorem]{Poznámka}
%     \newtheorem{example}[theorem]{Příklad}
%     \newtheorem{exercise}{Cvičení}[chapter]
%     \newtheorem*{solution}{Řešení}

%% macros and definitions
\DeclareMathOperator{\Aut}{Aut}
\DeclareMathOperator{\Conseq}{Csq}
\DeclareMathOperator{\DeLO}{DeLO}
\DeclareMathOperator{\dom}{dom}
\DeclareMathOperator{\Fm}{Fm}
\DeclareMathOperator{\M}{M}
%\DeclareMathOperator{\Proof}{Proof}
\DeclareMathOperator{\rng}{rng}
\DeclareMathOperator{\Term}{Term}
\DeclareMathOperator{\Th}{Th}
\DeclareMathOperator{\Thm}{Thm}
\DeclareMathOperator{\Tree}{Tree}
\DeclareMathOperator{\Var}{Var}
\DeclareMathOperator{\VF}{VF}

\newcommand{\A}{\mystructure{A}}
\newcommand{\B}{\mystructure{B}}
\newcommand{\Con}{\mathit{Con}}
\newcommand{\disjointunion}{\mathbin{\dot{\sqcup}}}
\newcommand{\F}{\ensuremath{\mathrm{F}}}
\newcommand{\landsymb}{{\land}}
\newcommand{\lbin}{\mathbin{\square}}
\newcommand{\lbinsymb}{{\lbin}}
\newcommand{\liff}{\mathbin{\leftrightarrow}}
\newcommand{\liffsymb}{{\liff}}
\newcommand{\limplies}{\mathbin{\rightarrow}}
\newcommand{\limpliessymb}{{\limplies}}
\newcommand{\lorsymb}{{\lor}}
\newcommand{\Prf}{\mathit{Prf}}
%\newcommand{\structure}[1]{\mathcal{#1}}
\newcommand{\todo}{[TODO]}
\newcommand{\T}{\ensuremath{\mathrm{T}}}
\newcommand{\union}{\mathbin{\cup}}

\DeclareRobustCommand\proves{\mathrel{|}\joinrel\mkern-.5mu\mathrel{-}}

\title{Desátá přednáška}
\subtitle{NAIL062 Výroková a predikátová logika}
\author{Jakub Bulín (KTIML MFF UK)}
% \institute{KTIML MFF UK}
\date{Zimní semestr 2023}


\begin{document}


\frame{\titlepage}


\begin{frame}{Desátá přednáška}

    \textbf{Program}
        \begin{itemize}
            \item grounding, Herbrandova věta
            \item unifikace, unifikační algoritmus
            \item rezoluční pravidlo, rezoluční důkaz
        \end{itemize}    

    \textbf{Materiály}

        \href{https://github.com/jbulin-mff-uk/nail062/raw/main/lecture/lecture-notes/lecture-notes.pdf}{\alert{\textbf{Zápisky z přednášky}}}, Sekce 8.3-8.5 z Kapitoly 8

\end{frame}


\section{8.3 Grounding}


\begin{frame}{Grounding}

    \begin{itemize}
        \item \alert{základní (ground) instance} otevřené $\varphi$ ve volných proměnných $x_1,\dots,x_n$ je $\varphi(x_1/t_1,\dots,x_n/t_n)$, kde vš. $t_i$ jsou konstantní
        
        \bigskip
        
        \myblock{\textbf{Herbrandova věta} říká, že je-li \alert{otevřená} teorie \alert{nesplnitelná}, lze to doložit ``na konkrétních prvcích'': existuje konečně mnoho \alert{základních instancí} axiomů, jejichž konjunkce je nesplnitelná}
        
        \bigskip
        \item např. pro \myexampleinline{
            $T=\{P(x,y)\lor R(x,y),\neg P(c,y),\neg R(x,f(x))\}$
        } substituujeme \alert{konstantní} termy $\{x/c,y/f(c)\}$:
        
        \myalertmath{\vspace{-12pt}
            $$
            (P(c,f(c))\lor R(c,f(c)))\ \land\ \neg P(c,f(c))\ \land\ \neg R(c,f(c))
            $$
        } 

        \item základní atomické sentence chápeme jako prvovýroky: 
        
        \myalertmath{
            $$
            (p_1 \lor p_2) \land \neg p_1 \land \neg p_2
            $$
        }

        \item to už snadno zamítneme výrokovou rezolucí
        \item $p_1$ znamená ``platí $P(c,f(c))$'', $p_2$ znamená ``platí $R(c,f(c))$''        
    \end{itemize}
    
\end{frame}


\begin{frame}{Přímá redukce do výrokové logiky}

    Herbrandova věta + korektnost a úplnost výrokové rezoluce dává následující, neefektivní postup ($S'$ je moc velká, i nekonečná):
    
    \medskip

    \myalert{
    \begin{enumerate}
        \item $S$ $\rightsquigarrow$ $S'$ = množina všech základních instancí klauzulí z $S$
        \item atomické sentence v $S'$ chápeme jako prvovýroky
        \item $S$ nesplnitelná $\Leftrightarrow$ $S'$ zamítnutelná `na úrovni výrokové logiky'
    \end{enumerate}
    }
    
    \medskip

    Např. pro \myexampleinline{
        $S=\{\{P(x,y),R(x,y)\},\{\neg P(c,y)\},\{\neg R(x,f(x))\}\}$
    }
    {\footnotesize
    \begin{align*}
        S'=\{&\{P(c,c),R(c,c)\},\{P(c,f(c)),R(c,f(c))\},\{P(f(c),c),R(f(c),c)\},\dots,\\ 
        &\{\neg P(c,c)\}, \{\neg P(c,f(c))\},\{\neg P(c,f(f(c)))\},\{\neg P(c,f(f(f(c))))\}, \dots,\\
        &\{\neg R(c,f(c))\}, \{\neg R(f(c),f(f(c)))\},\{\neg R(f(f(c)),f(f(f(c))))\},\dots\}    
    \end{align*}
    }
    $S'$ je nesplnitelná obsahuje konečnou nesplnitelnou podmnožinu:
    $$
    \{\{P(c,f(c)),R(c,f(c))\},\{\neg P(c,f(c))\},\{\neg R(c,f(c))\}\}\proves_R\square
    $$

    \myalertinline{\textbf{Efektivnější} je hledat vhodné základní instance \alert{unifikací} [za chvíli]}

\end{frame}


\begin{frame}{Herbrandův model}

    \myblock{
    Mějme jazyk $L=\langle\mathcal R,\mathcal F\rangle$ s alespoň jedním konstantním symbolem. $L$-struktura $\A=\langle A,\mathcal R^\A,\mathcal F^\A\rangle$ je \alert{Herbrandův model}, jestliže:
    \begin{itemize}
        \item $A$ je množina všech konst. $L$-termů (\alert{Herbrandovo univerzum})
        \item pro každý $n$-ární $f\in\mathcal F$ a (konstantní) $\text{``$t_1$''},\dots,\text{``$t_n$''}\in A$:
        \myalertmath{
        $$
        f^\A(\text{``$t_1$''},\dots,\text{``$t_n$''})=\text{``$f(t_1,\dots,t_n)$''}
        $$
        }
        \item speciálně, pro konstantní symbol $c\in\mathcal F$ je $c^\A=\text{``$c$''}$
        \item na relační symboly neklademe podmínky
    \end{itemize}
    }

    \medskip

    Např. \myexampleinline{
        $L=\langle P,f,c\rangle$
    } ($P$ unární rel., $f$ binární funkční, $c$ konstantní)
    \alert{Herbrandův model} je každá struktura $\A=\langle A,P^\A,f^\A,c^\A\rangle$, kde
    \begin{itemize}\small
        \item  $A=\{\text{``$c$''},\text{``$f(c,c)$''},\text{``$f(c,f(c,c))$''},\text{``$f(f(c,c),c)$''}\dots\}$
        \item $c^\A=\text{``$c$''}$
        \item $f^\A(\text{``$c$''},\text{``$c$''})=\text{``$f(c,c)$''}$, $f^\A(\text{``$c$''},\text{``$f(c,c)$''})=\text{``$f(c,f(c,c))$''}$, $f^\A(\text{``$f(c,c)$''},\text{``$c$''})=\text{``$f(f(c,c),c)$''}$, atd.
        \item $P^\A\subseteq A$ může být libovolná
    \end{itemize}

\end{frame}


\begin{frame}{Herbrandova věta}

    \myblock{
    \textbf{Věta (Herbrandova):}
        Je-li $T$ otevřená, v jazyce bez rovnosti a s alespoň jedním konstantním symbolem, potom:
        \begin{itemize}
            \item buď má $T$ \alert{Herbrandův model}, nebo
            \item existuje konečně mnoho základních instancí axiomů $T$, jejichž konjunkce je nesplnitelná.
        \end{itemize}
    }
    
    \textbf{Důkaz:}
        \alert{$T_\text{ground}$} = \myalertinline{množina všech základních instancí axiomů $T$}

        \vspace{-3pt}
        
        Zkonstruujeme \alert{``systematické tablo''} $\tau$ z $T_\text{ground}$ s $\F\bot$ v kořeni, ale z jazyka $L$, bez rozšíření o pomocné konstantní symboly na $L_C$. (Nepotřebujeme je, protože v $T_\text{ground}$ nejsou kvantifikátory.)
        
        Pokud má $\tau$ bezespornou větev, je \alert{``kanonický model''} (opět bez pomocných symbolů) Herbrandovým modelem $T$. 
        
        Jinak je $\tau$ \alert{důkaz sporu}, $T_\text{ground}$ (a tedy i $T$) je nesplnitelná. Tablo $\tau$ je konečné, používá jen konečně mnoho $\alpha_\text{ground}\in T_\text{ground}$, jejich konjunkce už je nesplnitelná.\hfill\qedsymbol
    

\end{frame}


\begin{frame}{Poznámky}

    \begin{itemize}
        \item konstatní symbol potřebujeme, aby existovaly vůbec nějaké konstantní termy (ale není-li v $L$ žádný, můžeme ho přidat)
        \item Herbrandův model je podobný kanonickému, ale nepřidáváme pomocné symboly, a neříkáme nic o relacích
        \item je-li jazyk s rovností, najdeme Herbrandův model pro $T^*$ (přidané axiomy rovnosti) a faktorizujeme podle $=^A$
    \end{itemize}

\end{frame}


\begin{frame}{Důsledky Herbrandovy věty}

    \myblock{
    \textbf{Důsledek:}
    Je-li $T$ otevřená v jazyce s konstantním symbolem, potom $T$ má model, právě když má model teorie $T_\text{ground}$.
    }

    \textbf{Důkaz:}
    \alert{\Large$\Rightarrow$} V modelu $T$ platí i všechny základní instance axiomů. Je tedy i modelem $T_\text{ground}$. 
    
    \alert{\Large$\Leftarrow$} Pokud $T$ nemá model, podle Herbrandovy věty je nějaká konečná podmnožina teorie $T_\text{ground}$ nesplnitelná.\hfill\qedsymbol

    \bigskip

    \myblock{
    \textbf{Důsledek:}
    Mějme otevřenou $\varphi(x_1,\dots,x_n)$ v $L$ s konst. symbolem. Potom existuje $m\in\mathbb N$ a konstantní $L$-termy $t_{ij}$ ($i\in[m],j\in[n]$), že sentence \myalertinline{
        $(\exists x_1)\dots(\exists x_n)\varphi(x_1,\dots,x_n)$
    } je pravdivá, právě když je následující formule (výroková) tautologie:

    \vspace{-12pt}
    $$
    \varphi(x_1/t_{11},\dots,x_n/t_{1n})\lor \dots \lor \varphi(x_1/t_{m1},\dots,x_n/t_{mn})
    $$
    \vspace{-16pt}
    }

    \textbf{Důkaz:}
    Je \alert{pravdivá}, právě když $(\forall x_1)\dots(\forall x_n)\neg\varphi$ neboli $\neg\varphi$ je \alert{nesplnitelná}. Stačí aplikovat Herbrandovu větu na $T=\{\neg\varphi\}$.\hfill\qedsymbol

\end{frame}


\section{8.4 Unifikace}


\begin{frame}{Příklady substitucí}

    Místo \alert{všech základních} použijeme \alert{`vhodné'} substituce (\alert{unifikace}):

    \myexampleinline{
        1. $\{P(x),Q(x,a)\}$ a $\{\neg P(y),\neg Q(b,y)\}$
        } \vspace{-6pt}
        \begin{itemize}
            \item substitucí \myalertinline{
            $\{x/b,y/a\}$
            } získáme $\{P(b),Q(b,a)\}$ a $\{\neg P(a),\neg Q(b,a)\}$, z nich rezolucí $\{P(b),\neg P(a)\}$
            \item nebo \myalertinline{
            $\{x/y\}$
            } a rezolucí přes $P(y)$ máme $\{Q(y,a),\neg Q(b,y)\}$
            \item šlo by např. \myalertinline{
                $\{x/a\}$
            }, získat $\{Q(a,a),\neg Q(b,a)\}$, ale to je \alert{horší}   
        \end{itemize}         
          
    \myexampleinline{
        2. $\{P(x),Q(x,z)\}$ a $\{\neg P(y),\neg Q(f(y),y)\}$
    } \vspace{-6pt}
        \begin{itemize}
            \item lze použít \myalertinline{
                $\{x/f(a),y/a,z/a\}$
                }, získat $\{P(f(a)),Q(f(a),a)\}$ a $\{\neg P(a),\neg Q(f(a),a)\}$, rezolucí \alert{$\{P(f(a)),\neg P(a)\}$}
            \item \alert{lepší} je \myalertinline{
                $\{x/f(z),y/z\}$
            }, dává $\{P(f(z)),Q(f(z),z)\}$ a $\{\neg P(z),\neg Q(f(z),z)\}$, rezolventu \alert{$\{P(f(z)),\neg P(z)\}$}
            \item proč lepší? \alert{obecnější}, rezolventa `říká více': {\small$\{P(f(a)),\neg P(a)\}$} je důsledkem {\small$\{P(f(z)),\neg P(z)\}$}, ale nejsou ekvivalentní
            \item $\{x/f(a),y/a,z/a\}$ získáme \alert{složením} $\{x/f(z),y/z\}$ a $\{z/a\}$
        \end{itemize}            

\end{frame}


\begin{frame}{Substituce formálně}

    \begin{itemize}
        \item \alert{substituce} je konečná množina \alert{$\sigma=\{x_1/t_1,\dots,x_n/t_n\}$}, kde $x_i$ jsou navzájem různé proměnné, $t_i$ jsou termy, $t_i$ není $x_i$
        \begin{itemize}
            \item \alert{základní}: všechny termy $t_i$ jsou konstantní
            \item \alert{přejmenování proměnných}: vš. $t_i$ navzájem různé proměnné
        \end{itemize}
        \item \alert{výraz} je term nebo literál (atomická formule nebo její negace)
        \item \alert{instance} výrazu $E$ \alert{při substituci} $\sigma=\{x_1/t_1,\dots,x_n/t_n\}$, \alert{$E\sigma$}: simultánně nahradíme  všechny výskyty $x_i$ za termy $t_i$
        \item pro množinu výrazů $S$ je \alert{$S\sigma=\{E\sigma\mid E\in S\}$}
        
        \bigskip
        
        \item simultánně, aby výskyt $x_i$ v termu $t_j$ nevedl ke zřetězení
        \item např. \myexampleinline{
            $S=\{P(x),R(y,z)\}$, $\sigma=\{x/f(y,z),y/x,z/c\}$
        }
        $$
        S\sigma=\{P(f(y,z)),R(x,c)\}
        $$
    \end{itemize}

\end{frame}


\begin{frame}{Skládání substitucí}
    

\end{frame}


\begin{frame}{Vlastnosti skládání}

    
\end{frame}


\begin{frame}{Unifikace}
    

\end{frame}


\begin{frame}{Unifikační algoritmus}
   

\end{frame}


\begin{frame}{Ukázkový běh}
    

\end{frame}


\begin{frame}{Důkaz korektnosti}
    

\end{frame}


\section{8.5 Rezoluční metoda}


\end{document}


