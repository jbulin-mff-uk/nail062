
\section{Korektnost a úplnost}

V této sekci dokážeme, že tablo metoda je i v predikátové logice korektní a úplná. Důkazy obou vět mají stejnou strukturu jako ve výrokové logice, liší se jen v implementačních detailech.

\subsection{Věta o korektnosti}

Model (struktura) $\A$ se \alert{shoduje} s položkou $P$, pokud
$P=\mathrm{T}\varphi$ a $\A\models\varphi$, nebo $P=\mathrm{F}\varphi$ a $\A\not\models\varphi$. Dále $\A$ se shoduje s větví $V$, pokud se shoduje s každou položkou na této větvi.

Ukážeme nejprve pomocné lemma analogické Lemmatu \ref{lemma:agrees-with-branch}:
\begin{lemma}\label{lemma:agrees-with-branch-predicate}
    Shoduje-li se model $\A$ teorie $T$ s položkou v kořeni tabla z teorie $T$ (v jazyce $L$), potom lze $\A$ expandovat do jazyka $L_C$ tak, že se shoduje s některou větví v tablu.
\end{lemma}
Všimněte si, že stačí expandovat $\A$ o nové konstanty $c^\A$ vyskytující se na větvi $V$. Ostatní konstantní symboly lze interpretovat libovolně.

\begin{proof}
    Mějme tablo $\tau=\bigcup_{i\geq 0}\tau_i$ z teorie $T$ a model $\A\in\M_L(T)$ shodující se s kořenem $\tau$, tedy s (jednoprvkovou) větví $V_0$ v (jednoprvkovém) $\tau_0$.
    
    Indukcí podle $i$ najdeme posloupnost větví $V_i$ a expanzí $\A_i$ modelu $\A$ o konstanty $c^\A\in C$ vyskytující se na $V_i$ takových, že $V_i$ je větev v tablu $\tau_i$ shodující se s modelem $\A_i$, $V_{i+1}$ je prodloužením $V_i$, a $\A_{i+1}$ je expanzí $\A_i$ (mohou si být i rovny). Požadovaná větev tabla $\tau$ je potom $V=\bigcup_{i\geq 0}V_i$. Expanzi modelu $\A$ do jazyka $L_C$ získáme jako `limitu' expanzí $\A_i$, tj. vyskytuje-li se symbol $c\in C$ na $V$, vyskytuje se na nějaké z větví $V_i$ a interpretujeme ho stejně jako v $\A_i$ (ostatní pomocné symboly interpretujeme libovolně).
    \begin{itemize}
        \item Pokud $\tau_{i+1}$ vzniklo z $\tau_i$ bez prodloužení větve $V_i$, definujeme $V_{i+1}=V_i$ a $\A_{i+1}=\A_{i}$.
        \item Pokud $\tau_{i+1}$ vzniklo z $\tau_i$ připojením položky $\mathrm{T}\alpha$ (pro nějaký axiom $\alpha\in T$) na konec větve  $V_i$, definujeme $V_{i+1}$ jako tuto prodlouženou větev a $\A_{i+1}=\A_i$ (nepřidali jsme žádný nový pomocný konstantní symbol). Protože $\A_{i+1}$ je modelem $T$, platí v něm axiom $\alpha$, tedy shoduje se i s novou položkou $\mathrm{T}\alpha$.
        \item Nechť $\tau_{i+1}$ vzniklo z $\tau_i$ připojením atomického tabla pro nějakou položku $P$ na konec větve $V_i$. Protože se model $\A_i$ shoduje s položkou $P$ (která leží na větvi $V_i$), shoduje se i s kořenem připojeného atomického tabla.
        \begin{itemize}
            \item Pokud jsme připojili atomické tablo pro logickou spojku, položíme $\A_{i+1}=\A_i$ (nepřidali jsme nový pomocný symbol). Protože $\A_{i+1}$ se shoduje s kořenem atomického tabla, shoduje se i s některou z jeho větví (stejně jako ve výrokové logice); definujeme $V_{i+1}$ jako prodloužení $V_i$ o tuto větev.
            \item Je-li položka $P$ typu `svědek': Pokud je $P=\T(\exists x)\varphi(x)$, potom $\A_i\models(\exists x)\varphi(x)$, tedy existuje $a\in A$ takové, že $\A_i\models\varphi(x)[e(x/a)]$. Větev $V_{i+1}$ definujeme jako prodloužení $V_i$ o nově přidanou položku $\T\varphi(x/c)$ a model $\A_{i+1}$ jako expanzi $\A_i$ o konstantu $c^A=a$. Případ $P=\F(\forall x)\varphi(x)$ je obdobný.
            \item Je-li položka $P$ typu `všichni', větev $V_{i+1}$ definujeme jako prodloužení $V_i$ o atomické tablo. Nově přidaná položka je $\T\varphi(x/t)$ nebo $\F\varphi(x/t)$ pro nějaký $L_C$-term $t$. Předpokládejme, že jde o první z těchto dvou možností, pro druhou je důkaz analogický. 
            Model $\A_{i+1}$ definujeme jako \alert{libovolnou} expanzi $\A_i$ o nové konstanty vyskytující se v $t$.    
            Protože $\A_i\models(\forall x)\varphi(x)$, platí i $\A_{i+1}\models(\forall x)\varphi(x)$ a tedy i $\A_{i+1}\models\varphi(x/t)$; model $\A_{i+1}$ se tedy shoduje s větví $V_i$.
        \end{itemize}       
    \end{itemize}
\end{proof}

Připomeňme stručně myšlenku důkazu Věty o korektnosti: Pokud by existoval důkaz a zároveň protipříklad, protipříklad by se musel shodovat s některou větví důkazu, ty jsou ale všechny sporné. Důkaz je tedy téměř stejný jako ve výrokové logice.

\begin{theorem}[O korektnosti]
Je-li sentence $\varphi$ tablo dokazatelná z teorie $T$, potom je $\varphi$ pravdivá v $T$, tj. $T\proves\varphi\ \Rightarrow\ T\models\varphi$.    
\end{theorem}

\begin{proof}
Předpokládejme pro spor, že $T\not\models\varphi$, tj. existuje $\A\in\M(T)$ takový, že $\A\not\models\varphi$. Protože $T\proves\varphi$,  existuje sporné tablo z $T$ s $\mathrm{F}\varphi$ v kořeni. Model $\A$ se shoduje s $\mathrm{F}\varphi$, tedy podle Lemmatu \ref{lemma:agrees-with-branch-predicate} lze expandovat do jazyka $L_C$ tak, že se expanze shoduje s nějakou větví $V$. Všechny větve jsou ale sporné.
\end{proof}

\subsection{Věta o úplnosti}

Stejně jako ve výrokové logice ukážeme, že \alert{bezesporná} větev v \alert{dokončeném} tablo důkazu poskytuje protipříklad: model teorie $T$, který se shoduje s položkou $\mathrm{F}\varphi$ v kořeni tabla, tj. neplatí v něm $\varphi$. Takových modelů může být více, definujeme proto opět jeden konkrétní, \alert{kanonický}.

Model musí mít nějakou doménu. Jak ji získat z tabla, což je čistě sémantický objekt? Využijeme standardní (v matematice) trik: ze syntaktických objektů uděláme sémantické. Konkrétně, za doménu zvolíme množinu všech \alert{konstantních termů} jazyka $L_C$.\footnote{Tj. termů zbudovaných aplikací funkčních symbolů jazyka $L$ na konstantní symboly jazyka $L$ (má-li nějaké) a pomocné konstantní symboly z $C$.} Ty chápeme jako konečné řetězce. V následujícím výkladu budeme někdy (neformálně) místo termu $t$ psát ``$t$'', abychom zdůraznili, že v daném místě chápeme $t$ jako řetězec znaků, a ne např. jako termovou funkci, kterou je třeba vyhodnotit.\footnote{Srovnejte aritmetický výraz ``1+1'' a 1+1=2.}

\begin{definition}[Kanonický model]\label{definition:canonical-model-predicate}
Mějme teorii $T$ v jazyce $L=\langle\mathcal F,\mathcal R\rangle$ a nechť $V$ je bezesporná větev nějakého dokončeného tabla z teorie $T$. Potom \alert{kanonický model} pro $V$ je $L_C$-struktura $\A=\langle A,\mathcal F^\mathcal A\cup C^\mathcal A,\mathcal R^\mathcal A\rangle$ definovaná následovně:

Je-li jazyk $L$ bez rovnosti, potom:
\begin{itemize}
    \item Doména $A$ je množina všech konstantních $L_C$-termů.
    \item Pro každý $n$-ární relační symbol $R\in\mathcal R$ a ``$s_1$'', \dots, ``$s_n$'' z $A$:
    $$
    (\text{``$s_1$''},\dots,\text{``$s_n$''})\in R^\mathcal A\text{ právě když na větvi $V$ je položka $\T R(s_1,\dots,s_n)$}
    $$
    \item Pro každý $n$-ární funkční symbol $f\in\mathcal F$ a ``$s_1$'', \dots, ``$s_n$'' z $A$:
    $$
    f^\mathcal A(\text{``$s_1$''},\dots,\text{``$s_n$''})=\text{``$f(s_1,\dots,s_n)$''}
    $$
    Speciálně, pro konstantní symbol $c$ máme $c^\mathcal A=\text{``$c$''}$.
\end{itemize}
Funkci $f^\mathcal A$ tedy interpretujeme jako `vytvoření' nového termu ze symbolu $f$ a vstupních termů (řetězců). 

Nechť je $L$ jazyk s rovností. Připomeňme, že naše tablo je nyní z teorie $T^*$, tj. z rozšíření~$T$ o axiomy rovnosti pro $L$. Nejprve vytvoříme kanonický model $\mathcal B$ pro $V$ jakoby byl $L$ bez rovnosti (jeho doména $B$ je tedy množina všech konstantních $L_C$-termů). Dále definujeme relaci $=^B$ stejně jako pro ostatní relační symboly:
$$
\text{``$s_1$''}=^B\text{``$s_2$''}\text{ právě když na větvi $V$ je položka $\T s_1=s_2$}
$$
\alert{Kanonický model} pro $V$ potom definujeme jako faktorstrukturu $\A=\B/_{=^B}$.
\end{definition}

Jak plyne z diskuze v Sekci \ref{section:tableaux-equality}, relace $=^B$ je opravdu kongruence struktury $\B$, definice je tedy korektní, a relace $=^\A$ je identita na $A$. Platí následující jednoduché pozorování:


\begin{observation}\label{canonical-with-equality-satisfies-the-same}
    Pro každou formuli $\varphi$ máme $\B\models\varphi$ (kde symbol $=$ je interpretován jako binární relace $=^B$), právě když $\A=\B/_{=^B}\models\varphi$ (kde $=$ je interpretován jako identita).    
\end{observation}


Všimněte si, že v jazyce bez rovnosti je kanonický model vždy spočetně nekonečný. V jazyce s rovností může ale být konečný, jak uvidíme v následujících příkladech.

\begin{example}
    Nejprve si ukažme příklad kanonického modelu v jazyce bez rovnosti. Mějme teorii $T=\{(\forall x)R(f(x))\}$ v jazyce $L=\langle R,f,d \rangle$ bez rovnosti, kde $R$ je unární relační, $f$ unární funkční, a $d$ konstantní symbol. Najděme protipříklad ukazující, že $T\not\models\neg R(d)$. 
    
    Systematické tablo z $T$ s položkou $\F\neg R(d)$ v kořeni není sporné, obsahuje jedinou větev $V$, která je bezesporná. (Sestrojte si tablo sami!) Kanonický model pro $V$ je $L_C$-struktura $\A=\langle A,R^\A,f^\A,d^\A,c_0^\A,c_1^\A,c_2^\A,\dots\rangle$, jejíž doména je
    $$
    A=\{\text{``$d$''},\text{``$f(d)$''},\text{``$f(f(d))$''},\dots,\text{``$c_0$''},\text{``$f(c_0)$''},\text{``$f(f(c_0))$''},\dots,\text{``$c_1$''},\text{``$f(c_1)$''},\text{``$f(f(c_1))$''},\dots\}
    $$
    a interpretace symbolů jsou následující:
    \begin{itemize}
        \item $d^\A=\text{``$d$''}$,
        \item $c^\A_i=\text{``$c_i$''}$ pro všechna $i\in \mathbb N$,
        \item $f^\A(\text{``$d$''})=\text{``$f(d)$''}$, $f^\A(\text{``$f(d)$''})=\text{``$f(f(d))$''}$, \dots
        \item $R^\A=A\setminus C=\{\text{``$d$''},\text{``$f(d)$''},\text{``$f(f(d))$''},\dots,\text{``$f(c_0)$''},\text{``$f(f(c_0))$''},\dots,\text{``$f(c_1)$''},\text{``$f(f(c_1))$''},\dots\}$.
    \end{itemize}
    Redukt kanonického modelu $\A$ na původní jazyk $L$ je potom $\A'=\langle A, R^\A, f^\A, d^\A\rangle$.
\end{example}

\begin{example}
    Nyní příklad v jazyce s rovností: Mějme teorii $T=\{(\forall x)R(f(x)),(\forall x)(x=f(f(x)))\}$ v jazyce $L=\langle R,f,d \rangle$ s rovností. Opět najděme protipříklad ukazující, že $T\not\models\neg R(d)$. 

    Systematické tablo z teorie $T^*$ (tj. z $T$ rozšířené o axiomy rovnosti pro $L$) s položkou $\F\neg R(d)$ v kořeni obsahuje bezespornou větev $V$. (Sestrojte si tablo sami!) Nejprve sestrojíme kanonický model $\B$ pro tuto větev, jako by byl jazyk bez rovnosti:
    $$
    \B=\langle B,R^\B,f^\B,d^\B,c_0^\B,c_1^\B,c_2^\B,\dots\rangle
    $$
    kde $B$ je množina všech konstantních $L_C$-termů. Relace $=^B$ je definovaná, jako by symbol `$=$' byl `obyčejným' relačním symbolem v $L$. Je to kongruence struktury $\B$, a platí pro ni, že $s_1=^B s_2$ právě když $s_1=f(\cdots (f(s_2))\cdots)$ nebo $s_2=f(\cdots (f(s_1))\cdots)$ pro sudý počet aplikací $f$. Jako reprezentanty jednotlivých tříd tedy můžeme vybrat termy s žádným nebo jedním výskytem symbolu $f$:
    $$
        B/_{=^B} = \{[\text{``$d$''}]_{=^B},[\text{``$f(d)$''}]_{=^B},[\text{``$c_0$''}]_{=^B},[\text{``$f(c_0)$''}]_{=^B},[\text{``$c_1$''}]_{=^B},[\text{``$f(c_1)$''}]_{=^B},\dots\}
    $$
    Kanonický model pro větev $V$ je potom $L_C$-struktura 
    $$
    \A=\B/_{=^B}=\langle A,R^\A,f^\A,d^\A,c_0^\A,c_1^\A,c_2^\A,\dots\rangle
    $$
    kde $A=B/_{=^B}$ a interpretace symbolů jsou následující:
    \begin{itemize}
        \item $d^\A=[\text{``$d$''}]_{=^B}$,
        \item $c^\A_i=[\text{``$c_i$''}]_{=^B}$ pro všechna $i\in \mathbb N$,
        \item $f^\A([\text{``$d$''}]_{=^B})=[\text{``$f(d)$''}]_{=^B}$, $f^\A([\text{``$f(d)$''}]_{=^B})=[\text{``$f(f(d))$''}]_{=^B}=[\text{``$d$''}]_{=^B}$, \dots
        \item $R^\A=A=B/_{=^B}$.
    \end{itemize}
    Redukt kanonického modelu $\A$ na původní jazyk $L$ je opět $\A'=\langle A, R^\A, f^\A, d^\A\rangle$.
\end{example}

\begin{exercise}
    \begin{enumerate}[(a)]
        \item Sestrojte dokončené tablo s položkou $\T (\forall x)(\forall y)(x=y)$ v kořeni. Sestrojte kanonický model pro (jedinou, bezespornou) větev tohoto tabla.
        \item Sestrojte dokončené tablo s položkou $\T (\forall x)(\forall y)(\forall z)(x=y\lor x=z \lor y=z)$ v kořeni. Sestrojte kanonické modely pro několik bezesporných větví a porovnejte je.
    \end{enumerate}
\end{exercise} 

Nyní jsme připraveni dokázat Větu o úplnosti. Použijeme opět následující pomocné lemma, jehož znění je zcela stejné, jako znění Lemmatu \ref{lemma:canonical-model-agrees} a důkaz se liší jen v technických detailech.

\begin{lemma}\label{lemma:canonical-model-agrees-predicate}
    Kanonický model pro (bezespornou dokončenou) větev $V$ se shoduje s $V$.
\end{lemma}
\begin{proof}
Nejprve uvažme jazyky bez rovnosti. Ukážeme indukcí podle struktury sentencí v položkách, že kanonický model $\A$ se shoduje se všemi položkami $P$ na větvi $V$. 

Základ indukce, tj. případ, kdy $\varphi=R(s_1,\dots,s_n)$ je atomická sentence, je jednoduchý: Je-li na $V$ položka $\T\varphi$, potom $(s_1,\dots,s_n)\in R^\A$ plyne přímo z definice kanonického modelu, máme tedy $\A\models\varphi$.  Je-li na $V$ položka $\F\varphi$, potom na $V$ není položka $\T\varphi$ ($V$ je bezesporná), $(s_1,\dots,s_n)\not\in R^\A$, a $\A\not\models\varphi$

Nyní indukční krok. Rozebereme jen několik případů, ostatní se dokáží obdobně. 

Pro logické spojky je důkaz zcela stejný jako ve výrokové logice, například je-li $P=\mathrm{F}\varphi\land\psi$, potom protože je $P$ na $V$ redukovaná, vyskytuje se na $V$ položka $\mathrm{F}\varphi$ nebo položka $\mathrm{F}\psi$. Platí tedy $\A\not\models\varphi$ nebo $\A\not\models\psi$, z čehož plyne $\A\not\models\varphi\land\psi$ a $\A$ se shoduje s $P$.

Máme-li položku typu ``všichni'', například $P=\T(\forall x)\varphi(x)$ 
(případ $P=\F(\exists x)\varphi(x)$ je obdobný), potom jsou na $V$ i položky $T\varphi(x/t)$ pro každý konstantní $L_C$-term, tj. pro každý prvek $\text{``$t$''}\in A$. Dle indukčního předpokladu je $\A\models\varphi(x/t)$ pro každé $\text{``$t$''}\in A$, tedy $\A\models(\forall x)\varphi(x)$.

Máme-li položku typu ``svědek'', například $P=\T(\exists x)\varphi(x)$ 
(případ $P=\F(\forall x)\varphi(x)$ je obdobný), potom je na $V$ i položka $T\varphi(x/c)$ pro nějaké $\text{``$c$''}\in A$. Dle indukčního předpokladu je $\A\models\varphi(x/c)$, tedy i $\A\models(\exists x)\varphi(x)$.

Je-li jazyk s rovností, máme kanonický model $\A=\B/_{=^B}$, důkaz výše platí pro $\B$, a zbytek plyne z Pozorování \ref{canonical-with-equality-satisfies-the-same}.
\end{proof}

\begin{exercise}
    Ověřte zbývající případy v důkazu Lemmatu \ref{lemma:canonical-model-agrees-predicate}.
\end{exercise}

Důkaz Věty o úplnosti je také analogický její verzi pro výrokovou logiku:

\begin{theorem}[O úplnosti]\label{theorem:completeness-theorem-predicate}
    Je-li sentence $\varphi$ pravdivá v teorii $T$, potom je tablo dokazatelná z $T$, tj. $T\models\varphi\ \Rightarrow\ T\proves\varphi$.    
\end{theorem} 

\begin{proof}
Ukážeme, že libovolné \alert{dokončené} tablo z $T$ s položkou $\mathrm{F}\varphi$ v kořeni je nutně sporné. Důkaz provedeme sporem: kdyby takové tablo nebylo sporné, existovala by v něm bezesporná (dokončená) větev $V$. Uvažme kanonický model $\A$ pro tuto větev, a označme jako $\A'$ jeho redukt na jazyk $L$. Protože je $V$ dokončená, obsahuje $\mathrm{T}\alpha$ pro všechny axiomy $\alpha\in T$. Model $\A$ se podle Lemmatu \ref{lemma:canonical-model-agrees-predicate} shoduje se všemi položkami na $V$, splňuje tedy všechny axiomy a máme i $\A'\models T$. Protože se ale $\A$ shoduje i s položkou $\mathrm{F}\varphi$ v kořeni, platí i $\A'\not\models\varphi$, což znamená, že $\A'\in \M_L(T)\setminus\M_L(\varphi)$, tedy $T\not\models\varphi$, a to je spor. Tablo tedy muselo být sporné, tj. být tablo důkazem $\varphi$ z $T$.
\end{proof}








\section{Důsledky korektnosti a úplnosti}

Stejně jako ve výrokové logice, Věty o korektnosti a úplnosti dohromady říkají, že \alert{dokazatelnost} je totéž, co \alert{platnost}. To nám umožňuje obdobně zformulovat syntaktické analogie sémantických pojmů a vlastností.

Analogií \alert{důsledků} jsou \alert{teorémy} teorie $T$:
$$
\Thm_L(T)=\{\varphi\mid \varphi\text{ je $L$-sentence a } T\proves\varphi\}
$$

\begin{corollary}[Dokazatelnost = platnost]\label{corollary:corollary-of-soundness-and-completeness-predicate}
    Pro libovolnou teorii $T$ a sentence $\varphi,\psi$ platí:
    \begin{itemize}
        \item $T\proves\varphi$ právě když $T\models\varphi$
        \item $\Thm_L(T)=\Conseq_L(T)$
    \end{itemize}
\end{corollary}

Platí například:
\begin{itemize}
    \item Teorie je \alert{sporná}, jestliže je v ní dokazatelný spor (tj. $T\proves\bot$).
    \item Teorie je \alert{kompletní}, jestliže pro každou sentenci $\varphi$ je buď $T\proves\varphi$ nebo $T\proves\neg\varphi$ (ale ne obojí, jinak by byla sporná).
    \item Věta o dedukci: Pro teorii $T$ a sentence $\varphi,\psi$ platí $T,\varphi\proves\psi$, právě když $T\proves\varphi\to\psi$.
\end{itemize}

Na závěr této sekce si ukážeme několik aplikací Vět o úplnosti a korektnosti.

\subsection{Löwenheim-Skolemova věta}\label{subsection:loewenheim-skolem-theorem}

\begin{theorem}[Löwenheim-Skolemova]
    Je-li $L$ spočetný jazyk bez rovnosti, potom každá bezesporná $L$-teorie má spočetně nekonečný model.
\end{theorem}

\begin{proof}
Vezměme nějaké dokončené (např. systematické) tablo z teorie $T$ s položkou $\F\bot$ v kořeni. Protože $T$ je bezesporná, není v ní dokazatelný spor, tedy tablo musí obsahovat bezespornou větev. Hledaný spočetně nekonečný model je $L$-redukt kanonického modelu pro tuto větev.
\end{proof}

K této větě se ještě vrátíme v Kapitole \ref{chapter:model-theory}, kde si ukážeme silnější verzi zahrnující i jazyky s rovností (v nich je kanonický model spočetný, ale může být i konečný).

\subsection{Věta o kompaktnosti}

Stejně jako ve výrokové logice platí Věta o kompaktnosti, stejný je i její důkaz:

\begin{theorem}[O kompaktnosti]\label{theorem:compactness-theorem-predicate}
    Teorie má model, právě když každá její konečná část má model.    
\end{theorem}
\begin{proof}
Model teorie je zřejmě modelem každé její části. Naopak, pokud $T$ nemá model, je sporná, tedy $T\proves\bot$. Vezměme nějaký \alert{konečný} tablo důkaz $\bot$ z $T$. K jeho konstrukci stačí konečně mnoho axiomů $T$, ty tvoří konečnou podteorii $T'\subseteq T$, která nemá model.
\end{proof}


\subsection{Nestandardní model přirozených čísel}

Na závěr této sekce si ukážeme, že existuje tzv. \alert{nestandardní model} přirozených čísel. Klíčem je Věta o kompaktnosti.
    
Nechť $\underline{\mathbb N}=\langle\mathbb N,S,+,\cdot,0,\leq\rangle$ je standardní model přirozených čísel. Označme $\Th(\underline{\mathbb N})$ množinu všech sentencí \alert{pravdivých} ve struktuře $\underline{\mathbb N}$ (tzv. \alert{teorii struktury} $\underline{\mathbb N}$). Pro $n\in \mathbb N$ definujme \alert{$n$-tý numerál} jako term $\underline n=S(S(\cdots (S(0)\cdots))$, kde $S$ je aplikováno $n$-krát.

Vezměme nový konstantní symbol $c$ a vyjádřeme, že je ostře větší než každý $n$-tý numerál:
$$
T=\Th(\underline{\mathbb N})\cup\{\underline n<c\mid n\in \mathbb N\}
$$
Všimněte si, že každá konečná část teorie $T$ má model. Z věty o kompaktnosti tedy plyne, že i teorie $T$ má model. Říkáme mu \alert{nestandardní model} (označme ho $\A$). Platí v něm tytéž sentence, které platí ve standardním modelu, ale zároveň obsahuje prvek $c^\A$, který je větší než každé $n\in \mathbb N$ (čímž zde myslíme hodnotu termu $\underline n$ v nestandardním modelu $\A$).
    

\section{(draft) Hilbertovský kalkulus v predikátové logice}
\todo

Na závěr kapitoly si ukážeme, jak lze adaptovat Hilbertův kalkulus, představený v Sekci \ref{section:hilbert-calculus-propositional} pro použití v predikátové logice. To není těžké, abychom se vypořádali s kvantifikátory, stačí přidat dvě nová schémata logických axiomů a jedno nové inferenční pravidlo. Opět si ukážeme korektnost tohoto dokazovacího systému, a jen zmíníme, že je také úplný.

% from slides:
\subsubsection*{Hilbertovský kalkul}
    \begin{itemize}
    \item základní logické spojky a kvantifikátory:\ \ $\neg$, $\to$, $(\forall x)$ (ostatní odvozené)
    
    \item dokazují se libovolné formule (nejen sentence)
    
    \item \mdef{logické axiomy} (\myblue{schémata} logických axiomů)
    \vspace{-2mm}\begin{align*}(i)& &\varphi &\to (\psi \to \varphi) \\
    (ii)& &(\varphi\to (\psi \to \chi))&\to ((\varphi \to \psi)\to(\varphi \to \chi))\qquad\qquad\qquad\qquad\phantom{\ } \\
    (iii)& &(\neg \varphi \to \neg \psi)&\to(\psi \to \varphi)\\
    (iv)& &(\forall x)\varphi &\to\varphi(x/t)\quad\quad\quad\ \ \ \text{ je-li $t$ substituovatelný za $x$ do $\varphi$}\\
    (v)& &(\forall x)(\varphi \to \psi)&\to(\varphi \to (\forall x)\psi)\quad\text{není-li $x$ volná proměnná ve $\varphi$}\\
    \end{align*}
    
    \vspace{-6mm}
    kde $\varphi$, $\psi$, $\chi$ jsou libovolné formule (daného jazyka), $t$ je libovolný term a
    \vspace{0.5mm}
    
    $x$ je libovolná proměnná.
    \smallskip
    
    \item je-li jazyk s rovností, mezi logické axiomy patří navíc \myblue{axiomy rovnosti}
    
    \item \mdef{odvozovací (deduktivní) pravidla}
    \vspace{-2mm}
    $$\frac{\varphi,\ \varphi \to \psi}{\psi}\quad\text{\myblue{(modus ponens)},}\qquad\frac{\varphi}{(\forall x)\varphi}\quad\text{\myblue{(generalizace)}}$$
    \end{itemize}
    
    
    
    %%%%%%%%%%%%%%%%%%%%%%%%%%%%%%%%%%%%%%%%%%%%%%%%%%%%%%5
    \subsubsection*{Pojem důkazu}
    \mdef{Důkaz} (\alert{Hilbertova stylu}) formule $\varphi$ z teorie $T$ je \myblue{konečná} posloupnost
    \smallskip
    
    $\varphi_0, \dots, \varphi_n=\varphi$ formulí taková, že pro každé $i\le n$
    \smallskip
    
    \begin{itemize}
    \item $\varphi_i$ je logický axiom nebo $\varphi_i \in T$ (axiom teorie), nebo
    \smallskip
    
    \item $\varphi_i$ lze odvodit z předchozích formulí pomocí odvozovacích pravidel.
    \end{itemize}
    \smallskip
    
    Formule $\varphi$ je \mdef{dokazatelná} v $T$, má-li důkaz z $T$, značíme $T \proves_{H} \varphi$.
    \bigskip
    
    {\bf \myblue{Věta} (o korektnosti Hilbertova kalkulu)}\ \ {\it  Pro každou teorii $T$ a formuli $\varphi$,\ \ $T\proves_H \varphi\ \Rightarrow\ T\models \varphi$.}
    \medskip
    
    {\it \myblue{Důkaz}}
    \begin{itemize}
    \item Je-li $\varphi\in T$ nebo logický axiom, je $T \models \varphi$ (logické axiomy jsou tautologie),
    \item jestliže $T \models \varphi$ a $T \models \varphi \to \psi$, pak $T \models \psi$, \alert{tj. modus ponens je korektní},
    \item jestliže $T \models \varphi$, pak $T \models (\forall x)\varphi$, \alert{tj. pravidlo generalizace je korektní},
    \item tedy každá formule vyskytující se v důkazu z $T$ platí v $T$. $\qed$
    \end{itemize}
    \medskip
    
    {\it \myblue{Poznámka}\ \ Platí i \myblue{úplnost}, tj. $T\models \varphi \Rightarrow T\proves_H \varphi$ pro každou teorii $T$ a formuli $\varphi$.}
% :from slides