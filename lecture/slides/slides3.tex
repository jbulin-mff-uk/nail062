\documentclass{beamer}

%% slide-specific

\usetheme[progressbar=frametitle]{metropolis}
%\usecolortheme{spruce}
%\metroset{block=fill}

% block indentation workaround
% map defaulf block to oldblock
\let\oldblock\block
\let\endoldblock\endblock
% change block by adding smallskip
\renewenvironment{block}[1]
    {\begin{oldblock}{#1}
        \smallskip
    }
    { 
    \end{oldblock}
    }

% define Metropolis colors    
\definecolor{mAlert}{HTML}{EB811B}
\definecolor{mExample}{HTML}{14B03D}
\definecolor{mBlock}{HTML}{23373b}

\usepackage[most]{tcolorbox}

%\newcommand{\myexample}[1]{\leavevmode\textcolor{mExample}{#1}}
\newcommand{\myalert}[1]{
\begin{tcolorbox}[colback=mAlert!10, enhanced, boxrule=0pt, boxsep=-1mm, frame hidden, left=2mm, right=2mm]
    {#1}  
\end{tcolorbox}
}
\newcommand{\myexample}[1]{
\begin{tcolorbox}[colback=mExample!10, enhanced, boxrule=0pt, boxsep=-1mm, frame hidden, left=2mm, right=2mm]
    {#1}  
\end{tcolorbox}
}
\newcommand{\myblock}[1]{
\begin{tcolorbox}[colback=mBlock!10, enhanced, boxrule=0pt, boxsep=-1mm, frame hidden, left=2mm, right=2mm]
    {#1}  
\end{tcolorbox}
}

\newcommand{\mystructure}[1]{\mathcal{#1}}



% \newcommand{\myexamplemath}[1]{
% \begin{tcolorbox}[colback=mExample!10, enhanced, boxrule=0pt, frame hidden]
%     \ensuremath{#1}  
% \end{tcolorbox}
% }


%% packages
\usepackage{amsmath,amssymb,amsthm}
\usepackage{booktabs}
\usepackage[czech]{babel}
\usepackage{enumerate}
\usepackage{forest}
\usepackage{multicol}
% \usepackage{tcolorbox}
\usepackage{tikz}
    \usetikzlibrary{arrows.meta}
%\usepackage[unicode]{hyperref}
\usepackage[utf8]{inputenc}
\usepackage{xfrac}

% %% theorems
% \theoremstyle{plain}
%     \newtheorem{theorem}{Věta}[section]
%     \newtheorem*{theorem-unnumbered}{Věta}
%     \newtheorem{proposition}[theorem]{Tvrzení}
%     \newtheorem{corollary}[theorem]{Důsledek}
%     \newtheorem{lemma}[theorem]{Lemma}
%     \newtheorem{observation}[theorem]{Pozorování}
% \theoremstyle{definition}
%     \newtheorem{definition}[theorem]{Definice}
%     \newtheorem*{algorithm}{Algoritmus}
% \theoremstyle{remark}
%     \newtheorem{remark}[theorem]{Poznámka}
%     \newtheorem{example}[theorem]{Příklad}
%     \newtheorem{exercise}{Cvičení}[chapter]
%     \newtheorem*{solution}{Řešení}

%% macros and definitions
\DeclareMathOperator{\Aut}{Aut}
\DeclareMathOperator{\Conseq}{Csq}
\DeclareMathOperator{\DeLO}{DeLO}
\DeclareMathOperator{\dom}{dom}
\DeclareMathOperator{\Fm}{Fm}
\DeclareMathOperator{\M}{M}
%\DeclareMathOperator{\Proof}{Proof}
\DeclareMathOperator{\rng}{rng}
\DeclareMathOperator{\Term}{Term}
\DeclareMathOperator{\Th}{Th}
\DeclareMathOperator{\Thm}{Thm}
\DeclareMathOperator{\Tree}{Tree}
\DeclareMathOperator{\Var}{Var}
\DeclareMathOperator{\VF}{VF}

\newcommand{\A}{\structure{A}}
\newcommand{\B}{\structure{B}}
\newcommand{\Con}{\mathit{Con}}
\newcommand{\disjointunion}{\mathbin{\dot{\sqcup}}}
\newcommand{\F}{\ensuremath{\mathrm{F}}}
\newcommand{\landsymb}{{\land}}
\newcommand{\lbin}{\mathbin{\square}}
\newcommand{\lbinsymb}{{\lbin}}
\newcommand{\liff}{\mathbin{\leftrightarrow}}
\newcommand{\liffsymb}{{\liff}}
\newcommand{\limplies}{\mathbin{\rightarrow}}
\newcommand{\limpliessymb}{{\limplies}}
\newcommand{\lorsymb}{{\lor}}
\newcommand{\Prf}{\mathit{Prf}}
\newcommand{\proves}{\vdash}
%\newcommand{\structure}[1]{\mathcal{#1}}
\newcommand{\todo}{[TODO]}
\newcommand{\T}{\ensuremath{\mathrm{T}}}
\newcommand{\union}{\mathbin{\cup}}


\title{Třetí přednáška}
\subtitle{NAIL062 Výroková a predikátová logika}
\author{Jakub Bulín (KTIML MFF UK)}
% \institute{KTIML MFF UK}
\date{Zimní semestr 2023}


\begin{document}


\frame{\titlepage}


\begin{frame}{Třetí přednáška}

    \textbf{Program}
        \begin{itemize}
            \item algebra výroků
            \item problém splnitelnosti
            \item 2-SAT a implikační grafů
            \item Horn-SAT a jednotková propagace
            \item algoritmus DPLL
        \end{itemize}        
    

    \textbf{Materiály}

        \href{https://github.com/jbulin-mff-uk/nail062/raw/main/lecture/lecture-notes/lecture-notes.pdf}{\alert{\textbf{Zápisky z přednášky}}}, Sekce 2.5 z Kapitoly 2, Kapitola 3

\end{frame}


\section{2.5 Algebra výroků}


\begin{frame}{Výroky až na ekvivalenci}
    Kolik existuje výroků nad $\mathbb P=\{p,q,r\}$? Nekonečně mnoho. \alert{Až na ekvivalenci?} Tolik, kolik je možných množin modelů: $2^{2^3}=256$.

    \myalertinline{Výroky až na ekvivalenci studujeme pomocí jejich množin modelů.}

    Ekvivalenční třídy: \alert{$\sfrac{\VF_\mathbb P}{\sim}$}, např. \myexampleinline{\small
        $[p\limplies q]_\sim=\{p\limplies q,\neg p \lor q,\dots\}$
    }

    Přiřazení modelů: \myalertinline{
        \small $h:\sfrac{\VF_\mathbb P}{\sim}\to\mathcal P(\M_\mathbb P)$ definované $h([\varphi]_\sim)=\M(\varphi)$
    }
    (je dobře definované, prosté, pro konečný jazyk bijekce)

    Na $\sfrac{\VF_\mathbb P}{\sim}$ zavedeme operace $\neg,\landsymb,\lorsymb$ \alert{pomocí reprezentantů}:
    {\small
    \begin{align*}
        \neg [\varphi]_\sim &=[\neg\varphi]_\sim\\
        [\varphi]_\sim \land [\psi]_\sim &= [\varphi\land\psi]_\sim\\
        [\varphi]_\sim \lor [\psi]_\sim &= [\varphi\lor\psi]_\sim\\
    \end{align*}
    }

    \vspace{-18pt}
    přidáme konstanty {\small $\bot=[\bot]_\sim,\top=[\top]_\sim$}, máme \emph{Booleovu algebru}: \alert{algebru výroků} jazyka $\mathbb P$; totéž relativně k teorii $T$ (\alert{použijeme $\sim_T$})
\end{frame}


\begin{frame}{Algebra výroků}

\myblock{
\alert{Algebra výroků} jazyka $\mathbb P$ resp. teorie $T$: \vspace{-6pt}
\begin{align*}
    \mathbf{AV}_\mathbb P&=\langle \sfrac{\VF_\mathbb P}{\sim}; \neg, \landsymb, \lorsymb, \bot, \top\rangle\\
    \mathbf{AV}_\mathbb P(T)&=\langle \sfrac{\VF_\mathbb P}{\sim_T}; \neg_T, \landsymb_T, \lorsymb_T, \bot_T, \top_T\rangle
\end{align*}
}

přiřazení modelů $h$ je prosté zobrazení algebry výroků jazyka do \alert{potenční algebry} \myalertinline{
    $\mathbf{\mathcal P(\M_\mathbb P)}=\langle \mathcal P(\M_\mathbb P); \overline{\phantom{x}}, \cap, \cup, \emptyset, \M_\mathbb P\rangle$
} \alert{zachovávající} operace a konstanty: $h(\bot)=\emptyset$, $h(\top)=\M_\mathbb P$, a
{\small
\begin{align*}
    h(\neg[\varphi]_\sim)&=\overline{h([\varphi]_\sim)}=\overline{\M(\varphi)}=\M_\mathbb P\setminus\M(\varphi)\\
    h([\varphi]_\sim\land[\psi]_\sim)&=h([\varphi]_\sim)\cap h([\psi]_\sim)=\M(\varphi)\cap\M(\psi)\\
    h([\varphi]_\sim\lor[\psi]_\sim)&=h([\varphi]_\sim)\cup h([\psi]_\sim)=\M(\varphi)\cup\M(\psi)
\end{align*}
}
tj. je to  \alert{homomorfismus} Booleových algeber, a nad konečným jazykem bijekce, tzv. \alert{izomorfismus}; stejně pro algebru výroků teorie

\myblock{
\textbf{Důsledek:}
Pro bezespornou teorii $T$ nad \emph{konečným} jazykem $\mathbb P$ je algebra výroků   $\mathbf{AV}_\mathbb P(T)$ izomorfní potenční algebře $\mathbf{\mathcal P(\M_\mathbb P(T))}$ prostřednictvím zobrazení $h([\varphi]_{\sim_T})=M(T,\varphi)$.
}

\end{frame}


\begin{frame}{Počítání až na ekvivalenci}

    \myblock{
    \textbf{Tvrzení:}
        Mějme $n$-prvkový jazyk $\mathbb P$ a bezespornou teorii $T$ mající právě $k$ modelů. Potom v jazyce $\mathbb P$ existuje \alert{až na ekvivalenci}:
        \begin{itemize}
            \item $2^{2^n}$ výroků (resp. teorií),
            \item $2^{2^n-k}$ výroků pravdivých (resp. lživých) v $T$,
            \item $2^{2^n}-2\cdot 2^{2^n-k}$ výroků nezávislých v $T$,
            \item $2^k$ jednoduchých extenzí teorie $T$ (z toho 1 sporná),
            \item $k$ kompletních jednoduchých extenzí $T$.
        \end{itemize}
        Dále \alert{až na $T$-ekvivalenci} existuje:
        \begin{itemize}
            \item $2^k$ výroků,
            \item $1$ výrok pravdivý v $T$, $1$ lživý v $T$,
            \item $2^k-2$ výroků nezávislých v $T$.
        \end{itemize}
    }

    \textbf{Důkaz:} stačí spočítat možné množiny modelů\hfill\qedsymbol
        
\end{frame}


\section{\sc Kapitola 3: Problém splnitelnosti}


\begin{frame}{Problém splnitelnosti Booleovských formulí}
   
\textbf{Problém SAT:}
\begin{itemize}
    \item vstup: výrok $\varphi$ v CNF
    \item otázka: je $\varphi$ splnitelný? 
\end{itemize}

\alert{univerzální problém}: každou teorii nad konečným jazykem lze převést do CNF

\alert{Cook-Levinova věta}: SAT je NP-úplný (důkaz: formalizuj výpočet nedeterministického Turingova stroje ve výrokové logice)

ale některé \emph{fragmenty} jsou v P, efektivně řešitelné, např. 2-SAT a Horn-SAT (viz Sekce 3.2 a 3.3)

\alert{praktický problém}: moderní \emph{SAT solvery} (viz Sekce 3.1) se používají v řadě odvětví aplikované informatiky, poradí si s obrovskými instancemi
    
\end{frame}


\section{3.1 SAT solvery}


\begin{frame}{SAT solvery}

    \begin{itemize}
        \item existují od 60. let 20. století, v 21. století dramatický rozvoj dnes až $~10^8$ proměnných, viz \href{http://www.satcompetition.org}{\alert{www.satcompetition.org}}.
        \item nejčastěji založeny na jednoduchém \alert{algoritmu DPLL} (viz Sekce 3.4), umí i najít řešení (model)
        \item řada technologií pro efektivnější řešení instancí pocházejících z různých aplikačních domén, heuristiky pro řízení prohledávání (za použití ML, NN) --- desítky tisíc řádků kódu
    \end{itemize}
    
    \textbf{Praktická ukázka:}

    \myexample{\emph{Boardomino:}
    Lze pokrýt šachovnici s chybějícími dvěma protilehlými rohy perfektně pokrýt kostkami domina?
    }

    \vspace{-6pt}
    těžká instance SATu (proč?), jak zakódovat? 
    
    řešič \href{http://www.labri.fr/perso/lsimon/glucose/}{\alert{Glucose}}, formát vstupu: \href{http://people.sc.fsu.edu/~jburkardt/data/cnf/cnf.html}{\alert{DIMACS CNF}}
    
\end{frame}


\section{3.2 2-SAT a implikační graf}


\section{3.3 Horn-SAT a jednotková propagace}


\section{3.4 DPLL algoritmus pro řešení problému SAT}



\end{document}



    \item Nejlepší řešiče pro SAT: \href{http://www.satcompetition.org}{\myblue{\texttt{www.satcompetition.org}}}.
    
