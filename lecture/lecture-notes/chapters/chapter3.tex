\chapter{The Boolean Satisfiability Problem}

\emph{The boolean satisfiability problem}, also known as the \emph{SAT problem}, is the following computational task: The input is a proposition $\varphi$ in CNF (in some reasonable encoding\footnote{For example, the \href{http://people.sc.fsu.edu/~jburkardt/data/cnf/cnf.html}{DIMACS-CNF} format.}), and the goal is to decide whether $\varphi$ is \emph{satisfiable}.\footnote{Beware, in some literature SAT means satisfiability of an \emph{arbitrary} proposition, the restriction of inputs to CNF is done only for the problem $k$-SAT (see below).}

As we demonstrated in the previous chapter, we can convert every proposition, or even every propositional theory in a finite language, into a CNF formula. The SAT problem is thus, in a sense, universal; it answers the question of whether there exists a model.

The well-known Cook-Levin theorem states that the SAT problem is \emph{NP-complete}, meaning it is in the NP class (if an oracle provides the right truth assignment, we can easily verify that all clauses are satisfied) and that every problem in the NP class can be reduced to it in polynomial time (specifically, the computation of a Turing machine can be described using a CNF formula).\footnote{See the course \href{https://is.cuni.cz/studium/predmety/index.php?do=predmet&kod=NTIN090}{NTIN090 Introduction to Complexity and Computability}.}

However, practical SAT solvers can handle instances containing many (up to tens of millions) propositional variables and clauses. In this chapter, we will first demonstrate the practical application of a SAT solver to a `real-life' problem, then show two fragments of the SAT problem, namely \emph{2-SAT} and \emph{Horn-SAT}, for which there are polynomial algorithms, and finally, we will also present the DPLL algorithm, which is the basis of (almost) all SAT solvers. (Later, in Chapter~\ref{chapter:tableau-method-propositional}, we will also see the connection with the \emph{resolution method}.)

\section{SAT Solvers}

The first SAT solvers were developed in the 1960s. They are almost always based on the DPLL (Davis–Putnam–Logemann–Loveland) algorithm, which we will introduce in Section \ref{section:DPLL}, or some of its improvements. After 2000, there has been a somewhat surprising, dramatic development of technologies for SAT solvers, leading to a rapid increase in their utility in various areas of applied computer science.

Modern SAT solvers use a range of technologies for efficiently solving typical instances from various application domains, strategies, and heuristics for exploring the solution space (including, for examle, the use of machine learning and neural networks), and other enhancements. These modern tools typically have several tens of thousands of lines of code. The availability of efficient SAT solvers has significantly impacted developments in areas such as software verification, program analysis, optimization, and artificial intelligence. The best SAT solvers regularly compete in the \href{http://www.satcompetition.org}{SAT competition}.

To try out SAT solving, we will use the solver \href{https://github.com/mi-ki/glucose-syrup}{\texttt{Glucose}}. It accepts input in the simple \href{http://people.sc.fsu.edu/~jburkardt/data/cnf/cnf.html}{DIMACS CNF} format. Let's demonstrate the usage on the following puzzle called \emph{boardomino}:

\begin{example}[Boardomino]
Can a chessboard with two opposite corners cut out be perfectly covered with domino tiles?    
\end{example}

How to formalize this problem? Let us choose propositional variables $h_{i,j}, v_{i,j}$ ($1 \leq i,j \leq n$), where $h_{i,j}$ means ``the left half of a horizontally oriented domino lies at position $(i,j)$'' and similarly $v_{i,j}$ for the top half of a vertically oriented domino. Here $n=8$, but we can try it for other (even) dimensions of the chessboard. Now we axiomatize all required properties:
\begin{itemize}
    \item the upper left and lower right corners are missing: $\neg h_{11}, \neg v_{11}, \neg h_{n,n-1}, \neg v_{n-1,n}$
    \item dominos do not extend beyond the chessboard (to the right or down): $\neg h_{i,n}, \neg v_{n,i}$ for $1 \leq i \leq n$
    \item each square is covered by at least one domino (the first row and column separately):
    \begin{align*}
        h_{i,j-1} \lor h_{i,j} \lor v_{i-1,j} \lor v_{i,j} &\text{ for } 1<i,j\leq n \\
        h_{1,j-1} \lor h_{1,j} \lor v_{1,j} &\text{ for } 1<j\leq n \\
        h_{i,1} \lor v_{i-1,1} \lor v_{i,1} &\text{ for } 1<i\leq n
    \end{align*}
    \item each square is covered by at most one domino (the first row and column separately):
    \begin{align*}
        (\neg h_{i,j-1} \lor \neg h_{i,j}) \land
        (\neg h_{i,j-1} \lor \neg v_{i-1,j}) \land        
        (\neg h_{i,j-1} \lor \neg v_{i,j}) \land &\\
        (\neg h_{i,j} \lor \neg v_{i-1,j}) \land
        (\neg h_{i,j} \lor \neg v_{i,j}) \land
        (\neg v_{i-1,j} \lor \neg v_{i,j}) &\text{ for } 1<i,j\leq n \\
        (\neg h_{1,j-1} \lor \neg h_{1,j}) \land (\neg h_{1,j-1} \lor \neg v_{1,j}) \land (\neg h_{1,j} \lor \neg v_{1,j}) &\text{ for } 1<j\leq n \\
        (\neg h_{i,1} \lor \neg v_{i-1,1}) \land (\neg h_{i,1} \lor \neg v_{i,1}) \land (\neg v_{i-1,1} \lor \neg v_{i,1}) &\text{ for } 1<i\leq n
    \end{align*}    
\end{itemize}

The resulting theory is already in CNF, and we can easily write it in the DIMACS CNF format and solve it using the Glucose solver. In practice, we might program this conversion or use one of the many high-level languages from the \emph{constraint programming} area that allow translation to SAT.

We will see that such instances of the SAT problem will be difficult for the solver, and for relatively small dimensions of the chessboard, we will not get a solution. As mathematicians, we can easily see that there is no solution: Each domino covers one white and one black square, but we removed two white squares, so necessarily two black squares will remain. However, this view is not available in the CNF encoding. It is possible to find partial truth assignments of almost all variables without violating any condition. The solver will have to search almost the entire solution space before proving unsatisfiability.\footnote{Similar properties also hold for the encoding of the \emph{pigeonhole principle} into SAT.} The key insight into SAT solving is that such difficult instances almost never occur in practice.


\section{2-SAT and Implication Graph}

A proposition $\varphi$ is in $k$-CNF if it is in CNF and each clause has at most $k$ literals. The $k$-SAT problem asks whether a given $k$-CNF proposition is satisfiable. For $k \geq 3$, $k$-SAT remains NP-complete; any CNF proposition can be encoded into a 3-CNF proposition:

\begin{exercise}
Show that for any proposition $\varphi$ in CNF, there exists an \emph{equisatisfiable} proposition $\varphi'$ in 3-CNF (i.e., $\varphi$ is satisfiable if and only if $\varphi'$ is satisfiable), which can be constructed in linear time.
\end{exercise}

For the 2-SAT problem, however, there is a polynomial (linear, even) algorithm, which we will now introduce. The algorithm utilizes the notion of \emph{implication graph}. We will demonstrate the procedure with an example:
\begin{example}
    Consider the following 2-CNF proposition $\varphi$:
    $$
    (\neg p_1 \lor p_2) \land (\neg p_2 \lor \neg p_3) \land (p_1\lor p_3) \land (p_3\lor \neg p_4)\land (\neg p_1\lor p_5)\land (p_2\lor p_5)\land p_1\land \neg p_4
    $$
\end{example}

\subsubsection{Implication Graph}

The implication graph of a 2-CNF proposition $\varphi$ is based on the idea that a 2-clause $\ell_1 \lor \ell_2$ (where $\ell_1$ and $\ell_2$ are literals) can be viewed as a pair of implications: $\overline{\ell_1} \rightarrow \ell_2$ and $\overline{\ell_2} \rightarrow \ell_1$.\footnote{In the previous chapter, we expressed $p_1 \rightarrow p_2$ as $\neg p_1 \lor p_2$; here, we are performing the reverse procedure.} For example, from the clause $\neg p_1 \lor p_2$, we get the implications $p_1 \rightarrow p_2$ and $\neg p_2 \rightarrow \neg p_1$. Therefore, if $p_1$ holds in some model, $p_2$ must hold as well, and if $p_2$ does not hold, then $p_1$ must not hold either. A unit clause $\ell$ can also be expressed as an implication, namely $\overline{\ell} \rightarrow \ell$, e.g., from $p_1$ we get $\neg p_1 \rightarrow p_1$.

The implication graph $\mathcal{G}_\varphi$ is thus a directed graph, whose vertices are all the literals (variables from $\Var(\varphi)$ and their negations), and edges are given by the implications described above:
\begin{itemize}
    \item $V(\mathcal{G}_\varphi) = \{p, \neg p \mid p \in \Var(\varphi)\}$,
    \item $E(\mathcal{G}_\varphi) = \{(\overline{\ell_1}, \ell_2), (\overline{\ell_2}, \ell_1) \mid \ell_1 \lor \ell_2 \text{ is a clause in } \varphi\} \cup \{(\overline{\ell}, \ell) \mid \ell \text{ is a unit clause in } \varphi\}$
\end{itemize}

In our example, the set of vertices is
$$
V(\mathcal{G}_\varphi) = \{p_1, p_2, p_3, p_4, p_5, \neg p_1, \neg p_2, \neg p_3, \neg p_4, \neg p_5\}
$$
and the edges are:
\begin{align*}
    E(\mathcal{G}_\varphi) = \{&(p_1, p_2), (\neg p_2, \neg p_1), (p_2, \neg p_3), (p_3, \neg p_2), (\neg p_1, p_3), (\neg p_3, p_1), (\neg p_3, \neg p_4), \\
    &(p_4, p_3), (p_1, p_5), (\neg p_5, \neg p_1), (\neg p_2, p_5), (\neg p_5, p_2), (\neg p_1, p_1), (p_4, \neg p_4)\}
\end{align*}
The resulting graph is shown in Figure \ref{figure:implication-graph}.

\begin{figure} 
    \small   
    \centering
        \begin{tikzpicture}[scale=2,
            every node/.style={circle,fill=gray!5,draw,minimum width=1cm,node distance=2cm}
        ]
        \node[fill=red!10] (-5) {$\neg p_5$};
        \node[fill=blue!10,right of=-5] (-1) {$\neg p_1$};
        \node[fill=blue!10,above of=-1] (-2) {$\neg p_2$};
        \node[fill=blue!10,below of=-1] (3) {$p_3$};    
        \node[fill=green!10,left of=3] (4) {$p_4$};
        \node[fill=blue!70,right of=-1] (1) {$p_1$}; 
        \node[fill=blue!70,above of=1] (2) {$p_2$};
        \node[fill=blue!70,below of=1] (-3) {$\neg p_3$};       
        \node[fill=red!70,right of=1] (5) {$p_5$};            
        \node[fill=green!70,right of=-3] (-4) {$\neg p_4$};
        \draw[-{Latex[length=2.5mm]}] (-5) -- (-1);
        \draw[-{Latex[length=2.5mm]}] (-1) -- (3);
        \draw[-{Latex[length=2.5mm]}] (3) edge[bend right] (-2);
        \draw[-{Latex[length=2.5mm]}] (-2) -- (-1);
        \draw[-{Latex[length=2.5mm]}] (4) -- (3);
        \draw[-{Latex[length=2.5mm]}] (-1) -- (1);
        \draw[-{Latex[length=2.5mm]}] (-3) -- (1);
        \draw[-{Latex[length=2.5mm]}] (2) edge[bend left] (-3);
        \draw[-{Latex[length=2.5mm]}] (1) -- (2);
        \draw[-{Latex[length=2.5mm]}] (-5) edge[bend left=70] (2);
        \draw[-{Latex[length=2.5mm]}] (4) edge[bend right] (-4);
        \draw[-{Latex[length=2.5mm]}] (1) -- (5);
        \draw[-{Latex[length=2.5mm]}] (-2) edge[bend left=70] (5);
        \draw[-{Latex[length=2.5mm]}] (-3) -- (-4);
        \end{tikzpicture}
        \caption{Implication graph $\mathcal{G}_\varphi$. Strongly connected components are distinguished by colors.}\label{figure:implication-graph}
\end{figure}

\subsection{Strongly Connected Components}

We now need to find the strongly connected components\footnote{\emph{Strong connectivity} means that there is a directed path from $u$ to $v$ and from $v$ to $u$, i.e., every two vertices in one component lie in a directed cycle. Conversely, every directed cycle lies within some component.} of this graph. In our example, we get the following components: $C_1=\{\neg p_5\}$, $C_2=\{p_4\}$, $C_3=\{\neg p_1,\neg p_2,p_3\}$, $\overline{C_3}=\{p_1,p_2,\neg p_3\}$, $\overline{C_2}=\{\neg p_4\}$, $\overline{C_1}=\{p_5\}$.

All literals in one component must be assigned the same value. Therefore, if we find a pair of opposite literals in one component, it means that the proposition is unsatisfiable. Otherwise, we can always find a satisfying assignment, as we will prove in Proposition \ref{proposition:2-sat-algorithm}. We need to ensure that no component assigned 1 has an edge leading to a component assigned 0. If we contract the components (and remove loops), the resulting graph $\mathcal{G}_\varphi^\ast$ is acyclic (every cycle was within some component), see Figure~\ref{figure:implication-graph-components}. Thus we can draw it in a \emph{topological ordering} (i.e., an ordering on a line where edges only go to the right), see Figure \ref{figure:implication-graph-topological-order} below.

\begin{figure} 
    \small   
    \centering
        \begin{tikzpicture}[scale=2,
            every node/.style={circle,fill=gray!5,draw,minimum width=1cm,node distance=2cm}
        ]
        \node[fill=red!10] (C1) {$C_1$};
        \node[fill=blue!10,below right of=C1] (C3) {$C_3$};
        \node[fill=green!10,below left of=C3] (C2) {$C_2$};
        \node[fill=blue!70,right of=C3] (barC3) {$\overline{C_3}$}; 
        \node[fill=red!70,above right of=barC3] (barC1) {$\overline{C_1}$};                    
        \node[fill=green!70,below right of=barC3] (barC2) {$\overline{C_2}$};
        \draw[-{Latex[length=2.5mm]}] (C1) -- (C3);
        \draw[-{Latex[length=2.5mm]}] (C2) -- (C3);
        \draw[-{Latex[length=2.5mm]}] (C3) -- (barC3);
        \draw[-{Latex[length=2.5mm]}] (barC3) -- (barC1);
        \draw[-{Latex[length=2.5mm]}] (barC3) -- (barC2);
        \draw[-{Latex[length=2.5mm]}] (C1) edge[bend left] (barC3);
        \draw[-{Latex[length=2.5mm]}] (C3) edge[bend left] (barC1);
        \draw[-{Latex[length=2.5mm]}] (C2) edge[bend right] (barC2);
        \end{tikzpicture}
        \caption{Implication graph $\mathcal{G}_\varphi$. Graph of strongly connected components $\mathcal{G}_\varphi^\ast$.}\label{figure:implication-graph-components}
\end{figure}

When searching for a satisfying assignment (if we are not content with the information that the proposition is satisfiable), we proceed by taking the leftmost unassigned component, assigning it 0, assigning the opposite component 1, and repeating this process until no unassigned components remain. For example, the topological ordering in Figure \ref{figure:implication-graph-topological-order} corresponds to the model $v = (1,1,0,0,1)$.

\begin{figure} 
    \small   
    \centering
        \begin{tikzpicture}[scale=2,
            every node/.style={circle,fill=gray!5,draw,minimum width=1cm,node distance=2cm}
        ]
        \node[fill=red!10,label={below:0}] (C1) {$C_1$};
        \node[fill=green!10,right of=C1,label={below:0}] (C2) {$C_2$};
        \node[fill=blue!10,right of=C2,label={below:0}] (C3) {$C_3$};
        \node[fill=blue!70,right of=C3,label={below:1}] (barC3) {$\overline{C_3}$}; 
        \node[fill=green!70,right of=barC3,label={below:1}] (barC2) {$\overline{C_2}$};
        \node[fill=red!70,right of=barC2,label={below:1}] (barC1) {$\overline{C_1}$};                    
        
        \draw[-{Latex[length=2.5mm]}] (C1) edge[bend left] (C3);
        \draw[-{Latex[length=2.5mm]}] (C2) -- (C3);
        \draw[-{Latex[length=2.5mm]}] (C3) -- (barC3);
        \draw[-{Latex[length=2.5mm]}] (barC3) edge[bend left] (barC1);
        \draw[-{Latex[length=2.5mm]}] (barC3) -- (barC2);
        \draw[-{Latex[length=2.5mm]}] (C1) edge[bend left] (barC3);
        \draw[-{Latex[length=2.5mm]}] (C3) edge[bend left] (barC1);
        \draw[-{Latex[length=2.5mm]}] (C2) edge[bend left] (barC2);
        \end{tikzpicture}
        \caption{Implication graph $\mathcal{G}_\varphi$. Topological ordering of the graph $\mathcal{G}_\varphi^\ast$ and satisfying assignment of the components.}\label{figure:implication-graph-topological-order}
\end{figure}

Finally, we summarize our reasoning in the following proposition:

\begin{proposition}\label{proposition:2-sat-algorithm}
    A proposition $\varphi$ is satisfiable if and only if no strongly connected component in $\mathcal{G}_\varphi$ contains a pair of opposite literals $\ell, \overline{\ell}$.
\end{proposition}

\begin{proof}
    Every model, i.e., valid truth assignment, must assign all literals in the same component the same truth value. (Otherwise, there would necessarily be an implication $\ell_1 \rightarrow \ell_2$ where $\ell_1$ is valid in the model but $\ell_2$ is not.) Therefore, there cannot be opposite literals in one component.

    Conversely, assume that no component contains a pair of opposite literals, and let us show that there exists a model. Let $\mathcal{G}_\varphi^\ast$ be the graph obtained from $\mathcal{G}_\varphi$ by contracting the strongly connected components. This graph is acyclic; choose a topological ordering. We construct a model by choosing the first unassigned component in our topological ordering, assigning all literals in it the value 0, and assigning the opposite literals the value 1. We continue this process until all components are assigned.

    Why does the proposition $\varphi$ hold in the model thus obtained? If it did not, some clause would not hold. A unit clause $\ell$ must hold because there is an edge $\overline{\ell} \rightarrow \ell$ in the graph $\mathcal{G}_\varphi$. The same edge is also in the graph of components, so $\overline{\ell}$ precedes the component containing $\ell$ in the topological ordering. In constructing the model, we must have assigned $\overline{\ell} = 0$, so $\ell = 1$. Similarly, a 2-clause $\ell_1 \lor \ell_2$ must also hold: there are edges $\overline{\ell_1} \rightarrow \ell_2$ and $\overline{\ell_2} \rightarrow \ell_1$. If we assigned $\ell_1$ first, we must have assigned $\overline{\ell_1} = 0$ due to the edge $\overline{\ell_1} \rightarrow \ell_2$, so $\ell_1$ holds. Similarly, if we assigned $\ell_2$ first, $\overline{\ell_2} = 0$ and $\ell_2 = 1$.
\end{proof}

\begin{corollary}
    The 2-SAT problem is solvable in linear time. We can also construct a model in linear time, if one exists.
\end{corollary}

\begin{proof}
The strongly connected components can be found in $\mathcal{O}(|V| + |E|)$ time, and the topological ordering can also be constructed in $\mathcal{O}(|V| + |E|)$ time.
\end{proof}

\begin{exercise}
    Find an unsatisfiable 2-CNF proposition, construct its implication graph, and verify that there is a pair of opposite literals in the same strongly connected component.
\end{exercise}

\begin{exercise}
    Find all topological orderings of the graph $\mathcal{G}_\varphi^\ast$ from the example above and the corresponding models. Think about why we obtain exactly all models of the proposition $\varphi$ in this way.
\end{exercise}

\begin{exercise}
    Explain how to find the components and topological ordering in $\mathcal{O}(|V| + |E|)$ time.
\end{exercise}


\section{Horn-SAT and Unit Propagation}\label{section:horn-sat}

Next, we will introduce another fragment of SAT that can be solved in polynomial time, the so-called \emph{Horn-SAT}, the \emph{Horn satisfiability problem}. A propositional formula is \emph{Horn} (\emph{in Horn form})\footnote{Mathematician Alfred Horn discovered the significance of this form of logical formulas (thus laying the foundation for logic programming) in 1951.} if it is a conjunction of \emph{Horn clauses}, i.e., clauses containing \emph{at most one *positive* literal}. The significance of Horn clauses is evident from their equivalent expression in the form of an implication:
$$
\neg p_1\lor \neg p_2\lor \dots \lor \neg p_n\lor q\ \sim\ (p_1\land p_2\land\dots \land p_n)\limplies q
$$
Horn formulas are thus well suited to model systems where the satisfaction of certain conditions guarantees the satisfaction of another condition. Note that a unit clause $\ell$ is also Horn. In the context of logic programming, it is called a \emph{fact} if the literal is positive, and a \emph{goal} if it is negative.\footnote{Because we are proving by contradiction, more on this in Chapter~\ref{chapter:tableau-method-propositional} and Section~\ref{subsection:resolution-in-prolog}.} Horn formulas with at least one positive and at least one negative literal are \emph{rules}.

\begin{example}
    An example of a proposition that is in CNF but not Horn is, for instance, $(p_1\lor p_2\lor\neg p_3)\land (\neg p_1\lor p_3)$. As an example to illustrate the algorithm, we will use the following Horn formula:
    $$
    \varphi=(\neg p_1\lor p_2)\land(\neg p_1\lor\neg p_2\lor p_3)\land(\neg p_2\lor\neg p_3)\land(\neg p_5\lor \neg p_4)\land p_4
    $$
\end{example}

The polynomial algorithm for solving the Horn-SAT problem is based on the simple idea of \emph{unit propagation}: If our proposition contains a \emph{unit} clause, we know how the propositional variable in this clause must be valued. And this knowledge can be \emph{propagated}---exploited to simplify the proposition.

Our proposition $\varphi$ contains the unit clause $p_4$. We know, therefore, that every model $v\in\M(\varphi)$ must satisfy $v(p_4)=1$. This means that in any model of the proposition $\varphi$:
\begin{itemize}
    \item Every clause containing the positive literal $p_4$ is satisfied, so we can remove it from the proposition,
    \item The negative literal $\neg p_4$ cannot be satisfied, so we can remove it from all clauses that contain it.
\end{itemize}
This step is called \emph{unit propagation}. The result is the following simplified proposition, denoted as $\varphi^{p_4}$ (in general $\varphi^\ell$ if we have a unit clause $\ell$):
$$
\varphi^{p_4}=(\neg p_1\lor p_2)\land(\neg p_1\lor\neg p_2\lor p_3)\land (\neg p_2\lor\neg p_3)\land\neg p_5
$$
\begin{observation}
Note that $\varphi^\ell$ no longer contains the literal $\ell$ nor $\overline{\ell}$, and it is obvious that the models of $\varphi$ are exactly the models of $\{\varphi^{\ell},\ell\}$, i.e., the models of $\varphi^{\ell}$ in the original language $\mathbb P$, in which $\ell$ is valid.
\end{observation}

By unit propagation, we obtained a new unit clause $\neg p_5$ in the proposition $\varphi^{p_4}$, so we can continue by setting $v(p_5)=0$ and performing another unit propagation:
$$
(\varphi^{p_4})^{\neg p_5}=(\neg p_1\lor p_2)\land(\neg p_1\lor\neg p_2\lor p_3)\land(\neg p_2\lor\neg p_3)
$$
The resulting proposition no longer contains a unit clause. This means that each clause contains at least two literals, and at most one of them can be positive! (Here we need the `Horn property' of the proposition.) Because each clause contains a negative literal, it is sufficient to evaluate all remaining variables as 0, and the proposition will be satisfied: $v(p_1)=v(p_2)=v(p_3)=0$. Thus, we get the model $v=(0,0,0,1,0)$.

\begin{example}
    What would happen if the proposition was not satisfiable? Let us look at the proposition 
    $$
    \psi=p\land (\neg p\lor q)\land (\neg q\lor r)\land\neg r
    $$ 
    and perform unit propagation as in the previous example: we have $v(p)=1$ and 
$\psi^p=q\land (\neg q\lor r)\land\neg r$, further $v(q)=1$ and $(\psi^p)^q=r\land\neg r$. 
This proposition is unsatisfiable because it contains a pair of opposite unit clauses. \footnote{In other words, in the next step, we would perform unit propagation on $r$, remove the unit clause $r$, and from the remaining unit clause $\neg r$, we would remove the literal $\neg r$, resulting in an \emph{empty clause}, which is unsatisfiable.}
\end{example}

Let us summarize the algorithm for solving the Horn-SAT problem:

\begin{algorithm}[Horn-SAT]
\textbf{input}: a proposition $\varphi$ in Horn form, \textbf{output}: a model of $\varphi$ or information that $\varphi$ is unsatisfiable
\begin{enumerate}
    \item If $\varphi$ contains a pair of opposite unit clauses $\ell,\overline{\ell}$, it is unsatisfiable.
    \item If $\varphi$ does not contain any unit clause, it is satisfiable, evaluate remaining variables to 0.
    \item If $\varphi$ contains a unit clause $\ell$, evaluate the literal $\ell$ to 1, perform unit propagation, replace $\varphi$ with the proposition $\varphi^\ell$, and return to the beginning.
\end{enumerate}
\end{algorithm}

\begin{proposition}
The algorithm is correct.
\end{proposition}
\begin{proof}
Correctness follows from the Observation and the preceding discussion.
\end{proof}

\begin{corollary}
Horn-SAT can be solved in linear time.
\end{corollary}

\begin{proof}
In each step, it is sufficient to traverse the proposition once, and unit propagation always shortens the proposition. This easily gives a quadratic upper bound, but with a suitable implementation, we can achieve linear time with respect to the length of $\varphi$.
\end{proof}

\begin{exercise}
Propose an implementation of the algorithm for Horn-SAT in linear time.
\end{exercise}

\begin{exercise}
Propose a modification of the Horn-SAT algorithm that finds all models.
\end{exercise}


\section{The DPLL Algorithm}\label{section:DPLL}

To conclude the chapter on the satisfiability problem, we will introduce the most widely used algorithm, by far, for solving the general SAT problem: the DPLL algorithm.\footnote{Named after its creators, Davis-Putnam-Logemann-Loveland, it was invented in 1961.} Although it has exponential complexity in the worst case, it performs very efficiently in practice.

The algorithm uses unit propagation along with the following observation: We say that a literal $\ell$ has \emph{pure occurrence} in $\varphi$ if it occurs in $\varphi$ but the opposite literal $\overline{\ell}$ does not occur in $\varphi$. If we have a literal with pure occurrence, we can set its value to 1, and thus satisfy (and remove) all clauses containing it. If the proposition cannot be simplified neither by unit propagation nor by pure literal, we branch the computation by assigning both possible values to a selected propositional variable.

\begin{algorithm}[DPLL]
    \textbf{input}: a proposition $\varphi$ in CNF, \textbf{output}: a model of $\varphi$ or information that $\varphi$ is unsatisfiable
    \begin{enumerate}                
        \item While $\varphi$ contains a unit clause $\ell$, set the literal $\ell$ to 1, perform unit propagation, and replace $\varphi$ with the proposition $\varphi^\ell$.
        \item While there is a literal $\ell$ with pure occurrence in $\varphi$, set $\ell$ to 1, and remove clauses containing $\ell$.
        \item If $\varphi$ contains no clauses, it is satisfiable.
        \item If $\varphi$ contains an empty clause, it is unsatisfiable.
        \item Otherwise, choose an unassigned propositional variable $p$, and recursively call the algorithm on $\varphi\land p$ and on $\varphi\land \neg p$.
    \end{enumerate}
\end{algorithm}

The algorithm runs in exponential time: the number of branching points in the computation cannot exceed the number of variables. It can be shown that in the worst case, exponential time is indeed needed. The correctness of the algorithm is not difficult to verify.

\begin{proposition}
    The DPLL algorithm solves the SAT problem.
\end{proposition}
    
\begin{example}
    We will demonstrate the algorithm on the following example:
    $$
    (\neg p\lor q\lor \neg r)\land(\neg p\lor\neg q\lor\neg s)\land(p\lor \neg r\lor \neg s)\land(q\lor \neg r\lor s)\land (p\lor s)\land(p\lor\neg s)\land(q\lor s)
    $$
    The proposition does not contain any unit clauses. The literal $\neg r$ has pure occurrence, so we set $v(r)=0$ and remove the clauses containing $\neg r$:
    $$
    (\neg p\lor\neg q\lor\neg s)\land (p\lor s)\land(p\lor\neg s)\land(q\lor s)
    $$
    No other literal has pure occurrence. We therefore recursively run the algorithm:
    \begin{itemize}
        \item[(p=1)] Add the unit clause $p$:
        $$
        (\neg p\lor\neg q\lor\neg s)\land (p\lor s)\land(p\lor\neg s)\land(q\lor s)\land p
        $$
        Set $v(p)=1$ and perform unit propagation: $(\neg q\lor\neg s)\land(q\lor s)$. Now we branch on the variable $q$:
        \begin{itemize}
            \item[(q=1)] $(\neg q\lor\neg s)\land(q\lor s)\land q$. After setting $v(q)=1$ and unit propagation, we get $\neg s$, and after setting $v(s)=0$ and unit propagation, we get a proposition containing no clauses, which is thus satisfiable by the model $(1,1,0,0)$. We already have the answer to the satisfiability problem, so we do not need to complete the other branches of the computation. For illustration, we will do so.
            \item[(q=0)] $(\neg q\lor\neg s)\land(q\lor s)\land \neg q$. By unit propagation with $v(q)=0$, we get $s$, and after setting $v(s)=1$ and unit propagation, we get an empty set of clauses. We obtain the model $(1,0,0,1)$.
        \end{itemize}
        \item[(p=0)] Add the unit clause $\neg p$:
        $$
        (\neg p\lor\neg q\lor\neg s)\land (p\lor s)\land(p\lor\neg s)\land(q\lor s)\land \neg p
        $$
        After performing unit propagation on $\neg p$, we have $s\land \neg s\land(q\lor s)$. After unit propagation on $s$, we have $\square\land q$, where $\square$ is an empty clause. The proposition is thus unsatisfiable, and we do not get any models in this branch. 
    \end{itemize}

We found that the original proposition is satisfiable. We found 2 models: $(1,1,0,0)$ and $(1,0,0,1)$. However, there may be other models, the valuation $v(r)=0$ for the literal $\neg r$ with pure occurrence may not be necessary to satisfy all clauses; this step does not preserve the set of models, only satisfiability.
\end{example}

What's next? The basic DPLL algorithm, which systematically explores the solution space, was enhanced and extended in various ways at the end of 1990s. Let us mention the algorithm called \emph{Conflict-driven clause learning (CDCL)}. It is based on the idea that we can \emph{learn} a new clause from the failure of a branch in the search tree, which prevents future repetitions of the same failure (``conflict''). Additionally, we can backtrack in the tree by more than one level at once (called \emph{back-jumping}) to the point where we started evaluating variables in this new clause. Thus we prevent repeatedly discovering the ``same'' conflict. You can learn much more about SAT solvers in the course \href{https://is.cuni.cz/studium/predmety/index.php?do=predmet&kod=NAIL094}{NAIL094 Decision procedures and SAT/SMT solvers}.
