\documentclass[a4paper,12pt]{article}

%% slide-specific

\usetheme[progressbar=frametitle]{metropolis}
%\usecolortheme{spruce}
%\metroset{block=fill}

% block indentation workaround
% map defaulf block to oldblock
\let\oldblock\block
\let\endoldblock\endblock
% change block by adding smallskip
\renewenvironment{block}[1]
    {\begin{oldblock}{#1}
        \smallskip
    }
    { 
    \end{oldblock}
    }

% define Metropolis colors    
\definecolor{mAlert}{HTML}{EB811B}
\definecolor{mExample}{HTML}{14B03D}
\definecolor{mBlock}{HTML}{23373b}

\usepackage[most]{tcolorbox}

%\newcommand{\myexample}[1]{\leavevmode\textcolor{mExample}{#1}}
\newcommand{\myalert}[1]{
\begin{tcolorbox}[colback=mAlert!10, enhanced, boxrule=0pt, boxsep=-1mm, frame hidden, left=2mm, right=2mm]
    {#1}  
\end{tcolorbox}
}
\newcommand{\myexample}[1]{
\begin{tcolorbox}[colback=mExample!10, enhanced, boxrule=0pt, boxsep=-1mm, frame hidden, left=2mm, right=2mm]
    {#1}  
\end{tcolorbox}
}
\newcommand{\myblock}[1]{
\begin{tcolorbox}[colback=mBlock!10, enhanced, boxrule=0pt, boxsep=-1mm, frame hidden, left=2mm, right=2mm]
    {#1}  
\end{tcolorbox}
}

\newcommand{\mystructure}[1]{\mathcal{#1}}



% \newcommand{\myexamplemath}[1]{
% \begin{tcolorbox}[colback=mExample!10, enhanced, boxrule=0pt, frame hidden]
%     \ensuremath{#1}  
% \end{tcolorbox}
% }


%% packages
\usepackage{amsmath,amssymb,amsthm}
\usepackage{booktabs}
\usepackage[czech]{babel}
\usepackage{enumerate}
\usepackage{forest}
\usepackage{multicol}
% \usepackage{tcolorbox}
\usepackage{tikz}
    \usetikzlibrary{arrows.meta}
%\usepackage[unicode]{hyperref}
\usepackage[utf8]{inputenc}
\usepackage{xfrac}

% %% theorems
% \theoremstyle{plain}
%     \newtheorem{theorem}{Věta}[section]
%     \newtheorem*{theorem-unnumbered}{Věta}
%     \newtheorem{proposition}[theorem]{Tvrzení}
%     \newtheorem{corollary}[theorem]{Důsledek}
%     \newtheorem{lemma}[theorem]{Lemma}
%     \newtheorem{observation}[theorem]{Pozorování}
% \theoremstyle{definition}
%     \newtheorem{definition}[theorem]{Definice}
%     \newtheorem*{algorithm}{Algoritmus}
% \theoremstyle{remark}
%     \newtheorem{remark}[theorem]{Poznámka}
%     \newtheorem{example}[theorem]{Příklad}
%     \newtheorem{exercise}{Cvičení}[chapter]
%     \newtheorem*{solution}{Řešení}

%% macros and definitions
\DeclareMathOperator{\Aut}{Aut}
\DeclareMathOperator{\Conseq}{Csq}
\DeclareMathOperator{\DeLO}{DeLO}
\DeclareMathOperator{\dom}{dom}
\DeclareMathOperator{\Fm}{Fm}
\DeclareMathOperator{\M}{M}
%\DeclareMathOperator{\Proof}{Proof}
\DeclareMathOperator{\rng}{rng}
\DeclareMathOperator{\Term}{Term}
\DeclareMathOperator{\Th}{Th}
\DeclareMathOperator{\Thm}{Thm}
\DeclareMathOperator{\Tree}{Tree}
\DeclareMathOperator{\Var}{Var}
\DeclareMathOperator{\VF}{VF}

\newcommand{\A}{\structure{A}}
\newcommand{\B}{\structure{B}}
\newcommand{\Con}{\mathit{Con}}
\newcommand{\disjointunion}{\mathbin{\dot{\sqcup}}}
\newcommand{\F}{\ensuremath{\mathrm{F}}}
\newcommand{\landsymb}{{\land}}
\newcommand{\lbin}{\mathbin{\square}}
\newcommand{\lbinsymb}{{\lbin}}
\newcommand{\liff}{\mathbin{\leftrightarrow}}
\newcommand{\liffsymb}{{\liff}}
\newcommand{\limplies}{\mathbin{\rightarrow}}
\newcommand{\limpliessymb}{{\limplies}}
\newcommand{\lorsymb}{{\lor}}
\newcommand{\Prf}{\mathit{Prf}}
\newcommand{\proves}{\vdash}
%\newcommand{\structure}[1]{\mathcal{#1}}
\newcommand{\todo}{[TODO]}
\newcommand{\T}{\ensuremath{\mathrm{T}}}
\newcommand{\union}{\mathbin{\cup}}



\begin{document}

\section*{NAIL062 V\&P Logika: 5. cvičení}
% po 4. přednášce

% 2023: [4 z 4. cvičení], 1a, 2, 3, 5ab, 6ab, 7ab

\textbf{Témata:} 
Algoritmus DPLL. Kódování problémů do SAT. Tablo metoda ve výrokové logice. 


\medskip\begin{problem}
    Pomocí algoritmu DPLL rozhodněte, zda je následující CNF formule splnitelná.
    \begin{enumerate}
        \item $$ (\neg p_1 \lor \neg p_2)\land( \neg p_1 \lor p_2)\land( p_1 \lor \neg p_2)\land( p_2 \lor \neg p_3)\land( p_1 \lor p_3)$$
        \item \begin{align*}
            &(\neg p_1 \lor p_3 \lor p_4)\land( \neg p_2 \lor p_6 \lor p_4)\land( \neg p_2 \lor \neg p_6 \lor \neg p_3)\land(
                \neg p_4 \lor \neg p_2)\land \\ & ( p_2 \lor \neg p_3 \lor \neg p_1)\land ( p_2 \lor p_6 \lor p_3)\land
                ( p_2 \lor \neg p_6 \lor \neg p_4)\land
                ( p_1 \lor p_5)\land \\ & ( p_1 \lor p_6)\land(
                \neg p_6 \lor p_3 \lor \neg p_5)\land( p_1 \lor \neg p_3 \lor \neg p_5)    
        \end{align*}
    \end{enumerate}
\end{problem}


% \medskip\begin{problem}
%     Lze šachovnici $4\times 4$ bez dvou protilehlých rohů perfektně pokrýt kostkami domina? Zakódujte tento problém do SAT. Zobecněte na všechna sudá~$n$.
%     \end{problem}
        
        
    \medskip\begin{problem}
        Lze obarvit čísla od 1 do $n$ dvěma barvami tak, že neexistuje monochromatické řešení rovnice
        $a+b=c$ pro žádná $1\leq a<b<c\leq n$? Sestrojte výrokovou formuli $\varphi_n$ v CNF která je splnitelná, právě když to lze. Zkuste nejprve $n=8$.
        
        Zkuste si doma: Napište skript generující $\varphi_n$ v DIMACS CNF formátu. Použijte SAT solver k nalezení nejmenšího $n$ pro které takové obarvení neexistuje (tj. každé 2-obarvení obsahuje monochromatickou trojici $a<b<c$ takovou, že $a+b=c$).
    \end{problem}
    
        
    \medskip\begin{problem}
        Zakódujte problém setřídění trojice celých čísel do SAT.
    \end{problem}
        
            
    \medskip\begin{problem}
    Věta o čtyřech barvách říká, že následující mapy lze obarvit 4 barvami tak, že žádné dva sousedící regiony nemají stejnou barvu. Najděte takové obarvení pomocí SAT solveru.
    \begin{multicols}{2}
    \begin{enumerate}
        \item Mapa krajů Česka  
        
        \vfill \includegraphics[width=0.5\textwidth]{files/map-coloring-czechia.png} \vfill
        
        \item Těžší instance  
        
        \vfill \includegraphics[width=0.33\textwidth]{files/map-coloring-hard.png} \vfill
    \end{enumerate}
    \end{multicols}
    \end{problem}
    
    
\newpage   


\medskip\begin{problem}
    Pomocí tablo metody dokažte následující výroky:
    \begin{enumerate}
    \item $(p\to (q \to q))$
    \item $p \leftrightarrow \neg \neg  p$
    \item $\neg (p \vee q) \leftrightarrow (\neg p \wedge \neg q)$
    \item $(p \to q) \leftrightarrow (\neg q \to \neg p)$    
    \end{enumerate}
\end{problem} 
   

\medskip\begin{problem}
    Pomocí tablo metody dokažte nebo najděte protipříklad ve formě \emph{kanonického} modelu pro bezespornou větev.
    \begin{enumerate}
    \item $\{ \neg q,\ p \vee q\} \models p$
    \item $\{ q \to p,\ r \to q,\ (r \to p) \to s\} \models s$
    \item $\{ p \to r,\ p \vee q,\ \neg s \to \neg q\} \models r \to s$
    \end{enumerate}
\end{problem}
  

\medskip\begin{problem}
    Pomocí tablo metody určete všechny modely následujících teorií:
    \begin{enumerate}
    \item $\{(\neg p \vee q) \to (\neg q \wedge r)\}$
    \item $\{\neg q \to (\neg p \vee q),\ \neg p \to q,\ r \to q\}$
    \item $\{ q \to p,\ r \to q,\ (r \to p) \to s\}$
    \end{enumerate}
\end{problem}


\end{document}