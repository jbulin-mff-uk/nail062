\documentclass{beamer}

%% slide-specific

\usetheme[progressbar=frametitle]{metropolis}
%\usecolortheme{spruce}
%\metroset{block=fill}

% define Metropolis colors    
\definecolor{mAlert}{HTML}{EB811B}
\definecolor{mExample}{HTML}{14B03D}
\definecolor{mBlock}{HTML}{23373b}

% my blocks
\setlength\fboxsep{0pt}%

\newcommand{\myblock}[1]{\colorbox{mBlock!12}{\begin{minipage}{\linewidth}#1\end{minipage}}}
\newcommand{\myblockmath}[1]{\colorbox{mBlock!12}{\begin{minipage}{\linewidth}\vspace{-6pt}#1\end{minipage}}}
\newcommand{\myblockamsmath}[1]{\colorbox{mBlock!12}{\begin{minipage}{\linewidth}\vspace{-6pt}#1\end{minipage}}}
\newcommand{\myblockinline}[1]{\colorbox{mBlock!12}{#1}}
\newcommand{\myexample}[1]{\colorbox{mExample!12}{\begin{minipage}{\linewidth}#1\end{minipage}}}
\newcommand{\myexamplemath}[1]{\colorbox{mExample!12}{\begin{minipage}{\linewidth}\vspace{-6pt}#1\end{minipage}}}
\newcommand{\myexampleamsmath}[1]{\colorbox{mExample!12}{\begin{minipage}{\linewidth}\vspace{-18pt}#1\end{minipage}}}
\newcommand{\myexampleinline}[1]{\colorbox{mExample!12}{#1}}
\newcommand{\myalert}[1]{\colorbox{mAlert!12}{\begin{minipage}{\linewidth}#1\end{minipage}}}
\newcommand{\myalertmath}[1]{\colorbox{mAlert!12}{\begin{minipage}{\linewidth}\vspace{-6pt}#1\end{minipage}}}
\newcommand{\myalertamsmath}[1]{\colorbox{mAlert!12}{\begin{minipage}{\linewidth}\vspace{-18pt}#1\end{minipage}}}
\newcommand{\myalertinline}[1]{\colorbox{mAlert!12}{#1}}

%% other
\newcommand{\mystructure}[1]{\mathcal{#1}}


%% packages
\usepackage{amsmath,amssymb,amsthm}
\usepackage{bookmark}
\usepackage{booktabs}
\usepackage{cancel}
\usepackage[czech]{babel}
\usepackage{enumerate}
\usepackage[T1]{fontenc}
\usepackage{forest}
\usepackage{lmodern}
\usepackage{multicol}
% \usepackage{tcolorbox}
\usepackage{tikz}
    \usetikzlibrary{arrows.meta}
%\usepackage[unicode]{hyperref}
\usepackage[utf8x]{inputenc}
\usepackage{xfrac}


% %% theorems
% \theoremstyle{plain}
%     \newtheorem{theorem}{Věta}[section]
%     \newtheorem*{theorem-unnumbered}{Věta}
%     \newtheorem{proposition}[theorem]{Tvrzení}
%     \newtheorem{corollary}[theorem]{Důsledek}
%     \newtheorem{lemma}[theorem]{Lemma}
%     \newtheorem{observation}[theorem]{Pozorování}
% \theoremstyle{definition}
%     \newtheorem{definition}[theorem]{Definice}
%     \newtheorem*{algorithm}{Algoritmus}
% \theoremstyle{remark}
%     \newtheorem{remark}[theorem]{Poznámka}
%     \newtheorem{example}[theorem]{Příklad}
%     \newtheorem{exercise}{Cvičení}[chapter]
%     \newtheorem*{solution}{Řešení}

%% macros and definitions
\DeclareMathOperator{\Aut}{Aut}
\DeclareMathOperator{\Conseq}{Csq}
\DeclareMathOperator{\DeLO}{DeLO}
\DeclareMathOperator{\dom}{dom}
\DeclareMathOperator{\Fm}{Fm}
\DeclareMathOperator{\M}{M}
%\DeclareMathOperator{\Proof}{Proof}
\DeclareMathOperator{\rng}{rng}
\DeclareMathOperator{\Term}{Term}
\DeclareMathOperator{\Th}{Th}
\DeclareMathOperator{\Thm}{Thm}
\DeclareMathOperator{\Tree}{Tree}
\DeclareMathOperator{\Var}{Var}
\DeclareMathOperator{\VF}{VF}

\newcommand{\A}{\mystructure{A}}
\newcommand{\B}{\mystructure{B}}
\newcommand{\Con}{\mathit{Con}}
\newcommand{\disjointunion}{\mathbin{\dot{\sqcup}}}
\newcommand{\F}{\ensuremath{\mathrm{F}}}
\newcommand{\landsymb}{{\land}}
\newcommand{\lbin}{\mathbin{\square}}
\newcommand{\lbinsymb}{{\lbin}}
\newcommand{\liff}{\mathbin{\leftrightarrow}}
\newcommand{\liffsymb}{{\liff}}
\newcommand{\limplies}{\mathbin{\rightarrow}}
\newcommand{\limpliessymb}{{\limplies}}
\newcommand{\lorsymb}{{\lor}}
\newcommand{\Prf}{\mathit{Prf}}
%\newcommand{\structure}[1]{\mathcal{#1}}
\newcommand{\todo}{[TODO]}
\newcommand{\T}{\ensuremath{\mathrm{T}}}
\newcommand{\union}{\mathbin{\cup}}

\DeclareRobustCommand\proves{\mathrel{|}\joinrel\mkern-.5mu\mathrel{-}}

\title{Devátá přednáška}
\subtitle{NAIL062 Výroková a predikátová logika}
\author{Jakub Bulín (KTIML MFF UK)}
% \institute{KTIML MFF UK}
\date{Zimní semestr 2025}


\begin{document}


\maketitle


\begin{frame}{Devátá přednáška}

    \textbf{Program}
        \begin{itemize}
            \item úvod do rezoluce v predikátové logice
            \item skolemizace
            \item grounding, Herbrandova věta
        \end{itemize}

    \textbf{Materiály}

        \href{https://github.com/jbulin-mff-uk/nail062/raw/main/lecture/lecture-notes/lecture-notes.pdf}{\alert{\textbf{Zápisky z přednášky}}}, Sekce 8.1-8.3 z Kapitoly 8

\end{frame}


\section{\sc Kapitola 8: Rezoluce v predikátové logice}


\section{8.1 Úvod}


\begin{frame}{Rezoluce v predikátové logice}

    $T\models\varphi$? {\Large$\rightsquigarrow$} $T\cup\{\neg \varphi\}$ {\Large$\rightsquigarrow$} CNF formule $S$ {\Large$\rightsquigarrow$} rezoluční zamítnutí

    \begin{itemize}[<+->]
        \item[] (pozor: $\varphi$ musí být \alert{sentence}!)        
        \item \alert{literál} je \alert{atomická formule} $R(t_1,\dots,t_n)$ nebo její negace
        \item \alert{klauzule} je konečná množina literálů, \alert{formule} množina klauzulí
        \item otevřenou formuli snadno převedeme do CNF, i univerzální kvantifikátor na začátku:{\small\myexampleinline{
            $(\forall x)(P(x)\lor \neg Q(x))\sim \{P(x),\neg Q(x)\}$
            }}
        \item co s existenčními kvantifikátory? nové symboly pro `svědky'
        {\small\myexampleinline{
        $(\exists x)(P(x)\lor \neg Q(x))\rightsquigarrow \{P(c),\neg Q(c)\}$
        }} ``\alert{skolemizace}''
        \item není ekvivalentní, ale zachovává \alert{[ne]splnitelnost}, to nám stačí
        \item rezoluční krok? literály nemusí být stejné, stačí \alert{unifikovatelné}
        {\myexampleinline{
        z klauzulí $\{P(x),\neg Q(x)\}$ a $\{Q(f(c))\}$ odvodíme $\{P(f(c))\}$
        }}
        \item \alert{unifikace} je substituce $\{x/f(c)\}$
    \end{itemize}

\end{frame}


\begin{frame}{Příklady}
    
    \pause
    1. \myexampleinline{
        $T=\{(\forall x)P(x),(\forall x)(P(x)\limplies Q(x))\}$, $\varphi=(\exists x)Q(x)$
    }
    \pause
    $$\neg\varphi=\neg(\exists x)Q(x)\sim(\forall x)\neg Q(x)\sim\neg Q(x)$$ 
    \pause
    $T\cup\{\neg \varphi\}$ je \alert{ekvivalentní} \myalertinline{\small
        $S = \{\{P(x)\},\{\neg P(x),Q(x)\},\{\neg Q(x)\}\}$
    }
    \pause
    rezoluční zamítnutí: představte si $p$ místo $P(x)$, $q$ místo $Q(x)$

    \pause
    2. \myexampleinline{
        $T=\{(\forall x)(\exists y)R(x,y),R(x,y)\limplies Q(x)\}$, $\varphi=(\exists x)Q(x)$
    }
    \pause
    $$
    T\cup\{\neg \varphi\}\sim\{(\forall x)(\exists y)R(x,y),\neg R(x,y)\lor Q(x),\neg Q(x)\}
    $$
    \pause
    formuli \alert{$(\forall x)(\exists y)R(x,y)$} nahradíme \alert{$R(x,f(x))$}, kde $f$ je nový unární funkční symbol (reprezentuje \alert{výběr svědka}):

    \pause
    \myalertmath{
    $$
    S = \{\{R(x,f(x))\},\{\neg R(x,y),Q(x)\},\{\neg Q(x)\}\}
    $$
    }

    \pause
    není ekvivalentní, ale \alert{ekvisplnitelná} (zde obě nesplnitelné), vidíme po \alert{substituci $y/f(x)$}, která \alert{unifikuje} $R(x,f(x))$ a $R(x,y)$
 
\end{frame}


\begin{frame}{Unifikace}

    \pause
    \myexamplemath{
        $$
        S = \{\{R(x,f(x))\},\{\neg R(x,y),Q(x)\},\{\neg Q(x)\}\}
        $$
    }

    \pause
    \begin{itemize}[<+->]
        \item na \alert{úrovni výrokové logiky} (\alert{ground level}):
        \myalertmath{ 
        $$
        \{\{r\},\{\neg p,q\},\{\neg q,p\},\{\neg q\}\}
        $$
        }
        není nesplnitelné! musíme využít, že $R(x,f(x))$ a $R(x,y)$ mají `podobnou strukturu' (jsou \alert{unifikovatelné})
        \item klauzule $\{\neg R(x,y),Q(x)\}$ platí i po provedení libovolné substituce: \alert{$\{\neg R(x/t),Q(x/t)\}$} je důsledek $S$ pro lib. term $t$
        \item představme si `přidání' všech takto získaných klauzulí do $S$: potom už je na ground level nesplnitelné (ale nekonečné)
        \item \alert{unifikační algoritmus} nám dá správnou substituci \myalertinline{
            $y/f(x)$
        }
        \item zahrneme už do \alert{rezolučního pravidla}, tedy \alert{rezolventou} klauzulí $\{P(c)\}$ a $\{\neg P(x),Q(x)\}$ bude klauzule $\{Q(c)\}$.
    \end{itemize}
 
\end{frame}

\begin{frame}{Rezoluční pravidlo}
    
    \begin{itemize}[<+->]

        \item zahrnuje aplikaci unifikace
        \item lze vybrat \alert{více literálů najednou}, ale musí být unifikovatelné:
        
        \bigskip

        např. z \myexampleinline{
        $\{R(x,f(x)),R(g(y),z)\},\{\neg R(g(c),u),P(u)\}$
        } odvodíme rezolventu \myalertinline{
            $\{P(f(g(c)))\}$
        } za použití \alert{unifikace} 

        \medskip

        \myalertmath{
            $$
            \{x/g(c),y/c,z/f(g(c)),u/f(g(c))\}
            $$
        }

        \bigskip

        \item budeme vyžadovat disjunktní množiny proměnných v klauzulích; lze přejmenovat, proměnné mají \alert{lokální význam}:
        $$
        \models(\forall x)(\psi \land \chi) \leftrightarrow (\forall x)\psi \land (\forall x)\chi
        $$
        
    \end{itemize}

\end{frame}


\section{8.2 Skolemizace}


\begin{frame}{Ekvisplnitelná otevřená teorie}

    \begin{itemize}
        \item teorie $T$ v jazyce $L$ a $T'$ v (ne nutně stejném) jazyce $L'$ jsou  \alert{ekvisplnitelné}, pokud platí: $T$ má model $\Leftrightarrow$ $T'$ má model \pause
        \item zajímá nás jen [ne]splnitelnost (dokazujeme sporem)\pause
        \item pro převod do CNF a rezoluci potřebujeme otevřené formule\pause 
    \end{itemize}

    \myblock{
    \textbf{Cíl:} Ke každé teorii $T$ sestrojíme \alert{ekvisplnitelnou, otevřenou} $T'$.
    }

    \pause
    \begin{enumerate}
        \item převod do \alert{prenexní normální formy} (vytkneme kvantifikátory)\pause
        \item nahradíme generálními uzávěry \myalertinline{(potřebujeme sentence!)}\pause
        \item nahradíme sentence \alert{Skolemovými variantami} (odstranění $\exists$)\pause
        \item odstraníme zbývající $\forall$, máme otevřené formule
    \end{enumerate}

\end{frame}


\begin{frame}{Prenexní normální forma}

    \pause
    \myblock{Formule $\varphi$ je v \alert{prenexní normální formě (PNF)}, je-li následujícího tvaru, kde $Q_i\in\{\forall,\exists\}$ a formule $\varphi'$ je otevřená:
    $$
    (Q_1x_1)\dots(Q_nx_n)\varphi'
    $$
    \vspace{-16pt}
    }

    \pause
    \begin{itemize}
        \item $(Q_1x_1)\dots(Q_nx_n)$ je \alert{kvantifikátorový prefix}, $\varphi'$ \alert{otevřené jádro}\pause
        \item \alert{univerzální} formule: v PNF a všechny kvantifikátory jsou $\forall$
    \end{itemize}

    \pause
    \myblock{
    \textbf{Tvrzení:}
        Ke každé formuli $\varphi$ existuje \alert{ekvivalentní} formule v PNF.
    }    
    
    \pause
    \textbf{Důkaz:} nahrazujeme podformule ekvivalentními s cílem posunout kvantifikátory blíž kořeni $\Tree(\varphi)$, dle pravidel z násl. Lemmatu.\hfill\qedsymbol

    \pause
    \myblock{
    \textbf{Důsledek:} Existuje i ekvivalentní PNF \alert{sentence} (generální uzávěr).
    }

\end{frame}


\begin{frame}{Pravidla vytýkání kvantifikátorů}

    \pause
    \myblock{
    \textbf{Lemma:} Označme $\overline{Q}$ opačný kvantifikátor ke $Q$. Jsou-li $\varphi$ a $\psi$ formule, kde \alert{$x$ není volná v $\psi$}, potom:
    \begin{align*}
        \neg (Qx)\varphi\ &\sim\ (\overline{Q}x)\neg\varphi\\
        (Qx)\varphi \land \psi\ &\sim\ (Qx)(\varphi \land \psi)\\
        (Qx)\varphi \lor \psi\ &\sim\ (Qx)(\varphi \lor \psi)\\
        (Qx)\varphi \limplies \psi\ &\sim\ \alert{(\overline{Q}x)}(\varphi \limplies \psi)\\
        \psi \limplies (Qx)\varphi\ &\sim\ (Qx)(\psi \limplies \varphi)
    \end{align*}
    }

    \pause
    \textbf{Důkaz:} snadno ověříme sémanticky, nebo tablo metodou (potom ale nejsou-li sentence, musíme nahradit generálními uzávěry)\hfill\qedsymbol

    \smallskip

    \pause
    \textbf{Pozorování:} Nahradíme-li ve $\varphi$ podformuli $\psi$ ekvivalentní $\psi'$, je i výsledná formule $\varphi'$ ekvivalentní $\varphi$. (Připomeňme: $\varphi\sim\varphi'$ právě když mají stejné modely, tj. $\models\varphi\liff\varphi'$)
    
\end{frame}


\begin{frame}{Převod do PNF: příklad}

    \pause
    \myexamplemath{
        $$
        (\forall z)P(x,z)\land P(y,z)\ \limplies\ \neg(\exists x)P(x,y)
        $$
    }      
    \begin{align*}
        \sim &\ 
        \alert{(\forall u)} P(x,\alert{u})\land P(y,z)\ \limplies\ (\forall x)\neg P(x,y)\\ \sim &\ 
        (\forall u)(P(x,u)\land P(y,z))\ \limplies\ \alert{(\forall v)}\neg P(\alert{v},y)\\ \sim &\ 
        (\exists u)( P(x,u)\land P(y,z)\ \limplies\ (\forall v)\neg P(v,y))\\ \sim &\ 
        (\exists u)(\forall v)( P(x,u)\land P(y,z)\ \limplies\ \neg P(v,y))
    \end{align*}
    
    \pause
    \begin{itemize}[<+->]
        \item v prvním kroku přejmenujeme $z$ na $u$, \alert{nesmí být volná v $P(y,z)$}
        \item podobně ve druhém kroku $x$ na $v$
        \item která pravidla používáme? sledujte postup na stromu formule
        \item chceme-li sentenci:
        $$
        \alert{(\forall x)(\forall y)(\forall z)}(\exists u)(\forall v)( P(x,u)\land P(y,z)\limplies\neg P(v,y))
        $$
    \end{itemize}

\end{frame}


\begin{frame}{Poznámky}

    \begin{enumerate}
        \pause
        \item proč se při vytýkání z \alert{antecedentu} mění kvantifikátor?
        \begin{align*}
            \alert{(Qx)\varphi \limplies \psi}&\sim\neg(Qx)\varphi \lor\psi\\&\sim\alert{(\overline{Q}x)}(\neg\varphi)\lor\psi\\
            &\sim(\overline{Q}x)(\neg\varphi\lor\psi)\sim \alert{(\overline{Q}x)(\varphi \limplies \psi)}    
        \end{align*}

        \medskip

        \pause
        \item proč nesmí být $x$ volná v $\psi$? neplatilo by, např:
        \myexamplemath{
        $$
        (\exists x)P(x)\land Q(x)\ \not\sim\ (\exists x)(P(x)\land Q(x))
        $$
        }
        \pause
        musíme přejmenovat vázanou proměnnou $x$ na novou: 
        
        \myexampleinline{
        $
        (\exists x)P(x)\land Q(x)\ \sim\ (\exists y)P(y)\land Q(x) \sim (\exists y)(P(y)\land Q(x))
        $
        }

        \medskip

        \pause
        \item PNF není jednoznačná, lze vytýkat v různém pořadí; lepší je nejprve vytknout ty, \alert{ze kterých se nakonec stanou existenční:}
        
        \medskip

        \myalertmath{
            $$
            (\exists y)(\forall x)\varphi(x,y)\text{ je lepší než }(\forall x)(\exists y)\varphi(x,y)
            $$
        }
        
        \medskip

        (protože ``$y$ nezávisí na $x$'')
    \end{enumerate}

\end{frame}


\begin{frame}{Skolemova varianta}

    \pause
    Je-li PNF sentence \alert{univerzální}, tvaru $(\forall x_1)\dots(\forall x_n)\psi(x_1,\dots,x_n)$, nahradíme otevřeným jádrem $\psi$. Jinak musíme provést \alert{skolemizaci}:

    \medskip

    \pause
    \myblock{
    Buď $\varphi$ $L$-\alert{sentence} v PNF, všechny vázané proměnné různé. Nechť
    \begin{itemize}
        \item existenční kvantifikátory jsou $(\exists y_1),\dots,(\exists y_n)$ (v tom pořadí)
        \item pro každé $i$ jsou $(\forall x_1),\dots,(\forall x_{n_i})$ právě všechny univerzální kvantifikátory předcházející $(\exists y_i)$ v prefixu $\varphi$
    \end{itemize}
    \pause
    Buď $L'$ rozšíření $L$ o \alert{nové} funkční symboly $f_1,\dots,f_n$, kde $f_i$ je $n_i$-ární. \alert{Skolemova varianta} $\varphi$ je $L'$-sentence $\varphi_S$ vzniklá \myalertinline{
        odstraněním $(\exists y_i)$
    } a substitucí termu \myalertinline{
        $f_i(x_1,\dots,x_{n_i})$ za $y_i$
    }, postupně pro $i=1,\dots,n$. 
    }

    \medskip

    \pause
    \myexampleamsmath{
    \begin{align*}
        \varphi&=\textcolor{red}{(\exists y_1)}(\forall x_1)(\forall x_2)\textcolor{blue}{(\exists y_2)}(\forall x_3)\ R(\textcolor{red}{y_1},x_1,x_2,\textcolor{blue}{y_2},x_3)\\
        \varphi_S&=(\forall x_1)(\forall x_2)(\forall x_3)\ R(\textcolor{red}{f_1},x_1,x_2,\textcolor{blue}{f_2(x_1,x_2)},x_3) 
    \end{align*}    
    }

    \pause
    \begin{itemize}
        \item \myalertinline{musí být sentence!} pro \myexampleinline{
            $(\exists y)E(x,y)$} ne \xcancel{$E(x,c)$} ale $E(x,f(x))$\pause
        \item \myalertinline{nové symboly!} (jedinou rolí je reprezentovat `svědky' ve $\varphi$)
    \end{itemize}

\end{frame}


\begin{frame}{Je to konzervativní extenze}

    \pause
    \myblock{
    \textbf{Lemma:} Buď $\varphi$ $L$-sentence $(\forall x_1)\dots(\forall x_n)(\exists y)\psi$, $f$ nový funkční symbol, a $\varphi'$ sentence 
    $(\forall x_1)\dots(\forall x_n)\psi(y/f(x_1,\dots,x_n))$. Potom:
    \begin{enumerate}[(i)]
        \item $L$-redukt každého modelu $\varphi'$ je modelem $\varphi$, a 
        \item každý model $\varphi$ lze expandovat na model $\varphi'$.
    \end{enumerate}
    }

    \pause
    \textbf{Důkaz:} \pause \alert{(i)} Buď $\A'$ model $\varphi'$, $\A$ jeho $L$-redukt, $e:\Var\to\A$. $\A\models\varphi[e]$ platí neboť $\A\models\psi[e(y/a)]$ pro $a=(f(x_1,\dots,x_n))^{\A'}[e]$.
    
    \pause
    \alert{(ii)} Protože $\A\models\varphi$, existuje funkce $f^A:A^n\to A$, že pro každé ohodnocení $e$ platí $\A\models \psi[e(y/a)]$ pro $a=f^A(e(x_1),\dots,e(x_n))$. To znamená, že expanze o funkci $f^A$ splňuje $\varphi'$.\hfill\qedsymbol 
    \begin{itemize}
        \item \pause říká, že $\{\varphi'\}$ je konzervativní extenze $\{\varphi\}$, opakovaná aplikace dává \alert{Skolemovu větu} (výsledek skolemizace je otevřená konzervativní extenze, speciálně je ekvisplnitelná)  
        \item \pause expanze v (ii) není jednoznačná (na rozdíl od extenze o definici nového funkčního symbolu)      
    \end{itemize}
   
\end{frame}


\begin{frame}{Skolemova věta (shrnutí postupu)}

    \pause
    \myblock{
    \textbf{Věta:} Každá teorie má otevřenou konzervativní extenzi.
    }

    \pause
    \textbf{Důkaz} Mějme $L$-teorii $T$. Axiomy nahradíme generálními uzávěry a převedeme do PNF, máme ekvivalentní $L$-teorii $T'$. V ní každý axiom nahradíme jeho Skolemovou variantou. 
    
    \pause
    Tím získáme teorii $T''$ v rozšířeném jazyce~$L'$. Lemma říká:
    \begin{itemize}
        \item $L$-redukt každého modelu $T''$ je model $T'$
        \item každý model $T'$ lze expandovat do $L'$ na model $T''$
    \end{itemize}
    \pause
    Neboli $T''$ je konzervativní extenzí $T'$, tedy i $T$. Je axiomatizovaná  univerzálními sentencemi, odstraníme kvantifikátorové prefixy (vezmeme jádra) a máme ekvivalentní otevřenou teorii $T'''$.    
    \hfill\qedsymbol

    \medskip

    \pause
    \myblock{
    \textbf{Důsledek:} Ke každé teorii můžeme pomocí skolemizace najít ekvisplnitelnou otevřenou teorii. (A tu už snadno převedeme do CNF.)
    }

\end{frame}


\section{8.3 Grounding}


\begin{frame}{Grounding}

    \pause
    \begin{itemize}
        \item \alert{základní (ground) instance} otevřené $\varphi$ ve volných proměnných $x_1,\dots,x_n$ je $\varphi(x_1/t_1,\dots,x_n/t_n)$, kde vš. $t_i$ jsou konstantní
        
        \bigskip
        
        \pause
        \myblock{\textbf{Herbrandova věta} říká, že je-li \alert{otevřená} teorie \alert{nesplnitelná}, lze to doložit ``na konkrétních prvcích'': existuje konečně mnoho \alert{základních instancí} axiomů, jejichž konjunkce je nesplnitelná}
        \pause
        
        \bigskip
        \item např. pro \myexampleinline{
            $T=\{P(x,y)\lor R(x,y),\neg P(c,y),\neg R(x,f(x))\}$
        } substituujeme \alert{konstantní} termy $\{x/c,y/f(c)\}$:
        \pause
        
        \myalertmath{\vspace{-12pt}
            $$
            (P(c,f(c))\lor R(c,f(c)))\ \land\ \neg P(c,f(c))\ \land\ \neg R(c,f(c))
            $$
        }
        \pause

        \item základní atomické sentence chápeme jako prvovýroky: 
        
        \myalertmath{
            $$
            (p_1 \lor p_2) \land \neg p_1 \land \neg p_2
            $$
        }
        \pause

        \item to už snadno zamítneme výrokovou rezolucí\pause
        \item $p_1$ znamená ``platí $P(c,f(c))$'', $p_2$ znamená ``platí $R(c,f(c))$''        
    \end{itemize}
    
\end{frame}


\begin{frame}{Přímá redukce do výrokové logiky}

    \pause
    Herbrandova věta + korektnost a úplnost výrokové rezoluce dává následující, neefektivní postup ($S'$ je moc velká, i nekonečná):
    
    \medskip

    \myalert{
    \begin{enumerate}
        \item \pause $S$ $\rightsquigarrow$ $S'$ = množina všech základních instancí klauzulí z $S$
        \item \pause atomické sentence v $S'$ chápeme jako prvovýroky
        \item \pause $S$ nesplnitelná $\Leftrightarrow$ $S'$ zamítnutelná `na úrovni výrokové logiky'
    \end{enumerate}
    }
    
    \medskip

    \pause 
    Např. pro \myexampleinline{
        $S=\{\{P(x,y),R(x,y)\},\{\neg P(c,y)\},\{\neg R(x,f(x))\}\}$
    }
    {\footnotesize
    \pause 
    \begin{align*}
        S'=\{&\{P(c,c),R(c,c)\},\{P(c,f(c)),R(c,f(c))\},\{P(f(c),c),R(f(c),c)\},\dots,\\ 
        &\{\neg P(c,c)\}, \{\neg P(c,f(c))\},\{\neg P(c,f(f(c)))\},\{\neg P(c,f(f(f(c))))\}, \dots,\\
        &\{\neg R(c,f(c))\}, \{\neg R(f(c),f(f(c)))\},\{\neg R(f(f(c)),f(f(f(c))))\},\dots\}    
    \end{align*}
    }
    \pause 
    $S'$ je nesplnitelná obsahuje konečnou nesplnitelnou podmnožinu:
    $$
    \{\{P(c,f(c)),R(c,f(c))\},\{\neg P(c,f(c))\},\{\neg R(c,f(c))\}\}\proves_R\square
    $$

    \pause 
    \myalertinline{\textbf{Efektivnější} je hledat vhodné základní instance \alert{unifikací} [za chvíli]}

\end{frame}


\begin{frame}{Herbrandův model}

    \pause 
    \myblock{
    Mějme jazyk $L=\langle\mathcal R,\mathcal F\rangle$ s alespoň jedním konstantním symbolem. $L$-struktura $\A=\langle A,\mathcal R^\A,\mathcal F^\A\rangle$ je \alert{Herbrandův model}, jestliže:\pause 
    \begin{itemize}
        \item $A$ je množina všech konst. $L$-termů (\alert{Herbrandovo univerzum})\pause 
        \item pro každý $n$-ární $f\in\mathcal F$ a (konstantní) $\text{``$t_1$''},\dots,\text{``$t_n$''}\in A$:
        \myalertmath{
        $$
        f^\A(\text{``$t_1$''},\dots,\text{``$t_n$''})=\text{``$f(t_1,\dots,t_n)$''}
        $$
        }\pause 
        \item speciálně, pro konstantní symbol $c\in\mathcal F$ je $c^\A=\text{``$c$''}$\pause 
        \item na relační symboly neklademe podmínky
    \end{itemize}
    }

    \medskip

    \pause 
    Např. \myexampleinline{
        $L=\langle P,f,c\rangle$
    } ($P$ unární rel., $f$ binární funkční, $c$ konstantní)
    \alert{Herbrandův model} je každá struktura $\A=\langle A,P^\A,f^\A,c^\A\rangle$, kde\pause 
    \begin{itemize}\small
        \item  $A=\{\text{``$c$''},\text{``$f(c,c)$''},\text{``$f(c,f(c,c))$''},\text{``$f(f(c,c),c)$''}\dots\}$\pause 
        \item $c^\A=\text{``$c$''}$\pause 
        \item $f^\A(\text{``$c$''},\text{``$c$''})=\text{``$f(c,c)$''}$, $f^\A(\text{``$c$''},\text{``$f(c,c)$''})=\text{``$f(c,f(c,c))$''}$, $f^\A(\text{``$f(c,c)$''},\text{``$c$''})=\text{``$f(f(c,c),c)$''}$, atd.\pause 
        \item $P^\A\subseteq A$ může být libovolná
    \end{itemize}

\end{frame}


\begin{frame}{Herbrandova věta}

    \pause 
    \myblock{
    \textbf{Věta (Herbrandova):}
        Je-li $T$ otevřená, v jazyce bez rovnosti a s alespoň jedním konstantním symbolem, potom:\pause 
        \begin{itemize}
            \item buď má $T$ \alert{Herbrandův model}, nebo\pause 
            \item existuje konečně mnoho základních instancí axiomů $T$, jejichž konjunkce je nesplnitelná.\pause 
        \end{itemize}
    }
    
    \textbf{Důkaz:}
        \alert{$T_\text{ground}$} = \myalertinline{množina všech základních instancí axiomů $T$}

        \vspace{-3pt}
        \pause 

        Zkonstruujeme \alert{``systematické tablo''} $\tau$ z $T_\text{ground}$ s $\F\bot$ v kořeni, ale z jazyka $L$, bez rozšíření o pomocné konstantní symboly na $L_C$. \pause  (Nepotřebujeme je, protože v $T_\text{ground}$ nejsou kvantifikátory.)
        
        \pause 
        Pokud má $\tau$ bezespornou větev, je \alert{``kanonický model''} (opět bez pomocných symbolů) Herbrandovým modelem $T$. 
        
        \pause 
        Jinak je $\tau$ \alert{důkaz sporu}, $T_\text{ground}$ (a tedy i $T$) je nesplnitelná. Tablo $\tau$ je konečné, používá jen konečně mnoho $\alpha_\text{ground}\in T_\text{ground}$, jejich konjunkce už je nesplnitelná.\hfill\qedsymbol
    
\end{frame}


\begin{frame}{Poznámky}

    \pause
    \begin{itemize}[<+->]
        \item konstatní symbol potřebujeme, aby existovaly vůbec nějaké konstantní termy (ale není-li v $L$ žádný, můžeme ho přidat)
        \item Herbrandův model je podobný kanonickému, ale nepřidáváme pomocné symboly, a neříkáme nic o relacích
        \item je-li jazyk s rovností, najdeme Herbrandův model pro $T^*$ (přidané axiomy rovnosti) a faktorizujeme podle $=^A$
    \end{itemize}

\end{frame}


\begin{frame}{Důsledky Herbrandovy věty}

    \pause 
    \myblock{
    \textbf{Důsledek:}
    Je-li $T$ otevřená v jazyce s konstantním symbolem, potom $T$ má model, právě když má model teorie $T_\text{ground}$.
    }

    \pause 
    \textbf{Důkaz:}\pause 
    \alert{\Large$\Rightarrow$} V modelu $T$ platí i všechny základní instance axiomů. Je tedy i modelem $T_\text{ground}$. 
    
    \pause 
    \alert{\Large$\Leftarrow$} Pokud $T$ nemá model, podle Herbrandovy věty je nějaká konečná podmnožina teorie $T_\text{ground}$ nesplnitelná.\hfill\qedsymbol

    \bigskip

    \pause 
    \myblock{
    \textbf{Důsledek:}
    Mějme otevřenou $\varphi(x_1,\dots,x_n)$ v $L$ s konst. symbolem. Potom existuje $m\in\mathbb N$ a konstantní $L$-termy $t_{ij}$ ($i\in[m],j\in[n]$), že sentence \myalertinline{
        $(\exists x_1)\dots(\exists x_n)\varphi(x_1,\dots,x_n)$
    } je pravdivá, právě když je následující formule (výroková) tautologie:

    \pause 
    \vspace{-12pt}
    $$
    \varphi(x_1/t_{11},\dots,x_n/t_{1n})\lor \dots \lor \varphi(x_1/t_{m1},\dots,x_n/t_{mn})
    $$
    \vspace{-16pt}
    }

    \pause 
    \textbf{Důkaz:}
    Je \alert{pravdivá}, právě když $(\forall x_1)\dots(\forall x_n)\neg\varphi$ neboli $\neg\varphi$ je \alert{nesplnitelná}. Stačí aplikovat Herbrandovu větu na $T=\{\neg\varphi\}$.\hfill\qedsymbol

\end{frame}


\end{document}
