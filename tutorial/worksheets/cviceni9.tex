\documentclass[a4paper]{article}

\usepackage{a4wide}
\usepackage{amsmath}
\usepackage{amssymb}
\usepackage{amsthm}
\usepackage{enumerate}
\usepackage{tikz}
\usepackage[utf8]{inputenc}

\theoremstyle{plain}
\newtheorem*{theorem*}{Věta}
\theoremstyle{definition}
\newtheorem{problem}{Příklad}
\newtheorem*{ukol}{Domácí úkol}


\begin{document}

\section*{NAIL062 V\&P Logika: 9. cvičení}


\textbf{Témata:}
Tablo metoda v predikátové logice, jazyky s rovností.


\medskip\begin{problem}
    Předpokládejme, že
    \begin{enumerate}[(a)]
    \item všichni viníci jsou lháři,
    \item aspoň jeden z obviněných je také svědkem,
    \item žádný svědek nelže.
    \end{enumerate}
    Dokažte tablo metodou, že ne všichni obvinění jsou viníci.
\end{problem} 
    
    
    
\medskip\begin{problem} Nechť $L(x,y)$ reprezentuje \emph{``existuje let z $x$ do $y$''} a $S(x,y)$ reprezentuje \emph{``existuje spojení z $x$ do $y$''}. Předpokládejme, že
    \begin{enumerate}[(a)]
    \itemsep6pt
    \item Z Prahy lze letět do Bratislavy, Londýna a New Yorku, a z New Yorku do Paříže,
    \item $(\forall x)(\forall y)(L(x,y) \to L(y,x))$,
    \item $(\forall x)(\forall y)(L(x,y)\to S(x,y))$,
    \item $(\forall x)(\forall y)(\forall z)(S(x,y)\wedge L(y,z)\to S(x,z))$.
    \end{enumerate}
    Dokažte tablo metodou, že existuje spojení z Bratislavy do Paříže.
\end{problem} 


\medskip\begin{problem} Mějme teorii $T^*$ s axiomy rovnosti. Pomocí tablo metody ukažte, že 
\begin{enumerate}[(a)]
    \itemsep6pt
    \item $T^*\models x=y\ \to\ y=x$\hfill(symetrie)
    \item $T^*\models (x=y\ \wedge\ y=z)\ \to\ x=z$\hfill(tranzitivita)
\end{enumerate}
{\it Hint:} Pro (a) použijte axiom rovnosti $(iii)$ pro $x_1=x$, $x_2=x$, $y_1=y$ a $y_2=x$, \newline
    na (b) použijte $(iii)$ pro $x_1=x$, $x_2=y$, $y_1=x$ a $y_2=z$.
\end{problem} 


\medskip\begin{problem} % could be moved to tutorial 10
    Ukažme, že platí následující pravidla `vytýkání' kvantifikátorů. Používáme je při převodu do tzv. \emph{Prenexní normální formy}. V následujících příkladech jsou $\varphi$ a $\psi$ sentence nebo formule s volnou proměnnou $x$ (což značíme $\varphi(x)$, $\psi(x)$). Najděte tablo důkazy dané formule:
\begin{enumerate}[(a)]
    \item $\neg(\exists x)\varphi(x)\to (\forall x)\neg \varphi(x)$,
    \item $(\forall x)\neg \varphi(x)\to \neg(\exists x)\varphi(x)$,
    \item $(\exists x)(\varphi(x)\vee \psi(x))\leftrightarrow (\exists x)\varphi(x)\vee (\exists x)\psi(x)$,
    \item $(\forall x)(\varphi(x)\wedge\psi(x))\leftrightarrow (\forall x)\varphi(x)\wedge(\forall x)\psi(x)$,
    \item $(\varphi \vee (\forall x)\psi(x))\to (\forall x)(\varphi \vee \psi(x))$ kde $x$ není volná v $\varphi$,
    \item $(\varphi \wedge (\exists x)\psi(x))\to (\exists x)(\varphi \wedge \psi(x))$ kde $x$ není volná v $\varphi$.
    \item $(\exists x)(\varphi \to \psi(x))\to(\varphi \to (\exists x)\psi(x))$ kde $x$ není volná v $\varphi$,
    \item $(\exists x)(\varphi \wedge \psi(x))\to(\varphi \wedge (\exists x)\psi(x))$ kde $x$ není volná v $\varphi$,
    \item $(\exists x)(\varphi(x)\to\psi)\to((\forall x)\varphi(x)\to \psi)$ kde $x$ není volná v $\psi$,
    \item $((\exists x)\varphi(x)\to\psi)\to(\forall x)(\varphi(x)\to \psi)$ kde $x$ není volná v $\psi$.
\end{enumerate}
\end{problem}


\medskip\begin{problem} Dokažte větu o konstantách syntakticky, pomocí transformací tabel.
\begin{theorem*} Buď $\varphi$ formula v jazyce $L$ s volnými proměnnými $x_1,\dots,x_n$ a $T$ teorie v $L$. Označme $L'$ extenzi $L$ o nové konstantní symboly $c_1,\dots,c_n$ a $T'$ teorii $T$ v $L'$. Potom platí
    $$T \vdash (\forall x_1)\dots(\forall x_n)\varphi \quad\text{právě když}\quad T'\vdash\varphi(x_1/c_1,\dots,x_n/c_n).$$
\end{theorem*}
\end{problem} 
  

\medskip\begin{problem} Dokažte větu o dedukci syntakticky, pomocí transformací tabel.
\begin{theorem*} Pro každou teorii $T$ (v uzavřené formě) a sentence $\varphi$, $\psi$,
    $$T\vdash \varphi\to\psi\quad\text{právě když}\quad T,\varphi\vdash\psi.$$
\end{theorem*}
\end{problem} 
    



\medskip\begin{ukol}[3 body]

\end{ukol}

\end{document}