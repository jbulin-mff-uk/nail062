\section*{NAIL062 V\&P Logika: 9. sada příkladů -- Příprava na rezoluci v PL}
% po 5. přednášce


\subsection*{Výukové cíle:} Po absolvování cvičení student

    \begin{itemize}\setlength{\itemsep}{0pt}
        \item umí převádět formule do prenexní normální formy (PNF)
        \item rozumí pojmu Skolemova varianta, umí skolemizovat danou teorii
        \item umí převést danou otevřenou teorii do CNF, zapsat v množinové reprezentaci
        \item zná Herbrandovu větu, umí ji demonstrovat na příkladě, popsat Herbrandův model
    \end{itemize}
    

\section*{Příklady na cvičení}


\begin{problem} 
    
    Převeďte následující formule do PNF. Poté najděte jejich Skolemovy varianty.
    \begin{enumerate}[(a)]
        \item $(\forall y)((\exists x)P(x,y)\to Q(y,z))\wedge (\exists y)((\forall x)R(x,y)\vee Q(x,y))$
        \item $(\exists x)R(x,y)\leftrightarrow (\forall y)P(x,y)$
        \item $\neg((\forall x)(\exists y)P(x,y)\to (\exists x)(\exists y)R(x,y))\wedge(\forall x)\neg(\exists y)Q(x,y)$
    \end{enumerate}

    \begin{solution}
                    
    \end{solution}

\end{problem}


\begin{problem} 
    
    Převeďte na ekvisplnitelnou CNF formuli, zapište v množinové reprezentaci.

    \begin{enumerate}[(a)]
        \item $(\forall y)(\exists x)P(x,y)$
        \item $\neg (\forall y)(\exists x)P(x,y)$
        \item $\neg (\exists x)((P(x)\to P(c))\wedge (P(x)\to P(d)))$
        \item $(\exists x)(\forall y)(\exists z)(P(x,z)\wedge P(z,y) \to R(x,y))$
    \end{enumerate}

    \begin{solution}
                    
    \end{solution}

\end{problem}

      
\begin{problem}
    
    Nechť $T=\{(\exists x)R(x), (\exists y)\neg P(x,y), (\exists y)(\forall z)(\neg R(x)\vee P(y,z))\}$ je teorie jazyka $L=\langle P,R\rangle$ bez rovnosti. Najděte otevřenou teorii $T'$ ekvisplnitelnou s $T$. Převeďte $T'$ do CNF a výslednou formuli $S$ zapište v množinové reprezentaci.

    \begin{solution}
                    
    \end{solution}

\end{problem}


\begin{problem}

    Nechť $T=\{\varphi_1,\varphi_2\}$ je teorie v jazyce $L=\langle R\rangle$ s~rovností, kde:
    \begin{align*}
    \varphi_1=&\quad (\exists y)R(y,x)\\
    \varphi_2=&\quad (\exists z)(R(z,x)\wedge R(z,y)\wedge (\forall w)(R(w,x) \wedge R(w,y)\to R(w,z)))
    \end{align*}
    \begin{enumerate}[(a)]
        \item Pomocí skolemizace sestrojte otevřeně axiomatizovanou teorii $T'$ (případně v širším jazyce $L'$) ekvisplnitelnou s $T$.
        \item Buď $\mathcal{A}=\langle\mathbb{N}\cup\{0\},R^A\rangle$, kde $(n,m)\in R^A$ právě když $n$ dělí $m$.  Nalezněte expanzi $\mathcal{A}'$ $L$-struktury $\mathcal{A}$ do jazyka $L'$ takovou, že $\mathcal{A}'\models T'$.
    \end{enumerate}

\end{problem}


\begin{problem} 
    
    Sestrojte Herbrandův model dané teorie, nebo najděte nesplnitelnou konjunkci základních instancí jejích axiomů ($c,d$ jsou konstantní symboly v daném jazyce).    
    \begin{enumerate}[(a)]
        \item $T=\{\neg P(x)\vee Q(f(x),y), \neg Q(x,d), P(c)\}$
        \item $T=\{\neg P(x)\vee Q(f(x),y), Q(x,d), P(c)\}$
        \item $T=\{P(x,f(x)),\neg P(x,g(x))\}$
        \item $T=\{P(x,f(x)),\neg P(x,g(x)), P(g(x),f(y)) \to P(x,y)\}$
    \end{enumerate}

    \begin{solution}
                    
    \end{solution}

\end{problem}

        
\section*{Další příklady k procvičení}


\begin{problem}

    Teorie těles $T$ jazyka $L=\langle +,-,\cdot,0,1\rangle$ obsahuje jeden axiom $\varphi$, který není otevřený: $x\neq 0\ \to\ (\exists y)(x\cdot y=1)$. Víme, že $T\models 0\cdot y=0$ a $T\models\ (x\ne 0\ \wedge\ x\cdot y=1\ \wedge\ x\cdot z=1)\ \to\ y=z$.
    \begin{enumerate}[(a)]
        \item Najděte Skolemovu variantu $\varphi_S$ formule $\varphi$ s novým funkčním symbolem $f$.
        \item Uvažme teorii $T'$ vzniklou z $T$ nahrazením $\varphi$ za $\varphi_S$. Platí $\varphi$ v $T'$?
        \item Lze každý model $T$ \emph{jednoznačně} rozšířit na model $T'$?
    \end{enumerate}
    Nyní uvažme formuli $\psi=x\cdot y=1\vee  (x=0 \wedge y=0)$.
    \begin{enumerate}[(a)]
        \setcounter{enumi}{3}
        \item Platí v $T$ axiomy existence a jednoznačnosti pro $\psi(x,y)$ a proměnnou $y$?
        \item Sestrojte extenzi $T''$ teorie $T$ o definici symbolu $f$ formulí $\psi$.
        \item Je $T''$ ekvivalentní teorii $T'$?
        \item Najděte $L$-formuli, která je v $T''$-ekvivalentní s formulí:
        $f(x\cdot y)=f(x)\cdot f(y)$
    \end{enumerate}

\end{problem}


\medskip\begin{problem} Víme, že platí následující:
    \begin{itemize}\it
        \item Je-li cihla na (jiné) cihle, potom není na zemi.
        \item Každá cihla je na (jiné) cihle nebo na zemi.
        \item Žádná cihla není na cihle, která by byla na (jiné) cihle.
    \end{itemize}
    Chceme dokázat rezolucí následující tvrzení: {\it ``Je-li cihla na (jiné) cihle, spodní cihla je na zemi.''}. Sestrojte příslušnou CNF formuli $S$, a pokuste se najít i její rezoluční zamítnutí.
\end{problem}

        
\section*{K zamyšlení}

    
\begin{problem} 
    
    Skolemova varianta nemusí být ekvivalentní původní formuli, ověřte, že platí:
    \begin{enumerate}[(a)]
        \item $\models (\forall x)P(x,f(x)) \to (\forall x)(\exists y)P(x,y)$
        \item $\not\models (\forall x)(\exists y)P(x,y)\to (\forall x)P(x,f(x))$
    \end{enumerate}

\end{problem}















 


