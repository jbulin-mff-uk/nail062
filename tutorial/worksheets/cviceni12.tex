\documentclass[a4paper,12pt]{article}

%% slide-specific

\usetheme[progressbar=frametitle]{metropolis}
%\usecolortheme{spruce}
%\metroset{block=fill}

% block indentation workaround
% map defaulf block to oldblock
\let\oldblock\block
\let\endoldblock\endblock
% change block by adding smallskip
\renewenvironment{block}[1]
    {\begin{oldblock}{#1}
        \smallskip
    }
    { 
    \end{oldblock}
    }

% define Metropolis colors    
\definecolor{mAlert}{HTML}{EB811B}
\definecolor{mExample}{HTML}{14B03D}
\definecolor{mBlock}{HTML}{23373b}

\usepackage[most]{tcolorbox}

%\newcommand{\myexample}[1]{\leavevmode\textcolor{mExample}{#1}}
\newcommand{\myalert}[1]{
\begin{tcolorbox}[colback=mAlert!10, enhanced, boxrule=0pt, boxsep=-1mm, frame hidden, left=2mm, right=2mm]
    {#1}  
\end{tcolorbox}
}
\newcommand{\myexample}[1]{
\begin{tcolorbox}[colback=mExample!10, enhanced, boxrule=0pt, boxsep=-1mm, frame hidden, left=2mm, right=2mm]
    {#1}  
\end{tcolorbox}
}
\newcommand{\myblock}[1]{
\begin{tcolorbox}[colback=mBlock!10, enhanced, boxrule=0pt, boxsep=-1mm, frame hidden, left=2mm, right=2mm]
    {#1}  
\end{tcolorbox}
}

\newcommand{\mystructure}[1]{\mathcal{#1}}



% \newcommand{\myexamplemath}[1]{
% \begin{tcolorbox}[colback=mExample!10, enhanced, boxrule=0pt, frame hidden]
%     \ensuremath{#1}  
% \end{tcolorbox}
% }


%% packages
\usepackage{amsmath,amssymb,amsthm}
\usepackage{booktabs}
\usepackage[czech]{babel}
\usepackage{enumerate}
\usepackage{forest}
\usepackage{multicol}
% \usepackage{tcolorbox}
\usepackage{tikz}
    \usetikzlibrary{arrows.meta}
%\usepackage[unicode]{hyperref}
\usepackage[utf8]{inputenc}
\usepackage{xfrac}

% %% theorems
% \theoremstyle{plain}
%     \newtheorem{theorem}{Věta}[section]
%     \newtheorem*{theorem-unnumbered}{Věta}
%     \newtheorem{proposition}[theorem]{Tvrzení}
%     \newtheorem{corollary}[theorem]{Důsledek}
%     \newtheorem{lemma}[theorem]{Lemma}
%     \newtheorem{observation}[theorem]{Pozorování}
% \theoremstyle{definition}
%     \newtheorem{definition}[theorem]{Definice}
%     \newtheorem*{algorithm}{Algoritmus}
% \theoremstyle{remark}
%     \newtheorem{remark}[theorem]{Poznámka}
%     \newtheorem{example}[theorem]{Příklad}
%     \newtheorem{exercise}{Cvičení}[chapter]
%     \newtheorem*{solution}{Řešení}

%% macros and definitions
\DeclareMathOperator{\Aut}{Aut}
\DeclareMathOperator{\Conseq}{Csq}
\DeclareMathOperator{\DeLO}{DeLO}
\DeclareMathOperator{\dom}{dom}
\DeclareMathOperator{\Fm}{Fm}
\DeclareMathOperator{\M}{M}
%\DeclareMathOperator{\Proof}{Proof}
\DeclareMathOperator{\rng}{rng}
\DeclareMathOperator{\Term}{Term}
\DeclareMathOperator{\Th}{Th}
\DeclareMathOperator{\Thm}{Thm}
\DeclareMathOperator{\Tree}{Tree}
\DeclareMathOperator{\Var}{Var}
\DeclareMathOperator{\VF}{VF}

\newcommand{\A}{\structure{A}}
\newcommand{\B}{\structure{B}}
\newcommand{\Con}{\mathit{Con}}
\newcommand{\disjointunion}{\mathbin{\dot{\sqcup}}}
\newcommand{\F}{\ensuremath{\mathrm{F}}}
\newcommand{\landsymb}{{\land}}
\newcommand{\lbin}{\mathbin{\square}}
\newcommand{\lbinsymb}{{\lbin}}
\newcommand{\liff}{\mathbin{\leftrightarrow}}
\newcommand{\liffsymb}{{\liff}}
\newcommand{\limplies}{\mathbin{\rightarrow}}
\newcommand{\limpliessymb}{{\limplies}}
\newcommand{\lorsymb}{{\lor}}
\newcommand{\Prf}{\mathit{Prf}}
\newcommand{\proves}{\vdash}
%\newcommand{\structure}[1]{\mathcal{#1}}
\newcommand{\todo}{[TODO]}
\newcommand{\T}{\ensuremath{\mathrm{T}}}
\newcommand{\union}{\mathbin{\cup}}



\begin{document}

\section*{NAIL062 V\&P Logika: 12. cvičení}
% po 10. přednášce


\textbf{Témata:}
Rezoluce v predikátové logice.



\medskip\begin{problem} Víme, že platí následující:
    \begin{itemize}\it
        \item Je-li cihla na (jiné) cihle, potom není na zemi.
        \item Každá cihla je na (jiné) cihle nebo na zemi.
        \item Žádná cihla není na cihle, která by byla na (jiné) cihle.
    \end{itemize}
    Vyjádřete tato fakta ve vhodném jazyce logiky prvního řádu a dokažte rezolucí následující tvrzení: {\it ``Je-li cihla na (jiné) cihle, spodní cihla je na zemi.''}
\end{problem}
        
    
\medskip\begin{problem} Víme, že platí následující:
    \begin{itemize}\it
        \item Každý holič holí všechny, kdo neholí sami sebe
        \item Žádný holič neholí nikoho, kdo holí sám sebe.
    \end{itemize}
    Formalizujte ve vhodném jazyce predikátové logiky a dokažte rezolucí, že: {\it Neexistují žádní holiči.}
\end{problem}


\medskip\begin{problem}
Jsou dána následující tvrzení o~proběhlém genetickém experimentu:
\begin{enumerate}[label=(\roman*)]\it
    \item Každá ovce byla buď porozena jinou ovcí, nebo byla naklonována (avšak nikoli oboje zároveň).
    \item Žádná naklonovaná ovce neporodila.
\end{enumerate}
Chceme ukázat rezolucí, že pak: {\it (iii) Pokud ovce porodila, byla sama porozena.} Konkrétně:
\begin{enumerate}
    \item Uvedená tvrzení vyjádřete \underline{sentencemi} $\varphi_1$, $\varphi_2$, $\varphi_3$ v jazyce $L=\langle P,K\rangle$ bez rovnosti, kde $P$ je binární relační symbol, $K$ je unární relační symbol a $P(x,y)$, $K(x)$ značí, že \emph{``ovce $x$ porodila ovci $y$''} a \emph{``ovce $x$ byla naklonována''}.    
    \item S využitím Skolemizace těchto formulí nebo jejich negací sestrojte množinu klauzulí $S$ (může být ve větším jazyce), která je nesplnitelná, právě když  $\{\varphi_1, \varphi_2\} \models \varphi_3$. Zapište ji v množinové reprezentaci.
    \item Najděte rezoluční zamítnutí $S$, znázorněte je rezolučním stromem. U každého kroku uveďte použitou unifikaci.
\end{enumerate}
\end{problem}
        

\medskip\begin{problem}
Mějme jazyk $L=\langle <, j, h, s\rangle$ bez rovností, kde $j,h,q$ jsou konstantní symboly značící (po řadě) jablka, hrušky, švestky, dále $<$ je binární relační symbol a $x < y$ značí, že {\it ``ovoce $y$ je lepší než ovoce $x$''}. Víme, že:
\begin{enumerate}[label=(\roman*)]\it
    \item Relace ``být lepší'' je ostré částečné uspořádání (ireflexivní, asymetrická, tranzitivní relace).
    \item Hrušky jsou lepší než jablka.
\end{enumerate}
Chceme rezolucí dokázat následující tvrzení.
\begin{enumerate}[label=(\roman*)]\it
    \setcounter{enumi}{2}
    \item Jsou-li švestky lepší než hrušky, nejsou jablka lepší než švestky.
\end{enumerate}
\smallskip

Konkrétně:
\begin{enumerate}
\item Tvrzení $(i)$, $(ii)$, $(iii)$ vyjádřete otevřenými formulemi jazyka $L$.
\item Pomocí předchozích formulí či jejich negací nalezněte otevřenou teorii $T$ nad $L$ axiomatizovanou klauzulemi, která je nesplnitelná, právě když z $(i)$, $(ii)$ vyplývá $(iii)$. Napište $T$ v množinové reprezentaci.
\item Rezolucí dokažte, že $T$ není splnitelná. Rezoluční zamítnutí znázorněte rezolučním stromem. U každého kroku uveďte použitou unifikaci. {\it Nápověda: stačí čtyři rezoluční kroky.}
\item Nalezněte konjunkci základních instancí axiomů $T$, která je nesplnitelná. {\it Nápověda: využijte unifikace z (c).}
\item Je $T$ zamítnutelná LI-rezolucí? Uveďte zdůvodnění.
\end{enumerate}
\end{problem}


\medskip\begin{problem}
Nechť $T=\{\neg(\exists x) R(x), (\exists x)(\forall y)(P(x,y)\to P(y,x)), (\forall x)((\exists y)(P(x,y)\wedge P(y,x))\to R(x)),$ $(\forall x)(\exists y)P(x,y)\}$ je teorie jazyka $L=\langle P,R\rangle$ bez rovnosti.
\begin{enumerate}
\item Skolemizací nalezněte k $T$ otevřenou ekvisplnitelnou teorii $T'$ (nad vhodně rozšířeným jazykem).
\item Převeďte $T'$ na ekvivalentní teorii $S$ v CNF. Zapište $S$ v množinové reprezentaci.
\item Nalezněte rezoluční zamítnutí teorie $S$. U každého kroku uveďte použitou unifikaci.
\item Nalezněte konjunkci základních instancí axiomů $S$, která je nesplnitelná.
\item Má teorie $T$ jednoduchou kompletní extenzi? Uveďte zdůvodnění.
\end{enumerate}
\end{problem}



\end{document}