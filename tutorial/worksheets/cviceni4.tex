\documentclass[a4paper,12pt]{article}

%% slide-specific

\usetheme[progressbar=frametitle]{metropolis}
%\usecolortheme{spruce}
%\metroset{block=fill}

% block indentation workaround
% map defaulf block to oldblock
\let\oldblock\block
\let\endoldblock\endblock
% change block by adding smallskip
\renewenvironment{block}[1]
    {\begin{oldblock}{#1}
        \smallskip
    }
    { 
    \end{oldblock}
    }

% define Metropolis colors    
\definecolor{mAlert}{HTML}{EB811B}
\definecolor{mExample}{HTML}{14B03D}
\definecolor{mBlock}{HTML}{23373b}

\usepackage[most]{tcolorbox}

%\newcommand{\myexample}[1]{\leavevmode\textcolor{mExample}{#1}}
\newcommand{\myalert}[1]{
\begin{tcolorbox}[colback=mAlert!10, enhanced, boxrule=0pt, boxsep=-1mm, frame hidden, left=2mm, right=2mm]
    {#1}  
\end{tcolorbox}
}
\newcommand{\myexample}[1]{
\begin{tcolorbox}[colback=mExample!10, enhanced, boxrule=0pt, boxsep=-1mm, frame hidden, left=2mm, right=2mm]
    {#1}  
\end{tcolorbox}
}
\newcommand{\myblock}[1]{
\begin{tcolorbox}[colback=mBlock!10, enhanced, boxrule=0pt, boxsep=-1mm, frame hidden, left=2mm, right=2mm]
    {#1}  
\end{tcolorbox}
}

\newcommand{\mystructure}[1]{\mathcal{#1}}



% \newcommand{\myexamplemath}[1]{
% \begin{tcolorbox}[colback=mExample!10, enhanced, boxrule=0pt, frame hidden]
%     \ensuremath{#1}  
% \end{tcolorbox}
% }


%% packages
\usepackage{amsmath,amssymb,amsthm}
\usepackage{booktabs}
\usepackage[czech]{babel}
\usepackage{enumerate}
\usepackage{forest}
\usepackage{multicol}
% \usepackage{tcolorbox}
\usepackage{tikz}
    \usetikzlibrary{arrows.meta}
%\usepackage[unicode]{hyperref}
\usepackage[utf8]{inputenc}
\usepackage{xfrac}

% %% theorems
% \theoremstyle{plain}
%     \newtheorem{theorem}{Věta}[section]
%     \newtheorem*{theorem-unnumbered}{Věta}
%     \newtheorem{proposition}[theorem]{Tvrzení}
%     \newtheorem{corollary}[theorem]{Důsledek}
%     \newtheorem{lemma}[theorem]{Lemma}
%     \newtheorem{observation}[theorem]{Pozorování}
% \theoremstyle{definition}
%     \newtheorem{definition}[theorem]{Definice}
%     \newtheorem*{algorithm}{Algoritmus}
% \theoremstyle{remark}
%     \newtheorem{remark}[theorem]{Poznámka}
%     \newtheorem{example}[theorem]{Příklad}
%     \newtheorem{exercise}{Cvičení}[chapter]
%     \newtheorem*{solution}{Řešení}

%% macros and definitions
\DeclareMathOperator{\Aut}{Aut}
\DeclareMathOperator{\Conseq}{Csq}
\DeclareMathOperator{\DeLO}{DeLO}
\DeclareMathOperator{\dom}{dom}
\DeclareMathOperator{\Fm}{Fm}
\DeclareMathOperator{\M}{M}
%\DeclareMathOperator{\Proof}{Proof}
\DeclareMathOperator{\rng}{rng}
\DeclareMathOperator{\Term}{Term}
\DeclareMathOperator{\Th}{Th}
\DeclareMathOperator{\Thm}{Thm}
\DeclareMathOperator{\Tree}{Tree}
\DeclareMathOperator{\Var}{Var}
\DeclareMathOperator{\VF}{VF}

\newcommand{\A}{\structure{A}}
\newcommand{\B}{\structure{B}}
\newcommand{\Con}{\mathit{Con}}
\newcommand{\disjointunion}{\mathbin{\dot{\sqcup}}}
\newcommand{\F}{\ensuremath{\mathrm{F}}}
\newcommand{\landsymb}{{\land}}
\newcommand{\lbin}{\mathbin{\square}}
\newcommand{\lbinsymb}{{\lbin}}
\newcommand{\liff}{\mathbin{\leftrightarrow}}
\newcommand{\liffsymb}{{\liff}}
\newcommand{\limplies}{\mathbin{\rightarrow}}
\newcommand{\limpliessymb}{{\limplies}}
\newcommand{\lorsymb}{{\lor}}
\newcommand{\Prf}{\mathit{Prf}}
\newcommand{\proves}{\vdash}
%\newcommand{\structure}[1]{\mathcal{#1}}
\newcommand{\todo}{[TODO]}
\newcommand{\T}{\ensuremath{\mathrm{T}}}
\newcommand{\union}{\mathbin{\cup}}



\begin{document}

\section*{NAIL062 V\&P Logika: 4. cvičení}
% po 3. přednášce

\textbf{Témata:} 
Vlastnosti a extenze teorií. Počítání výroků až na ekvivalenci (Lindenbaum-Tarského algebra). 2-SAT a implikační graf. Horn-SAT a jednotková propagace.



\medskip\begin{problem}
    Dokažte nebo vyvraťte (nebo uveďte správný vztah), že pro každou teorii $T$ a výroky $\varphi$, $\psi$ v jazyce $\mathbb{P}$ platí:
    \begin{enumerate}
        \item $T \models \varphi$,\ \  právě když \ \ $T \not\models \neg \varphi$
        \item $T \models \varphi$ a $T \models \psi$,\ \ právě když \ \ $T \models \varphi \wedge \psi$
        \item $T \models \varphi$ nebo $T \models \psi$,\ \ právě když \ \ $T \models \varphi \vee \psi$
        \item $T \models \varphi \to \psi$ and $T \models \psi \to \chi$,\ \ právě když \ \ $T \models \varphi \to \chi$
    \end{enumerate}
    \end{problem}

\medskip\begin{problem}
    Uvažte následující dvě teorie:
    \begin{enumerate}[label=(\Roman*)]
        \item $T=\{p\wedge q,p\to\neg q,q\}$ v jazyce $\mathbb P=\{p,q\}$
        \item $T=\{(p\wedge q)\to r, \neg r\vee(p\wedge q)\}$ v jazyce $\mathbb P=\{p,q,r\}$        
    \end{enumerate}
    \begin{enumerate}
        \item Rozhodněte, zda je teorie $T$ [konzistentní/splnitelná/kompletní]. (konzistentní=bezesporná, kompletní=úplná)
        \item Uveďte příklad výroku $\varphi$, který je [platný/nesplnitelný/nezávislý] v $T$
        \item Uveďte příklad extenze $T'$ teorie $T$ (pokud existuje, a pokud možno neekvivalentní s $T$), která je [jednoduchá / konzervativní/kompletní/konzervativní jednoduchá/kompletní jednoduchá/kompletní konzervativní].
    \end{enumerate}
    
    \end{problem}



    \medskip\begin{problem}
        Dokažte nebo vyvraťte (nebo uveďte správný vztah), že pro libovolné teorie $T$, $S$ nad~$\mathbb{P}$ platí:
        \begin{enumerate}
            \item $S\subseteq T \Rightarrow \Conseq(T) \subseteq \Conseq(S)$
            \item $\Conseq(S\cup T)=\Conseq(S) \cup \Conseq(T)$
            \item $\Conseq(S\cap T)=\Conseq(S) \cap \Conseq(T)$
        \end{enumerate}
        \end{problem}
        

\medskip\begin{problem}
    Nechť $|\mathbb{P}|=n$ a mějme výrok $\varphi\in\mathrm{VF}_{\mathbb{P}}$ takový, že $|M(\varphi)|=k$. Určete počet až na ekvivalenci:
    \begin{enumerate}
    \item výroků $\psi$ takových, že $\varphi \models \psi$ nebo $\psi \models \varphi$,
    \item teorií nad $\mathbb{P}$, ve kterých platí $\varphi$,
    \item úplných teorií nad $\mathbb{P}$ ve kterých platí $\varphi$,
    \item teorií $T$ nad $\mathbb{P}$ takových, že $T \cup \{\varphi\}$ je bezesporná.
    \end{enumerate}
    Uvažme navíc spornou teorii $\{\varphi,\psi\}$ kde $|M(\psi)|=p$. Spočtěte až na ekvivalenci:
    \begin{enumerate}
    \setcounter{enumi}{4}
    \item výroky $\chi$ takové, že $\varphi \vee \psi \models \chi$, 
    \item teorie, ve kterých platí $\varphi \vee \psi$.
    \end{enumerate}
\end{problem}



\medskip\begin{problem} Pro danou formuli $\varphi$ v CNF najděte a 3-CNF formuli $\varphi'$ takovou, že $\varphi'$ je splnitelná, právě když $\varphi$ je splnitelná. Popište efektivní algoritmus konstrukce $\varphi'$ je-li dána $\varphi$ (tj. redukci z problému SAT do problému 3-SAT).
\end{problem}

    
\medskip\begin{problem} Sestrojte implikační graf daného 2-CNF výroku. Je splnitelný? Pokud ano, najděte nějaké řešení.
\begin{enumerate}
    \item $(p_1\vee \neg p_2)\wedge (p_2\vee p_3)\wedge (\neg p_3\vee \neg p_1)\wedge (\neg p_3\vee \neg p_4)\wedge (p_4\vee p_5)\wedge (\neg p_5\vee \neg p_1)$
    \item $(p_1\vee \neg p_2)\wedge (p_2\vee p_3)\wedge (\neg p_3\vee p_1)\wedge (\neg p_3\vee \neg p_4)\wedge (p_4\vee p_5)\wedge (\neg p_5\vee p_1)$
    \item $(p_0 \vee  p_2) \wedge  (p_0 \vee  \neg p_3) \wedge  (p_1 \vee  \neg p_3) 
    \wedge  (p_1 \vee  \neg p_4) \wedge  (p_2 \vee  \neg p_4) 
    \wedge  (p_0 \vee  \neg p_5)
    \wedge 
    (p_1 \vee  \neg p_5) \wedge  (p_2 \vee  \neg p_5) \wedge  (\neg p_1 \vee  \neg p_6) \wedge  (p_4 \vee  p_6) \wedge  (p_5 \vee  p_6) \wedge  p_1\wedge \neg p_7$
\end{enumerate}
\end{problem}


\medskip\begin{problem}
Pomocí jednotkové propagace zjistěte, zda je následující Hornův výrok splnitelný. Pokud ano, najděte nějaké splňující ohodnocení.
\begin{align*}
    &(\neg p_1 \vee \neg p_3 \vee p_2)\wedge(\neg p_1 \vee p_2)\wedge p_1 \wedge (\neg p_1 \vee \neg p_2 \vee p_3)\wedge \\
    &(\neg p_2 \vee \neg p_4 \vee p_1)\wedge(\neg p_4 \vee \neg p_3 \vee \neg p_2)\wedge(p_4\vee \neg p_5 \vee\neg p_6)
\end{align*}
\end{problem}
    



\end{document}