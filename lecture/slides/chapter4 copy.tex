


\section{Věta o kompaktnosti}

Důležitým důsledkem vět o korektnosti a úplnosti je také tzv. \emph{Věta o kompaktnosti}.\footnote{Slovo \emph{kompaktnost} pochází z kompaktních (tj. omezených a uzavřených) množin v Euklidovských prostorech, ve kterých lze z každé posloupnosti vybrat konvergentní podposloupnost. Můžete si představit posloupnost zvětšujících se konečných částí `konvergující' k nekonečnému celku.} Tento princip umožňuje převádět tvrzení o nekonečných objektech/procesech na tvrzení o (všech) jejich konečných částech.

\begin{theorem}[O kompaktnosti]\label{theorem:compactness-theorem}
Teorie má model, právě když každá její konečná část má model.    
\end{theorem}

\begin{proof}
Každý model teorie $T$ je zjevně modelem každé její části. Druhou implikaci dokážeme nepřímým důkazem: Předpokládejme, že $T$ nemá model, tj. je sporná, a najděme konečnou část $T'\subseteq T$, která je také sporná.

Protože je $T$ sporná, platí $T\proves\bot$ (zde potřebujeme Větu o úplnosti). Podle Důsledku \ref{corollary:finiteness-of-proofs} potom existuje \emph{konečný} tablo důkaz $\tau$ výroku $\bot$ z $T$. Konstrukce tohoto důkazu má jen konečně mnoho kroků, použili jsme tedy jen konečně mnoho axiomů z $T$. Definujeme-li $T'=\{\alpha\in T\mid \mathrm{T}\alpha\text{ je položka v tablu $\tau$}\}$, potom $\tau$ je také tablo důkaz sporu z teorie $T'$. Teorie $T'$ je tedy sporná konečná část $T$.
\end{proof}

\subsection{Aplikace kompaktnosti}

Následující jednoduchou aplikaci Věty o kompaktnosti můžete chápat jako šablonu, kterou následuje i mnoho dalších, složitějších aplikací této věty. 

\begin{corollary}\label{corollary:infinite-bipartite-compactness}
Spočetně nekonečný graf je bipartitní, právě když je každý jeho konečný podgraf bipartitní.    
\end{corollary}

\begin{proof}
    Každý podgraf bipartitního grafu je zjevně také bipartitní. Ukažme opačnou implikaci. Graf je bipartitní, právě když je obarvitelný 2 barvami. Označme barvy $0,1$.

    Sestrojíme výrokovou teorii $T$ v jazyce $\mathbb P=\{p_v\mid v\in V(G)\}$, kde hodnota výrokové proměnné $p_v$ reprezentuje barvu vrcholu $v$.
    $$  
        T=\{p_u\limplies\neg p_v\mid \{u,v\}\in E(G)\}
    $$
    Zřejmě platí, že $G$ je bipartitní, právě když $T$ má model. Podle Věty o kompaktnosti stačí ukázat, že každá konečná část $T$ má model. Vezměme tedy konečnou $T'\subseteq T$. Buď $G'$ podgraf $G$ indukovaný na množině vrcholů, o kterých se zmiňuje teorie $T'$, tj. $V(G')=\{v\in V(G)\mid p_v\in\Var(T')\}$. Protože je $T'$ konečná, je $G'$ také konečný, a podle předpokladu je 2-obarvitelný. Libovolné 2-obarvení $V(G')$ ale určuje model teorie $T'$.
\end{proof}

Základem této techniky je popis požadované vlastnosti nekonečného objektu pomocí (nekonečné) výrokové teorie. Dále si všimněte, jak z konečné části teorie sestrojíme konečný podobjekt mající danou vlastnost (v našem případě konečný podgraf, který je bipartitní).

\begin{exercise}
    Zobecněte Důsledek \ref{corollary:infinite-bipartite-compactness} pro více barev, tj. ukažte, že spočetně nekonečný graf je $k$-obarvitelný, právě když je každý jeho konečný podgraf $k$-obarvitelný. (Viz Sekce \ref{subsection:example-graph-coloring}.)
\end{exercise}

\begin{exercise}
    Ukažte, že každé částečné uspořádání na spočetné množině lze rozšířit na lineární uspořádání.
\end{exercise}

\begin{exercise}
    Vyslovte a dokažte `spočetně nekonečnou' analogii Hallovy věty.
\end{exercise}

\section{(draft) Hilbertovský kalkulus}\label{section:hilbert-calculus-propositional}
\todo
Na závěr kapitoly o tablo metodě si pro srovnání ukážeme jiný dokazovací systém, tzv. \emph{hilbertovský deduktivní systém} neboli \emph{hilbertovský kalkulus}. Jde o nejstarší dokazovací systém, modelovaný podle matematických důkazů. Jak uvidíme na příkladě, dokazování je v něm poměrně pracné, hodí se tedy spíše pro teoretické účely. Jde také o korektní a úplný dokazovací systém (to ale necháme bez důkazu).

% from slides:

\begin{itemize}
    \item základní logické spojky: $\neg$, $\to$ (ostatní z nich odvozené)
    
    \item \mdef{logické axiomy} (\myblue{schémata} logických axiomů):
    \vspace{-2mm}\begin{align*}(i)& &\varphi &\to (\psi \to \varphi) \\
    (ii)& &(\varphi\to (\psi \to \chi))&\to ((\varphi \to \psi)\to(\varphi \to \chi))\qquad\qquad\qquad\qquad\phantom{\ } \\
    (iii)& &(\neg \varphi \to \neg \psi)&\to(\psi \to \varphi)
    \end{align*}
    
    \vspace{-2mm}
    kde $\varphi$, $\psi$, $\chi$ jsou libovolné formule (daného jazyka).
    \item \mdef{odvozovací pravidlo}:
    \vspace{-3mm}
    $$\frac{\varphi,\ \varphi \to \psi}{\psi}\qquad\text{\myblue{(modus ponens)}}$$
    \end{itemize}
    
    \vspace{-1mm}
    \mdef{Důkaz} (\emph{Hilbertova stylu}) formule $\varphi$ v teorii $T$ je \myblue{konečná} posloupnost
    \smallskip
    
    $\varphi_0, \dots, \varphi_n=\varphi$ formulí taková, že pro každé $i\le n$
    \begin{itemize}
    \item $\varphi_i$ je logický axiom nebo $\varphi_i \in T$ (axiom teorie), nebo
    \item $\varphi_i$ lze odvodit z předchozích formulí pomocí odvozovacího pravidla.
    \end{itemize}
    
    %$\varphi$ je \mdef{dokazatelná} v $T$, má-li důkaz z $T$. Značíme $T _{H} \varphi$, popř. $_H \varphi$ pro $T=\emptyset$.
    \medskip
    
    {\it \myblue{Poznámka}\ \ Volba axiomů a odvozovacích pravidel se v může v různých dokazovacích systémech Hilbertova stylu lišit.}
    
    \subsubsection*{Příklad a korektnost}
    Formule $\varphi$ je \mdef{dokazatelná} v $T$, má-li důkaz z $T$, značíme $T _{H} \varphi$.
    \smallskip
    
    Je-li $T=\emptyset$, značíme $_H \varphi$. Např. pro \mygreen{$T=\{\neg \varphi\}$} je \mygreen{$T _H \varphi \to \psi$} pro každé $\psi$.
    \mygreen{\vspace{-2mm}\begin{align*}
    1)& &\neg \varphi& & &\text{axiom z $T$} \\
    2)& &\neg \varphi& \to (\neg \psi \to \neg \varphi)& &\text{logický axiom $(i)$}\\
    3)& &\neg \psi &\to \neg \varphi& &\text{modus ponens z 1), 2)}\\
    4)& &(\neg \psi \to \neg \varphi)&\to(\varphi \to \psi)& &\text{logický axiom $(iii)$}\\
    5)& &\varphi &\to \psi&  &\text{modus ponens z 3), 4)}
    \end{align*}}
    
    \vspace{-4mm}
    {\bf \myblue{Věta}}\ \ {\it  Pro každou teorii $T$ a formuli $\varphi$,\ \ $T_H \varphi\ \Rightarrow\ T\models \varphi$.}
    \smallskip
    
    {\it \myblue{Důkaz}}
    \begin{itemize}
    \item Je-li $\varphi\in T$ nebo logický axiom, je $T \models \varphi$ (logické axiomy jsou tautologie),
    \item jestliže $T \models \varphi$ a $T \models \varphi \to \psi$, pak $T \models \psi$, tj. modus ponens je \myblue{korektní},
    \item tedy každá formule vyskytující se v důkazu z $T$ platí v $T$. \qed
\end{itemize}
    \medskip
    
    {\it \myblue{Poznámka}\ \ Platí i \myblue{úplnost}, tj. $T\models \varphi \Rightarrow T_H \varphi$ pro každou teorii $T$ a formuli $\varphi$.}

% :from slides


% \subsection*{Hilbert's calculus}
%     The \emph{Hilbert's propositional calculus} is a proof system for propositional logic where 
%     \begin{itemize}
%         \item we only use the logical connectives $\neg,\to$
%         \item we have the following (schemes of) \emph{logical axioms}:
%         \begin{enumerate}[(i)]
%             \item $\varphi \to (\psi \to \varphi)$
%             \item $(\varphi\to (\psi \to \chi)) \to ((\varphi \to \psi)\to(\varphi \to \chi))$
%             \item $(\neg \varphi \to \neg \psi)\to(\psi \to \varphi)$
%         \end{enumerate}
%         \item and the following \emph{rule of inference}:
%         $$\frac{\varphi,\ \varphi \to \psi}{\psi}$$
%         i.e. ``from $\varphi$ and $\varphi\to\psi$ infer $\psi$'' (called ``modus ponens'')
%     \end{itemize}
%     In Hilbert's calculus, a \emph{proof} of a proposition $\varphi$ from a theory $T$ is a finite sequence $\varphi_0,\dots,\varphi_n=\varphi$ of formulas such that for every $i\leq n$,
%     \begin{itemize}
%     \item $\varphi_i$ is a logical axiom, or 
%     \item $\varphi_i \in T$ (an axiom of the theory), or
%     \item $\varphi_i$ can be inferred from a pair of preceding propositions $\varphi_j$, $\varphi_k$ ($j<i,k<i$) by applying the rule of inference.
%     \end{itemize}
%     If such a proof exists, we write $T\ _H\ \varphi$.