\documentclass{beamer}

%% slide-specific

\usetheme[progressbar=frametitle]{metropolis}
%\usecolortheme{spruce}
%\metroset{block=fill}

% define Metropolis colors    
\definecolor{mAlert}{HTML}{EB811B}
\definecolor{mExample}{HTML}{14B03D}
\definecolor{mBlock}{HTML}{23373b}

% my blocks
\setlength\fboxsep{0pt}%

\newcommand{\myblock}[1]{\colorbox{mBlock!12}{\begin{minipage}{\linewidth}#1\end{minipage}}}
\newcommand{\myblockmath}[1]{\colorbox{mBlock!12}{\begin{minipage}{\linewidth}\vspace{-6pt}#1\end{minipage}}}
\newcommand{\myblockamsmath}[1]{\colorbox{mBlock!12}{\begin{minipage}{\linewidth}\vspace{-6pt}#1\end{minipage}}}
\newcommand{\myblockinline}[1]{\colorbox{mBlock!12}{#1}}
\newcommand{\myexample}[1]{\colorbox{mExample!12}{\begin{minipage}{\linewidth}#1\end{minipage}}}
\newcommand{\myexamplemath}[1]{\colorbox{mExample!12}{\begin{minipage}{\linewidth}\vspace{-6pt}#1\end{minipage}}}
\newcommand{\myexampleamsmath}[1]{\colorbox{mExample!12}{\begin{minipage}{\linewidth}\vspace{-18pt}#1\end{minipage}}}
\newcommand{\myexampleinline}[1]{\colorbox{mExample!12}{#1}}
\newcommand{\myalert}[1]{\colorbox{mAlert!12}{\begin{minipage}{\linewidth}#1\end{minipage}}}
\newcommand{\myalertmath}[1]{\colorbox{mAlert!12}{\begin{minipage}{\linewidth}\vspace{-6pt}#1\end{minipage}}}
\newcommand{\myalertamsmath}[1]{\colorbox{mAlert!12}{\begin{minipage}{\linewidth}\vspace{-18pt}#1\end{minipage}}}
\newcommand{\myalertinline}[1]{\colorbox{mAlert!12}{#1}}

%% other
\newcommand{\mystructure}[1]{\mathcal{#1}}


%% packages
\usepackage{amsmath,amssymb,amsthm}
\usepackage{bookmark}
\usepackage{booktabs}
\usepackage{cancel}
\usepackage[czech]{babel}
\usepackage{enumerate}
\usepackage[T1]{fontenc}
\usepackage{forest}
\usepackage{lmodern}
\usepackage{multicol}
% \usepackage{tcolorbox}
\usepackage{tikz}
    \usetikzlibrary{arrows.meta}
%\usepackage[unicode]{hyperref}
\usepackage[utf8x]{inputenc}
\usepackage{xfrac}


% %% theorems
% \theoremstyle{plain}
%     \newtheorem{theorem}{Věta}[section]
%     \newtheorem*{theorem-unnumbered}{Věta}
%     \newtheorem{proposition}[theorem]{Tvrzení}
%     \newtheorem{corollary}[theorem]{Důsledek}
%     \newtheorem{lemma}[theorem]{Lemma}
%     \newtheorem{observation}[theorem]{Pozorování}
% \theoremstyle{definition}
%     \newtheorem{definition}[theorem]{Definice}
%     \newtheorem*{algorithm}{Algoritmus}
% \theoremstyle{remark}
%     \newtheorem{remark}[theorem]{Poznámka}
%     \newtheorem{example}[theorem]{Příklad}
%     \newtheorem{exercise}{Cvičení}[chapter]
%     \newtheorem*{solution}{Řešení}

%% macros and definitions
\DeclareMathOperator{\Aut}{Aut}
\DeclareMathOperator{\Conseq}{Csq}
\DeclareMathOperator{\DeLO}{DeLO}
\DeclareMathOperator{\dom}{dom}
\DeclareMathOperator{\Fm}{Fm}
\DeclareMathOperator{\M}{M}
%\DeclareMathOperator{\Proof}{Proof}
\DeclareMathOperator{\rng}{rng}
\DeclareMathOperator{\Term}{Term}
\DeclareMathOperator{\Th}{Th}
\DeclareMathOperator{\Thm}{Thm}
\DeclareMathOperator{\Tree}{Tree}
\DeclareMathOperator{\Var}{Var}
\DeclareMathOperator{\VF}{VF}

\newcommand{\A}{\mystructure{A}}
\newcommand{\B}{\mystructure{B}}
\newcommand{\Con}{\mathit{Con}}
\newcommand{\disjointunion}{\mathbin{\dot{\sqcup}}}
\newcommand{\F}{\ensuremath{\mathrm{F}}}
\newcommand{\landsymb}{{\land}}
\newcommand{\lbin}{\mathbin{\square}}
\newcommand{\lbinsymb}{{\lbin}}
\newcommand{\liff}{\mathbin{\leftrightarrow}}
\newcommand{\liffsymb}{{\liff}}
\newcommand{\limplies}{\mathbin{\rightarrow}}
\newcommand{\limpliessymb}{{\limplies}}
\newcommand{\lorsymb}{{\lor}}
\newcommand{\Prf}{\mathit{Prf}}
%\newcommand{\structure}[1]{\mathcal{#1}}
\newcommand{\todo}{[TODO]}
\newcommand{\T}{\ensuremath{\mathrm{T}}}
\newcommand{\union}{\mathbin{\cup}}

\DeclareRobustCommand\proves{\mathrel{|}\joinrel\mkern-.5mu\mathrel{-}}

\title{Sedmá přednáška}
\subtitle{NAIL062 Výroková a predikátová logika}
\author{Jakub Bulín (KTIML MFF UK)}
% \institute{KTIML MFF UK}
\date{Zimní semestr 2023}


\begin{document}


\frame{\titlepage}


\begin{frame}{Sedmá přednáška}

    \textbf{Program}
        \begin{itemize}
            \item podstruktury, expanze, redukty           
            \item extenze teorií, extenze o definice
            \item definovatelnost a databázové dotazy
            \item vztah výrokové a predikátové logiky
        \end{itemize}

    \textbf{Materiály}

        \href{https://github.com/jbulin-mff-uk/nail062/raw/main/lecture/lecture-notes/lecture-notes.pdf}{\alert{\textbf{Zápisky z přednášky}}}Sekce 6.6-6.9 z Kapitoly 6

\end{frame}


\section{6.6 Podstruktura, expanze, redukt}


\begin{frame}{Podstruktura}

    \begin{itemize}
        \item \alert{podstruktura} zobecňuje podgrupu, podprostor vektorového prostoru, (indukovaný) podgraf: na podmnožině $B$ univerza vytvoříme strukturu, která \myalertinline{``zdědí'' relace, operace a konstanty}
        \item $B$ musí být \alert{uzavřená} na všechny operace (vč. konstant)
    \end{itemize}

    \myblock{
    Struktura $\B=\langle B,\mathcal R^\mathcal B,\mathcal F^\mathcal B\rangle$ je \alert{(indukovaná) podstruktura} struktury $\A=\langle A,\mathcal R^\mathcal A,\mathcal F^\mathcal A\rangle$ (v též signatuře $\langle\mathcal R,\mathcal F\rangle$), značíme \alert{$\B\subseteq\A$}, jestliže:

    \begin{itemize}
        \item $\emptyset\neq B\subseteq A$
        \item $R^\B=R^\A\cap B^{\mathrm{ar(R)}}$ pro každý relační symbol $R\in \mathcal R$
        \item $f^\B=f^\A\cap (B^{\mathrm{ar(f)}}\times B)$ pro každý funkční symbol $f\in \mathcal F$ (tj. $f^\B$ je restrikce $f^\A$ na množinu $B$, a výstupy jsou všechny z $B$)
        \item speciálně, pro konstantní symbol $c\in\mathcal F$ máme $c^\B=c^\A\in B$.
    \end{itemize}
    }
    


\end{frame}


\begin{frame}{Restrikce na podmnožinu, příklady}
    
    Množina $C\subseteq A$ je \alert{uzavřená} na funkci $f:A^n\to A$, pokud $f(x_1,\dots,x_n)\in C$ pro všechna $x_i\in C$.

    \medskip

    \myblock{
    \textbf{Pozorování:} Množina $\emptyset\neq C\subseteq A$ je univerzem podstruktury, právě když je uzavřená na všechny funkce struktury $\A$ (včetně konstant). V tom případě je to \alert{restrikce} $\A$ na množinu $C$, značíme \alert{$\A\restriction C$}.
    }

    \medskip
    
    \begin{itemize}
        \item \myexampleinline{
            $\underline{\mathbb Z}=\langle \mathbb Z,+,\cdot,0\rangle$
         } je podstrukturou \myexampleinline{
            $\underline{\mathbb Q}=\langle \mathbb Q,+,\cdot,0\rangle$
            }, můžeme psát: \alert{$\underline{\mathbb Z}=\underline{\mathbb Q}\restriction\mathbb Z$}
        \item \myexampleinline{
            $\underline{\mathbb N}=\langle \mathbb N,+,\cdot,0\rangle$
         } je podstrukturou obou těchto struktur, platí: \alert{$\underline{\mathbb N}=\underline{\mathbb Q}\restriction\mathbb N=\underline{\mathbb Z}\restriction\mathbb N$}
        \item Množina \myexampleinline{
            $\{k\in\mathbb Z\mid k\leq 0\}$
         } není univerzem podstruktury $\underline{\mathbb Z}$ ani~$\underline{\mathbb Q}$, není uzavřená na násobení.
    \end{itemize}
\end{frame}


\begin{frame}{Platnost v podstruktuře}
    
    
\end{frame}


\begin{frame}{Generovaná podstruktura}
    
    
\end{frame}


\begin{frame}{Expanze a redukt}
    
\end{frame}


\begin{frame}{Věta o konstantách}


\end{frame}


\section{6.7 Extenze teorií}


\section{6.8 Definovatelnost ve struktuře}


\section{6.9 Vztah výrokové a predikátové logiky}


\end{document}


