\documentclass{beamer}

%% slide-specific

\usetheme[progressbar=frametitle]{metropolis}
%\usecolortheme{spruce}
%\metroset{block=fill}

% define Metropolis colors    
\definecolor{mAlert}{HTML}{EB811B}
\definecolor{mExample}{HTML}{14B03D}
\definecolor{mBlock}{HTML}{23373b}

% my blocks
\setlength\fboxsep{0pt}%

\newcommand{\myblock}[1]{\colorbox{mBlock!12}{\begin{minipage}{\linewidth}#1\end{minipage}}}
\newcommand{\myblockmath}[1]{\colorbox{mBlock!12}{\begin{minipage}{\linewidth}\vspace{-6pt}#1\end{minipage}}}
\newcommand{\myblockamsmath}[1]{\colorbox{mBlock!12}{\begin{minipage}{\linewidth}\vspace{-6pt}#1\end{minipage}}}
\newcommand{\myblockinline}[1]{\colorbox{mBlock!12}{#1}}
\newcommand{\myexample}[1]{\colorbox{mExample!12}{\begin{minipage}{\linewidth}#1\end{minipage}}}
\newcommand{\myexamplemath}[1]{\colorbox{mExample!12}{\begin{minipage}{\linewidth}\vspace{-6pt}#1\end{minipage}}}
\newcommand{\myexampleamsmath}[1]{\colorbox{mExample!12}{\begin{minipage}{\linewidth}\vspace{-18pt}#1\end{minipage}}}
\newcommand{\myexampleinline}[1]{\colorbox{mExample!12}{#1}}
\newcommand{\myalert}[1]{\colorbox{mAlert!12}{\begin{minipage}{\linewidth}#1\end{minipage}}}
\newcommand{\myalertmath}[1]{\colorbox{mAlert!12}{\begin{minipage}{\linewidth}\vspace{-6pt}#1\end{minipage}}}
\newcommand{\myalertamsmath}[1]{\colorbox{mAlert!12}{\begin{minipage}{\linewidth}\vspace{-18pt}#1\end{minipage}}}
\newcommand{\myalertinline}[1]{\colorbox{mAlert!12}{#1}}

%% other
\newcommand{\mystructure}[1]{\mathcal{#1}}


%% packages
\usepackage{amsmath,amssymb,amsthm}
\usepackage{bookmark}
\usepackage{booktabs}
\usepackage{cancel}
\usepackage[czech]{babel}
\usepackage{enumerate}
\usepackage[T1]{fontenc}
\usepackage{forest}
\usepackage{lmodern}
\usepackage{multicol}
% \usepackage{tcolorbox}
\usepackage{tikz}
    \usetikzlibrary{arrows.meta}
%\usepackage[unicode]{hyperref}
\usepackage[utf8x]{inputenc}
\usepackage{xfrac}


% %% theorems
% \theoremstyle{plain}
%     \newtheorem{theorem}{Věta}[section]
%     \newtheorem*{theorem-unnumbered}{Věta}
%     \newtheorem{proposition}[theorem]{Tvrzení}
%     \newtheorem{corollary}[theorem]{Důsledek}
%     \newtheorem{lemma}[theorem]{Lemma}
%     \newtheorem{observation}[theorem]{Pozorování}
% \theoremstyle{definition}
%     \newtheorem{definition}[theorem]{Definice}
%     \newtheorem*{algorithm}{Algoritmus}
% \theoremstyle{remark}
%     \newtheorem{remark}[theorem]{Poznámka}
%     \newtheorem{example}[theorem]{Příklad}
%     \newtheorem{exercise}{Cvičení}[chapter]
%     \newtheorem*{solution}{Řešení}

%% macros and definitions
\DeclareMathOperator{\Aut}{Aut}
\DeclareMathOperator{\Conseq}{Csq}
\DeclareMathOperator{\DeLO}{DeLO}
\DeclareMathOperator{\dom}{dom}
\DeclareMathOperator{\Fm}{Fm}
\DeclareMathOperator{\M}{M}
%\DeclareMathOperator{\Proof}{Proof}
\DeclareMathOperator{\rng}{rng}
\DeclareMathOperator{\Term}{Term}
\DeclareMathOperator{\Th}{Th}
\DeclareMathOperator{\Thm}{Thm}
\DeclareMathOperator{\Tree}{Tree}
\DeclareMathOperator{\Var}{Var}
\DeclareMathOperator{\VF}{VF}

\newcommand{\A}{\mystructure{A}}
\newcommand{\B}{\mystructure{B}}
\newcommand{\Con}{\mathit{Con}}
\newcommand{\disjointunion}{\mathbin{\dot{\sqcup}}}
\newcommand{\F}{\ensuremath{\mathrm{F}}}
\newcommand{\landsymb}{{\land}}
\newcommand{\lbin}{\mathbin{\square}}
\newcommand{\lbinsymb}{{\lbin}}
\newcommand{\liff}{\mathbin{\leftrightarrow}}
\newcommand{\liffsymb}{{\liff}}
\newcommand{\limplies}{\mathbin{\rightarrow}}
\newcommand{\limpliessymb}{{\limplies}}
\newcommand{\lorsymb}{{\lor}}
\newcommand{\Prf}{\mathit{Prf}}
%\newcommand{\structure}[1]{\mathcal{#1}}
\newcommand{\todo}{[TODO]}
\newcommand{\T}{\ensuremath{\mathrm{T}}}
\newcommand{\union}{\mathbin{\cup}}

\DeclareRobustCommand\proves{\mathrel{|}\joinrel\mkern-.5mu\mathrel{-}}

\title{Šestá přednáška}
\subtitle{NAIL062 Výroková a predikátová logika}
\author{Jakub Bulín (KTIML MFF UK)}
% \institute{KTIML MFF UK}
\date{Zimní semestr 2024}


\begin{document}


\maketitle


\begin{frame}{Šestá přednáška}

    \textbf{Program}
        \begin{itemize}
            \item sémantika predikátové logiky 
            \item vlastnosti teorií
        \end{itemize}

    \textbf{Materiály}

        \href{https://github.com/jbulin-mff-uk/nail062/raw/main/lecture/lecture-notes/lecture-notes.pdf}{\alert{\textbf{Zápisky z přednášky}}}, Sekce 6.1-6.5 z Kapitoly 6

\end{frame}


\section{6.4 Sémantika}


\begin{frame}{Sémantika neformálně}

    \begin{itemize}
        \item \alert{modely jsou struktury} dané signatury,
        \item formule \alert{platí} ve struktuře, pokud platí při každém ohodnocení volných proměnných prvky z domény,
        \item \alert{hodnoty termů} (jsou to prvky z domény) se vyhodnocují podle jejich stromů, kde symboly nahradíme jejich interpretacemi (relacemi, funkcemi, a konstantami z domény),
        \item z hodnot termů získáme \alert{pravdivostní hodnoty atomických formulí}: je výsledná $n$-tice v relaci?
        \item hodnoty složených formulí vyhodnocujeme také podle jejich stromu, přičemž \alert{$(\forall x)$ hraje roli `konjunkce přes všechny prvky'} a  \alert{$(\exists y)$ hraje roli `disjunkce přes všechny prvky'} z domény struktury
    \end{itemize}    

\end{frame}


\begin{frame}{Modely jazyka}

    \medskip

    \myblock{
    \alert{Model jazyka $L$}, nebo také \alert{$L$-struktura}, je libovolná struktura v signatuře jazyka $L$. \alert{Třídu} všech modelů jazyka označíme \alert{$\M_L$}.
    }

    \begin{itemize}
        \item zda je jazyk s rovností nebo bez nehraje roli
        \item proč \alert{třída} a ne \alert{množina} všech modelů $\M_L$? doména je libovolná neprázdná množina, `množina všech množin' neexistuje; třída je \alert{`soubor'} všech množin splňujících danou vlastnost (popsatelnou v \alert{jazyce teorie množin})
    \end{itemize}
    
    Mezi \myexampleinline{modely jazyka uspořádání} $L=\langle \leq \rangle$ patří: \begin{itemize}
        \item částečně uspořádané množiny $\langle \mathbb N,\leq\rangle$, $\langle \mathbb Q, > \rangle$, $\langle\mathcal P(X),\subseteq\rangle$
        \item libovolný orientovaný graf $G=\langle V,E\rangle$, typicky není částečné uspořádání, tj. nesplňuje axiomy \alert{teorie uspořádání}
        \item $\langle \mathbb C,R^\mathbb C\rangle$ kde $(z_1,z_2)\in R^\mathbb C$ právě když $|z_1|=|z_2|$ (není č. usp.)
        %\item $\langle \{0,1\},\emptyset\rangle$
    \end{itemize}

\end{frame}


\begin{frame}{Hodnota termu}

    Mějme term $t$ jazyka $L=\langle \mathcal R,\mathcal F\rangle$ a $L$-strukturu $\A=\langle A,\mathcal R^\A,F^\A\rangle$. \alert{Ohodnocení proměnných} v množině $A$ je lib. funkce $e:\Var\to A$.
    
    \myblock{
        \alert{Hodnota termu} $t$ \alert{ve struktuře} $\A$ \alert{při ohodnocení} $e$, značíme \alert{$t^\A[e]$}, je definovaná induktivně:
        \begin{itemize}
            \item $x^\A[e]=e(x)$ pro proměnnou $x\in\Var$,
            \item $c^\A[e]=c^\A$ pro konstantní symbol $c\in\mathcal F$, a
            \item je-li $t=f(t_1,\dots,t_n)$ složený term, kde $f\in\mathcal F$, potom:
            $$
            t^\A[e]=f^\A(t_1^\A[e],\dots,t_n^\A[e])
            $$
        \end{itemize}
    }

    \begin{itemize}
        \item závisí pouze na ohodnocení proměnných vyskytujících se v $t$
        \item obecně, term $t$ reprezentuje \alert{termovou funkci} $f_t^\A\colon A^k\to A$, kde $k$ je počet proměnných v $t$
        \item speciálně, hodnota konstantního termu na ohodnocení nezávisí, konstantní termy reprezentují konstantní funkce
    \end{itemize}
    
\end{frame}


\begin{frame}{Hodnota termu: příklady}
    
    1. Hodnota termu \myexampleinline{
            $t=-(x\lor \bot)\land y$
         } v Booleově algebře $\A=\underline{\mathcal P(\{0,1,2\})}$ při ohodnocení $e$ ve kterém:
        \begin{itemize}
            \item $e(x)=\{0,1\}$
            \item $e(y)=\{1,2\}$
        \end{itemize}
  
        \myalertmath{
            $$t^\A[e]=\{2\}$$
        }
    
    \bigskip\bigskip

    2. Hodnota termu \myexampleinline{
            $x+1$
         } ve struktuře $\mathcal N=\langle\mathbb N,\cdot,3\rangle$ jazyka $L=\langle +,1\rangle$ při ohodnocení $e$ ve kterém $e(x)=2$ 

        \myalertmath{
            $$(x+1)^\mathcal N[e]=6$$
        }
    
\end{frame}


\begin{frame}{Pravdivostní hodnota formule}

    \myblock{\smallskip
    Buď $\varphi$ v jazyce $L$, $\A\in\M_L$, $e:\Var\to A$ ohodnocení proměnných. \alert{Pravdivostní hodnota} $\varphi$ v $\A$ při ohodnocení $e$, \alert{$\mathrm{PH}^\A(\varphi)[e]$}:
    \begin{itemize}
        \item pro atomickou formuli $R(t_1,\dots,t_n)$:
        $$
        \mathrm{PH}^\A(R(t_1,\dots,t_n))[e]=
        \begin{cases}
            1 & \text{pokud }(t_1^\A[e],\dots,t_n^\A[e])\in R^\A\\
            0 & \text{jinak}    
        \end{cases}
        $$        
        \item pro formuli tvaru $(\neg\varphi)$:
        $$
        \mathrm{PH}^\A(\neg \varphi)[e]=f_\neg(\mathrm{PH}^\A(\varphi)[e])=1-\mathrm{PH}^\A(\varphi)[e]
        $$
        \item pro formuli tvaru $(\varphi\lbin\psi)$ kde $\lbinsymb\in\{\landsymb,\lorsymb,\limpliessymb,\liffsymb\}$:
        $$
        \mathrm{PH}^\A(\varphi\lbin\psi)[e]=f_\lbinsymb(\mathrm{PH}^\A(\varphi)[e],\mathrm{PH}^\A(\psi)[e])
        $$
    \end{itemize}
    \smallskip
    }
    
\end{frame}


\begin{frame}{Pravdivostní hodnota formule: zbytek definice a poznámky}
    
    \myblock{
    \smallskip
    \begin{itemize}
        \item pro formuli tvaru $(Qx)\varphi$ kde $Q\in\{\forall,\exists\}$:    
        \begin{align*}
            \mathrm{PH}^\A((\forall x)\varphi)[e]&=\min_{a\in A}(\mathrm{PH}^\A(\varphi)[e(x/a)])\\ 
            \mathrm{PH}^\A((\exists x)\varphi)[e]&=\max_{a\in A}(\mathrm{PH}^\A(\varphi)[e(x/a)])
        \end{align*}
        kde \alert{$e(x/a)$} je ohodnocení získané z $e$ změnou $e(x)$ na $a$
    \end{itemize}
    \smallskip
    }

    \textbf{Pozorování:} Závisí pouze na ohodnocení volných proměnných. Speciálně, pro sentenci nezávisí na ohodnocení.

    \begin{itemize}
        \item tedy v ohodnocení $e$ nastavíme hodnotu proměnné $x$ postupně na všechny prvky $a\in A$ a požadujeme, aby PH byla jedna vždy (v případě $\forall$) nebo alespoň jednou (v případě $\exists$)
        \item speciálně, $\mathrm{PH}^\A(t_1=t_2)[e]=1$ $\Leftrightarrow$ $(t_1^\A[e],t_2^\A[e])\in {=^\A}$ (\alert{identita} na $A$), tj. $t_1^\A[e]=t_2^\A[e]$ (je to stejný prvek $A$)
    \end{itemize}
    
\end{frame}


\begin{frame}{Příklady}
    
    Vezměme si uspořádané těleso $\underline{\mathbb Q}$. Potom:
    \begin{itemize}
        \item $\mathrm{PH}^{\underline{\mathbb Q}}(x\leq 1 \land \neg (x\leq 0))[e]=1$ právě když $e(x)\in (0,1]$
        \item $\mathrm{PH}^{\underline{\mathbb Q}}((\forall x)(x\cdot y = y))[e]=1$ právě když $e(y)=0$
        \item $\mathrm{PH}^{\underline{\mathbb Q}}((\exists x)(x \leq 0 \land \neg x=0))[e]=1$ pro každé ohodnocení $e$ (je to sentence)
    \end{itemize}   
    Ale pro strukturu $\A=\langle \mathbb N,+,-,0,\cdot,1,\leq\rangle$ máme: 
    \begin{itemize}
        \item $\mathrm{PH}^{\A}((\exists x)(x \leq 0 \land \neg x=0))[e]=0$ 
    \end{itemize}     

\end{frame}


\begin{frame}{Platnost ve struktuře}
    
    \myblock{
        Mějme formuli $\varphi$, strukturu $\A$ (ve stejném jazyce), a ohodnocení $e$.

        \begin{itemize}
            \item je-li $\mathrm{PH}^\A(\varphi)[e]=1$, $\varphi$ \alert{platí} v $\A$ \alert{při ohodnocení} $e$, \alert{$\A\models\varphi[e]$}
            \item je-li $\mathrm{PH}^\A(\varphi)[e]=0$, $\varphi$ \alert{neplatí} v $\A$ \alert{při ohodnoc.} $e$, \alert{$\A\not\models\varphi[e]$} 
            \item $\varphi$ je \alert{pravdivá} (\alert{platí}) v $\A$, \alert{$\A\models\varphi$}, pokud platí při každém ohodnocení $e:\Var\to A$
            \item $\varphi$ je \alert{lživá} v $\A$, pokud neplatí při žádném ohodnocení (v tom případě $\A\models\neg\varphi$)                       
        \end{itemize} 
    }
    
    \begin{itemize}
        \item pozor, \alert{lživá} není totéž, co \alert{není pravdivá} (\alert{neplatí})! \\(je to pravda jen pro sentence)
        \item \alert{platnost} je klíčový pojem sémantiky a celé logiky
    \end{itemize}
    
\end{frame}


\begin{frame}{Zřejmé vlastnosti platnosti ve struktuře při ohodnocení}

    \begin{itemize}
        \item $\A\models\neg\varphi[e]$ právě když $\A\not\models\varphi[e]$
        \item $\A\models(\varphi\land\psi)[e]$ právě když $\A\models\varphi[e]$ a $\A\models\psi[e]$
        \item $\A\models(\varphi\lor\psi)[e]$ právě když $\A\models\varphi[e]$ nebo $\A\models\psi[e]$
        \item $\A\models(\varphi\limplies\psi)[e]$ právě když platí: jestliže $\A\models\varphi[e]$ potom $\A \models\psi[e]$
        \item $\A\models(\varphi\liff\psi)[e]$ právě když platí: $\A\models\varphi[e]$ právě když $\A\models\psi[e]
        $
        \item \myalertinline{
            $\A\models(\forall x)\varphi[e]$ právě když $\A\models\varphi[e(x/a)]$ pro každé $a\in A$
            }
        \item \myalertinline{
            $\A\models(\exists x)\varphi[e]$ právě když $\A\models\varphi[e(x/a)]$ pro nějaké $a\in A$
        }
        \item je-li term $t$ substituovatelný za proměnnou $x$ do $\varphi$, potom:
        \myalertamsmath{
        $$
        \A\models\varphi(x/t)[e]\text{ právě když }\A\models\varphi[e(x/a)]\text{ pro }a=t^\A[e]
        $$
        }
        \item je-li $\psi$ varianta $\varphi$, potom $\A\models\varphi[e]$ právě když $\A\models\psi[e]$
    \end{itemize}

    (dokažte si snadno z definic, najděte protipříklady)  

\end{frame}


\begin{frame}{Zřejmé vlastnosti platnosti ve struktuře}

    \begin{itemize}
        \item pokud $\A\models\varphi$, potom $\A\not\models\neg\varphi$; je-li $\varphi$ sentence, platí i opačná implikace
        \item $\A\models\varphi\land\psi$ právě když $\A\models\varphi$ a $\A\models\psi$
        \item pokud $\A\models\varphi$ nebo $\A\models\psi$, potom $\A\models\varphi\lor\psi$; je-li $\varphi$ sentence,  platí i opačná implikace.
        \item $\A\models\varphi$ právě když $\A\models
        (\forall x)\varphi$
        \item speciálně, $\varphi(x_1,\dots,x_n)$ platí ve struktuře $\A$, právě když v $\A$ platí její \alert{generální uzávěr}, tj. sentence $(\forall x_1)\cdots(\forall x_n)\varphi$ 
    \end{itemize}

    (dokažte si snadno z definic, najděte protipříklady)

\end{frame}


\section{6.5 Vlastnosti teorií}


\begin{frame}{Platnost v teorii}

    \begin{itemize}
        \item \alert{teorie} jazyka $L$ je množina $L$-formulí, její prvky jsou \alert{axiomy}
        \item \alert{model} teorie $T$ je $L$-struktura, ve které platí všechny axiomy $T$, tj. $\A\models\varphi$ pro všechna $\varphi\in T$, značíme $\A\models T$
        \item \alert{třída modelů} teorie $T$ je:
        $$
        \alert{\M_L(T)}=\{\A\in\M_L\mid\A\models T\}
        $$
    \end{itemize}
    
    Je-li $T$ teorie v jazyce $L$ a $\varphi$ $L$-formule, potom $\varphi$ je:
    \begin{itemize}
        \item \alert{pravdivá (platí) v $T$}, značíme \alert{$T\models\varphi$}, pokud $\A\models\varphi$ pro všechna $\A\in\M(T)$ (neboli: \myalertinline{
            $\M(T)\subseteq\M(\varphi)$
            })
        \item \alert{lživá v $T$}, pokud $T\models\neg\varphi$, tj. pokud je lživá v každém modelu $T$ (neboli: \myalertinline{$\M(T)\cap\M(\varphi)=\emptyset$})
        \item \alert{nezávislá v $T$}, pokud není pravdivá v $T$ ani lživá v $T$
        \item je-li $T=\emptyset$ (tj. $\M(T)=\M_L$), píšeme jen \alert{$\models\varphi$}, a říkáme, že $\varphi$ \alert{je pravdivá (v logice), (logicky) platí, je tautologie}, apod.
    \end{itemize}
    
\end{frame}


\begin{frame}{Další sémantické pojmy o teorii}
    
    \begin{itemize}
        \item $T$ je \alert{sporná}, pokud v ní platí \alert{spor} $\bot$ (definujeme jako $R(x_1,\dots,x_n)\land \neg R(x_1,\dots,x_n)$, kde $R$ je lib. relační symbol)
        \item $T$ je sporná, právě když v ní platí každá formule (ekvivalentně, nemá žádný model), jinak je \alert{bezesporná} (neplatí-li v ní spor, má-li alespoň jeden model)
        \item \alert{důsledky} $T$ jsou \alert{sentence} pravdivé v $T$, množina všech důsledků $T$ v jazyce $L$ je
        $$
        \alert{\Conseq_L(T)}=\{\varphi\mid\text{$\varphi$ je sentence a }T\models \varphi\}
        $$
    \end{itemize}

\end{frame}


\begin{frame}{Kompletnost v predikátové logice}

    %Připomeňme, že \alert{výroková} teorie je kompletní, je-li bezesporná a každý výrok v ní buď platí, nebo platí jeho negace. Ekvivalentně, má právě jeden model.

    
    \begin{itemize}
        \item $T$ je \alert{kompletní}, je-li bezesporná a každá \alert{sentence} je v ní buď pravdivá, nebo lživá. \myalertinline{Pozor: neplatí, že má jediný model!}
        \item máme-li jeden model, máme i nekonečně mnoho \alert{izomorfních} modelů (liší se jen pojmenováním prvků, definujeme později)
        \item uvažovat jediný model \alert{až na izomorfismus} ale také \alert{nestačí}!
    \end{itemize}
    
    \myblock{ 
        Struktury $\A,\B$ (v témž jazyce) jsou \alert{elementárně ekvivalentní}, píšeme \alert{$\A\equiv\B$}, pokud v nich platí tytéž sentence.
    }
 
    \textbf{Pozorování:} Teorie je kompletní, právě když má právě jeden model \alert{až na elementární ekvivalenci}.   

    Příklad: uspořádané množiny \myexampleinline{
        $\A=\langle\mathbb Q,\leq\rangle$
     } a \myexampleinline{
        $\B=\langle\mathbb R,\leq\rangle$
     }.
     \begin{itemize}
        \item \alert{nejsou izomorfní}, $\mathbb Q$ je spočetná a $\mathbb R$ nespočetná množina, neexistuje dokonce žádná \alert{bijekce} mezi doménami
        \item \alert{ale $\A\equiv\B$}: indukcí dle struktury sentence $\varphi$ lze ukázat $\A\models\varphi\Leftrightarrow\B\models\varphi$; netriviální případ je $\exists$, klíčová je \alert{hustota} uspořádání $(x\leq y\land \neg x=y)\limplies(\exists z)(x\leq z\land z\leq y\land \neg x=z\land\neg y=z)$
     \end{itemize}     

\end{frame}


\begin{frame}{Platnost pomocí nesplnitelnosti}

    Otázku platnosti v teorii lze převést na problém existence modelu:

    \medskip

    \myblock{
    \textbf{Tvrzení (O nesplnitelnosti a pravdivosti):}
        Je-li $T$ teorie a $\varphi$ \alert{sentence} (v témž jazyce), potom: $T\models\varphi$ $\Leftrightarrow$ $T\cup\{\neg\varphi\}$ nemá model.
    }

    \medskip

    \textbf{Důkaz:}
        Platí následující ekvivalence: 
        \begin{itemize}
            \item $T\cup\{\neg\varphi\}$ nemá model,
            \item právě když $\neg\varphi$ neplatí v žádném modelu $T$,
            \item právě když $\varphi$ platí v každém modelu $T$  ($\varphi$ je sentence!). \hfill\qedsymbol
        \end{itemize}


    NB: Předpoklad, že $\varphi$ je sentence, je nutný: pro $T=\{P(c)\}$ a formuli $\varphi=P(x)$ je $P(c)\not\models P(x)$ ale $\{P(c),\neg P(x)\}$ nemá model.

\end{frame}


\begin{frame}{Příklady teorií: Teorie grafů}
    
    \myexampleinline{Teorie grafů}: $L=\langle E\rangle$ s rovností, axiomy \alert{ireflexivity} a \alert{symetrie}
    
    \myalertmath{
    $$
    T_\text{graph}=\{\neg E(x, x),E(x,y)\limplies E(y,x)\}
    $$
    }

    \textbf{Modely:} $\mathcal G=\langle G,E^\mathcal G\rangle$, kde $E^\mathcal G$ je symetrická ireflexivní relace, tj. \alert{jednoduché} grafy, hranu $\{x,y\}$ reprezentuje dvojice $(x,y),(y,x)$
    \begin{itemize}
        \item Formule \myexampleinline{
            $\neg x=y\limplies E(x,y)$
         } platí v grafu, právě když je \alert{úplný}. Je tedy nezávislá v $T_\text{graph}$.
        \item Formule \myexampleinline{
            $(\exists y_1)(\exists y_2)(\neg y_1=y_2\land E(x,y_1)\land E(x,y_2)\land$
         } \myexampleinline{
            $(\forall z)(E(x,z)\limplies z=y_1\lor z=y_2)$
        } vyjadřuje, že každý vrchol má stupeň právě 2. Platí tedy právě v grafech, které jsou disjunktní sjednocení kružnic, a je nezávislá v teorii $T_\text{graph}$.
    \end{itemize}

\end{frame}


\begin{frame}{Příklady teorií: Teorie uspořádání}

    \myexampleinline{Teorie uspořádání}: v jazyce uspořádání $L=\langle\leq\rangle$ s rovností, axiomy \alert{reflexivity}, \alert{antisymetrie}, a \alert{tranzitivity}

    \myalertamsmath{
    $$
    T=\{ x\leq x,\ x\leq y\land y\leq x\limplies x=y,\ x\leq y\land y\leq z\limplies x\leq z\}
    $$
    }

    \textbf{Modely:} $\langle S,\leq^S\rangle$, kde $\leq^S$ je \alert{částečné uspořádání}. 
    
    Příklad: \myexampleinline{
        $\A=\langle\mathbb N,\leq\rangle$
    }, \myexampleinline{
        $\B=\langle\mathcal P(X),\subseteq\rangle$
     } pro $X=\{0,1,2\}$.
    \begin{itemize}
        \item Formule \myexampleinline{
            $x\leq y\lor y\leq x$
         } (\alert{linearita}) platí v $\A$, ale neplatí v $\B$: neplatí např. při ohodnocení kde $e(x)=\{0\}$, $e(y)=\{1\}$ (píšeme $\B\not\models\varphi[e]$). Je tedy nezávislá v $T$.
        \item Sentence \myexampleinline{
            $(\exists x)(\forall y)(y\leq x)$
         } (označme $\psi$) je pravdivá v $\B$ a lživá v $\A$, píšeme $\B\models\psi$, $\A\models\neg\psi$. Je také nezávislá v $T$.
        \item Formule \myexampleinline{
            $(x\leq y\land y\leq z\land z\leq x)\limplies (x=y\land y=z)$
         } (označme  $\chi$) je pravdivá v $T$, píšeme $T\models\chi$. Totéž platí pro její \alert{generální uzávěr} $(\forall x)(\forall y)(\forall z)\chi$.
    \end{itemize}

\end{frame}


\begin{frame}{Příklady teorií: Algebraické teorie 1/2}

    \myexampleinline{Teorie grup}: $L=\langle +,-,0\rangle$ s rovností, axiomy 
    \alert{asociativita $+$}, \alert{neutralita $0$ vůči $+$}, a \alert{$-x$ je inverzní prvek k $x$ (vůči $+$ a $0$)}
    {\small
    \begin{align*}
        T_1=\{& x + (y + z) = (x + y) + z,\\
            & 0 + x = x,\ x + 0 = 0,\\
            & x + (-x) = 0,\ (-x) + x = 0\}\\
    \end{align*}
    }
   
    \vspace{-24pt}

    \myexampleinline{Teorie komutativních grup}: navíc \alert{komutativita $+$} 
    {\small
    $$
    T_2=T_1\cup\{x+y=y+x\}
    $$
    }

    \vspace{-6pt}

    \myexampleinline{Teorie okruhů}: $L=\langle +,-,0,\cdot,1\rangle$ s rovností, navíc \alert{neutralita $1$ vůči~$\cdot$}, \alert{asociativita~$\cdot$}, a \alert{(levá i pravá) distributivita $\cdot$ vůči $+$}
    {\small
    \begin{align*}
        T_3=T_2\cup\{   & 1 \cdot x = x \cdot 1,\\
        & x \cdot (y \cdot z) = (x \cdot y) \cdot z,\\
        & x \cdot (y + z) = x \cdot y + x \cdot z,\\
        & (x + y) \cdot z = x \cdot z + y \cdot z\}
    \end{align*}
    }
\end{frame}
    

\begin{frame}{Příklady teorií: Algebraické teorie 2/2}
    
    \myexampleinline{Teorie komutativních okruhů}: navíc axiom \alert{komutativity $\cdot$}:
    {\small
    $$
    T_4 = T_3 \cup \{x \cdot y = y \cdot x\}
    $$
    }
    
    \myexampleinline{Teorie těles} je ve stejném jazyce, ale má navíc axiomy \alert{existence inverzního prvku k $\cdot$} a \alert{netriviality}:
    {\small
    $$
    T_5 = T_4 \cup \{\neg\,x=0 \limplies (\exists y)(x\cdot y = 1),\ \neg\,0=1\}
    $$
    }

    \myexampleinline{Teorie uspořádaných těles} je v jazyce $\langle +, -, 0,\cdot,1,\leq\rangle$ s rovností, sestává z axiomů teorie těles, teorie uspořádání spolu s axiomem linearity, a z následujících axiomů \alert{kompatibility uspořádání}: 

    \begin{itemize}
        \item $x\leq y\limplies (x+z\leq y+z)$
        \item $(0\leq x\land 0\leq y)\limplies 0\leq x\cdot y$
    \end{itemize}

    Modely jsou tělesa s \alert{lineárním (totálním)} uspořádáním, které je kompatibilní s tělesovými operacemi.

\end{frame}


\end{document}
