\documentclass[a4paper,12pt]{article}

%% slide-specific

\usetheme[progressbar=frametitle]{metropolis}
%\usecolortheme{spruce}
%\metroset{block=fill}

% block indentation workaround
% map defaulf block to oldblock
\let\oldblock\block
\let\endoldblock\endblock
% change block by adding smallskip
\renewenvironment{block}[1]
    {\begin{oldblock}{#1}
        \smallskip
    }
    { 
    \end{oldblock}
    }

% define Metropolis colors    
\definecolor{mAlert}{HTML}{EB811B}
\definecolor{mExample}{HTML}{14B03D}
\definecolor{mBlock}{HTML}{23373b}

\usepackage[most]{tcolorbox}

%\newcommand{\myexample}[1]{\leavevmode\textcolor{mExample}{#1}}
\newcommand{\myalert}[1]{
\begin{tcolorbox}[colback=mAlert!10, enhanced, boxrule=0pt, boxsep=-1mm, frame hidden, left=2mm, right=2mm]
    {#1}  
\end{tcolorbox}
}
\newcommand{\myexample}[1]{
\begin{tcolorbox}[colback=mExample!10, enhanced, boxrule=0pt, boxsep=-1mm, frame hidden, left=2mm, right=2mm]
    {#1}  
\end{tcolorbox}
}
\newcommand{\myblock}[1]{
\begin{tcolorbox}[colback=mBlock!10, enhanced, boxrule=0pt, boxsep=-1mm, frame hidden, left=2mm, right=2mm]
    {#1}  
\end{tcolorbox}
}

\newcommand{\mystructure}[1]{\mathcal{#1}}



% \newcommand{\myexamplemath}[1]{
% \begin{tcolorbox}[colback=mExample!10, enhanced, boxrule=0pt, frame hidden]
%     \ensuremath{#1}  
% \end{tcolorbox}
% }


%% packages
\usepackage{amsmath,amssymb,amsthm}
\usepackage{booktabs}
\usepackage[czech]{babel}
\usepackage{enumerate}
\usepackage{forest}
\usepackage{multicol}
% \usepackage{tcolorbox}
\usepackage{tikz}
    \usetikzlibrary{arrows.meta}
%\usepackage[unicode]{hyperref}
\usepackage[utf8]{inputenc}
\usepackage{xfrac}

% %% theorems
% \theoremstyle{plain}
%     \newtheorem{theorem}{Věta}[section]
%     \newtheorem*{theorem-unnumbered}{Věta}
%     \newtheorem{proposition}[theorem]{Tvrzení}
%     \newtheorem{corollary}[theorem]{Důsledek}
%     \newtheorem{lemma}[theorem]{Lemma}
%     \newtheorem{observation}[theorem]{Pozorování}
% \theoremstyle{definition}
%     \newtheorem{definition}[theorem]{Definice}
%     \newtheorem*{algorithm}{Algoritmus}
% \theoremstyle{remark}
%     \newtheorem{remark}[theorem]{Poznámka}
%     \newtheorem{example}[theorem]{Příklad}
%     \newtheorem{exercise}{Cvičení}[chapter]
%     \newtheorem*{solution}{Řešení}

%% macros and definitions
\DeclareMathOperator{\Aut}{Aut}
\DeclareMathOperator{\Conseq}{Csq}
\DeclareMathOperator{\DeLO}{DeLO}
\DeclareMathOperator{\dom}{dom}
\DeclareMathOperator{\Fm}{Fm}
\DeclareMathOperator{\M}{M}
%\DeclareMathOperator{\Proof}{Proof}
\DeclareMathOperator{\rng}{rng}
\DeclareMathOperator{\Term}{Term}
\DeclareMathOperator{\Th}{Th}
\DeclareMathOperator{\Thm}{Thm}
\DeclareMathOperator{\Tree}{Tree}
\DeclareMathOperator{\Var}{Var}
\DeclareMathOperator{\VF}{VF}

\newcommand{\A}{\structure{A}}
\newcommand{\B}{\structure{B}}
\newcommand{\Con}{\mathit{Con}}
\newcommand{\disjointunion}{\mathbin{\dot{\sqcup}}}
\newcommand{\F}{\ensuremath{\mathrm{F}}}
\newcommand{\landsymb}{{\land}}
\newcommand{\lbin}{\mathbin{\square}}
\newcommand{\lbinsymb}{{\lbin}}
\newcommand{\liff}{\mathbin{\leftrightarrow}}
\newcommand{\liffsymb}{{\liff}}
\newcommand{\limplies}{\mathbin{\rightarrow}}
\newcommand{\limpliessymb}{{\limplies}}
\newcommand{\lorsymb}{{\lor}}
\newcommand{\Prf}{\mathit{Prf}}
\newcommand{\proves}{\vdash}
%\newcommand{\structure}[1]{\mathcal{#1}}
\newcommand{\todo}{[TODO]}
\newcommand{\T}{\ensuremath{\mathrm{T}}}
\newcommand{\union}{\mathbin{\cup}}



\begin{document}

\section*{NAIL062 V\&P Logika: 7. cvičení}
% po 5. přednášce


\textbf{Témata:}
Rezoluce ve výrokové logice. Aplikace věty o kompaktnosti. Hilbertův kalkulus.


\medskip\begin{problem}
    Označme jako $\varphi$ výrok $\neg (p \vee q) \to (\neg p \wedge \neg q)$. Ukažte, že $\varphi$ je tautologie:
    \begin{enumerate}
        \item Převeďte $\neg \varphi$ do CNF a zapište výsledný výrok jako formuli $S$ v množinové reprezentaci.
        \item Najděte rezoluční zamítnutí $S$.
    \end{enumerate}
    \end{problem}
    
    
    \medskip\begin{problem}
    Najděte rezoluční zamítnutí následujících výroků:
    \begin{enumerate}
        \item $\neg(((p\to q)\to \neg q)\to \neg q)$
        \item $(p\leftrightarrow (q\to r))\wedge((p\leftrightarrow q)\wedge(p\leftrightarrow \neg r))$
        
    \end{enumerate}
    \end{problem}
        
        
    \medskip\begin{problem}
    Dokažte rezolucí, že v teorii $T=\{\neg p \to \neg q,\neg q \to \neg r, (r\to p)\to s\}$ platí výrok $s$.
    \end{problem}
    
    
    \medskip\begin{problem}Nechť prvovýroky $r$, $s$, $t$  reprezentují (po řadě), že \emph{``Radka / Sára / Tom je ve škole''} a označme $\mathbb{P}=\{r,s,t\}$. Víme, že
        \begin{itemize}\it
        \item Není-li Tom ve škole, není tam ani Sára.
        \item Radka bez Sáry do školy nechodí.
        \item Není-li Radka ve škole, je tam Tom.
        \end{itemize}
        \begin{enumerate}
        \item Formalizujte naše znalosti jako teorii $T$ v jazyce $\mathbb P$.
        \item Rezoluční metodou dokažte, že z $T$ vyplývá, že \emph{Tom je ve škole}: Napište formuli $S$ v množinové reprezentaci, která je nesplnitelná, právě když to platí, a najděte rezoluční zamítnutí $S$. Nakreslete rezoluční strom.
        \item Určete množinu modelů teorie $T$.
        \end{enumerate}
    \end{problem}
    
    
    \medskip\begin{problem} Máme k dispozici MgO, H$_2$, O$_2$, a C, a můžeme provádět následující reakce:
        \begin{itemize}
            \item MgO\ +\ H$_2$\ \ $\to$\ \ Mg\ +\ H$_2$O
            \item C\ +\ O$_2$\ \ $\to$\ \ CO$_2$
            \item CO$_2$\ +\ H$_2$O\ \ $\to$\ \ H$_2$CO$_3$
        \end{itemize}
        \begin{enumerate}
            \item Reprezentujte naše možnosti výrokem %(nad vhodně zvoleným jazykem) 
            a převeďte ho do množinové reprezentace.
            \item Pomocí rezoluce dokažte, že můžeme získat H$_2$CO$_3$. Lze najít LI-důkaz téhož?
        \end{enumerate}
    \end{problem}
    
    
    \medskip\begin{problem}
        Najděte rezoluční uzávěry $\mathcal{R}(S)$ pro následující výroky $S$:
        \begin{enumerate}
            \item $\{\{p,q\},\{p,\neg q\},\{\neg p,\neg q\}\}$
            \item $\{\{p,\neg q,r\},\{q,r\},\{\neg p, r\},\{q,\neg r\},\{\neg q\}\}$
        \end{enumerate}
    \end{problem}
        
        
    \medskip\begin{problem}
        Zkonstruujte \emph{strom dosazení} pro následující formuli: 
        $$
        S=\{\{p,r\},\{q,\neg r\},\{\neg q\},\{\neg p,t\},\{\neg s\},\{s,\neg t\}\}
        $$
    \end{problem}
    
    
    \medskip\begin{problem}
        Dokažte podrobně, že je-li $S=\{C_1,C_2\}$ splnitelná a $C$ je rezolventa $C_1$ a $C_2$, potom je i $C$ splnitelná.
    \end{problem}
    
        
    \medskip\begin{problem} Dokažte pomocí věty o kompaktnosti a variant tvrzení pro konečné objekty:
    \begin{enumerate}
        \item Každý spočetný rovinný graf je obarvitelný čtyřmi barvami.
        \item Každé spočetné částečné uspořádání lze rozšířit na úplné (lineární) uspořádání.
        %\item Hallova věta platí i pro nekonečné množiny.
    \end{enumerate}
    
    \end{problem}
        
    
    \medskip\begin{problem}
    V Hilbertově kalkulu dokažte pro libovolné formule následující vztahy:
    \begin{enumerate}
        %\item $_H\ p\to p$
        \item $\{\neg p\}\ _H\ p\to q$
        \item $\{\neg(\neg p)\}\ _H\ p$
        \item $\{p\to q,q \to r\}\ _H\ p\to r$
    \end{enumerate}    
    \end{problem}
    
    \medskip\begin{problem}
        Dokažte korektnost Hilbertova kalkulu:
        \begin{itemize}
            \item Dokažte, že logické axiomy jsou tautologie.
            \item Dokažte, že modus ponens je korektní, tj. když $T\models\varphi$ a $T\models\varphi\to\psi$, tak $T\models\psi$.
            \item Ukažte, že $T\ _H\ \varphi$ implikuje $T\models\varphi$.
        \end{itemize}
        \end{problem}
        
    \medskip\begin{problem}
        Vyslovte a dokažte větu o dedukci pro Hilbertův kalkul.
    \end{problem}
       



\end{document}