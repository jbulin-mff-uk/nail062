\chapter{Undecidability and Incompleteness}

In this final chapter, we will explore how theories can be worked with algorithmically. The highlight will be the \emph{Gödel's Incompleteness Theorems} from 1931, which show the limits of the formal approach and halted the decades-long program of formalizing mathematics. We do not have enough space to provide formal definitions and complete proofs here, so we will sometimes operate on a somewhat intuitive level. We will focus on understanding the meaning of the statements and the ideas of the proofs.

We will understand the concept of an \emph{algorithm} intuitively as well. If we wanted to formalize it, the most common (but by no means the only) choice would be the concept of a \emph{Turing machine}.\footnote{See the lecture NTIN090 Foundations of Complexity and Computability.}

\section{Recursive Axiomatization and Decidability}

In the proof systems we have discussed (the tableau method, resolution, Hilbert calculus), we allowed the theory $T$ in which we prove to be infinite. However, we have not yet considered how it is given. If we want to verify that a given object (tableau, resolution tree, sequence of formulas) is a correct proof, we need some algorithmic access to all axioms of $T$.

One possibility would be to require an \emph{enumerator} for $T$, i.e., an algorithm that lists the axioms of $T$ on the output, and each axiom is eventually listed.\footnote{A necessary condition is that $T$ is countable. It suffices to assume that the language is countable.} Then it would be easy to confirm that a given proof is correct. However, if we received a proof that used an incorrect axiom that is not in $T$, we would never know: we would wait infinitely long to see if the enumerator lists it. Therefore, we require a stronger property that allows recognizing incorrect proofs as well: \emph{recursive axiomatization}.\footnote{The word \emph{recursive} here does not mean the commonly known recursion, but refers to the formalization of an algorithm using `recursive functions' by Gödel. Recursive functions here mean the same as computable by some Turing machine, and the theory of computability is sometimes called \emph{recursion theory}.}

\begin{definition}[Recursive Axiomatization]
    A theory $T$ is \emph{recursively axiomatized} if there is an algorithm that, for every input formula $\varphi$, halts and answers whether $\varphi \in T$.
\end{definition}

\begin{remark}
    In fact, an enumerator for $T$ would suffice if it is guaranteed to list the axioms in lexicographic order. That is equivalent to recursive axiomatization. (Think about why.)
\end{remark}

We focus on the question of whether we can `algorithmically decide the truth' in a given theory $T$ (i.e., the validity of an input formula). If so, we say the theory is \emph{decidable}. This is quite a strong property, so we also define \emph{partial decidability}, which means that if the formula is valid, the algorithm will tell us, but if it is not, we may never get an answer.

\begin{definition}[Decidability]
A theory $T$ is said to be
\begin{itemize}
    \item \emph{decidable} if there is an algorithm that, for every input formula $\varphi$, halts and answers whether $T \models \varphi$,
    \item \emph{partially decidable} if there is an algorithm that, for every input formula:
    \begin{itemize}
        \item if $T \models \varphi$, halts and answers "yes",
        \item if $T \not\models \varphi$, either does not halt or halts and answers "no".
    \end{itemize}
\end{itemize}
\end{definition}
We can usually assume that $\varphi$ in the definition is a sentence. Let us show a simple proposition:

\begin{proposition}
    Let $T$ be recursively axiomatized. Then:
    \begin{enumerate}[(i)]
        \item $T$ is partially decidable,
        \item if $T$ is also complete, then it is decidable.
    \end{enumerate}
\end{proposition}
\begin{proof}
An algorithm showing partial decidability is the construction of a systematic tableau for $\F \varphi$.\footnote{Here, an enumerator of the axioms of $T$ suffices, or we can systematically generate all sentences (e.g., in lexicographic order) and test whether they are axioms.} If $\varphi$ is valid in $T$, the construction will end in finitely many steps and we can easily verify that the tableau is contradictory; otherwise, it may not halt.

If $T$ is complete, we know that exactly one of the following holds: either $T \proves \varphi$ or $T \proves \neg \varphi$. We will then construct parallel tableaux for $\F \varphi$ and $\T \varphi$ (proof and refutation of $\varphi$ from $T$): one of the constructions will halt after finitely many steps.
\end{proof}


\subsection{Recursively Enumerable Completion}

The requirement of completeness is too strong; we will show that it suffices if we can effectively describe all complete simple extensions.\footnote{I.e., `all models up to elementary equivalence'.}

\begin{definition}[Recursively Enumerable Completion]
We say that a theory $T$ has a \emph{recursively enumerable completion} if (some) set up to equivalence of all simple complete extensions of the theory $T$ is \emph{recursively enumerable}, i.e., there is an algorithm that, for a given pair of natural numbers $(i,j)$, outputs the $i$-th axiom of the $j$-th extension (in some fixed order\footnote{Here, we need the language to be countable.}), or answers that such an axiom does not exist.\footnote{If there are fewer than $j$ extensions or if the $j$-th extension has fewer than $i$ axioms.}
\end{definition}

\begin{proposition}\label{propositon:recursively-enumerable-completion}    
    If a theory $T$ is recursively axiomatized and has a recursively enumerable completion, then $T$ is decidable.
\end{proposition}
\begin{proof}
For a given sentence $\varphi$, either $T \proves \varphi$, or there is a counterexample $\A \not\models \varphi$, i.e., a complete simple extension $T_i$ of the theory $T$ such that $T_i \not\proves \varphi$. From completeness, it follows that $T_i \proves \neg \varphi$. Our algorithm will construct in parallel the tableau proof of $\varphi$ from $T$ and (sequentially) the tableau proofs of $\neg \varphi$ from all complete simple extensions $T_1, T_2, \dots$ of the theory $T$.\footnote{It does not matter that there are infinitely many of them; we can use so-called \emph{dovetailing}: Perform the 1st step of constructing the 1st tableau, then the 2nd step of the 1st tableau and the 1st step of the 2nd tableau, the 3rd step of the 1st tableau, the 2nd step of the 2nd tableau, the 1st step of the 3rd tableau, etc.} We know that at least one of the parallel constructed tableaux is contradictory, and we can assume it is finite (if we do not extend contradictory branches of the tableau), so the algorithm will construct it after finitely many steps.
\end{proof}

\begin{exercise}
Show that the following theories have a recursively enumerable completion:
\begin{itemize}
\item The theory of pure equality (the empty theory in the language $L = \langle \rangle$ with equality),
\item The theory of a unary predicate (the empty theory in the language $L = \langle U \rangle$ with equality, where $U$ is a unary relational symbol),
\item The theory of dense linear orders DeLO* (the complete simple extensions are described in Corollary \ref{corollary:complete-simple-extensions-of-delo}),
\end{itemize}
These are recursively axiomatized theories (as they are finite), so they are decidable.
\end{exercise}

\begin{example}
    Finally, without proof, let us mention a few more examples of decidable theories:
    \begin{itemize}  
        \item The theory of Boolean algebras (Alfred Tarski 1940),
        \item The theory of algebraically closed fields (Tarski 1949),
        \item The theory of commutative groups (Wanda Szmielew 1955).
    \end{itemize}
    These theories are also incomplete, but recursively axiomatized and have a recursively enumerable completion.    
\end{example}

 
\subsection{Recursive Axiomatizability}

In the previous chapter, specifically in Section \ref{section:axiomatizability}, we addressed the question of when a class of structures [or a theory] can be described using axioms [of a certain form]. Now, let us focus on the question of when this can be done \emph{algorithmically}.

\begin{definition}[Recursive Axiomatizability]
A class of $L$-structures $K \subseteq \M_L$ is \emph{recursively axiomatizable} if there is a recursively axiomatized $L$-theory $T$ such that $K = \M_L(T)$. A theory $T'$ is \emph{recursively axiomatizable} if the class of its models is recursively axiomatizable, i.e., if $T'$ is equivalent to some recursively axiomatized theory.
\end{definition}
\begin{remark}
    Similarly, we could define \emph{recursively enumerable axiomatizability}.
\end{remark}

Let us show the following simple proposition:

\begin{proposition}
    If $\A$ is a finite structure in a finite language with equality, then the theory $\Th(\A)$ is recursively axiomatizable.
\end{proposition}
\begin{remark}
    It follows that $\Th(\A)$ is decidable, which is not surprising: the validity of a sentence $\varphi$ in a finite structure $\A$ can be easily verified.
\end{remark}
\begin{proof}
    Let us enumerate the elements of the domain as $A = \{a_1, \dots, a_n\}$. The theory $\Th(\A)$ can be axiomatized by a single sentence of the form ``there exist exactly $n$ elements $a_1, \dots, a_n$ satisfying precisely those \emph{basic relations} on function values and relations that hold in the structure $\A$''.\footnote{For example, if $f^\A(a_4, a_2) = a_{17}$, we add the atomic formula $f(x_{a_4}, x_{a_2}) = x_{a_{17}}$ to the conjunction (where $x_{a_i}$ are variables corresponding to the individual elements). And if $(a_3, a_3, a_1) \notin R^\A$, we add $\neg R(x_{a_3}, x_{a_3}, x_{a_1})$.}    
\end{proof}
 
Let us give some standard examples of structures that can be `described algorithmically':

\begin{example}\label{example:structures-recursively-axiomatizable}
For the following structures, $\Th(\A)$ is recursively axiomatizable and thus decidable:

\begin{itemize}
    \item $\langle \mathbb{Z}, \leq \rangle$, this is the so-called theory of \emph{discrete linear orders},        
    \item $\langle \mathbb{Q}, \leq \rangle$, this is the theory DeLO,
    \item $\langle \mathbb{N}, S, 0 \rangle$, the theory of \emph{successor with zero},
    \item $\langle \mathbb{N}, S, +, 0 \rangle$, \emph{Presburger arithmetic},
    \item $\langle \mathbb{R}, +, -, \cdot, 0, 1 \rangle$, the theory of \emph{real closed fields},\footnote{This significant result by A. Tarski (1949) also means that it is possible to algorithmically decide which properties hold in Euclidean geometry.}
    \item $\langle \mathbb{C}, +, -, \cdot, 0, 1 \rangle$, the theory of \emph{algebraically closed fields of characteristic 0}.
\end{itemize}
\end{example}
   
\begin{corollary}
    For the structures listed in Example \ref{example:structures-recursively-axiomatizable}, it holds that $\Th(\A)$ is decidable.
\end{corollary}


\begin{remark}\label{remark:std-arithmetic-not-recursively-axiomatizable}
    As follows from Gödel's First Incompleteness Theorem (see below), the \emph{standard model of arithmetic}, i.e., the structure $\underline{\mathbb{N}} = \langle \mathbb{N}, S, +, \cdot, 0, \leq \rangle$, does \emph{not} have a recursively axiomatizable theory.
\end{remark}


\section{Arithmetic}

The properties of natural numbers play an important role not only in mathematics but also in cryptography, for example. Recall that the language of arithmetic is the language $L = \langle S, +, \cdot, 0, \leq \rangle$ with equality. As mentioned in Remark \ref{remark:std-arithmetic-not-recursively-axiomatizable}, the so-called \emph{standard model of arithmetic} $\underline{\mathbb{N}} = \langle \mathbb{N}, S, +, \cdot, 0, \leq \rangle$ does not have a recursively axiomatizable theory. Therefore, we use recursively axiomatized theories that attempt to describe the properties of $\underline{\mathbb{N}}$ partially; these theories are called \emph{arithmetics}.

\subsection{Robinson and Peano Arithmetic}

We will mention only the two most important examples of arithmetics: \emph{Robinson arithmetic} and \emph{Peano arithmetic}.

\begin{definition}[Robinson Arithmetic]
\emph{Robinson arithmetic} is the theory $Q$ in the language of arithmetic consisting of the following (finite) axioms:
\begin{align*}
    &\neg S(x) = 0& &x \cdot 0 = 0\\
    &S(x) = S(y) \rightarrow x = y& &x \cdot S(y) = x \cdot y + x\\
    &x + 0 = x& &\neg x = 0 \rightarrow (\exists y)(x = S(y))\\
    &x + S(y) = S(x + y)& &x \le y \leftrightarrow (\exists z)(z + x = y)\qquad
\end{align*}
\end{definition}

Robinson arithmetic is very weak; it cannot prove, for example, the commutativity or associativity of addition or multiplication, or the transitivity of order.

On the other hand, it can prove all \emph{existential statements about numerals} that are true in $\underline{\mathbb{N}}$. By this, we mean formulas that, in prenex form, have only existential quantifiers, and into which we have substituted \emph{numerals} $\underline{n} = S(\dots S(0) \dots)$ for the free variables.

\begin{example}
For instance, for the formula $\varphi(x,y)$ of the form $(\exists z)(x + z = y)$, $Q \proves \varphi(\underline{1}, \underline{2})$, where $\underline{1} = S(0)$ and $\underline{2} = S(S(0))$.    
\end{example}

Thus, the following proposition holds, which we will leave without proof.

\begin{proposition}\label{proposition:robinson-satisfies-existence-about-numerals}
    If $\varphi(x_1, \dots, x_n)$ is an existential formula and $a_1, \dots, a_n \in \mathbb{N}$, then it holds that:
    $$
    Q \proves \varphi(x_1 / \underline{a_1}, \dots, x_n / \underline{a_n}) \text{ if and only if } \underline{\mathbb{N}} \models \varphi[e(x_1 / a_1, \dots, x_n / a_n)]
    $$
\end{proposition}

A useful extension of Robinson arithmetic is the so-called Peano arithmetic, in which \emph{induction} can be proved:

\begin{definition}[Peano Arithmetic]
\emph{Peano arithmetic} $\text{\it PA}$ is an extension of Robinson arithmetic $Q$ with the \emph{induction schema}, i.e., for each $L$-formula $\varphi(x, \overline{y})$, the following axiom is added:
$$
(\varphi(0, \overline{y}) \land (\forall x)(\varphi(x, \overline{y}) \limplies \varphi(S(x), \overline{y}))) \limplies (\forall x)\varphi(x, \overline{y})
$$
\end{definition}

Peano arithmetic is a much better approximation of the theory $\Th(\underline{\mathbb{N}})$; it can prove all `basic' properties valid in $\underline{\mathbb{N}}$ (such as the commutativity and associativity of addition). However, there are still sentences in the language of arithmetic that are valid in $\underline{\mathbb{N}}$ but are independent in Peano arithmetic.\footnote{As we will show in Gödel's First Incompleteness Theorem.} 



%todo







\begin{remark}
Pokud bychom se přesunuli do logiky \emph{2. řádu}, potom bychom už mohli strukturu $\underline{\mathbb N}$ axiomatizovat (až na izomorfismus), a to extenzí Peanovy aritmetiky o následující formuli 2. řádu, tzv. \emph{axiom indukce}:
$$
(\forall X)((X(0) \land (\forall x)(X(x) \limplies X(S(x)))) \limplies (\forall x)X(x))
$$
Připomeňme, že $X$ reprezentuje (libovolnou) unární relaci, neboli podmnožinu univerza. Použitím axiomu indukce na množinu následníků 0 získáme, že každý prvek (daného modelu) je následníkem nuly. Tak můžeme sestrojit izomorfismus s $\underline{\mathbb N}$.
\end{remark}

\section{Nerozhodnutelnost predikátové logiky}
    
V této sekci si ukážeme, že nelze (algoritmicky) rozhodovat logickou platnost formulí prvního řádu. (Jinými slovy, nerozhodnutelnost prázdné teorie nad jazykem daným na vstupu.)

\begin{theorem}[O nerozhodnutelnosti predikátové logiky]\label{theorem:undecidability-of-predicate-logic}
Neexistuje algoritmus, který by pro danou vstupní formuli $\varphi$ rozhodl, zda je logicky platná.\footnote{Tj. zda je formule $\varphi$ tautologie, neboli zda $\models\varphi$. Zde mluvíme o formulích 1. řádu, v libovolném jazyce.}
\end{theorem}

Protože zatím neznáme potřebný formalismus týkající se algoritmů, např. pojem Turingova stroje, zvolíme jako výchozí bod jiný \emph{nerozhodnutelný problém}. Nejznámějším je tzv. \emph{Halting problem}, tj. otázka, zda se daný program zastaví na daném vstupu.\footnote{Jeho nerozhodnutelnost si dokážete v předmětech NTIN071 Automaty a gramatiky a poté znovu v NTIN090 Základy složitosti a vyčíslitelnosti.} My si ale usnadníme práci tím, že zvolíme jiný nerozhodnutelný problém, tzv. \emph{Hilbertův desátý problém}.\footnote{Hilbert jej vyslovil v roce 1900, a publikoval v roce 1902 spolu s 22 dalšími problémy, které významně ovlivnily matematiku 20., i 21. století. Některé zůstávají nevyřešeny, např. Riemannova hypotéza, \href{https://https://en.wikipedia.org/wiki/Riemann_hypothesis}{viz Wikipedia}.}

\subsection{Hilbertův desátý problém}

Mějme polynom $p(x_1,\dots,x_n)$ s celočíselnými koeficienty. Hilbertův desátý problém se ptá po algoritmu, který rozhodne, zda má takový vstupní polynom celočíselný kořen, neboli zda má \emph{Diofantická rovnice}  $p(x_1,\dots,x_n)=0$ (celočíselné) řešení:
\begin{quote}
    ``Nalezněte algoritmus, který po konečně mnoha krocích určí, zda daná Diofantická rovnice s libovolným počtem proměnných a
    celočíselnými koeficienty má celočíselné řešení.''
\end{quote}

Kdyby se Hilbert dožil vyřešení svého desátého problému v roce 1970, byl by překvapen, že žádný takový algoritmus neexistuje.

\begin{theorem}[Matiyasevich, Davis, Putnam, Robinson]
Problém existence celočíselného řešení dané Diofantické rovnice s celočíselnými koeficienty je (algoritmicky) nerozhodnutelný.
\end{theorem}

Důkaz zde pro nedostatek místa neuvedeme. K důkazu nerozhodnutelnosti ve skutečnosti použijeme následující důsledek, který mluví o polynomech s přirozenými koeficienty, a o řešení v přirozených číslech. 

\begin{corollary}
Neexistuje algoritmus, který by pro danou dvojici polynomů $p(x_1,\dots,x_n)$, $q(x_1,\dots,x_n)$ s \emph{přirozenými} koeficienty rozhodl, zda mají přirozené řešení, tj. zda platí:
$$
\underline{\mathbb N}\models(\exists x_1)\dots(\exists x_n)\ p(x_1,\dots,x_n)=q(x_1,\dots,x_n)
$$
\end{corollary}
\begin{proof}[Důkaz důsledku]
Důkaz je snadný, využívá faktu, že každé celé číslo lze vyjádřit jako rozdíl dvojice přirozených čísel, a naopak, každé přirozené číslo lze vyjádřit jako součet čtyř čtverců (celých čísel).\footnote{Tzv. Lagrangeova věta o čtyřech čtvercích.} Každou Diofantickou rovnici lze tedy transformovat na rovnici z důsledku, a naopak.
\end{proof}


\subsection{Důkaz nerozhodnutelnosti}

Připomeňme, že Robinsonova aritmetika $Q$ má jen konečně mnoho axiomů, $\underline{\mathbb N}$ je jejím modelem, a lze v ní dokázat všechna \emph{existenční tvrzení o numerálech} platná v $\underline{\mathbb N}$. Nyní jsme připraveni dokázat Větu o nerozhodnutelnosti predikátové logiky.

\begin{proof}[Důkaz věty o nerozhodnutelnosti predikátové logiky]
Uvažme formuli $\varphi$ tvaru 
$$(\exists x_1)\dots(\exists x_n)\ p(x_1,\dots,x_n)=q(x_1,\dots,x_n)
$$ 
kde $p$ a $q$ jsou polynomy s přirozenými koeficienty. Dle Tvrzení \ref{proposition:robinson-satisfies-existence-about-numerals} platí:
$$
\underline{\mathbb N}\models \varphi\text{ právě když }Q\proves \varphi
$$

Označme jako $\psi_Q$ konjunkci (generálních uzávěrů) všech axiomů $Q$. Zřejmě $Q\proves\varphi$, právě když $\psi_Q\proves\varphi$, což platí právě když $\proves\psi_Q\limplies\varphi$. Dle Věty o úplnosti je to ale ekvivalentní $\models\psi_Q\limplies\varphi$. Dostáváme tedy následující ekvivalenci:
$$
\underline{\mathbb N}\models \varphi\text{ právě když }\models \psi_Q\limplies\varphi
$$
To znamená, že pokud existoval algoritmus rozhodující logickou platnost, mohli bychom rozhodovat i existenci přirozeného řešení rovnice $p(x_1,\dots,x_n)=q(x_1,\dots,x_n)$, neboli Hilbertův desátý problém by byl rozhodnutelný.\footnote{Ukazujeme, že existuje \emph{redukce} `těžkého' problému (Hilbertova desátého) na náš problém, tedy i náš problém je `těžký'.} Což by byl spor.   
\end{proof}

\section{Gödelovy věty}

Na závěr přednášky představíme slavné Gödelovy věty o neúplnosti, jejichž pochopení by mělo být samozřejmou součástí vzdělání každého informatika. Pokusíme se vysvětlit i princip důkazů, ale vynecháme veškeré technické detaily.

\subsection{První věta o neúplnosti}

Nejprve vyslovíme Gödelovu \emph{První větu o neúplnosti}, a vysvětlíme smysl jejích předpokladů.

\begin{theorem}[První věta o neúplnosti]
Pro každou bezespornou rekurzivně axiomatizovanou extenzi $T$ Robinsonovy aritmetiky existuje sentence, která je pravdivá v $\underline{\mathbb N}$, ale není dokazatelná v $T$.    
\end{theorem}

Takové sentenci se říká \emph{Gödelova sentence}. Velmi neformálně řečeno, Gödelova První věta o neúplnosti říká, že vlastnosti aritmetiky přirozených čísel nelze `rozumně', efektivně popsat (v logice 1. řádu), každý takový popis je nutně `neúplný'. Je důležité si uvědomit, že mluvíme o \emph{pravdivosti} ve standardním modelu aritmetiky, tj. ve struktuře $\underline{\mathbb N}$, zatímco \emph{dokazatelnost} je v teorii $T$. (Z Věty o úplnosti samozřejmě plyne, že každá sentence \emph{pravdivá v $T$} je v $T$ i dokazatelná.)

Bezespornost je nutným předpokladem, neboť ve sporné teorii je dokazatelná každá sentence. Připomeňme, že rekurzivní axiomatizovanost můžeme chápat jako `efektivní zadání' axiomů (pomocí algoritmu), bez této vlastnosti by taková teorie nebyla užitečná. Požadavek aby teorie byla extenzí Robinsonovy aritmetiky chápejte jako předpoklad, že má alespoň `základní aritmetickou sílu', že v ní lze určitým způsobem `mluvit' o přirozených číslech. Existují různé varianty tohoto předpokladu, s jinými teoriemi než je Robinsonova aritmetika, a není například nutné, aby šlo přímo o extenzi, stačí, když je v teorii Robinsonova aritmetika v jistém smyslu `definovatelná'. Ale teorie, ve které `nelze zakódovat přirozená čísla' (a zde je důležité, že můžeme mluvit nejen o sčítání, ale i o násobení), je `příliš slabá'.

Je dobré si uvědomit, že speciálně platí i následující tvrzení `o nekompletnosti':

\begin{corollary}
    Splňuje-li teorie $T$ předpoklady První věty o neúplnosti a je-li navíc $\underline{\mathbb N}$ modelem teorie $T$, potom $T$ není kompletní.
\end{corollary}
\begin{proof}
    Předpokládejme pro spor, že $T$ je kompletní. Vezměme sentenci $\varphi$, která je pravdivá v $\underline{\mathbb N}$ ale není dokazatelná v $T$. Díky kompletnosti víme, že $T\proves\neg\varphi$, potom ale Věta o korektnosti říká, že  $T\models\neg\varphi$, tedy $\varphi$ je lživá v $\underline{\mathbb N}$, což je spor.
\end{proof}

Zajímavé je nejen samotné tvrzení První věty o neúplnosti, ale také její důkaz: Gödel v něm přišel se zcela novou, na svou dobu převratnou důkazovou technikou. Sentence sestrojená v důkazu formalizuje tvrzení \emph{``Nejsem dokazatelná v $T$''}, důkaz je založen na následujících dvou principech, které níže poněkud neformálně popíšeme:
\begin{itemize}
    \item \emph{aritmetizace syntaxe}, tedy zakódování sentencí a jejich dokazatelnosti do přirozených čísel,
    \item \emph{self-reference}, tedy schopnost sentence `mluvit sama o sobě' (o svém kódu).
\end{itemize}

\subsubsection*{Aritmetizace dokazatelnosti}

Konečné syntaktické objekty, jako jsou termy, formule, konečná tabla, a tedy i tablo důkazy, lze `rozumně' zakódovat do přirozených čísel.\footnote{Představte si jakýkoliv rozumný způsob, jak daný objekt zapsat do souboru. Soubor v binárním kódu je posloupnost 0 a 1. Připíšeme na začátek jedničku, abychom nezačínali nulou, a máme binární zápis přirozeného čísla.} Konkrétní způsob jak to lze provést, tzv. \emph{Gödelovo číslování}, jako technický detail přeskočíme. Stačí nám, že jsme schopni objekty `algoritmicky' kódovat a dekódovat (případně `simulovat manipulaci s objekty' na jejich kódech).

Označme kód formule $\varphi$ jako $\lceil\varphi\rceil$, podobně pro jiné syntaktické objekty. Numerál odpovídající kódu $\varphi$, tedy $\lceil\varphi\rceil$-tý numerál, budeme značit $\underline{\varphi}$. Pro danou teorii $T$ definujme následující binární relaci na množině všech přirozených čísel:
$$
(n,m)\in\Proof_T\ \text{ právě když \ \ $n=\lceil\varphi\rceil$ a $m=\lceil\tau\rceil$, kde $\tau$ je tablo důkaz sentence $\varphi$ z $T$}
$$
Máme-li efektivní přístup k axiomům, umíme také efektivně zkontrolovat zda $\tau$ je opravdu důkazem $\varphi$ (kde $\tau$ a $\varphi$ získáme dekódováním $m$ a $n$), tedy platí:
\begin{observation}
Je-li $T$ rekurzivně axiomatizovaná, je relace $\Proof_T\subseteq\mathbb N^2$ \emph{rekurzivní}. 
\end{observation}

Klíčovou, ale velmi technickou částí důkazu První věty je následující tvrzení, které ponecháme bez důkazu.

\begin{proposition}
Je-li $T$ navíc extenzí Robinsonovy aritmetiky $Q$, potom existuje formule $\Prf_T(x,y)$ v jazyce aritmetiky, která \emph{reprezentuje} relaci $\Proof_T$, tj. pro každá $n,m\in\mathbb N$ platí:
\begin{itemize}
    \item Je-li $(n,m)\in\Proof_T$, potom $Q\proves\Prf_T(\underline{n},\underline{m})$,
    \item jinak $Q\proves\neg\Prf_T(\underline{n},\underline{m})$.
\end{itemize} 
\end{proposition}

Formule $\Prf_T(x,y)$ tedy vyjadřuje \emph{``$y$ je důkaz $x$ v $T$''}.\footnote{Přesněji, tablo jehož kódem je $y$ je důkazem sentence jejíž kódem je $x$.} Potom můžeme vyjádřit, že \emph{``$x$ je dokazatelná v $T$''}, a to formulí $(\exists y)\Prf_T(x,y)$. Všimněte si, že platí následující pozorování, neboť svědek poskytuje kód nějakého tablo důkazu, a $\underline{\mathbb N}$ splňuje axiomy $Q$:
\begin{observation}\label{observation:proof-predicate}
$T\proves\varphi$ právě když $\underline{\mathbb N}\models (\exists y)\Prf_T(\underline{\varphi},y)$.  
\end{observation}
Budeme potřebovat i následující důsledek, který vyslovíme také bez důkazu:
\begin{corollary}[O predikátu dokazatelnosti]\label{corollary:predicate-of-provability}
    Je-li $T\proves\varphi$, potom $T\proves (\exists y)\Prf_T(\underline{\varphi},y)$.
\end{corollary}

Umíme tedy vyjádřit, že daná sentence je, nebo není, dokazatelná. Jak ale může sentence říci `sama o sobě', že není dokazatelná? K tomu použijeme \emph{princip self-reference}.

\subsubsection*{Self-reference}

Abychom ilustrovali princip self-reference, pro názornost si místo logické sentence představme větu v češtině, a místo vlastnosti ``být dokazatelný'' tvrzení o počtu písmen. Podívejme se na následující větu:
\begin{quote}
    \texttt{Tato věta má 22 znaků.}
\end{quote}
V přirozeném jazyce snadno vyjádříme self-referenci zájmenem ``Tato'', z kontextu víme, že myslíme větu samou. Ve formálních systémech ale typicky nemáme self-referenci přímo k dispozici. \emph{Přímou referenci} obvykle máme k dispozici, stačí umět `mluvit' o posloupnostech symbolů, jako v následujícím příkladě:
\begin{quote}
    \texttt{Následující věta má 29 znaků. "Následující věta má 29 znaků."}
\end{quote}
Zde se ale není žádná self-reference. Pomůžeme si trikem, kterému budeme říkat `zdvojení':
\begin{quote}
    \texttt{Následující věta zapsaná jednou a ještě jednou v uvozovkách má 149\\ znaků. "Následující věta zapsaná jednou a ještě jednou v uvozovkách\\ má 149 znaků."}
\end{quote}
Pomocí přímé reference a zdvojení tedy můžeme získat self-referenci.\begin{remark}
    Stejný princip lze použít k sestrojení programu v C, jehož výstupem je jeho vlastní kód (34 je ASCII kód uvozovek):    
{\small
\begin{verbatim}
main(){char *c="main(){char *c=%c%s%c; printf(c,34,c,34);}"; printf(c,34,c,34);}    
\end{verbatim}
}
\end{remark}


\subsection{Důkaz a důsledky}

V této podsekci dokážeme První Gödelovu větu o neúplnosti a řekneme si i něco o jejích důsledcích. Budeme potřebovat následující větu, která popisuje, jak technicky využijeme princip self-reference. Lze na ní nahlížet jako na formu `diagonalizačního argumentu',\footnote{Diagonalizací se myslí argument připomínající \emph{Cantorův diagonální argument}, známý z důkazu nespočetnosti $\mathbb R$. Podobný argument, používající self-referenci, potkáme třeba v \emph{Holičově paradoxu}, nebo v důkazu nerozhodnutelnosti \emph{Halting problému}.} proto se tomuto tvrzení také někdy říká \emph{diagonální lemma}.

\begin{theorem}[Věta o pevném bodě]
Je-li $T$ extenzí Robinsonovy aritmetiky, potom pro každou formuli $\varphi(x)$ (v jazyce teorie $T$) existuje sentence $\psi$ taková, že platí: 
$$
T\proves \psi \liff \varphi(\underline{\psi})
$$
\end{theorem}
Sentence $\psi$ je tedy \emph{self-referenční}, říká o sobě: ``splňuji vlastnost $\varphi$''.\footnote{Přesněji, říká to o numerálu odpovídajícímu jejímu kódu.} Vysvětlíme si jen myšlenku důkazu. Všimněte si, jak se v důkazu použije přímá reference a zdvojení.
\begin{proof} Uvažme \emph{zdvojující funkci}, funkci $d\colon\mathbb N\to\mathbb N$ takovou, že pro každou formuli $\chi(x)$ platí:
$$
d(\lceil \chi(x)\rceil)=\lceil\chi(\underline{\chi(x)})\rceil
$$
Funkce $d$ tedy dostane na vstupu přirozené číslo $n$, které dekóduje jako formuli v jedné proměnné, dosadí do této formule numerál $\underline{n}$,\footnote{Zde \emph{numerál} odpovídá `uvozovkám' z předchozího neformálního popisu self-reference, a $d(\lceil\chi\rceil)$ znamená ``$\chi$ napsaná jednou a ještě jednou v uvozovkách.''} a výslednou sentenci znovu zakóduje.

S využitím předpokladu, že $T$ je extenzí $Q$, lze dokázat, že tato funkce je v $T$ \emph{reprezentovatelná}. Pro jednoduchost předpokládejme, že je reprezentovatelná termem,\footnote{Byť ve skutečnosti je reprezentovaná (složitou) formulí.} a označme ho také $d$. To znamená, že pro každou formuli $\chi(x)$ platí:
$$
T\proves d(\underline{\chi(x)})=\underline{\chi(\underline{\chi(x)})}
$$
Tedy Robinsonova aritmetika, a tím pádem i naše teorie $T$, dokazuje \emph{o numerálech}, že $d$ opravdu `zdvojuje'.

Hledaná self-referenční sentence $\psi$ je sentence:\footnote{Následující věta zapsaná jednou a ještě jednou v uvozovkách má vlastnost $\varphi$. ``Následující věta zapsaná jednou a ještě jednou v uvozovkách má vlastnost $\varphi$.''}
$$
\varphi(d(\underline{\varphi(d(x))}))
$$
Chceme dokázat, že platí $T\proves \psi \liff \varphi(\underline{\psi})$, neboli $T \proves \varphi(d(\underline{\varphi(d(x))}))\liff\varphi(\underline{\varphi(d(\underline{\varphi(d(x))}))})$. K~tomu stačí ověřit, že:
$$
T \proves d(\underline{\varphi(d(x))})=\underline{\varphi(d(\underline{\varphi(d(x))}))}
$$
To ale víme z reprezentovatelnosti $d$, kde za formuli $\chi(x)$ dosadíme $\varphi(d(x))$.
\end{proof}

Než přistoupíme k samotnému důkazu Gödelovy věty, ukážeme si jako rozcvičku jeden důsledek Věty o pevném bodě: \emph{Definicí pravdy} v aritmetické teorii $T$ myslíme formuli $\tau(x)$ takovou, že pro každou sentenci $\psi$ platí: 
$$
T\proves\psi\liff\tau(\underline{\psi})
$$
Pokud by definice pravdy existovala, znamenalo by to, že místo dokazování sentence stačí spočíst její kód, substituovat příslušný numerál do $\tau$, a vyhodnotit.

\begin{theorem}[Nedefinovatelnost pravdy]
    V žádném bezesporném rozšíření Robinsonovy aritmetiky neexistuje definice pravdy.
\end{theorem}
Důkaz využívá \emph{Paradox lháře}, vyjádříme větu ``Nejsem pravdivá v $T$''.
\begin{proof}
Předpokládejme pro spor, že existuje definice pravdy $\tau(x)$.
Použijeme Větu o pevném bodě, kde za formuli $\varphi(x)$ vezmeme $\neg\tau(x)$. Dostáváme existenci sentence $\psi$ takové, že:
$$
T\proves\psi\liff\neg\tau(\underline{\psi})
$$
Protože $\tau(x)$ je definice pravdy, platí ale i $T\proves\psi\liff\tau(\underline{\psi})$, tedy i $T\proves\tau(\underline{\psi})\liff\neg\tau(\underline{\psi})$. To by ale znamenalo, že $T$ je sporná.
\end{proof}

Důkaz Gödelovy věty používá tentýž trik, ale pro větu ``Nejsem dokazatelná v $T$''.

\begin{proof}[Důkaz První věty o neúplnosti]
Mějme bezespornou rekurzivně axiomatizovanou extenzi $T$ Robinsonovy aritmetiky. Chceme najít Gödelovu sentenci $\psi_T$, která je pravdivá v $\underline{\mathbb N}$, ale není dokazatelná v $T$.

Takovou sentenci získáme z Věty o pevném bodě jako sentenci vyjadřující ``Nejsem dokazatelná v $T$''. Nechť $\varphi(x)$ je formule $\neg(\exists y)\Prf_T(x,y)$ (``$x$ není dokazatelná v $T$''). Podle Věty o pevném bodě existuje sentence $\psi_T$ splňující:
$$
T\proves\psi_T\liff\neg(\exists y)\Prf_T(\underline{\psi_T},y)
$$
Sentence $\psi_T$ je tedy v $T$ ekvivalentní sentenci, která vyjadřuje, že $\psi_T$ není dokazatelná v $T$. Lze ukázat, že stejná ekvivalence platí i v $\underline{\mathbb N}$ (neboť tak jsme $\Prf_T$ a $\psi_T$ zkonstruovali):
$$
\underline{\mathbb N}\models\psi_T\ \text{ právě když }\ \underline{\mathbb N}\models\neg(\exists y)\Prf_T(\underline{\psi_T},y)
$$
Z Pozorování \ref{observation:proof-predicate} získáváme, že 
$$
\underline{\mathbb N}\models\psi_T\ \text{ právě když }\ T\not\proves\psi_T
$$
neboli $\psi_T$ je pravdivá v $\underline{\mathbb N}$, právě když není dokazatelná v $T$. Stačí tedy ukázat, že $\psi_T$ není dokazatelná v $T$. Předpokládejme pro spor, že $T\proves\psi_T$. Ze self-reference víme, že platí
$T\proves\neg(\exists y)\Prf_T(\underline{\psi_T},y)$.
Z Důsledku \ref{corollary:predicate-of-provability} o predikátu dokazatelnosti ale dostáváme $T\proves (\exists y)\Prf_T(\underline{\psi_T},y)$, což by znamenalo, že $T$ je sporná.
\end{proof}


Na závěr si ukážeme dva důsledky a jedno zesílení. Následující okamžitý důsledek už jsme zmínili dříve:

\begin{corollary}
Je-li $T$ rekurzivně axiomatizovaná extenze Robinsonovy aritmetiky a je-li navíc $\underline{\mathbb N}$ modelem teorie $T$, potom $T$ není kompletní.
\end{corollary}
\begin{proof}
Protože má $T$ model, není sporná. Splňuje tedy předpoklady První věty o neúplnosti, tedy v ní není dokazatelná Gödelova sentence $\psi_T$. Pokud by byla kompletní, musela by dokazovat $\neg\psi_T$. To by ale znamenalo, že platí i $\underline{\mathbb N}\models\neg\psi_T$, přičemž víme, že $\psi_T$ je v $\underline{\mathbb N}$ pravdivá.  
\end{proof}

Z toho plyne, že nelze rekurzivně axiomatizovat standardní model přirozených čísel:
\begin{corollary}
Teorie $\Th(\underline{\mathbb N})$ není rekurzivně axiomatizovatelná.    
\end{corollary}
\begin{proof}
Teorie $\Th(\underline{\mathbb N})$ je extenzí Robinsonovy aritmetiky a platí v modelu $\underline{\mathbb N}$. Pokud by byla rekurzivně axiomatizovatelná, její (libovolná) rekurzivní axiomatizace by podle předchozího důsledku nemohla být kompletní. Ale $\Th(\underline{\mathbb N})$ kompletní je.
\end{proof}

Jedním ze zesílení Gödelovy První věty je následující tvrzení, které uvedeme bez důkazu. Ukazuje, že předpoklad $\underline{\mathbb N}\models T$ v prvním důsledku výše je ve skutečnosti nadbytečný.

\begin{theorem}[Rosserův trik, 1936]
V každé bezesporné rekurzivně axiomatizované extenzi Robinsonovy aritmetiky existuje nezávislá sentence. Tedy taková teorie není kompletní.    
\end{theorem}


\subsection{Druhá věta o neúplnosti}

Druhá Gödelova věta o neúplnosti říká, neformálně řečeno, že efektivně daná, dostatečně bohatá teorie nemůže sama dokázat svou bezespornost. Bezespornost (``konzistenci'') vyjádříme následující sentencí, kterou označíme jako $\Con_T$:
$$
\neg(\exists y)\Prf_T(\underline{0=S(0)},y)
$$
Všimněte si, že platí $\underline{\mathbb N}\models\Con_T$, právě když $T\not\proves 0=S(0)$. Neboli sentence $\Con_T$ opravdu vyjadřuje, že \emph{``Teorie $T$ je bezesporná''.}

\begin{theorem}[Druhá věta o neúplnosti]
Pro každou bezespornou rekurzivně axiomatizovanou extenzi $T$ Peanovy aritmetiky platí, že $\Con_T$ není dokazatelná v $T$.    
\end{theorem}

Všimněte si, že sentence $\Con_T$ je přitom pravdivá v $\underline{\mathbb N}$ (neboť $T$ je opravdu bezesporná). Zmiňme také, že není třeba plná síla Peanovy aritmetiky, stačí slabší předpoklad. Nyní si ukážeme hlavní myšlenku důkazu Druhé věty:

\begin{proof}[Důkaz Druhé věty o neúplnosti]
Vezměme Gödelovu sentenci $\psi_T$ vyjadřující ``nejsem dokazatelná v $T$''. V důkazu První věty o neúplnosti (konkrétně v první části) jsme ukázali, že:
\begin{quote}
    ``Pokud je $T$ bezesporná, potom $\psi_T$ není dokazatelná v $T$.''
\end{quote}
Z toho jednak plyne, že $T\not\proves\psi_T$, neboť $T$ bezesporná je. Na druhou stranu to lze formulovat jako ``platí $\Con_T\to \psi_T$'' a je-li $T$ extenze Peanovy aritmetiky, lze důkaz tohoto tvrzení zformalizovat v rámci teorie $T$, tedy ukázat, že:
$$
T\proves\Con_T\to\psi_T
$$
Kdyby platilo $T\proves\Con_T$, dostali bychom i $T\proves\psi_T$, což by byl spor.
\end{proof}

Na závěr si ukážeme tři důsledky Druhé věty.

\begin{corollary}
    Existuje model {\it PA}, ve kterém platí sentence $(\exists y)\Prf_{\text{\it PA}}(\underline{0=S(0)},y)$.
\end{corollary}
\begin{proof}
    Sentence $\Con_{\text{\it PA}}$ není dokazatelná, tedy ani pravdivá v $\text{\it PA}$. Platí ale v $\underline{\mathbb N}$ (neboť $\text{\it PA}$ je bezesporná), což znamená, že je $\Con_{\text{\it PA}}$ nezávislá v $\text{\it PA}$. V nějakém modelu tedy musí platit její negace, která je ekvivalentní $(\exists y)\Prf_{\text{\it PA}}(\underline{0=S(0)},y)$.
        
\end{proof}
Uvědomme si, že musí jít o nestandardní model $\text{\it PA}$, svědkem musí být    
\emph{nestandardní} prvek (tj. takový, který není hodnotou žádného numerálu).

\begin{corollary}
    Existuje bezesporná rekurzivně axiomatizovaná extenze $T$    Peanovy aritmetiky, která `dokazuje svou spornost', tj. taková, že $T\proves \neg \Con_T$.
\end{corollary}
\begin{proof}
Uvažme teorii $T=\text{\it PA} \cup \{\neg \Con_{\text{\it PA}}\}$. Tato teorie je bezesporná, neboť $\text{\it PA}\not\proves\Con_{\text{\it PA}}$. Také triviálně platí $T\proves\neg\Con_{\text{\it PA}}$ (tj. $T$ `dokazuje spornost' teorie $\text{\it PA}$). Protože je $\text{\it PA}\subseteq T$, platí i $T\proves\neg\Con_T$.
\end{proof}
Zde si uvědomme, že $\underline{\mathbb{N}}$ nemůže být modelem teorie $T$.

Nakonec se podívejme na teorii ZFC, tj. Zermelovu–Fraenkelovu teorii množin s axiomem výběru, na které je založena formalizace matematiky. Tato teorie není formálně vzato extenzí $\text{\it PA}$, ale není problém v ní Peanovu aritmetiku (v jistém smyslu) `interpretovat'. To znamená, že ani tato teorie neumí dokázat svou vlastní bezespornost.

\begin{corollary}
    Je-li teorie množin ZFC bezesporná, nemůže být sentence $\Con_{ZFC}$ v teorii ZFC dokazatelná.
\end{corollary}

Pokud by tedy někdo v rámci teorie ZFC dokázal, že je ZFC bezesporná, znamenalo by to, že je ZFC sporná. Což bude taková pěkná tečka za naší přednáškou.
