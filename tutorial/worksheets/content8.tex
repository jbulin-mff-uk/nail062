\section*{NAIL062 V\&P Logika: 8. sada příkladů -- Tablo metoda v predikátové logice}
% po 5. přednášce


\subsection*{Cíle výuky:} Po absolvování cvičení student

    \begin{itemize}\setlength{\itemsep}{0pt}
        \item rozumí tomu, jak se liší tablo metoda v predikátové logice od výrokové logiky, umí formálně definovat všechny potřebné pojmy
        \item zná atomická tabla pro kvantifikátory, rozumí jejich použití
        \item umí sestrojit dokončené tablo pro danou položku z dané (i nekonečné) teorie
        \item umí popsat kanonický model pro danou dokončenou bezespornou větev tabla
        \item zná axiomy rovnosti a rozumí jejich souvislosti s pojmy kongruence, faktorstruktura
        \item umí aplikovat tablo metodu k řešení daného problému (slovní úlohy, aj.)
        \item rozumí tablo metodě pro jazyky s rovností, umí aplikovat na jednoduchých příkladech
        \item zná větu o kompaktnosti predikátové logiky, umí ji aplikovat
    \end{itemize}
    

\section*{Příklady na cvičení}
        
       
\begin{problem}

    Předpokládejme, že:
    \begin{itemize}\it
        \item Všichni viníci jsou lháři.
        \item Alespoň jeden z obviněných je také svědkem.
        \item Žádný svědek nelže.
    \end{itemize}

    Dokažte tablo metodou, že: {\it Ne všichni obvinění jsou viníci.} Konkrétně:
    \begin{enumerate}[(a)]
        \item Zvolte vhodný jazyk $\mathcal L$. Bude s rovností, nebo bez rovnosti?        
        \item Formalizujte naše znalosti a dokazované tvrzení jako sentence $\alpha_1,\alpha_2,\alpha_3,\varphi$ v jazyce $\mathcal L$.
        \item Sestrojte tablo důkaz sentence $\varphi$ z teorie $T=\{\alpha_1,\alpha_2,\alpha_3\}$.
        %, tj. sporné tablo z teorie $T$ s položkou $\mathrm{F}\varphi$ v kořeni.
    \end{enumerate}

    \begin{solution}

        \begin{enumerate}[(a)]
            \item Zvolme jazyk $\mathcal L=\langle V,L,O,S\rangle$ bez rovnosti, kde $V$, $L$, a $S$ jsou unární relační symboly o významu ``být viníkem/lhářem/svědkem''.
            \item 
            \begin{align*}
                \alpha_1 & = (\forall x)(V(x)\to L(x)) \\
                \alpha_2 & = (\exists x)(O(x)\land S(x)) \\
                \alpha_3 & = \neg(\exists x)(S(x)\land L(x)) \\
                \varphi & = \neg (\forall x)(O(x)\limplies V(x))
            \end{align*}
            \item Sestrojíme dokončené tablo z teorie $T=\{\alpha_1,\alpha_2,\alpha_3\}$ s položkou $F\varphi$ v kořeni. Uvidíme, že všechny větve budou sporné, půjde tedy o tablo důkaz. (\textcolor{blue}{Modře} je vyznačeno připojení axiomů, \textcolor{red}{červeně} jsou kořeny atomických tabel položek typu `všichni', které bychom mohli nekreslit, kdyby nám to konvence dovolila.)
            
            \begin{center}
                \scalebox{1}{
                \begin{forest}
                    for tree={math content}
                    [\F\neg (\forall x)(O(x)\limplies V(x))
                        [\T (\forall x)(O(x)\limplies V(x))
                            [\textcolor{blue}{\T(\exists x)(O(x)\land S(x))}
                                [\T O(c_0)\land S(c_0)
                                    [\T O(c_0)
                                        [\T S(c_0)
                                            [\textcolor{red}{\T (\forall x)(O(x)\limplies V(x))}
                                                [\T O(c_0)\limplies V(c_0)
                                                    [\F O(c_0), tikz={\node[fit to=tree,label=below:$\otimes$] {};}]
                                                    [\T V(c_0)
                                                        [\textcolor{blue}{\T(\forall x)(V(x)\limplies L(x))}
                                                            [\textcolor{red}{\T(\forall x)(V(x)\limplies L(x))}
                                                                [\T V(c_0)\limplies L(c_0)
                                                                    [\F V(c_0), tikz={\node[fit to=tree,label=below:$\otimes$] {};}]
                                                                    [\T L(c_0)
                                                                        [\textcolor{blue}{\T\neg(\exists x)(S(x)\land L(x))}
                                                                            [\F(\exists x)(S(x)\land L(x))
                                                                                [\textcolor{red}{\F(\exists x)(S(x)\land L(x))}
                                                                                    [\F S(c_0)\land L(c_0)
                                                                                        [\F S(c_0), tikz={\node[fit to=tree,label=below:$\otimes$] {};}]           [\F L(c_0), tikz={\node[fit to=tree,label=below:$\otimes$] {};}]
                                                                                    ]
                                                                                ]
                                                                            ]
                                                                        ]
                                                                    ]
                                                                ]
                                                            ]
                                                        ]
                                                    ]       
                                                ]
                                            ]
                                        ]
                                    ]                                                
                                ]
                            ]
                        ]
                    ]
                \end{forest}
                }
            \end{center}            
        \end{enumerate}
                    
    \end{solution}

\end{problem} 
    

\begin{problem}

    Uvažte následující tvrzení:
    \begin{enumerate}[(i)] \it 
        \item Nula je malé číslo.
        \item Číslo je malé, právě když je blízko nuly.
        \item Součet dvou malých čísel je malé číslo.
        \item Je-li $x$ blízko $y$, potom $f(x)$ je blízko $f(y)$.
    \end{enumerate}

    Chceme dokázat, že platí: {\it (v) Jsou-li $x$ a $y$ malá čísla, potom $f(x+y)$ je blízko $f(0)$.}

    \begin{enumerate}[(a)]
        \item Formalizujte tvrzení jako sentence $\varphi_1,\dots,\varphi_5$ v jazyce $L=\langle M,B,f,+,0\rangle$ bez rovnosti.        
        \item Sestrojte dokončené tablo z teorie $T=\{\varphi_1,\varphi_2,\varphi_3,\varphi_4\}$ s položkou $F\varphi_5$ v kořeni. Rozhodněte, zda platí $T\models \varphi_5$.
        \item Pokud existují, uveďte alespoň dvě kompletní jednoduché extenze teorie $T$.
    \end{enumerate}

    \begin{solution}

        \begin{enumerate}[(a)]
            \item \begin{align*}
                \varphi_1 & = M(0) \\
                \varphi_2 & = (\forall x)(M(x)\leftrightarrow B(x,0)) \\
                \varphi_3 & = (\forall x)(\forall y)(M(x)\wedge M(y)\to M(x+y)) \\
                \varphi_4 & = (\forall x)(\forall y)(B(x,y)\to B(f(x),f(y))) \\
                \varphi_5 & = (\forall x)(\forall y)(M(x)\wedge M(y)\to B(f(x+y),f(0)))
            \end{align*}

            \item Tablo vyjde sporné, máme tedy $T\proves \varphi_5$ a z úplnosti $T\models \varphi_5$. Všimněte si, že axiom $\varphi_1 = M(0)$ není potřeba:
            
            \begin{center}
            \scalebox{0.95}{
            \begin{forest}
                for tree={math content}
                [{\F(\forall x)(\forall y)(M(x)\wedge M(y)\to B(f(x+y),f(0)))}
                  [{\F(\forall y)(M(c_0)\wedge M(y)\to B(f(c_0+y),f(0)))}
                    [{\F M(c_0)\wedge M(c_1)\to B(f(c_0+c_1),f(0))}
                      [\T M(c_0)\wedge M(c_1)
                        [{\F B(f(c_0+c_1),f(0))}
                          [\T M(c_0)
                            [\T M(c_1)
                              [\textcolor{blue}{\T(\forall x)(\forall y)(M(x)\wedge M(y)\to M(x+y))}
                                [\textcolor{red}{\T(\forall x)(\forall y)(M(x)\wedge M(y)\to M(x+y))}
                                  [\T(\forall y)(M(c_0)\wedge M(y)\to M(c_0+y))
                                    [\textcolor{red}{\T(\forall y)(M(c_0)\wedge M(y)\to M(c_0+y))}
                                      [\T M(c_0)\wedge M(c_1)\to M(c_0+c_1)
                                        [\F M(c_0)\wedge M(c_1)
                                          [\F M(c_0), tikz={\node[fit to=tree,label=below:$\otimes$] {};}]
                                          [\F M(c_1), tikz={\node[fit to=tree,label=below:$\otimes$] {};}]
                                        ]
                                        [\T M(c_0+c_1)
                                          [\textcolor{blue}{\T(\forall x)(M(x)\leftrightarrow B(x,0))}
                                            [\textcolor{red}{\T(\forall x)(M(x)\leftrightarrow B(x,0))}
                                              [{\T M(c_0+c_1)\leftrightarrow B(c_0+c_1,0)}
                                                [\T M(c_0+c_1)
                                                  [{\T B(c_0+c_1,0)}
                                                    [\textcolor{blue}{\T (\forall x)(\forall y)(B(x,y)\to B(f(x),f(y)))}
                                                      [\textcolor{red}{\T (\forall x)(\forall y)(B(x,y)\to B(f(x),f(y)))}
                                                        [{\T (\forall y)(B(c_0+c_1,y)\to B(f(c_0+c_1),f(y)))}
                                                          [\textcolor{red}{\T (\forall y)(B(c_0+c_1,y)\to B(f(c_0+c_1),f(y)))}
                                                            [{\T B(c_0+c_1,0)\to B(f(c_0+c_1),f(0))}
                                                              [{\F B(c_0+c_1,0)}, tikz={\node[fit to=tree,label=below:$\otimes$] {};}]
                                                              [{\T B(f(c_0+c_1),f(0))}, tikz={\node[fit to=tree,label=below:$\otimes$] {};}]
                                                            ]
                                                          ]
                                                        ]
                                                      ]
                                                    ]
                                                  ]
                                                ]
                                                [\F M(c_0+c_1)
                                                  [{\F B(c_0+c_1,0)}, tikz={\node[fit to=tree,label=below:$\otimes$] {};}]
                                                ]
                                              ]
                                            ]
                                          ]
                                        ]
                                      ]
                                    ]
                                  ]
                                ]
                              ]                              
                            ]
                          ]
                        ]
                      ]
                    ]
                  ]
                ]
            \end{forest}
            }
            \end{center} 
            \item Najdeme dva elementárně neekvivalentní modely $T$:
            \begin{itemize}
                \item $\mathcal A=\langle\{0\};M^\mathcal A,B^\mathcal A,f^\mathcal A,+^\mathcal A, 0^\mathcal A\rangle$ kde $M^\mathcal A=\{0\}$, $B^\mathcal A=\{(0,0)\}$, $f^\mathcal A=\{(0,0)\}$, $+^\mathcal A=\{((0,0),0)\}$, a $0^\mathcal A=0$
                \item $\mathcal B=\langle\{0,1\};M^\mathcal B,B^\mathcal B,f^\mathcal B,+^\mathcal B, 0^\mathcal B\rangle$ kde $M^\mathcal B=\{0\}$, $B^\mathcal B=\{(0,0),(1,1)\}$, $f^\mathcal B=\{(0,0),(1,1)\}$, $+^\mathcal B=\{((0,0),0),((0,1),1),((1,0),1),((1,1),0)\}$, a $0^\mathcal B=0$
            \end{itemize}
            Kompletní jednoduché extenze jsou potom $\mathrm{Th}(\mathcal A)$ a $\mathrm{Th}(\mathcal B)$ (tj. všechny $L$-sentence, které platí v $\mathcal A$ resp. $\mathcal B$). Teorie struktury je vždy kompletní teorie. Nejsou ekvivalentní například proto, že $(\forall x)M(x)$ platí v $\mathcal A$ ale ne v $\mathcal B$. (Uvědomte si, že jazyk je bez rovnosti, potřebujeme tedy sentenci bez rovnosti.)
        \end{enumerate}
                    
    \end{solution}

\end{problem}


\begin{problem}

    Uvažme jazyk $L=\langle c\rangle$ s rovností, kde $c$ je konstantní symbol. Tablo metodou dokažte, že v teorii $T=\{(\exists x)(\forall y)x=y\}$ platí formule $x=c$.

    \begin{solution}

        Sestrojíme dokončené tablo z teorie $T$ s položkou $\F(\forall x)x=c$ v kořeni (formule v položkách tabla musí být sentence). Protože je jazyk s rovností, můžeme v tablu používat i axiomy rovnosti pro jazyk $L$, resp. jejich generální uzávěry: $(\forall x)x=x$ a $(\forall x_1)(\forall x_2)(\forall y_1)(\forall y_2)(x_1=y_1\wedge x_2=y_2\to (x_1=x_2\to y_1=y_2))$.

        \begin{center}
            \scalebox{0.95}{
            \begin{forest}
                for tree={math content}
                [{\F(\forall x)x=c}
                  [{\F c_0=c}
                    [\textcolor{blue}{\T(\exists x)(\forall y)x=y}
                      [{\T(\forall y)c_1=y}
                        [\textcolor{red}{\T(\forall y)c_1=y}
                          [{\T c_1=c}
                            [\textcolor{red}{\T(\forall y)c_1=y}
                              [{\T c_1=c_0}
                                [\textcolor{blue}{\T(\forall x_1)(\forall x_2)(\forall y_1)(\forall y_2)(x_1=y_1\wedge x_2=y_2\to (x_1=x_2\to y_1=y_2))}
                                  [\textcolor{red}{\T(\forall x_1)(\forall x_2)(\forall y_1)(\forall y_2)(x_1=y_1\wedge x_2=y_2\to (x_1=x_2\to y_1=y_2))}
                                    [{\T(\forall x_2)(\forall y_1)(\forall y_2)(c_1=y_1\wedge x_2=y_2\to (c_1=x_2\to y_1=y_2))}
                                      [\textcolor{red}{\T(\forall x_2)(\forall y_1)(\forall y_2)(c_1=y_1\wedge x_2=y_2\to (c_1=x_2\to y_1=y_2))}
                                        [{\T(\forall y_1)(\forall y_2)(c_1=y_1\wedge c_1=y_2\to (c_1=c_1\to y_1=y_2))}
                                          [\textcolor{red}{\T(\forall y_1)(\forall y_2)(c_1=y_1\wedge c_1=y_2\to (c_1=c_1\to y_1=y_2))}
                                            [{\T(\forall y_2)(c_1=c_0\wedge c_1=y_2\to (c_1=c_1\to c_0=y_2))}
                                              [\textcolor{red}{\T(\forall y_2)(c_1=c_0\wedge c_1=y_2\to (c_1=c_1\to c_0=y_2))}
                                                [{\T(c_1=c_0\wedge c_1=c\to (c_1=c_1\to c_0=c))}
                                                  [{\F c_1=c_0\wedge c_1=c}
                                                    [{\F c_1=c_0}, tikz={\node[fit to=tree,label=below:$\otimes$] {};}]
                                                    [{\F c_1=c}, tikz={\node[fit to=tree,label=below:$\otimes$] {};}]
                                                  ]
                                                  [{\T c_1=c_1\to c_0=c}
                                                    [{\F c_1=c_1}
                                                      [\textcolor{blue}{\T(\forall x)x=x}
                                                        [\textcolor{red}{\T(\forall x)x=x}
                                                          [{\T c_1=c_1
                                                          }, tikz={\node[fit to=tree,label=below:$\otimes$] {};}]
                                                        ]    
                                                      ]
                                                    ]
                                                    [{\T c_0=c}, tikz={\node[fit to=tree,label=below:$\otimes$] {};}]
                                                  ]
                                                ]
                                              ]
                                            ]
                                          ]
                                        ]
                                      ]
                                    ]
                                  ]
                                ]
                              ]
                            ]
                          ]
                        ]
                      ]
                    ]
                  ]
                ]        
            \end{forest}
            }            
        \end{center}
                    
    \end{solution}

\end{problem}


\begin{problem} 
    
    Buď $L$ jazyk s rovností obsahující binární relační symbol $\le$ a $T$ teorie v tomto jazyce taková, že $T$ má nekončený model a platí v ní axiomy lineárního uspořádání. Pomocí věty o kompaktnosti ukažte, že $T$ má model $\mathcal{A}$ s \emph{nekonečným klesajícím řetězcem}; tj. že v $A$ existují prvky $c_i$ pro každé $i\in \mathbb{N}$ takové, že: $\dots < c_{n+1} < c_n< \dots <c_0$.
    (Z toho plyne, že pojem \emph{dobrého uspořádání} není definovatelný v logice prvního řádu.)

    \begin{solution}

        Z předpokladu víme, že $T$ má nekonečný model $\mathcal B$, tj. nekonečné lineární uspořádání. To by ale mohlo být např. $\langle \mathbb N;\leq^\mathbb N\rangle$, které žádný nekonečný řetězec nemá. Potřebujeme model s nekonečným klesajícím řetězcem, ten získáme z Věty o kompaktnosti (verze pro predikátovou logiku):
                
        Jazyk $L$ rozšíříme přidáním spočetně mnoha nových konstantních symbolů $c_i$ ($i\in\mathbb{N}$). Označme rozšířený jazyk $L'$. Uvažme následující $L'$teorii $T'$:
        $$
        T' = T \cup \{c_{i+1}\leq c_i\land\neg c_{i+1}=c_i\mid i\in\mathbb{N}\}
        $$
        Stačí ukázat, že $T'$ má model. Ten zřejmě musí být nekonečný a jeho redukt na jazyk $L$ je hledaný model $\mathcal A$ teorie $T$, který má nekonečný klesající řetězec $\dots < c_{n+1}^\mathcal A < c_n^\mathcal A < \dots < c_0^\mathcal A$. 
        
        Z věty o kompaktnosti víme, že $T'$ má model, právě když každá konečná podmnožina $T'$ má model. Máme-li konečnou podteorii $S\subseteq T'$, ta obsahuje jen konečně mnoho formulí $c_{i+1}\leq c_i\land\neg c_{i+1}=c_i$, pro nějakou konečnou množinu indexů $I\subseteq\mathbb N$. Označme jako $\mathcal B$  nekonečný model $T$, který máme z přepokladu. (Tento model nemusí mít nekonečný klesající řetězec! Mohl by to být např. $\langle \mathbb N;\leq^\mathbb N\rangle$) V něm stačí vybrat libovolný konečný klesající řetězec délky $|I|$ jako interpretace konstantních symbolů $c_i$ pro $i\in I$ (symboly $c_j\notin I$ interpretujeme libovolně), a dostáváme model $S$.
                    
    \end{solution}

\end{problem}




\section*{Další příklady k procvičení}


\begin{problem}
    
    Uvažte následující tvrzení:
    \begin{enumerate}[(i)]\it
        \item Každý docent napsal alespoň jednu učebnici.
        \item Každou učebnici napsal nějaký docent.
        \item U každého docenta někdo studuje.
        \item Každý, kdo studuje u nějakého docenta, přečetl všechny učebnice od tohoto docenta.
        \item Každou učebnici někdo přečetl.
    \end{enumerate}    
    \begin{enumerate}[(a)]
        \item Formalizujte {\it(i)--(v)} jako sentence $\varphi_1,\varphi_2,\varphi_3,\varphi_4,\varphi_5$ v $L=\langle N, S, P, D, U\rangle$ bez rovnosti.
        %, kde $N,S,P$ jsou binární relační symboly ($N(x,y)$ znamená ``$x$ napsal $y$'', $S(x,y)$ znamená ``$x$ studuje u $y$'', $P(x,y)$ znamená ``$x$ přečetl $y$'') a $D,U$ jsou unární relační symboly (``být docentem'', ``být učebnicí'').
        \item Sestrojte dokončené tablo z teorie $T=\{\varphi_1,\varphi_2,\varphi_3,\varphi_4\}$ s položkou $F\varphi_5$ v kořeni.
        \item Je sentence $\varphi_5$ pravdivá v teorii $T$? Je lživá v $T$? Je nezávislá v $T$? Zdůvodněte.
        \item Má teorie $T$ kompletní konzervativní extenzi? Zdůvodněte.
        %\item Kolik má $T'=T\cup \{D(x),S(x,y),P(x,y)\}$ 2-prvkových modelů (až na izomorfismus)?
    \end{enumerate}

\end{problem}


\begin{problem}
    
    Tablo metodou dokažte následující pravidla `vytýkání' kvantifikátorů, kde $\varphi(x)$ je formule s jedinou volnou proměnnou $x$, a $\psi$ je sentence.

    \vspace{-6pt}
    \begin{multicols}{2}
        \begin{enumerate}[(a)]        
            \item $\neg(\exists x)\varphi(x)\to (\forall x)\neg \varphi(x)$
            \item $(\forall x)\neg \varphi(x)\to \neg(\exists x)\varphi(x)$
            \item $((\exists x)\varphi(x)\to\psi)\to(\forall x)(\varphi(x)\to \psi)$       
            \item $(\forall x)(\varphi(x)\to\psi)\to((\exists x)\varphi(x)\to \psi)$            
        \end{enumerate}
    \end{multicols}
    \vspace{-6pt}
    
\end{problem}


\begin{problem} 
    
    Nechť $L(x,y)$ reprezentuje \emph{``existuje let z $x$ do $y$''} a $S(x,y)$ reprezentuje \emph{``existuje spojení z $x$ do $y$''}. Předpokládejme, že z Prahy lze letět do Bratislavy, Londýna a New Yorku, a z New Yorku do Paříže, a platí
    \begin{itemize}  
        \item $(\forall x)(\forall y)(L(x,y) \to L(y,x))$,
        \item $(\forall x)(\forall y)(L(x,y)\to S(x,y))$,
        \item $(\forall x)(\forall y)(\forall z)(S(x,y)\wedge L(y,z)\to S(x,z))$.
    \end{itemize}
    Dokažte tablo metodou, že existuje spojení z Bratislavy do Paříže.

\end{problem}



\begin{problem} 

    Buď $T$ následující teorie v jazyce $L=\langle R,f,c,d\rangle$ s rovností, kde $R$ je binární relační symbol,  $f$ unární funkční symbol, a $c,d$ konstantní symboly:
    $$
    T=\{R(x,x),R(x,y)\wedge R(y,z)\to R(x,z),R(x,y)\wedge R(y,x)\to x=y,R(f(x),x)\}
    $$
    Označme jako $T'$ generální uzávěr $T$. Nechť $\varphi$ a $\psi$ jsou následující formule:
    $$
    \varphi = R(c,d) \wedge (\forall x)(x=c\vee x=d)\qquad\qquad
    \psi = (\exists x)R(x,f(x))
    $$
    \begin{enumerate}[(a)]
        \item Sestrojte tablo důkaz formule $\psi$ z teorie $T'\cup\{\varphi\}$. (Pro zjednodušení můžete kromě axiomů rovnosti v tablu přímo používat axiom $(\forall x)(\forall y)(x=y\to y=x)$, což je jejich důsledek.)
        \item Ukažte, že $\psi$ není důsledek teorie $T$, tím že najdete model $T$, ve kterém $\psi$ neplatí.
        \item Kolik kompletních jednoduchých extenzí (až na $\sim$) má teorie $T\cup \{\varphi\}$? Uveďte dvě.
        \item Nechť $S$ je následující teorie v jazyce $L'=\langle R\rangle$ s rovností. Je $T$ konzervativní extenzí $S$?
        $$S=\{R(x,x),R(x,y)\wedge R(y,z)\to R(x,z),R(x,y)\wedge R(y,x)\to x=y\}$$     
    \end{enumerate}

\end{problem}


\section*{K zamyšlení}


\begin{problem} 
    
    Dokažte syntakticky, pomocí transformací tabel:
    \begin{enumerate}[(a)]
        \item \emph{Větu o konstantách}: Buď $\varphi$ formule v jazyce $L$ s volnými proměnnými $x_1,\dots,x_n$ a $T$ teorie v $L$. Označme $L'$ extenzi $L$ o nové konstantní symboly $c_1,\dots,c_n$ a $T'$ teorii $T$ v $L'$. Potom platí:
        $T \vdash (\forall x_1)\dots(\forall x_n)\varphi$ právě když $T'\vdash\varphi(x_1/c_1,\dots,x_n/c_n)$
        \item \emph{Větu o dedukci}: Pro každou teorii $T$ (v uzavřené formě) a sentence $\varphi$, $\psi$ platí: $T\vdash \varphi\to\psi$ právě když $T,\varphi\vdash\psi$
    \end{enumerate}

\end{problem}


\begin{problem} 
    
    Mějme teorii $T^*$ s axiomy rovnosti. Pomocí tablo metody ukažte, že:
    \begin{enumerate}[(a)]
        \item $T^*\models x=y\ \to\ y=x$\hfill(symetrie)
        \item $T^*\models (x=y\ \wedge\ y=z)\ \to\ x=z$\hfill(tranzitivita)
    \end{enumerate}
    {\it Hint:} Pro (a) použijte axiom rovnosti $(iii)$ pro $x_1=x$, $x_2=x$, $y_1=y$ a $y_2=x$, \newline
        na (b) použijte $(iii)$ pro $x_1=x$, $x_2=y$, $y_1=x$ a $y_2=z$.
        
\end{problem}
  



