\documentclass[a4paper]{article}
\usepackage{a4wide}
\usepackage{amssymb}
\usepackage[utf8]{inputenc}
\usepackage[czech]{babel}

\begin{document}

\begin{center}
    \large{Vzorový zápočtový test: výroková logika}    
\end{center}

Časový limit: 45 minut. Celkem bodů: 100.

\bigskip

\begin{enumerate}
\item Mějme tři bratry, přičemž o každém víme, že buď vždy říká pravdu anebo vždy lže.
\begin{enumerate}
\item[$(i)$] Nejstarší říká: \emph{``Oba mí bratři jsou lháři.''}
\item[$(ii)$] Prostřední říká: \emph{``Nejmladší je lhář.''}
\item[$(iii)$] Nejmladší říká: \emph{``Nejstarší je lhář.''}
\end{enumerate}
Nechť prvovýroky $p_1$, $p_2$, $p_3$ reprezentují (po řadě), že \emph{``nejstarší / prostřední / nejmladší bratr je pravdomluvný''} a označme $\mathbb{P}=\{p_1,p_2,p_3\}$.
\begin{enumerate}
\item Napište výroky (ve tvaru ekvivalence) $\varphi_1$, $\varphi_2$, $\varphi_3$ nad $\mathbb{P}$ reprezentující znalosti vyplývající (po řadě) z $(i)$, $(ii)$, $(iii)$. {\it (15b)}
\item Napište teorii $S$ v množinové reprezentaci vzniklou převodem $\varphi_1$, $\varphi_2$, $\varphi_3$, $p_3$ či jejich negací na CNF, která je nesplnitelná, právě když z tvrzení $(i)$, $(ii)$, $(iii)$ plyne, že \emph{``nejmladší je pravdomluvný''}. {\it (15b)}
\item Rezolucí ukažte, že $S$ je nesplnitelná. Rezoluční zamítnutí znázorněte rezolučním stromem.    {\it (20b)}
\end{enumerate}

\item Nechť $T=\{(\neg p \wedge q) \to r,\ (q \to r) \leftrightarrow p\}$ je teorie nad $\mathbb{P}=\{p,q,r\}$.
\begin{enumerate}
    \item Tablo metodou určete všechny modely teorie $T$. {\it (20b)}
    \item Je $T$ extenzí teorie $S=\{q \to p\}$ nad $\{p,q\}$? Je $T$ konzervativní extenzí $S$? Uveďte zdůvodnění. {\it (15b)}
    \item Určete, kolik je navzájem neekvivalentních výroků nad $\mathbb{P}$, které jsou nezávislé v $S$ i v $T$. Uveďte zdůvodnění. {\it (15b)}
\end{enumerate}

\end{enumerate}

\end{document} 