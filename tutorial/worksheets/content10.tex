\section*{NAIL062 V\&P Logika: 10. sada příkladů -- Rezoluční metoda v PL}
% po 5. přednášce


\subsection*{Výukové cíle:} Po absolvování cvičení student

    \begin{itemize}\setlength{\itemsep}{0pt}
        \item zná potřebné pojmy z rezoluční metody v predikátové logice (rezoluční pravidlo, rezolventa, rezoluční důkaz/zamítnutí, rezoluční strom), umí je formálně definovat, uvést příklady, vysvětlit rozdíly oproti výrokové logice, 
        \item umí aplikovat rezoluční metodu k řešení daného problému (slovní úlohy, aj.), provést všechny potřebné kroky (převod do PNF, skolemizace, převod do CNF)
        \item umí sestrojit rezoluční zamítnutí dané (i nekonečné) CNF formule (existuje-li), a také nakreslit příslušný rezoluční strom, včetně uvedení použitých unifikací
        \item z rezolučního stromu umí sestrojit nesplnitelnou konjunkci základních instancí axiomů
        \item zná pojem LI-rezoluce, umí najít LI-zamítnutí dané teorie (existuje-li)
    \end{itemize}
    

\section*{Příklady na cvičení}


\begin{problem}


    \begin{solution}
                    
    \end{solution}

\end{problem}

        
        
\section*{Další příklady k procvičení}

        
\section*{K zamyšlení}


