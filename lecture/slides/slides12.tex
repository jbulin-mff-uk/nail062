\documentclass{beamer}

\input{slides-header.tex}

\title{Dvanáctá přednáška}
\subtitle{NAIL062 Výroková a predikátová logika}
\author{Jakub Bulín (KTIML MFF UK)}
% \institute{KTIML MFF UK}
\date{Zimní semestr 2024}


\begin{document}


\maketitle


\begin{frame}{Dvanáctá přednáška}

    \textbf{Program}
        \begin{itemize}            
            \item axiomatizovatelnost
            \item rekurzivní axiomatizace a rozhodnutelnost
            \item aritmetické teorie
            \item nerozhodnutelnost predikátové logiky
            \item Gödelovy věty o neúplnosti
        \end{itemize}

    \textbf{Materiály}

        \href{https://github.com/jbulin-mff-uk/nail062/raw/main/lecture/lecture-notes/lecture-notes.pdf}{\alert{\textbf{Zápisky z přednášky}}}, Sekce 9.4 z Kapitoly 9, Kapitola 10

\end{frame}


\section{9.4 Axiomatizovatelnost}


\begin{frame}{Axiomatizovatelnost}
    
    \pause
    Třída struktur $K\subseteq\M_L$ je:
    
    \pause
    \begin{itemize}
        \item \alert{axiomatizovatelná}, existuje-li teorie $T$ taková, že $\M_L(T)=K$\pause
        \item \alert{konečně}/\alert{otevřeně} axiomatiz., je-li ax. konečnou/otevřenou $T$\pause
        \item teorie $T'$ je \alert{konečně}/\alert{otevřeně} axiomatizovatelná, platí-li to o třídě jejích modelů $K=\M_L(T')$
    \end{itemize}

    \pause
    \textbf{Pozorování:} Je-li $K$ axiomatizovatelná, musí být uzavřená na $\equiv$.

    \medskip
    
    \pause
    \myexample{
    Například, jak ukážeme:\pause
    \begin{itemize}
        \item grafy a částečná uspořádání jsou konečně i otevřeně ax.\pause
        \item tělesa jsou konečně, ale ne otevřeně axiomatizovatelná\pause
        \item nekonečné grupy jsou axiomatizovatelné, ale ne konečně\pause
        \item konečné grafy nejsou axiomatizovatelné
    \end{itemize}
    }

\end{frame}


\begin{frame}{Neaxiomatizovatelnost konečných modelů}

    \smallskip

    \pause
    \myblock{
        \textbf{Věta:}
        Má-li $T$ libovolně velké konečné modely, má i nekonečný model. Potom není třída jejích konečných modelů axiomatizovatelná.
    }

    \pause
    \textbf{Důkaz:} \pause
    \alert{Je-li jazyk bez rovnosti,} vezmeme  kanonický model pro bezespornou větev v tablu z $T$ pro $\F\bot$ ($T$ je bezesporná).
    
    \pause
    \alert{Je-li jazyk s rovností,} přidáme spočetně mnoho nových konst. symbolů $c_i$ a vezmeme extenzi: \myalertinline{
        $T'=T \cup \{\neg c_i = c_j \mid i\neq j\in\mathbb N\}$
    }

    \vspace{-2pt}

    \pause
    Každá \alert{konečná část} $T'$ má model: buď $k$ největší, že $c_k$ je v této konečné části: lib. $\geq(k+1)$-prvkový model,21 interpretuj $c_0,\dots,c_k$ jako různé prvky.

    \vspace{-2pt}

    \pause
    \alert{Věta o kompaktnosti} dává model $T'$, ten je nekonečný, redukt na původní jazyk (zapomenutí $c_i^\A$) je nekonečný model $T$.\hfill\qedsymbol

    \pause
    \begin{itemize}
        \item např. \myexampleinline{konečné grafy} nejsou axiomatizovatelné\pause
        \item \alert{nekonečné modely} teorie jsou vždy axiomatizovatelné, máme-li rovnost: stačí přidat \myalertinline{`existuje alespoň $n$ prvků'} pro vš. $n\in\mathbb N$
    \end{itemize}

\end{frame}


\begin{frame}{Konečná axiomatizovatelnost}
    
    \smallskip

    \pause
    \myblock{
        \textbf{Věta (O konečné axiomatizovatelnosti):}
        $K\subseteq \M_L$ je konečně axiomatizovatelná, právě když $K$ i $\overline{K}=\M_L\setminus K$ jsou axiomatizovatelné.
    }

    \pause
    \textbf{Důkaz:} \alert{\Large$\Rightarrow$}
    Je-li $K$ axiomatizovatelná \alert{sentencemi} $\varphi_1,\dots,\varphi_n$ (vezmi gen. uzávěry), potom $\neg(\varphi_1\land\varphi_2\land\dots\land\varphi_n)$ axiomatizuje $\overline{K}$.\pause

    \alert{\Large$\Leftarrow$} Buď $K=\M(T)$ a $\overline{K}=\M(S)$. Potom  \alert{$T\cup S$ je sporná}, neboť:
    $$
    \M(T\cup S)=\M(T)\cap \M(S)=K\cap\overline{K}=\emptyset
    $$
    \pause \alert{Věta o kompaktnosti} dává konečné $T'\subseteq T$ a $S'\subseteq S$ takové, že:
    $$
    \emptyset = \M(T'\cup S')=\M(T')\cap \M(S')
    $$
    \pause Nyní si všimněme, že platí:
    $$
    \M(T)\subseteq \M(T')\subseteq \overline{\M(S')}\subseteq \overline{\M(S)}=\M(T)
    $$
    \pause Tím jsme dokázali, že \alert{$M(T)=M(T')$}, neboli $T'$ je konečná axiomatizace $K$. \hfill\qedsymbol

\end{frame}


\begin{frame}{Tělesa charakteristiky 0 nejsou konečně axiomatizovatelná}

    \pause
    Buď $T$ teorie těles. Těleso $\A=\langle A,+,-,0,\cdot,1 \rangle$ je\pause
    \vspace{-6pt}
    \begin{itemize}
        \item \alert{charakteristiky $p$}, je-li $p$ nejmenší prvočíslo takové, že $\A\models p1=0$, kde $p1$ je term $1+1+\dots+1$ (s $p$ jedničkami),\pause
        \item \alert{charakteristiky 0}, pokud není charakteristiky $p$ pro žádné $p$.\pause
        \item Tělesa charakteristiky $p$ jsou konečně axiomatizovatelná: \myalertmath{\small
            $$
            T_p=T\cup \{p1=0\}
            $$
        }
        \pause
        \item Tělesa char. 0 jsou axiomatizovatelná, ale ne konečně: \myalertmath{\small
            $$
            T_0=T\cup \{\neg\, p1=0\mid p\text{ prvočíslo}\}
            $$
        }
        \pause
    \end{itemize}
    \myblock{\vspace{2pt}
        \textbf{Tvrzení:}
        Třída $K$ těles char. $0$ není konečně axiomatizovatelná.
        \vspace{2pt}
    }
    
    \pause
    \textbf{Důkaz:} \pause
    Stačí ukázat, že $\overline{K}$ (tělesa nenulové char. a netělesa) není axiomatizovatelná. 
    \pause \alert{Sporem: $\overline{K}=\M(S)$.} 
    \pause Potom $S'=S\cup T_0$ má model, neboť každá konečná část má model: těleso charakteristiky větší než jakékoliv $p$ z axiomu $T_0$ tvaru $\neg\, p1=0$. 
    \pause Je-li $\A$ je model $S'$, potom $\A\in\M(S)=\overline{K}$. Zároveň ale $\A\in\M(T_0)=K$, spor.\hfill\qedsymbol

\end{frame}


\begin{frame}{Otevřená axiomatizovatelnost}

    \smallskip

    \pause
    \myblock{\vspace{2pt}
        \textbf{Tvrzení:}
        Je-li $T$ otevřeně axiomatizovatelná, potom je každá podstruktura modelu $T$ také modelem $T$.
        \vspace{2pt}  
    }
    
    \pause
    \textbf{Důkaz:}
    Buď $T'$ otevřená axiomatizace $T$, $\A$ model $T'$, $\B\subseteq\A$. Pro každou $\varphi\in T'$ platí $\B\models\varphi$ ($\varphi$ je otevřená), tedy i $\B\models T'$.  
    \hfill\qedsymbol

    \pause
    \textbf{Poznámka:} Platí i obráceně, je-li každá podstruktura modelu také model, potom je otevřeně axiomatizovatelná. (Důkaz neuvedeme.)

    \pause
    \begin{itemize}
        \item \myexampleinline{DeLO} není otevřeně axiomatizovatelná, např. žádná konečná podstruktura modelu DeLO není hustá\pause
        \item \myexampleinline{teorie těles} není otevřeně axiomatizovatelná, podstruktura $\mathbb Z\subseteq\mathbb Q$ není těleso, nemá inverzní prvek k $2$ vůči násobení\pause
        \item pro dané $n\in\mathbb N$ jsou \myexampleinline{nejvýše $n$-prvkové grupy} otevřeně axiomatizovatelné (i jejich podgrupy jsou nejvýše $n$-prvkové); k (otevřené) teorii grup stačí přidat: \myalertinline{
            $\bigvee_{1\leq i<j\leq n+1}x_i=x_j$
        }
    \end{itemize}

\end{frame}


\section{\sc Kapitola 10: Nerozhodnutelnost a neúplnost}


\begin{frame}{Nerozhodnutelnost a neúplnost}

    \pause
    Jak lze s teoriemi pracovat algoritmicky?

    \medskip

    \pause
    + zlatý hřeb přednášky: \alert{Gödelovy věty o neúplnosti} (1931)\pause
    \begin{itemize}
        \item ukazují limity formálního přístupu
        \item zastavily program formalizace matematiky
        \item pojem \alert{algoritmu} budeme chápat jen intuitivně
        \item technické podrobnosti důkazů vynecháme
        
    \end{itemize}

    \pause
    Typicky potřebujeme spočetný jazyk.
     
\end{frame}


\section{10.1 Rekurzivní axiomatizace a rozhodnutelnost}


\begin{frame}{Rekurzivní axiomatizace}

    \begin{itemize}
        \item v dokazování povolujeme nekonečné teorie, jak jsou zadané?\pause
        \item pro ověření že daný důkaz (např. tablo, rezoluční zamítnutí) je korektní potřebujeme algoritmický přístup ke všem axiomům\pause
        \item mohli bychom požadovat \alert{enumerátor} pro $T$, tj. algoritmus, který vypisuje axiomy z $T$, a každý axiom někdy vypíše\pause
        \item ale kdyby byl v důkazu chybný axiom, nikdy bychom se to nedozvěděli: stále bychom čekali, zda ho enumerátor vypíše\pause
        \item proto požadujeme silnější vlastnost:
    \end{itemize}

    \pause
    \myblock{
        $T$ je \alert{rekurzivně axiomatizovaná}, pokud existuje algoritmus, který pro každou vstupní formuli $\varphi$ doběhne a odpoví, zda $\varphi\in T$.
    }

    \pause
    (ekvivalentní enumerátoru vypisujícímu axiomy v lexikograf. pořadí)  

\end{frame}


\begin{frame}{Rozhodnutelnost}

    \pause
    \vspace{-3pt}
    Můžeme v dané teorii \alert{`algoritmicky rozhodovat pravdu'}?

    \pause
    \vspace{-3pt}
    \myblock{
        \begin{itemize}
            \item $T$ je \alert{rozhodnutelná}, pokud existuje algoritmus, který pro každou vstupní formuli $\varphi$ doběhne a odpoví, zda $T\models\varphi$,\pause
            \item $T$ je \alert{částečně rozhodnutelná}, existuje-li algoritmus, který: \pause%pro každou vstupní formuli:
            \begin{itemize}
                \item pokud $T\models\varphi$, doběhne a odpoví ``ano''\pause
                \item pokud $T\not\models\varphi$, buď nedoběhne, nebo doběhne a odpoví ``ne''
            \end{itemize}
        \end{itemize}
    }

    
    \pause
    \myblock{
        \textbf{Tvrzení:}
        Je-li $T$ je rekurzivně axiomatizovaná, potom:
                
        \pause (i) $T$ je část. rozhod. \hfill \pause (ii) je-li navíc kompletní, je rozhodnutelná
    }
    
    \pause
    \textbf{Důkaz:} \pause \alert{(i)} \pause Algoritmus konstruuje systematické tablo z $T$ pro $\F\varphi$; stačí enumerátor pro $T$, nebo postupně generovat vš. sentence a testovat, jsou-li v $T$. Je-li $T\models\varphi$, konstrukce skončí, ověříme, že je tablo sporné. (Jinak skončit nemusí.)
    
    \vspace{-3pt}  
    \pause \alert{(ii)} \pause
    Víme, že buď $T\proves\varphi$ nebo $T\proves\neg\varphi$. Paralelně konstruujeme tablo pro $\F\varphi$ a pro $\T\varphi$ (důkaz a zamítnutí $\varphi$ z $T$). Jedna z konstrukcí po konečně mnoha krocích skončí.
    \hfill\qedsymbol

\end{frame}


\begin{frame}{Rekurzivně spočetná kompletace}

    \pause
    \myblock{
        $T$ má \alert{rekurzivně spočetnou kompletaci}, je-li (nějaká) množina až na $\sim$ všech jednoduchých kompletních extenzí $T$ \alert{rekurzivně spočetná}, tj. existuje algoritmus, který pro vstup $(i,j)$ vypíše $i$-tý axiom $j$-té extenze (v nějakém uspořádání), nebo odpoví, že už neexistuje.
    }
     
    \pause
    \myblock{
        \textbf{Tvrzení:}
        Je-li $T$ rekurzivně axiomatizovaná a má rekurzivně spočetnou kompletaci, potom je rozhodnutelná.
    }
    
    \pause
    \textbf{Důkaz:} \pause
    Buď $T\proves\varphi$, nebo existuje protipříklad $\A\not\models\varphi$, tj. kompl. jedn. extenze $T_i$, že $T_i\not\proves\varphi$. Kompletnost $T_i$ dává $T_i\proves\neg\varphi$. 
    
    \pause
    Algoritmus paralelně konstruuje tablo důkaz $\varphi$ z $T$ a (postupně) tablo důkazy $\neg\varphi$ ze všech kompletních jedn. extenzí $T_1,T_2,\dots$. (Je-li jich nekonečně mnoho, uděláme \alert{dovetailing}: 1. krok 1. tabla, potom 2. krok 1., 1. krok 2., 3. krok 1., 2. krok 2., 1. krok 3., atd.)
    
    \pause
    Alespoň jedno z tabel je sporné, můžeme předpokládat konečné, algoritmus ho po konečně mnoha krocích zkonstruuje.
    \hfill\qedsymbol

\end{frame}


\begin{frame}{Příklady}
    
    Následující teorie jsou rekurzivně axiomatizované a mají rekurzivně spočetnou kompletaci, tedy jsou rozhodnutelné:

    \myexample{
    \begin{enumerate}[(a)]
    \item Teorie čisté rovnosti
    \item Teorie unárního predikátu ($T=\emptyset$, $L=\langle U \rangle$ s rovností)
    \item Teorie hustých lineárních uspořádání DeLO*
    \item Teorie Booleových algeber (Alfred Tarski 1940),
    \item Teorie algebraicky uzavřených těles (Tarski 1949),
    \item Teorie komutativních grup (Wanda Szmielew 1955).
    \end{enumerate}
    }

\end{frame}


\begin{frame}{Rekurzivní axiomatizovatelnost}

    \pause
    Kdy lze třídu struktur `efektivně (algoritmicky) popsat'?

    \pause
    \myblock{
        $K\subseteq\M_L$ je \alert{rek. axiomatizovatelná}, pokud existuje rek. axiomatizovaná $T$, že $K=M_L(T)$. $T'$ je \alert{rek. axiomatizovatelná}, platí-li to pro třídu jejích modelů (tj. je-li ekvivalentní rek. axiomatizované teorii).
    }

    \pause
    (podobně lze definovat \alert{rek. spočetnou axiomatizovatelnost})

    \pause
    \myblock{
        \textbf{Tvrzení:}
        Je-li $\A$ konečná struktura v konečném jazyce s rovností, potom je teorie $\Th(\A)$ rekurzivně axiomatizovatelná.
    }
        
    \pause
    (z toho plyne i rozhodnutelnost $\Th(\A)$, ale $\A\models\varphi$ lze ověřit přímo)

    \pause
    \textbf{Důkaz:} \pause
    Buď $A=\{a_1,\dots,a_n\}$. $\Th(\A)$ axiomatizujeme sentencí ``existuje právě $n$ prvků $a_1,\dots,a_n$ splňujících právě ty \alert{základní vztahy} o funkčních hodnotách a relacích, které platí v $\A$''.
    
    \pause
    Např. je-li $f^\A(a_4, a_2)=a_{17}$, přidej atom. formuli $f(x_{a_4},x_{a_2})=x_{a_{17}}$, je-li $(a_3,a_3,a_1)\notin R^\A$ přidej $\neg R(x_{a_3},x_{a_3},x_{a_1})$.\hfill\qedsymbol

\end{frame}


\begin{frame}{Příklady}

    Pro následující struktury je $\Th(\A)$ rekurzivně axiomatizovatelná:

    \myexample{
        \begin{itemize}
            \item $\langle\mathbb Z,\leq\rangle$, jde o tzv. teorii \alert{diskrétních lineárních uspořádání}        
            \item $\langle\mathbb Q,\leq\rangle$, jde o teorii DeLO
            \item $\langle\mathbb N,S,0\rangle$, teorie \alert{následníka s nulou}
            \item $\langle\mathbb N,S,+,0\rangle$, \alert{Presburgerova aritmetika}
            \item $\langle\mathbb R,+,-,\cdot,0,1\rangle$, teorie \alert{reálně uzavřených těles}, znamená že lze algoritmicky rozhodovat Euklid. geometrii (Tarski, 1949)
            \item $\langle \mathbb C,+,-,\cdot,0,1 \rangle$, teorie \alert{algebraicky uzavřených těles char. 0}
        \end{itemize}
    }

    \medskip

    \pause
    \myblock{
        \textbf{Důsledek:}
        Pro struktury výše platí, že $\Th(\A)$ je rozhodnutelná.
    }
    \pause
    \textbf{Důkaz:} $\Th(\A)$ je vždy kompletní.

    \pause
    \myexample{
        Teorie \alert{standardního modelu aritmetiky} $\underline{\mathbb N}=\langle\mathbb N,S,+,\cdot,0,\leq\rangle$ ale \alert{není} rekurzivně axiomatizovatelná (viz První Gödelova věta o neúplnosti).
    }

\end{frame}


\section{10.2 Aritmetika}


\begin{frame}{Aritmetika}

    \begin{itemize}[<+->]
        \item přirozená čísla hrají důležitou roli v matematice i v aplikacích
        \item \alert{jazyk aritmetiky} je $L=\langle S,+,\cdot,0,\leq\rangle$ s rovností
        \item \alert{standardní model aritmetiky}  $\underline{\mathbb N}=\langle\mathbb N,S,+,\cdot,0,\leq\rangle$ nemá rekurzivně axiomatizovatelnou teorii (První věta o neúplnosti)
        \item proto používáme rekurzivně axiomatizované teorie, které vlastnosti $\underline{\mathbb N}$ popisují částečně; říkáme jim \alert{aritmetiky}       
        \item představíme dvě: \alert{Robinsonovu} $Q$ a \alert{Peanovu} $PA$
    \end{itemize}

\end{frame}


\begin{frame}{Robinsonova aritmetika $Q$}    

    \pause
    \myblockamsmath{ 
        \vspace{-12pt}       
        \begin{align*}
            &\neg S(x) = 0& &x\cdot 0=0\\
            &S(x)=S(y)\rightarrow x=y& &x\cdot S(y)=x\cdot y+x\\
            &x+0=x& &\neg x=0 \rightarrow (\exists y)(x=S(y))\\
            &x+S(y)=S(x+y)& &x\le y \leftrightarrow (\exists z)(z+x=y)\qquad
        \end{align*}
    }
    
    \pause
    \begin{itemize}
        \item velmi slabá, nelze v ní dokázat např. komutativitu ani asociativitu $+$ či $\cdot$, nebo tranzitivitu $\leq$\pause
        \item ale lze dokázat všechna \alert{existenční tvrzení o numerálech} pravdivá v $\underline{\mathbb N}$, tj. formule v PNF, jen $\exists$, za volné proměnné substituujeme \alert{numerály} $\underline{n}=S(\dots S(0)\dots)$\pause
        \item např. pro \myexampleinline{
            $\varphi(x,y)=(\exists z)(x+z=y)$
         } je $Q\proves\varphi(\underline{1},\underline{2})$
    \end{itemize}

    \medskip
    
    \pause
    \myblock{
        \textbf{Tvrzení:}
        Je-li $\varphi(x_1,\dots,x_n)$ existenční formule, $a_1,\dots,a_n\in\mathbb N$, pak
        $Q\proves\varphi(x_1/\underline{a_1},\dots,x_n/\underline{a_n})$ právě když $\underline{\mathbb{N}}\models \varphi[e(x_1/a_1,\dots,x_n/a_n)]
        $
    }

    (Důkaz vynecháme.)

\end{frame}


\begin{frame}{Peanova aritmetika $PA$}    

    \pause
    \myblock{
        Extenze $Q$ o \alert{schéma indukce}, tj. pro každou $L$-formuli $\varphi(x,\overline{y})$:
        \vspace{-6pt}
        $$
        (\varphi(0,\overline{y}) \land (\forall x)(\varphi(x,\overline{y})\limplies \varphi(S(x),\overline{y}))) \limplies (\forall x)\varphi(x,\overline{y})
        $$
        \vspace{-21pt}
    }

    \pause
    \begin{itemize}
        \item mnohem lepší aproximace $\Th(\underline{\mathbb N})$\pause
        \item dokáže `základní' vlastnosti (např. komut. a asociativitu $+$) \item stále ale existují sentence platné v $\underline{\mathbb N}$ ale nezávislé v $PA$\\(opět dokážeme v První větě o neúplnosti)
    \end{itemize}

    \bigskip

    \pause
    \textbf{Poznámka:} strukturu $\underline{\mathbb N}$ lze axiomatizovat (až na $\simeq$) v predikátové logice \alert{2. řádu}, extenzí $PA$ o tzv. \alert{axiom indukce}:
    $$
    (\forall X)((X(0) \land (\forall x)(X(x) \limplies X(S(x)))) \limplies (\forall x)X(x))
    $$

    \pause
    \begin{itemize}
        \item $X$ reprezentuje (libovolnou) podmnožinu modelu\pause
        \item použijeme na množinu všech následníků 0\pause
        \item každý prvek je následník 0 $\Rightarrow$ izomorfismus s $\underline{\mathbb N}$
    \end{itemize}

\end{frame}


\section{10.3 Nerozhodnutelnost predikátové logiky}


\begin{frame}{Nerozhodnutelnost predikátové logiky}
    
    \pause
    \myblock{
        \textbf{Věta (O nerozhodnutelnosti predikátové logiky):}
        Neexistuje algoritmus, který pro vstupní formuli $\varphi$ rozhodne, zda je logicky platná.
    }

    \pause
    \begin{itemize}
        \item tj. zda je formule $\varphi$ [v lib. jazyce 1. řádu] tautologie ($\models\varphi$)\pause
        \item neboli $T=\emptyset$ není rozhodnutelná 
    \end{itemize}

    \pause
    Nemáme formalismus pro algoritmy (Turingovy stroje), dokážeme redukcí na jiný nerozhodnutelný problém: \alert{\href{https://en.wikipedia.org/wiki/Hilbert\%27s_problems}{Hilbertův 10. problém}}

    \bigskip

    \pause
    \myalert{
    \begin{quote}
        ``Najděte algoritmus, který po konečně mnoha krocích určí, zda daná diofantická rovnice s libovolným počtem proměnných a
        celočíselnými koeficienty má celočíselné řešení.''
    \end{quote}
    }

    \medskip

    \pause
    \alert{diofantická rovnice}: $p(x_1,\dots,x_n)=0$, kde $p$ je celočíselný polynom

    \pause
    ukážeme, že existuje \alert{redukce} `těžkého' Hilbertova 10. problému na náš problém, tedy i náš problém je `těžký'
    
\end{frame}


\begin{frame}{Nerozhodnutelnost Hilbertova desátého problému}

    \pause
    \myblock{
        \textbf{Věta (Matiyasevich 1970):}    
        Problém existence celočíselného řešení dané diofantické rovnice s celočís. koeficienty je nerozhodnutelný.
    }

    (Důkaz neuvedeme.)

    \medskip

    \pause
    \myblock{
        \textbf{Důsledek:}
        Neexistuje algoritmus rozhodující, mají-li dané polynomy $p(x_1,\dots,x_n),q(x_1,\dots,x_n)$ s \alert{přiroz.} koeficienty \alert{přirozené řešení}, tj.
        \vspace{-6pt}
        $$
        \underline{\mathbb N}\models(\exists x_1)\dots(\exists x_n)\ p(x_1,\dots,x_n)=q(x_1,\dots,x_n)
        $$
        \vspace{-16pt}
    }
    
    \pause
    \textbf{Důkaz:} \alert{\href{https://en.wikipedia.org/wiki/Lagrange\%27s_four-square_theorem}{Lagrangeova věta o čtyřech čtvercích}} říká, že každé přirozené číslo lze vyjádřit jako součet čtyř čtverců (celých čísel). Naopak, každé celé číslo je rozdíl dvou přirozených. Diofantickou rovnici lze tedy transformovat na rovnici z důsledku, a naopak.\hfill\qedsymbol

\end{frame}


\begin{frame}{Důkaz nerozhodnutelnosti predikátové logiky}

    \pause
    Uvažme $\varphi$ tvaru $(\exists x_1)\dots(\exists x_n)\ p(x_1,\dots,x_n)=q(x_1,\dots,x_n)$ 
    kde $p$ a $q$ jsou přirozené polynomy. Dle Tvrzení o Robinsonově aritmetice:
    $$
    \underline{\mathbb N}\models \varphi\ \Leftrightarrow\  Q\proves \varphi
    $$

    \pause
    Buď $\psi_Q$ konjunkce (gen. uzávěrů) axiomů $Q$ (je konečná). Zřejmě: 
    $$
    \alert{Q\proves\varphi}\ \Leftrightarrow\ \psi_Q\proves\varphi\ \Leftrightarrow\ \alert{\proves\psi_Q\limplies\varphi}
    $$
    \pause
    Dle Věty o úplnosti je to ale ekvivalentní \alert{$\models\psi_Q\limplies\varphi$}. Dostáváme:
    $$
    \underline{\mathbb N}\models \varphi\ \Leftrightarrow\ \models \psi_Q\limplies\varphi
    $$
    \pause
    \alert{Sporem:} Pokud bychom měli algoritmus rozhodující logickou platnost, mohli bychom rozhodovat i existenci přirozeného řešení rovnice $p(x_1,\dots,x_n)=q(x_1,\dots,x_n)$, tj. Hilbertův 10. problém.\hfill\qedsymbol

\end{frame}


\section{10.4 Gödelovy věty}


\begin{frame}{První věta o neúplnosti + důsledek o nekompletnosti}

    \pause
    \myblock{
        \textbf{Věta (Gödel 1931):}
        Je-li $T$ bezesporná rekurzivně axiomatizovaná extenze Robinsonovy aritmetiky, potom existuje sentence, která je pravdivá v~$\underline{\mathbb N}$, ale není dokazatelná v $T$.
    }

    \pause
    \begin{itemize}
        \item vlastnosti aritmetiky přir. čísel nelze `rozumně', efektivně popsat (v logice 1. řádu), takový popis je nutně `neúplný'\pause
        \item \alert{pravdivost} je ve standardním modelu $\underline{\mathbb N}$ zatímco \alert{dokazatelnost} v $T$ (samozřejmě pravdivá v $T$ je v $T$ i dokazatelná)\pause
        \item \alert{bezespornost} nutná (sporná teorie dokáže vše)\pause
        \item bez \alert{rekurzivní axiomatizovatelnosti} by teorie nebyla `užitečná'\pause
        \item extenze $Q$ znamená `základní aritmetická síla' (různé varianty předpokladu; nelze-li zakódovat přir. čísla s $+,\cdot$ je moc `slabá'
    \end{itemize}    

    \pause
    \myblock{
        \textbf{Důsledek:}
        Splňuje-li teorie $T$ předpoklady První věty o neúplnosti a je-li navíc $\underline{\mathbb N}$ modelem $T$, potom $T$ není kompletní.
    }
    
    \pause
    \textbf{Důkaz:}
        Vezměme Gödelovu sentenci $\varphi$ ($\underline{\mathbb N}\models\varphi$, $T\not\proves\varphi$). Je-li $T$ kompletní, víme $T\proves\neg\varphi$, z korektnosti $T\models\neg\varphi$, tedy $\underline{\mathbb N}\models\neg\varphi$.
    \hfill\qedsymbol    

\end{frame}


\begin{frame}{O důkazu}

    \begin{itemize}[<+->]
        \item Gödelova sentence formalizuje \alert{``Nejsem dokazatelná v $T$''}
        \item převratná důkazová technika, dva hlavní principy:
        \item \alert{aritmetizace syntaxe}, zakódování sentencí a jejich dokazatelnosti do přirozených čísel
        \item \alert{self-reference}, sentence `mluví sama o sobě' (o svém kódu)
        \item všechny technické detaily vynecháme, viz např. V. Švejdar: \emph{Logika -- neúplnost, složitost a nutnost}, Academia 2002
    \end{itemize}
    
\end{frame}


\begin{frame}{Aritmetizace syntaxe a dokazatelnosti}
    
    \begin{itemize}[<+->]
        \item \alert{Gödelovo číslování} `rozumně' kóduje konečné syntaktické objekty (termy, formule, tablo důkazy) do $\mathbb N$: lze algoritmicky [de-]kódovat, simulovat `manipulaci' s objekty na jejich kódech
        \item pro $\varphi$ bude \alert{$\lceil\varphi\rceil$} příslušný kód, \alert{$\underline{\varphi}$} odpovídající $\lceil\varphi\rceil$-tý numerál
        \item pro danou $T$ máme binární relaci $\MyProof_T\subseteq\mathbb N^2$ definovanou
        \hspace{-1cm}\myalertinline{
            $(n,m)\in\MyProof_T$ $\Leftrightarrow$
        $n=\lceil\varphi\rceil$, $m=\lceil\tau\rceil$, $\tau$ je tablo důkaz $\varphi$ z $T$
        }
        \item je-li $T$ rek. axiomatizovaná, je relace $\MyProof_T\subseteq\mathbb N^2$ \alert{rekurzivní} (lze algoritmicky ověřit korektnost tabla, tj. $(n,m)\in\MyProof_T$)
        \item klíčovou technickou částí důkazu První věty je fakt, že relaci $\MyProof_T$ lze \alert{reprezentovat} predikátem v Robinsonově aritmetice

    \end{itemize}    

\end{frame}


\begin{frame}{Predikát dokazatelnosti}

    \pause
    \myblock{
        \textbf{Tvrzení:}
        Je-li $T$ rekurzivně axiomatizovaná extenze Robinsonovy aritmetiky, potom existuje formule $\Prf_T(x,y)$ v jazyce aritmetiky, která \alert{reprezentuje} relaci $\MyProof_T$, tj. pro každá $n,m\in\mathbb N$:\pause
    \begin{itemize}
        \item je-li $(n,m)\in\MyProof_T$, potom $Q\proves\Prf_T(\underline{n},\underline{m})$\pause
        \item jinak $Q\proves\neg\Prf_T(\underline{n},\underline{m})$
    \end{itemize} 
    }

    \pause
    (Důkaz vynecháme!)

    \pause
    \begin{itemize}
        \item formule \myalertinline{
            $\Prf_T(x,y)$
            } vyjadřuje \alert{``$y$ je důkaz $x$ v $T$''}\pause
        \item formule \myalertinline{
            $(\exists y)\Prf_T(x,y)$
            } znamená \alert{``$x$ je dokazatelná v $T$''}\pause
        \item svědek poskytuje kód tablo důkazu, a $\underline{\mathbb N}$ splňuje $Q$, proto:
    \end{itemize}

    \pause
    \myblock{
        \textbf{Pozorování:} $T\proves\varphi$ právě když $\underline{\mathbb N}\models (\exists y)\Prf_T(\underline{\varphi},y)$.  
    }
    
    \pause
    Budeme potřebovat následující důsledek (také bez důkazu):
    
    \pause
    \myblock{
    \textbf{Důsledek:}
        Je-li $T\proves\varphi$, potom $T\proves (\exists y)\Prf_T(\underline{\varphi},y)$.
    }
    
\end{frame}


\begin{frame}[fragile]{Self-reference}

    \pause
    \vspace{-6pt}
    vyjádřili jsme \myalertinline{
        $\varphi$ je dokazatelná
    } ale chceme \myalertinline{já nejsem dokazatelná}
    
    \pause
    přirozené jazyky mají self-referenci:
    \myexampleinline{\texttt{Tato věta má 22 znaků.}};  
    formální systémy obvykle ne, umožňují ale \alert{přímou referenci} (mluvit o posloupnostech symbolů):
    
    \medskip

    \pause
    \myexample{
        \texttt{Následující věta má 29 znaků. "Následující věta má 29 znaků."}
    }

    \medskip

    \pause
    zde není žádná self-reference, pomůžeme si proto trikem \alert{zdvojení}:
    
    \medskip

    \pause
    \myexample{   
        \texttt{Následující věta zapsaná jednou a ještě jednou v uvozovkách má 149 znaků. "Následující věta zapsaná jednou a ještě jednou v uvozovkách má 149 znaků."}
    }  

    \medskip

    \pause
    \myalertinline{přímou referencí a zdvojením tedy získáme self-referenci}; \pause podobně program v C, který vypíše svůj kód (34 je ASCII kód uvozovek): 

    \vspace{-6pt}
    {\small
    \begin{verbatim}
main(){char *c="main(){char *c=%c%s%c; printf(c,34,c,34);}"; 
printf(c,34,c,34);}  
    \end{verbatim}
    }    

\end{frame}


\begin{frame}{Věta o pevném bodě}
    
    \pause
    \myblock{
        \textbf{Věta:}
        Je-li $T$ extenzí Robinsonovy aritmetiky, potom pro každou formuli $\varphi(x)$ (v jazyce teorie $T$) existuje sentence $\psi$ taková, že: 

        \vspace{-9pt}
        $$
        T\proves \psi \liff \varphi(\underline{\psi})
        $$
    }

    \pause
    \begin{itemize}
        \item také ``diagonalizační lemma'' nebo ``self-referenční'' lemma\pause
        \item $\psi$ je \alert{self-referenční}, říká o sobě: \myalertinline{``já splňuji vlastnost $\varphi$''}\pause
        \item v důkazu První věty bude $\varphi(x)$ formule $\neg(\exists y)\Prf_T(x,y)$\pause
        %\item důkaz využívá diagonalizační argument (jako Holičův paradox)
        \item všimněte si, jak se v důkazu použije přímá reference a zdvojení
    \end{itemize}

    \pause
    \textbf{Důkaz (myšlenka):} \alert{Zdvojující funkce} $d\colon\mathbb N\to\mathbb N$ dekóduje vstup $n$ jako $\varphi(x)$, dosadí numerál $\underline{n}$, znovu zakóduje: pro vš. $\chi(x)$ platí:
    $$
    \alert{d(\lceil \chi(x)\rceil)=\lceil\chi(\underline{\chi(x)})\rceil}
    $$
    \pause
    S využitím $T$ extenze $Q$ se dokáže, že $d$ je v $T$ \alert{reprezentovatelná}. \myexampleinline{Pro jednoduchost ať ji reprezentuje term}, označíme ho také $d$ (ale ve skutečnosti je to složitá formule).
    
\end{frame}


\begin{frame}{Pokračování důkazu}

    \pause
    Tedy $Q$, proto i $T$, dokazuje \alert{o numerálech}, že $d$ opravdu `zdvojuje':
    $$
    T\proves d(\underline{\chi(x)})=\underline{\chi(\underline{\chi(x)})}
    $$
        
    \pause
    Hledaná self-referenční sentence $\psi$ je sentence:

        \myalertmath{
        $$
        \varphi(d(\underline{\varphi(d(x))}))
        $$
        }

    \bigskip

    \pause
    Chceme dokázat, že $T\proves \psi \liff \varphi(\underline{\psi})$, neboli:
    $$
    T \proves \varphi(\alert{d(\underline{\varphi(d(x))})})\liff\varphi(\alert{\underline{\varphi(d(\underline{\varphi(d(x))}))}})
    $$
    
    \pause
    K~tomu stačí \myalertinline{
        $T \proves d(\underline{\varphi(d(x))})=\underline{\varphi(d(\underline{\varphi(d(x))}))}$
    } což máme z reprezentovatelnosti $d$, kde $\chi(x)$ je $\varphi(d(x))$.\hfill\qedsymbol

    \pause
    {\footnotesize
    $\psi$ tedy říká: >>Následující věta zapsaná jednou a ještě jednou v uvozovkách má vlastnost $\varphi$. ``Následující věta zapsaná jednou a ještě jednou v uvozovkách má vlastnost $\varphi$.''<< kde v uvozovkách znamená numerál kódu (přímá reference)
    }

\end{frame}


\begin{frame}{Nedefinovatelnost pravdy}

    \pause
    \myblock{
        \textbf{Věta:}
        V žádném bezesporném rozšíření Robinsonovy aritmetiky nemůže existovat definice pravdy.
    }
    \vspace{-12pt}
    \pause
    \begin{itemize}
        \item \alert{definice pravdy} v aritmetické teorii $T$ je formule $\tau(x)$ taková, že pro každou sentenci $\psi$ platí: 
        \myalertinline{
            $T\proves\psi\liff\tau(\underline{\psi})$
        }\pause
        \item kdyby existovala, místo dokazování by stačilo spočíst kód $\lceil \psi\rceil$, dosadit numerál $\underline{\psi}$ do $\tau$, a vyhodnotit\pause
        \item rozcvička pro důkaz Gödelovy První věty o neúplnosti\pause
        \item důkaz užívá \alert{Paradox lháře}, vyjádříme \myalertinline{``Nejsem pravdivá v $T$''}\pause
        \item důkaz První věty užívá stejný trik s ``Nejsem dokazatelná v $T$''\pause
    \end{itemize}
    \vspace{-3pt}
    \textbf{Důkaz:} \pause Sporem, ať existuje definice pravdy $\tau(x)$. Z Věty o pevném bodě kde $\varphi(x)$ je $\neg\tau(x)$ dostáváme sentenci $\psi$ takovou, že:
    $$
    T\proves\psi\liff\neg\tau(\underline{\psi})
    $$
    \pause
    Protože $\tau(x)$ je definice pravdy, platí ale i $T\proves\psi\liff\tau(\underline{\psi})$, tedy i $T\proves\tau(\underline{\psi})\liff\neg\tau(\underline{\psi})$. To by ale znamenalo, že $T$ je sporná.
    \hfill\qedsymbol
    
\end{frame}


\begin{frame}{Důkaz První věty o neúplnosti}

    \pause
    $T$ bezesp. rek. ax. ext. $Q$. Gödelovu sentenci ($\underline{\mathbb N}\models\psi_T,T\not\proves\psi_T$) získáme z Věty o pevném bodě kde $\varphi(x)$ je \alert{$\neg(\exists y)\Prf_T(x,y)$}:

    \pause
    \myalertmath{
    $$
    T\proves\psi_T\liff\neg(\exists y)\Prf_T(\underline{\psi_T},y)
    $$
    }

    \pause
    Tedy $\psi_T$ je v $T$ ekvivalentní ``$\psi_T$ není dokazatelná v $T$''. Ekvivalence platí i v $\underline{\mathbb N}$ (z konstrukce, protože $\underline{\mathbb N}$ splňuje $Q$), a spolu s ekvivalencí z Pozorování o predikátu dokazatelnosti: \pause
    $$
    \alert{\underline{\mathbb N}\models\psi_T}\ \Leftrightarrow\ 
    \underline{\mathbb N}\models\neg(\exists y)\Prf_T(\underline{\psi_T},y)\ \Leftrightarrow\ \alert{T\not\proves\psi_T}
    $$    

    \pause
    Stačí tedy ukázat nedokazatelnost $\psi_T$ v $T$. \alert{Sporem: ať $T\proves\psi_T$. }
    
    \pause
    \begin{itemize}
        \item Self-reference: $T\proves\neg(\exists y)\Prf_T(\underline{\psi_T},y)$\pause
        \item Důsledek o predikátu dokazatelnosti: $T\proves (\exists y)\Prf_T(\underline{\psi_T},y)$\pause
    \end{itemize}
    To by ale znamenalo, že $T$ je sporná.\hfill\qedsymbol
    
\end{frame}


\begin{frame}{Důsledky a zesílení}
    
    \pause
    \myblock{
        \textbf{Důsledek (už byl):}
        Je-li $T$ rekurzivně axiomatizovaná extenze Robinsonovy aritmetiky a je-li $\underline{\mathbb N}$ model $T$, potom $T$ není kompletní.        
    }
    
    \pause
    \textbf{Důkaz:}
        $T$ není sporná, tedy splňuje předpoklady První věty. Víme, že G. sentence splňuje $\underline{\mathbb N}\models\psi_T$ a $T\not\proves\psi_T$. Je-li $T$ kompletní, máme $T\proves\neg\psi_T$, z korektnosti $T\models\neg\psi_T$, tj. $\underline{\mathbb N}\models\neg\psi_T$, spor.
    \hfill\qedsymbol
    
    \medskip

    \pause
    \myblock{\textbf{Důsledek:}
    Teorie $\Th(\underline{\mathbb N})$ není rekurzivně axiomatizovatelná.    
    }

    \pause
    \textbf{Důkaz:}
    $\Th(\underline{\mathbb N})$ je extenze $Q$, platí v $\underline{\mathbb N}$. Kdyby byla rekurzivně axiomatizovatelná, podle Důsledku by [její rekurzivní axiomatizace] nebyla kompletní, ale je.
    \hfill\qedsymbol

    \medskip

    \pause
    Zesílení První věty: předpoklad $\underline{\mathbb N}\models T$ v Důsledku je nadbytečný.

    \pause
    \myblock{\textbf{Věta (Rosserův trik, 1936):}
    V bezesporné rekurzivně axiomatizované extenzi Robinsonovy aritmetiky existuje nezávislá sentence.  
    }

    (Bez důkazu.)

\end{frame}


\begin{frame}{Gödelova Druhá věta o neúplnosti}
    
    \pause
    \myalert{
        Efektivně daná, dostatečně bohatá $T$ nedokáže svou bezespornost.
    }

    \pause
    \begin{itemize}
        \item bezespornost vyjádří sentence \alert{$\Con_T$}:\hfill \myalertinline{\small
            $\neg(\exists y)\Prf_T(\underline{0=S(0)},y)$
        } \pause 
        \item všimněte si: $\underline{\mathbb N}\models\Con_T$ $\Leftrightarrow$ $T\not\proves 0=S(0)$\pause
        \item tj. $\Con_T$ opravdu vyjadřuje, že \alert{``$T$ je bezesporná''}
    \end{itemize}

    \bigskip
    \pause
    \myblock{\textbf{Věta (Gödel, 1931):} Je-li $T$ bezesporná rekurzivně axiomatizovaná extenze $PA$, potom $\Con_T$ není dokazatelná v $T$.
    }

    \medskip

    \pause
    \begin{itemize}
        \item všimněte si: \alert{$\Con_T$ je pravdivá v $\underline{\mathbb N}$} (neboť $T$ je bezesporná)\pause
        \item není třeba plná síla $PA$, stačí slabší předpoklad\pause
        \item ukážeme si hlavní myšlenku důkazu
    \end{itemize}
    
\end{frame}


\begin{frame}{Myšlenka důkazu}

    \pause
    Gödelova sentence $\psi_T$ vyjadřuje: \myalertinline{``Nejsem dokazatelná v $T$.''}
    
    \pause
    V důkazu První věty o neúplnosti jsme ukázali:

    \myalert{
        ``Pokud je $T$ bezesporná, potom $\psi_T$ není dokazatelná v $T$.''
    }
    
    \pause
    Z toho jednak plyne, že $T\not\proves\psi_T$, neboť $T$ bezesporná je. 
    
    \pause
    Na druhou stranu to lze formulovat jako: \myalertinline{``Platí $\Con_T\to \psi_T$.''}
    
    \pause
    Je-li $T$ extenze Peanovy aritmetiky, lze důkaz tohoto tvrzení zformalizovat v rámci teorie $T$, tedy ukázat, že:
    $$
    \alert{ T\proves\Con_T\to\psi_T}
    $$
    \pause
    Kdyby platilo $T\proves\Con_T$, dostali bychom i $T\proves\psi_T$, což je spor.
    \hfill\qedsymbol

\end{frame}


\begin{frame}{Důsledky}    

    \pause
    \myblock{\textbf{Důsledek:}
        $PA$ má model, ve kterém platí $(\exists y)\Prf_{PA}(\underline{0=S(0)},y)$.
    }

    \pause
    \textbf{Důkaz:}
        Sentence $\Con_{PA}$ není dokazatelná, tedy ani pravdivá v $PA$. Platí ale v $\underline{\mathbb N}$ (neboť $PA$ je bezesporná), což znamená, že je $\Con_{PA}$ nezávislá v $PA$. V nějakém modelu tedy musí platit její negace, která je ekvivalentní $(\exists y)\Prf_{PA}(\underline{0=S(0)},y)$.            
    \hfill\qedsymbol

    \pause
    \textbf{Poznámka:} Musí to být nestandardní model $PA$, svědek \alert{nestandardní} prvek (není hodnotou žádného numerálu).

    \bigskip

    \pause
    \myblock{\textbf{Důsledek:}
        $PA$ má bezespornou rekurzivně axiomatizovanou extenzi, která ``dokazuje svou spornost'', tj. $T\proves \neg \Con_T$.
    }

    \pause
    \textbf{Důkaz:}
    $T=PA \cup \{\neg \Con_{PA}\}$ je \alert{bezesporná}, neboť $PA\not\proves\Con_{PA}$. Také triviálně $T\proves\neg\Con_{PA}$, tj. $T$ `dokazuje spornost' $PA$. Protože $PA\subseteq T$, platí i $T\proves\neg\Con_T$.
    \hfill\qedsymbol

    \pause
    \textbf{Poznámka:} $\underline{\mathbb{N}}$ nemůže být modelem $T$.
    
\end{frame}


\begin{frame}{Bezespornost ZFC}

    \pause
    Formalizace matematiky je založena na \href{https://en.wikipedia.org/wiki/Zermelo\%E2\%80\%93Fraenkel_set_theory}{\alert{Zermelově–Fraenkelově teorii množin s axiomem výběru (ZFC)}}. Formálně vzato to není extenze $PA$, ale můžeme v ní Peanovu aritmetiku `interpretovat'.

    \pause
    \myblock{\textbf{Důsledek:}
        Je-li ZFC bezesporná, není $\Con_{ZFC}$ v ZFC dokazatelná.
    }

    \pause
    Pokud by tedy někdo v rámci ZFC dokázal, že je ZFC bezesporná, znamenalo by to, že je ZFC sporná.    

\end{frame}


\end{document}
