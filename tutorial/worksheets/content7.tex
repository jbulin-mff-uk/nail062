\section*{NAIL062 P\&P Logic: Worksheet 7 -- Properties of structures and theories}
% after Lecture 7


\subsection*{Teaching goals:} After completing, the student

    \begin{itemize}\setlength{\itemsep}{0pt}
        \item understands the notion of substructure, generated substructure, can find them
        \item understands the notion of expansion and reduct of a structure, can define them formally and give examples
        \item understands the notions of [simple, conservative] extension, can formulate the definitions and the corresponding semantic criterion (for both expansions and reducts), and apply it to an example
        \item understands the notion of extension by definition, can define it formally and give examples
        \item can decide whether a given theory is a extension by definition, construct an extension by a given definition
        \item understands the notion of definability in a structure, can find definable subsets/relations
    \end{itemize}

    

\section*{In-class problems}


\begin{problem}

    Consider $\underline{\mathbb Z_4}=\langle\{0,1,2,3\};+,-,0 \rangle$ where $+$ is binary addition modulo $4$ and $-$ is the unary function returning the \emph{inverse} for $+$ with respect to the \emph{neutral} element $0$.
    \begin{enumerate}[(a)]      
        \item Is $\underline{\mathbb Z_4}$ a model of the theory of groups (i.e. is it a \emph{group})?
        \item Determine all substructures $\underline{\mathbb Z_4}\langle a\rangle$ generated by some $a\in \mathbb Z_4$.
        \item Does $\underline{\mathbb Z_4}$ contain any other substructures?
        \item Is every substructure of $\underline{\mathbb Z_4}$ a model of the theory of groups?
        \item Is every substructure of $\underline{\mathbb Z_4}$ elementarily equivalent to $\underline{\mathbb Z_4}$?
        %\item Is every substructure of a \emph{commutative} group (i.e. a group satisfying $x+y=y+x$) also a commutative group?
    \end{enumerate}

    \begin{solution}

        \begin{enumerate}[(a)]
            \item Yes, one can check that $\underline{\mathbb Z_4}$ satisfies all axioms of the theory of groups ($+$ is associative, $0$ is neutral for $+$, $-x$ is the inverse of $x$ w.r.t.\ $+$ and $0$).
            \item $\underline{\mathbb Z_4}\langle 0\rangle=\underline{\mathbb Z_4}\restriction\{0\}$ (\emph{the trivial group}), $\underline{\mathbb Z_4}\langle 1\rangle=\underline{\mathbb Z_4}\langle 3\rangle=\underline{\mathbb Z_4}$, $\underline{\mathbb Z_4}\langle 2\rangle=\underline{\mathbb Z_4}\restriction\{0,2\}$ (a two-element group \emph{isomorphic} to $\underline{\mathbb Z_2}$).
            \item No, as soon as we have the element $1$ or $3$, the generated substructure is the whole $\underline{\mathbb Z_4}$.
            \item Yes, the theory of groups is \emph{universal} (closed under substructures), hence substructures of models (groups) are also models (\emph{subgroups}).
            \item No, the language of group theory is with equality, and any finite model size can be expressed by a sentence, so finite models of different sizes cannot be elementarily equivalent. However, we do not even need to express model size directly. It suffices to use ``group properties'' to distinguish them: e.g. the sentence $(\forall x)x=0$ distinguishes the trivial group $\underline{\mathbb Z_4}\restriction\{0\}$ from the two-element group $\underline{\mathbb Z_4}\restriction\{0,2\}$ and from $\underline{\mathbb Z_4}$; and e.g. $(\forall x)x+x=0$ is valid in $\underline{\mathbb Z_4}\restriction\{0,2\}$ but not in $\underline{\mathbb Z_4}$.
        \end{enumerate}
                    
    \end{solution}

\end{problem}


\begin{problem}

    Let $\underline{\mathbb{Q}}=\langle\mathbb{Q};+,-,\cdot,0,1 \rangle$ be the field of rationals with the standard operations.
    \begin{enumerate}[(a)]                
        \item Is there a reduct of $\underline{\mathbb{Q}}$ that is a model of the theory of groups?
        \item Can the reduct $\langle\mathbb{Q},\cdot,1\rangle$ be extended to a model of the theory of groups?
        \item Does $\underline{\mathbb{Q}}$ contain a substructure that is not elementarily equivalent to $\underline{\mathbb{Q}}$?
        \item Let $\mathrm{Th}(\underline{\mathbb{Q}})$ denote the set of all sentences true in $\underline{\mathbb{Q}}$. Is $\mathrm{Th}(\underline{\mathbb{Q}})$ a complete theory?
    \end{enumerate}

    \begin{solution}

        \begin{enumerate}[(a)]
            \item Yes, $\underline{\mathbb{Q}}=\langle\mathbb{Q};+,-,0\rangle$ (the additive group reduct).
            \item No, the element $1$ (interpretation of the symbol $0$ of the language of groups) is not a neutral element with respect to $\cdot$ (the interpretation of $+$), because $1\cdot 0=0\neq 1$.
            \item Yes, for example $\underline{\mathbb{Q}}\restriction\mathbb Z =\langle\mathbb{Z};+,-,\cdot,0,1 \rangle$ (the ring of integers): in it not every nonzero element has a multiplicative inverse, so the sentence $(\forall x)(\neg x=0\to (\exists y)x\cdot y=1)$ fails (e.g. $2$ has no inverse in $\mathbb Z$, while it does in $\mathbb Q$). (From this it follows that the theory of fields is not openly axiomatizable, otherwise substructures of fields would be fields.)
            \item Yes, the so-called \emph{theory of a structure} is always complete: for every sentence $\psi$ either $\mathrm{Th}(\underline{\mathbb{Q}})\models\psi$ or $\mathrm{Th}(\underline{\mathbb{Q}})\models\neg\psi$, because $\underline{\mathbb{Q}}\models\psi$ or $\underline{\mathbb{Q}}\models\neg\psi$.
        \end{enumerate}
                    
    \end{solution}

\end{problem}



\begin{problem}

    Consider the theory $T=\{x=c_1 \vee x=c_2 \vee x=c_3\}$ in the language $L=\langle c_1,c_2,c_3\rangle$ with equality.
    \begin{enumerate}[(a)]     
        \item Is $T$ complete?
        \item How many simple extensions of $T$ are there, up to equivalence? How many are complete? Write down all complete ones and at least three incomplete ones.
        \item Is the theory $T'=T\cup\{x=c_1 \vee x=c_4\}$ in the language $L'=\langle c_1,c_2,c_3,c_4\rangle$ an extension of $T$? Is $T'$ a simple extension of $T$? Is $T'$ a conservative extension of $T$?
    \end{enumerate}

    \begin{solution}
        
        The theory says that every element is one of the three constants. But these constants need not be distinct. First find all models up to isomorphism; there are five (draw them):
        \begin{itemize}
            \item $\mathcal A_1=\langle\{0\};0,0,0\rangle$ (one-element model, $c_1^{\mathcal A_1}=c_2^{\mathcal A_1}=c_3^{\mathcal A_1}=0$)
            \item $\mathcal A_2=\langle\{0,1\};0,0,1\rangle$ (two-element model, $c_1^{\mathcal A_2}=c_2^{\mathcal A_2}\neq c_3^{\mathcal A_2}$)
            \item $\mathcal A_3=\langle\{0,1\};0,1,1\rangle$ (two-element model, $c_1^{\mathcal A_3}\neq c_2^{\mathcal A_3}=c_3^{\mathcal A_3}$) 
            \item $\mathcal A_4=\langle\{0,1\};0,1,0\rangle$ (two-element model, $c_1^{\mathcal A_4}=c_3^{\mathcal A_4}\neq c_2^{\mathcal A_4}$)
            \item $\mathcal A_5=\langle\{0,1,2\};0,1,2\rangle$ (three-element model, constants are distinct)
        \end{itemize}
        \begin{enumerate}[(a)]
            \item It is not complete; for example the sentence $c_1=c_2$ is independent in $T$: it is valid in $\mathcal A_1$ but not in $\mathcal A_3$. (Equivalently, by the semantic criterion, models $\mathcal A_1$ and $\mathcal A_3$ are not elementarily equivalent.)
            \item Simple extensions correspond to subsets of $\{\mathcal A_1,\mathcal A_2,\mathcal A_3,\mathcal A_4,\mathcal A_5\}$, there are $2^5=32$ of them; complete ones correspond to singletons (complete theories of individual models), so there are 5.
            
            Simple extensions that are not complete:
            \begin{itemize}
                \item $T$ \hfill models $\mathcal A_1,\mathcal A_2,\mathcal A_3,\mathcal A_4,\mathcal A_5$
                \item $T\cup\{x=y\lor x=z\}$ \hfill models $\mathcal A_1,\mathcal A_2,\mathcal A_3,\mathcal A_4\phantom{,\mathcal A_5}$
                \item $T\cup\{(\exists x)(\exists y)\neg x=y\}$ \hfill models $\phantom{\mathcal A_1,}\mathcal A_2,\mathcal A_3,\mathcal A_4,\mathcal A_5$\\
                (Note: $(\exists x)(\exists y)\neg x=y\sim\neg(\forall x)(\forall y)x=y\not\sim\neg x=y\sim(\forall x)(\forall y)\neg x=y$.)
                
                \item[\vdots]
                
                \item $\{x=x\land\neg x=x\}$ \hfill the inconsistent theory, has no model
            \end{itemize}

            Complete simple extensions:
            \begin{itemize}
                \item $\mathrm{Th}(\mathcal A_1)\sim\{x=y\}$
                \item $\mathrm{Th}(\mathcal A_2)\sim\{(\exists x)(\exists y)\neg x=y,x=y\lor x=z,c_1=c_2,\neg c_2=c_3\}$
                \item $\mathrm{Th}(\mathcal A_3)\sim\{(\exists x)(\exists y)\neg x=y,x=y\lor x=z,\neg c_1=c_2,c_2=c_3\}$
                \item $\mathrm{Th}(\mathcal A_4)\sim\{(\exists x)(\exists y)\neg x=y,x=y\lor x=z,c_1=c_3,\neg c_1=c_2\}$
                \item $\mathrm{Th}(\mathcal A_5)\sim\{x=c_1 \vee x=c_2 \vee x=c_3,\neg (c_1=c_2\lor c_1=c_3\lor c_2=c_3)\}$
            \end{itemize}    
            
            \item The theory additionally says that every element is either the interpretation of $c_1$ or $c_4$. Thus models have at most two elements; up to isomorphism they are:
            \begin{itemize}
                \item $\mathcal A_1'=\langle\{0\};0,0,0,0\rangle$
                \item $\mathcal A_2'=\langle\{0,1\};0,0,1,1\rangle$
                \item $\mathcal A_3'=\langle\{0,1\};0,1,1,1\rangle$
                \item $\mathcal A_4'=\langle\{0,1\};0,1,0,1\rangle$                
            \end{itemize}
            The theory $T'$ is an extension of $T$. All consequences of $T$ are valid in $T'$; semantically: the reducts of models of $T'$ to the original language $L$ are models of $T$ (e.g. the reduct of $\mathcal A_1'$ to $L$ is $\mathcal A_1$). It is not a simple extension, since we enlarged the language.
            
            It is also not a conservative extension: for example the sentence $(\forall x)(\forall y)(\forall z)(x=y\lor x=z)$ is a sentence in the original language $L$, it is valid in $T'$ but it was not valid in $T$. Semantically: the three-element model $\mathcal A_5$ of $T$ cannot be expanded to an $L'$-structure that models $T'$, i.e. the reducts of models of $T'$ to $L$ do not yield all models of $T$.

        \end{enumerate}
                    
    \end{solution}

\end{problem}


\begin{problem}

    Let $T'$ be an extension of $T=\{(\exists y)(x+y=0),(x+y=0)\wedge (x+z=0)\rightarrow y=z\}$ in the language $L=\langle +,0,\le\rangle$ with equality by definitions of $<$ and unary $-$ with axioms
    \begin{align*}
        -x=y\ \ &\leftrightarrow\ \ x+y=0\\
        x<y\ \ &\leftrightarrow\ \ x\le y\ \wedge\ \neg(x=y)
    \end{align*}
    Find formulas in the language $L$ that are equivalent in $T'$ to the following formulas.
        
    (a) $(-x)+x=0$ \hfill (b) $x+(-y)<x$ \hfill (c) $-(x+y)<-x$\hfill{}

    \begin{solution}

        Note that the axioms express existence and uniqueness for the definition of the function symbol $-$, so this is a proper extension by definition. We proceed according to the (proof of the) claim from the lecture:
        \begin{enumerate}[(a)]
            \item $(\exists z)(x+z=0\land z+x=0)$ (The subformula $x+z=0$ says that `$z$ is $-x$' and the second subformula says that `$(-x)+x=0$'.)
            \item First replace the term $-y$ by its definition:
            $$
            (\exists z)(y+z=0\land x+z<x)
            $$
            Now replace the relation symbol $<$:
            $$
            (\exists z)(y+z=0\land x+z\leq z\land\neg(x+z=z))
            $$
            \item $(\exists u)(\exists v)((x+y)+u=0\land x+v=0\land u\leq v\land \neg u=v)$ (Where `$u$ is $-(x+y)$' and `$v$ is $-x$'.)
        \end{enumerate}
                    
    \end{solution}
    
\end{problem}


\begin{problem}

    Consider the language $L=\langle F \rangle$ with equality, where $F$ is a binary function symbol. Find formulas defining the following sets (without parameters):
    \begin{enumerate}[(a)]
        \item the interval $(0,\infty)$ in $\mathcal A=\langle\mathbb R, \cdot\rangle$ where $\cdot$ is multiplication of real numbers
        \item the set $\{(x, 1/x)\mid x\neq 0\}$ in the same structure $\mathcal A$
        \item the set of all singleton subsets of $\mathbb N$ in $\mathcal B=\langle\mathcal P(\mathbb N),\cup\rangle$
        \item the set of all prime numbers in $\mathcal C=\langle \mathbb N\cup\{0\}, \cdot\rangle$
    \end{enumerate}

    \begin{solution}

        \begin{enumerate}[(a)]
            \item $(\exists y)F(y,y)=x\land \neg (\forall y)F(x,y)=x$ (The number $x$ is a square, and it is not zero.)
            \item $(\exists z)(F(x,y)=z\land(\forall u)F(z,u)=u)$ (The product equals one.)
            \item $(\forall y)(\forall z)(F(y,z)=x\to y=x\lor z=x)\land\neg(\forall y)F(x,y)=y$ (Whenever the set is the union of two sets, it equals one of them. And it is not empty.)
            \item Same as (c), $(\forall y)(\forall z)(F(y,z)=x\to y=x\lor z=x)\land\neg(\forall y)F(x,y)=x$ (Whenever the product of two numbers equals a prime, one of them equals the prime, and a prime is not zero.)
        \end{enumerate}
                    
    \end{solution}

\end{problem}

        
        
\section*{Extra practice}


\begin{problem}

    Let $T=\{\neg E(x,x), E(x,y)\to E(y,x), (\exists x)(\exists y)(\exists z)(E(x,y)\wedge E(y,z)\wedge E(x,z)\wedge \neg(x=y\vee y=z\vee x=z)),\varphi\}$ be a theory in the language $L=\langle E\rangle$ with equality, where $E$ is a binary relation symbol and $\varphi$ expresses that “there are exactly four elements.”
    \begin{enumerate}[(a)]
        \item Consider the expansion $L'=\langle E,c\rangle$ of the language by a new constant symbol $c$. Determine the number (up to equivalence) of theories $T'$ in $L'$ that are extensions of $T$. 
        \item Does $T$ have any \emph{conservative} extension in the language $L'$? Justify your answer.
    \end{enumerate}

\end{problem}


\begin{problem}

    Let $T=\{x=f(f(x)),\varphi, \neg c_1 = c_2\}$ be a theory in the language $L=\langle f,c_1,c_2\rangle$ with equality, where $f$ is a unary function symbol, $c_1,c_2$ are constant symbols, and the axiom $\varphi$ expresses that “there are exactly three elements.”
    \begin{enumerate}[(a)]    
        \item Determine how many pairwise nonequivalent complete simple extensions the theory $T$ has. Write down two of them.
        \item Let $T'=\{x=f(f(x)),\varphi,\neg f(c_1)=f(c_2)\}$ be a theory in the same language, with $\varphi$ same as above. Is $T'$ an extension of $T$? Is $T$ an extension of $T'$? If so, is it a conservative extension? Provide justification.
    \end{enumerate}
    
\end{problem}


\begin{problem}

    Consider $L=\langle P,R,f,c,d\rangle$ with equality and the following two formulas:
    \begin{align*}
        \varphi:\quad P(x,y) &\leftrightarrow R(x,y) \wedge \neg x=y\\
        \psi:\quad P(x,y) &\to P(x,f(x,y)) \wedge P(f(x,y),y)
    \end{align*}
    Consider the following $L$-theory:
    \begin{align*}
        T=\{&\varphi,\ \psi,\ \neg c=d,\\
        &R(x,x),\\ 
        &R(x,y) \wedge R(y,x) \to x=y,\\
        &R(x,y) \wedge R(y,z) \to R(x,z),\\
        &R(x,y) \vee R(y,x)\}    
    \end{align*}
    

    \begin{enumerate}[(a)]
        \item Find an expansion of the structure $\langle \mathbb{Q},\le \rangle$ to the language $L$ that is a model of $T$.
        \item Is the sentence $(\forall x)R(c,x)$ valid/contradictory/independent in $T$? Justify all 3 answers.
        \item Find two nonequivalent complete simple extensions of $T$, or justify why they do not exist.
        \item Let $T'=T\setminus\{\varphi,\psi\}$ be a theory in the language $L'=\langle R,f,c,d\rangle$. Is the theory $T$ a conservative extension of the theory $T'$? Provide justification.
    \end{enumerate}

\end{problem}

        
\section*{For further thought}


\begin{problem}

    Let $T_n = \{\neg c_i = c_j \mid 1 \leq i < j \leq n\}$ be a theory in the language $L_n = \langle c_1, \dots, c_n \rangle$ with equality, where $c_1, \dots, c_n$ are constant symbols.
    \begin{enumerate}[(a)]   
        \item For a given finite $k \geq 1$, count $k$-element models of the theory $T_n$ up to isomorphism. 
        \item Determine the number of countable models of the theory $T_n$ up to isomorphism. 
        \item For which pairs of values $n$ and $m$ is $T_n$ an extension of $T_m$? For which pairs is it a conservative extension? Justify your answer.
    \end{enumerate}

\end{problem}




